%% This is file `DEMO-TUDaPhD.tex' version 3.21 (2022/01/12),
%% it is part of
%% TUDa-CI -- Corporate Design for TU Darmstadt
%% ----------------------------------------------------------------------------
%%
%%  Copyright (C) 2018--2021 by Marei Peischl <marei@peitex.de>
%%
%% ============================================================================
%% This work may be distributed and/or modified under the
%% conditions of the LaTeX Project Public License, either version 1.3c
%% of this license or (at your option) any later version.
%% The latest version of this license is in
%% http://www.latex-project.org/lppl.txt
%% and version 1.3c or later is part of all distributions of LaTeX
%% version 2008/05/04 or later.
%%
%% This work has the LPPL maintenance status `maintained'.
%%
%% The Current Maintainers of this work are
%%   Marei Peischl <tuda-ci@peitex.de>
%%   Markus Lazanowski <latex@ce.tu-darmstadt.de>
%%
% The development respository can be found at
%% https://github.com/tudace/tuda_latex_templates
%% Please use the issue tracker for feedback!
%%
%% If you need a compiled version of this document, have a look at
%% http://mirror.ctan.org/macros/latex/contrib/tuda-ci/doc
%% or at the documentation directory of this package (if installed)
%% <path to your LaTeX distribution>/doc/latex/tuda-ci
%% ============================================================================
%%
% !TeX program = pdflatex
%% TeX program = lualatex
%%

\documentclass[
%	draft,
	USenglish,
	ruledheaders=chapter,% Ebene bis zu der die Überschriften mit Linien abgetrennt werden, vgl. DEMO-TUDaPub
	class=book,% Basisdokumentenklasse. Wählt die Korrespondierende KOMA-Script Klasse
	thesis={
		% Für kleinere Abschlussarbeiten Siehe DEMO-TUDaThesis
		type=dr,
		dr=ing
	},
	accentcolor=7b,% Auswahl der Akzentfarbe
	custommargins=geometry,%true,% Ränder werden mithilfe von typearea automatisch berechnet
	marginpar=false,% Kopfzeile und Fußzeile erstrecken sich nicht über die Randnotizspalte
	%BCOR=5mm,%Bindekorrektur, falls notwendig
%	parskip=half-,%Absatzkennzeichnung durch Abstand vgl. KOMA-Script
	fontsize=11pt,%Basisschriftgröße laut Corporate Design ist mit 9pt häufig zu klein
%	logofile=example-image, %Falls die Logo Dateien nicht vorliegen
	instbox=true,				% Blendet Box mit FB/FG auf der Titelseite ein/aus
	pdfa=true,
	%	captions=oneline
	numbers=noenddot,			% Kein Punkt am Ende der Section-Nummerierung (KOMA-Script)
]{tudapub}

\geometry{
	reset,
	a4paper,
	twoside,
	outer=2.5cm,
	inner=3.0cm,
	top=2.0cm,
	bottom=2.0cm,
	includehead,
	includefoot,
	includemp,
	nomarginpar
}
%\usepackage{atbegshi}
%\newcommand\showtimer{%
%	\message{^^Jtimer: \the\numexpr\the\pdfelapsedtime*1000/65536\relax}%
%	\pdfresettimer}
%\AtBeginDocument{\showtimer}
%\AtBeginShipout {\showtimer}
%% Der folgende Block ist nur bei pdfTeX auf Versionen vor April 2018 notwendig
%\usepackage{iftex}
%\ifPDFTeX
%	\usepackage[utf8]{inputenc}%kompatibilität mit TeX Versionen vor April 2018
%\fi

%%%%%%%%%%%%%%%%%%%
%Sprachanpassung & Verbesserte Trennregeln
%%%%%%%%%%%%%%%%%%%d
%\usepackage[english, main=ngerman]{babel}
\usepackage[ngerman, main=USenglish]{babel}
\usepackage[autostyle]{csquotes}% Anführungszeichen vereinfacht
\usepackage{blindtext}
% Falls mit pdflatex kompiliert wird, wird microtype automatisch geladen, in diesem Fall muss diese Zeile entfernt werden, und falls weiter Optionen hinzugefügt werden sollen, muss dies über
% \PassOptionsToPackage{Optionen}{microtype}
% vor \documentclass hinzugefügt werden.
\usepackage{microtype}


%%%%%%%%%%%%%%%%%%%
%Literaturverzeichnis
%%%%%%%%%%%%%%%%%%%
\usepackage[
backend = biber,             		
natbib = true,                	% macht alte natbib-Befehle (\citep, \citet etc.) möglich
style = authoryear,       		% Autor-Jahr Zitierweisen (STANDARD)
%style = apa,
%bibstyle = authoryear,
%citestyle = authoryear,      
uniquelist = false,        		% Unterscheidung von gleichen Autoren+Jahren: Wenn identisch, ergänzt biblatex bei Zitierung 
% zunächst Namen bis maxcitenames erreicht, dann nimmt er Vornamen dazu. Befinden sich Einträge 
% mit "Cockburn, B." und welche mit "Cockburn, Bernardo" in der .bib Datei, meint Biblatex einen 
% Unterschied zu erkennen und gibt bei letzterem den Vornamen an. 
uniquename = false,      		% uniquename=false unterdrückt dies, sodass er dann a,b,c hinter die Jahreszahl hängt
maxcitenames = 2,         		% bei mehr als 2 Namen kürzt biblatex mit et al. ab
maxbibnames = 25,       		% im Literaturverzeichnis keine Abkürzung mit et al. (bzw. erst ab 25 Autoren)
firstinits = true,              % Nutzt Initialien statt vollem Namen im Literaturverzeichnis
dashed = false,                	% kommt ein Autor im Verzeichnis öfter vor, wird als default bei den nächsten Papern eine Linie als 
% Platzhalter eingefügt statt nochmal der Name. dashed=false lässt alle Namen wiederholen                        
date = year,                    % Als Datumsangabe in Klammern im Verzeichnis nur das Jahr und nicht das volle Datum verwenden
url = false,                    % Schaltet url-Angabe aus
doi = true,              		% Schaltet doi-Angabe aus
isbn = false,             		% Schaltet isbn-Angabe aus
abbreviate = true,				% Ob Aufl. oder Auflage im Literaturverzeichnis stehen soll. Gilt für auch für die anderen Abkürzungen. 
%	sorting=nyt,					% Sort by name, year, title
maxalphanames=1,				% Sort by name of first author, year, title
labelalpha,
sorting=anyt,					
]{biblatex}            
% Abstand zwischen Einträgen in der BIBLIOGRAPHY
\setlength\bibitemsep{1.1\itemsep}			% length between two different entries, preset to \itemsep
%\setlength\bibnamesep{1.5\itemsep} 		% length between two entries of different authors
%\setlength\bibinitsep{1.5\itemsep} 		% length between two entries of authors with different first letter 


\DeclareNameAlias{sortname}{last-first}			% sortiert im Literaturverzeichnis alle Namen mit erst last name, dann first name. 
% Default wäre: erster Autor last-first, alle anderen first-last

%\addbibresource{bibliography.bib}

\bibliography{bibliography}
%\nocite{*}  
%\bibliography{DEMO-TUDaBibliography}

%%%%%%%%%%%%%%%%%%%%%%%%%%%%%
%Fussnoten
%%%%%%%%%%%%%%%%%%%%%%%%%%%%%
\counterwithout{footnote}{chapter}	% Durchgaengige Nummerierung

%%%%%%%%%%%
%Figures & Captions
%%%%%%%%%%%
\definecolor{gray140}{RGB}{140,140,140}
\definecolor{fdygray}{RGB}{173,173,173}
\definecolor{fdyyellow}{RGB}{245,163,0}

\usepackage{subcaption}

% Caption OHNE Serifen, SMALL
\captionsetup{textfont={sf,small}, labelfont={bf,sf,small}}
\captionsetup[sub]{textfont={sf,small}, labelfont={sf,small}}
\captionsetup[algorithm]{textfont={sf,small}, labelfont={bf,sf,small}, labelsep=colon}

\usepackage{ragged2e}

%%%%%%%%%%%%%%%%%%%%%%%%%%%
%Paketvorschläge Mathematik
%%%%%%%%%%%%%%%%%%%%%%%%%%%
\usepackage{mathtools} 	% erweiterte Fassung von amsmath
\usepackage{amssymb}   	% erweiterter Zeichensatz
\usepackage{bm}			% bold math symbols (\boldsymbol)
\usepackage{dsfont}

\usepackage{amsmath}
\usepackage{amsfonts}
%%%%%%%%%%%%%%%%%%%%%%%%%%%
%Misc. packages
%%%%%%%%%%%%%%%%%%%%%%%%%%%
\usepackage{siunitx}
\usepackage{chemmacros} 
\usepackage{stmaryrd}
\usepackage{listings}
\usepackage{graphicx}
%%%%%%%%%
%Commands
%%%%%%%%%
\renewcommand\vec{\mathbf}
%\newcommand{\vectr}[1]{\boldsymbol{\mathbf{#1}}}

\DeclareMathOperator\supp{supp}
\DeclareMathOperator*{\argmax}{arg\,max}
\DeclareMathOperator*{\argmin}{arg\,min}
\DeclareMathOperator{\sign}{sign}

\newcommand{\xdgShockSolver}{\emph{XDGShock} solver}
\newcommand{\cnsSolver}{\emph{\gls{cns}} solver}
\newcommand{\bosssFramework}{\emph{\acrshort{bosss}} framework}

\newcommand{\leftSided}{\ensuremath{_{\mathrm{L}}}}
\newcommand{\rightSided}{\ensuremath{_{\mathrm{R}}}}

\newcommand{\diff}[1]{\,\mathrm{d} #1}
\newcommand{\dV}{\diff{V}}
\newcommand{\dt}{\diff{t}}
\newcommand{\dS}{\diff{S}}
\newcommand{\dl}{\diff{l}}
\newcommand{\de}{\mathrm{d}}
\newcommand{\D}{\mathrm{D}}
\newcommand{\Tr}{^\intercal}

\newcommand{\sourceCell}{\ensuremath{\gls{cell}^{\mathrm{src}}}}
\newcommand{\targetCell}{\ensuremath{\gls{cell}^{\mathrm{tar}}}}

\newcommand{\pre}{\ensuremath{_\mathrm{pre}}}
\newcommand{\post}{\ensuremath{_\mathrm{post}}}
\newcommand{\src}{\ensuremath{^{\mathrm{src}}}}
\newcommand{\tar}{\ensuremath{^{\mathrm{tar}}}}
\newcommand{\agg}{\ensuremath{^{\mathrm{agg}}}}

\newcommand{\set}[1]{\ensuremath{\mathcal{#1}}}
\newcommand{\jump}[1]{\left[\!\left[{{#1}}\right]\!\right]}
\newcommand{\aver}[1]{\left\{\!\!\left\{{#1}\right\}\!\!\right\}}
\newcommand{\norm}[1]{\left|\!\left| {#1} \right|\!\right|}
\newcommand{\normVector}[1]{\left| {#1} \right|}
\newcommand{\scalarProduct}[1]{\ensuremath{\langle {#1} \rangle}}

\newcommand{\basis}{\ensuremath{\phi}}
\newcommand{\velocity}{\ensuremath{u}}
\newcommand{\x}{\ensuremath{x}}
\newcommand{\scalarField}{\ensuremath{\psi}}
\newcommand{\testField}{\ensuremath{\vartheta}}
\newcommand{\statePrimitive}{\ensuremath{W}}
\newcommand{\coeffVec}{\ensuremath{\tilde{\boldsymbol{\scalarField}}}}
%\newcommand{\f}{\ensuremath{\vec{\gls{function}}}}
\newcommand{\scalarFieldVec}{\ensuremath{\boldsymbol{\scalarField}}}

\newcommand{\xt}{\ensuremath{(\gls{x_vec}, \gls{t})}}
\newcommand{\ps}{\ensuremath{\gls{poly_space}_{\gls{poly_degree}} (\set{\gls{cell}}_{\gls{h_length}})}}
\newcommand{\ns}{\ensuremath{\gls{scalar_field}_{\gls{h_length}}}}
\newcommand{\tn}{\ensuremath{\gls{t}_n}}
\newcommand{\tnone}{\ensuremath{\gls{t}_{n+1}}}

\newcommand{\cell}{\ensuremath{\gls{cell}_j}}
\newcommand{\cutCellSpace}{\ensuremath{\gls{poly_space}_{\gls{poly_degree}}^{\mathrm{X}}}}




\newcommand{\gasliq}{{}}
\newcommand{\bb}[1]{\mathcal{#1}}
\newcommand{\dza}[1]{\frac{d#1}{dz}}
\newcommand{\dzb}[1]{\frac{d}{dz}(#1)}
\newcommand{\BoSSS}{\textit{BoSSS }}
\newcommand{\gvec}[1]{\boldsymbol{#1}}
\renewcommand{\vec}{\textbf}
\newcommand{\unigasconst}{\mathcal{R}}
\newcommand{\pfrac}[2]{\frac{\partial #1}{\partial #2}}
\newcommand{\pfracTWO}[2]{\frac{\partial^2 #1}{\partial {#2}^2}}
\newcommand{\Dfrac}[1]{\frac{\text{D} #1}{\text{D} t}}
\newcommand{\ddfrac}[1]{\frac{\text{d} #1}{\text{d} t}}
\newcommand{\SumOvAllns}{\sum\limits_{k = 1}^{\gls{TotalNumberSpecies}}}
\newcommand{\Viscosity}{\mu}
\newcommand{\gray}{\rowcolor[gray]{.20}}
%\newcommand{\diff}[1]{\,d\mkern-1mu\text{#1}}
\newcommand{\mean}[1]{\left\{{#1}\right\}}
%\newcommand{\dV}{\diff{V}}
%\newcommand{\dS}{\diff{S}}
\newcommand{\domain}{\Omega}
\newcommand{\dA}{\diff{A}}
\newcommand{\Dt}[1]{\frac{\text{D}{#1}}{\text{D}t}}
\newcommand{\LTwoNorm}[1]{\left\lVert#1\right\rVert_{L^2}}
\newcommand{\Reynolds}{\text{Re}}
\newcommand{\Prandtl}{\text{Pr}}
\newcommand{\Froude}{\text{Fr}}
\newcommand{\Lewis}{\text{Le}}
\newcommand{\LewisAlpha}{\text{Le}_\alpha}
\newcommand{\Da}{\text{Da}}
\newcommand{\Sc}{\text{Sc}}

%\newcommand{\rateReac}{\dot\omega} 
\newcommand{\rateReac}{\omega} 
\newcommand{\stoicCoef}{\nu} 

\newcommand{\heatRelease}{\mathrm{Q}}

\newcommand{\MFVec}{\vec{Y}} 
\newcommand{\MFVecPrima}{\vec{Y}'} 
\renewcommand{\thefootnote}{\fnsymbol{footnote}}
\newcommand{\myvector}[1]{\textbf{#1}}
\newcommand{\mytensor}[1]{\boldsymbol #1}


%\newcommand{\norm}[1]{\left\lVert#1\right\rVert}
%\newcommand{\jump}[1]{\left\llbracket#1\right\rrbracket}
\newcommand{\pd}[2]{\frac{\partial #1}{\partial #2}}
\newcommand{\average}[1]{\left\{#1\right\}}
\newcommand{\flux}[1]{\widehat{#1}}
\newcommand{\rf}[1]{\tilde{#1}_\infty}
\newcommand{\DGvec}[1]{\underline{\tilde{#1}}}
\newcommand{\Nu}{\text{Nu}}
\newcommand{\Ra}{\text{Ra}}
\newcommand{\Rey}{\text{Re}}
\newcommand{\Pra}{\text{Pr}}
\newcommand{\Fr}{\text{Fr}}
\newcommand{\Mach}{\text{Ma}}

\newcommand{\Yi}{Y_{\alpha h}}

\newcommand{\RefVal}[1]{\hat #1_{\text{ref}}}
\newcommand{\RefValS}[2]{\hat #1_{#2,\text{ref}}} \newcommand{\LtwoNorm}[1]{\left \Vert #1 \right\Vert_2}

\newcommand{\GammaD}{\Gamma_{\text{D}}}
\newcommand{\GammaDW}{\Gamma_{\text{DW}}}
\newcommand{\GammaN}{\Gamma_{\text{N}}}
\newcommand{\GammaND}{\Gamma_{\text{ND}}}
\newcommand{\GammaI}{\Gamma_{\text{I}}}
\newcommand{\GammaP}{\Gamma_{\text{P}}}

\newcommand{\normalBoundary}{\vec{n}_{\Gamma}}

\newcommand{\cph}{c_p(T_h,\vec{Y}_h)}
\newcommand{\rhoh}{\rho(T_h,\vec{Y}_h)}
\newcommand{\visch}{\mu(T_h)}
\newcommand{\lambdah}{\lambda(T_h)}
\newcommand{\rhodh}{\rho\!D_\alpha(T_h)}
\newcommand{\cd}[1]{#1^{*}}
\newcommand{\dtpart}[1]{\frac{\partial}{\partial t}(#1)}
\newcommand{\dtpartcd}[1]{\frac{\partial}{\partial t^*}(#1)}
\newcommand{\nondimA}[1]{#1 = \frac{#1^*}{#1^*_\infty}}
\newcommand{\nondimB}[1]{#1 = \frac{#1^*}{p^*_\infty/\rho^*_\infty}}

\newcommand{\ContDis}{\mathcal{C}\left(\vec{u}_h,q_h, \rhoh\right) }
\newcommand{\MomConv}{\mathcal{U}^C\left(\vec{u}_h,\vec{u}_h,\vec{v}_h, \rhoh\right) } \newcommand{\MomConvDefinition}{\mathcal{U}^C\left(\vec{w}_h,\vec{u}_h,\vec{v}_h, \rhoh\right) }\newcommand{\MomPres}{\mathcal{U}^P\left(p_h,\vec{v}_h\right) }
\newcommand{\MomDiff}{\mathcal{U}^D\left(\vec{u}_h,\vec{v}_h,\visch\right)}
\newcommand{\MomSource}{\mathcal{U}^S\left(\rhoh, \vec{v}_h\right)}

\newcommand{\EnergyConv}{\mathcal{E}^C\left(\vec{u}_h,T_h,r_h, \rhoh\right)}
\newcommand{\EnergyDiff}{\mathcal{E}^D\left(T_h,r_h,\lambdah\right)} 
\newcommand{\EnergySource}{\mathcal{E}^S\left(r_h, Q_h(T_h,\vec{Y}_h), \rateReac_h(T_h,\vec{Y}_h)\right)}
\newcommand{\MFTestFunc}{s_h}
\newcommand{\MFConv}{\mathcal{M}^C_\alpha\left(\vec{u}_h,\Yi,\MFTestFunc, \rhoh\right)}
\newcommand{\MFDiff}{\mathcal{M}^D_\alpha\left(\Yi,\MFTestFunc,\rhodh\right) } 
\newcommand{\MFSource}{\mathcal{M}^S_\alpha\left(\MFTestFunc,\rateReac_h(T_h,\vec{Y}_h )\right)}

\newcommand{\glsHat}[1]{\hat{\gls{#1}}}

%%%%%%%%%%%%%%%%%%%%%%%%%%%%%%%%%%%%%%%%%%%%%%%%%%%%%%%%%%%%%%%%%%%%%%%%%%%%%%%%%%%%%%%%%

\newcommand{\vectr}[1]{\boldsymbol{\mathbf{#1}}}
\newcommand{\deriv}[2]{\frac{{\textrm{d}}{#1}}{{\textrm{d}}{#2}}}
\newcommand{\pDeriv}[2]{\frac{\partial{#1}}{\partial{#2}}}
\newcommand{\matDeriv}[1]{\frac{{\textrm{D}}{#1}}{\textrm{D}t}}
\renewcommand{\div}[1]{\nabla \cdot {#1}}
\newcommand{\lapl}[1]{\Delta{#1}}
\newcommand{\divH}[1]{\nabla_{\gls{grdSz}} \cdot {#1}}
\newcommand{\divI}[1]{\nabla_{\gls{interface}} \cdot {#1}}
\newcommand{\grad}[1]{\nabla{#1}}
\newcommand{\gradT}[1]{\nabla{#1}^{\textrm{T}}}
\newcommand{\gradH}[1]{\nabla_{\gls{grdSz}}{#1}}
\newcommand{\gradHT}[1]{\nabla_{\gls{grdSz}}{#1}^{\textrm{T}}}
\newcommand{\gradI}[1]{\nabla_{\gls{interface}}{#1}}
\newcommand{\gradIT}[1]{\nabla_{\gls{interface}}{#1}^{\textrm{T}}}
%\newcommand{\jump}[1]{\left[\!\left[{{#1}}\right]\!\right]}
%\newcommand{\aver}[1]{\left\{\!\left\{{#1}\right\}\!\right\}}
%\newcommand{\norm}[1]{\left|\!\left| {#1} \right|\!\right|}
\newcommand{\abs}[1]{\left| {#1} \right|}
\newcommand{\tensr}[1]{ \boldsymbol{\mathbf{#1}} }
\newcommand{\tensrT}[1]{ \boldsymbol{\mathbf{#1}}^{\textrm{T}} }
\newcommand{\trace}[1]{{\textrm{tr}}{#1}}
\newcommand{\cond}[1]{\textrm{cond}{\left(#1\right)}}
\renewcommand{\d}[1]{~\textrm{d}{#1}}
\newcommand{\nuM}[1]{#1_{\gls{grdSz}}}
\newcommand{\out}[1]{#1^{{-}}}
\newcommand{\inn}[1]{#1^{{+}}}
\newcommand{\diri}[1]{{#1}_{\textrm{D}}}
\newcommand{\neum}[1]{{#1}_{\textrm{N}}}
\newcommand{\wall}[1]{{#1}_{\textrm{wall}}}
\newcommand{\slip}[1]{{#1}_{\textrm{S}}}
\newcommand{\cL}[1]{{#1}_{\gls{cLine}}}
\newcommand{\indA}[1]{{#1}_{\gls{domainA}}}
\newcommand{\indB}[1]{{#1}_{\gls{domainB}}}
\newcommand{\indI}[1]{{#1}_{\gls{interface}}}
\newcommand{\brknPspace}[1]{\mathbb{P}_{#1}}
\newcommand{\cutcell}[1]{{\gls{cell}}_{#1}^{\gls{cut}}}
\newcommand{\basisX}[1]{{\gls{basis}}_{#1}^{\gls{cut}}}
\newcommand{\basisXvec}[1]{{\vectr{\gls{basis}}}_{#1}^{\gls{cut}}}
\newcommand{\vap}[1]{{#1}_{\textrm{v}}}
\newcommand{\liq}[1]{{#1}_{\textrm{l}}}

\newcommand{\scp}[3][]{\left( #2 , #3 \right)_{#1}}
\newcommand{\matrixDG}[1]{\underline{\underline{#1}}}
\newcommand{\basisDg}{\phi}


%%%%%%%%%%%
%Glossaries
%%%%%%%%%%%
\usepackage[automake,
nonumberlist, 	%do not show page numbers
acronym,      	%generate acronym listing
toc,          	%show listings as entries in table of contents
nomain,		  	%user-definded gloassaries
nopostdot=true	%remove the dot at the end of glossary descriptionsfsymbo
]{glossaries}

%Generate a list of symbols
\newglossary[slg]{symbols}{syi}{syg}{List of Symbols}

\newglossarystyle{mysymbolstyle}{%
	\glossarystyle{long}%
	\renewenvironment{theglossary}%
	{\begin{longtable}[l]{p{15mm}p{1.5\glsdescwidth}}}%
		{\end{longtable}}%
}

%Remove the dot at the end of glossary descriptions
%\renewcommand*{\glspostdescription}{}

%Format acronym glossary
\setacronymstyle{long-short}
\renewcommand{\glsnamefont}[1]{\textbf{\textsf{#1}}}%

\newglossarystyle{myacronymstyle}{%
	\glossarystyle{long}%
	\renewenvironment{theglossary}%
	{\begin{longtable}[l]{p{25mm}p{\glsdescwidth}}}%
		{\end{longtable}}%
}

%Activate glossary commands
\makeglossaries

%Load glossary files
\loadglsentries{chapters_others/abbreviations}
\loadglsentries{chapters_others/symbols}
\loadglsentries{chapters_others/SymbolsSM}

\glsaddall



%%%%%%%%%
%Todo notes
%%%%%%%%%
%\usepackage[
%colorinlistoftodos,
%textsize=footnotesize,
%]{todonotes}
%\usepackage{todo}

%%%%%%%%%
%Tikz
%%%%%%%%%
\usepackage{tikz}
\usetikzlibrary{shapes,arrows}
\usetikzlibrary{arrows.meta}
\usetikzlibrary{calc}
\usetikzlibrary{positioning}
\newcommand{\inputtikz}[1]{%
	\tikzsetnextfilename{#1}%
	\input{./tikzText/#1.tex}%
}

%%%%%%%%%
%pgfplots
%%%%%%%%%
\usepackage{pgfplots}
\usepackage{pgfplotstable}
\usetikzlibrary{spy,backgrounds}
% Excludes todo notes from externalize
\usepackage{letltxmacro}
\usepackage{layouts}			%textwidth in cm: \printinunitsof{cm}\prntlen{\textwidth}

% Options for pgfplots
\usepgfplotslibrary{units}

\pgfplotscreateplotcyclelist{mycycle}{%
	solid, every mark/.append style={solid, fill=gray}, mark=*\\%
	dashed, every mark/.append style={solid, fill=gray}, mark=square*\\%
	dotted, every mark/.append style={solid, fill=gray}, line width=0.8pt, mark=diamond*\\%
	dashdotted, every mark/.append style={solid, fill=gray, rotate=180},mark=triangle*\\%	
	densely dashed, every mark/.append style={solid, fill=gray}, mark=otimes*\\%
	loosely dashed, every mark/.append style={solid, fill=gray}, mark=triangle*\\%
	loosely dotted, every mark/.append style={solid, fill=gray}, mark=oplus*\\%
	densely dotted, every mark/.append style={solid, fill=gray}, mark=pentagon*\\%
}

\usepgfplotslibrary{groupplots}

\pgfplotsset{
	compat=newest,
%	grid=major,
	cycle list name=mycycle,
	yminorticks=false,
	tick align=outside,
	tick pos=left,
	legend pos=outer north east,
	legend cell align=left,
%	height=0.23\textwidth,
%	width=0.23\textwidth,
	every axis/.append style={scale only axis},
	legend style={font=\footnotesize},
	label style={font=\footnotesize},
	tick label style={font=\footnotesize},
}

% Color map that is identical to the VisIt default color map
\usepgfplotslibrary{colormaps}
\pgfplotsset{%
	/pgfplots/colormap={visit}{%
		rgb255(0cm)=(0,0,255)
		rgb255(1cm)=(0,255,255)
		rgb255(2cm)=(0,255,0)
		rgb255(3cm)=(255,255,0)
		rgb255(4cm)=(255,0,0)
	}
}
 
\pgfplotsset{
	/pgfplots/colormap = {turbo}{rgb=(0.18995,0.07176,0.23217),rgb=(0.19483,0.08339,0.26149),rgb=(0.19956,0.09498,0.29024),rgb=(0.20415,0.10652,0.31844),rgb=(0.20860,0.11802,0.34607),rgb=(0.21291,0.12947,0.37314),rgb=(0.21708,0.14087,0.39964),rgb=(0.22111,0.15223,0.42558),rgb=(0.22500,0.16354,0.45096),rgb=(0.22875,0.17481,0.47578),rgb=(0.23236,0.18603,0.50004),rgb=(0.23582,0.19720,0.52373),rgb=(0.23915,0.20833,0.54686),rgb=(0.24234,0.21941,0.56942),rgb=(0.24539,0.23044,0.59142),rgb=(0.24830,0.24143,0.61286),rgb=(0.25107,0.25237,0.63374),rgb=(0.25369,0.26327,0.65406),rgb=(0.25618,0.27412,0.67381),rgb=(0.25853,0.28492,0.69300),rgb=(0.26074,0.29568,0.71162),rgb=(0.26280,0.30639,0.72968),rgb=(0.26473,0.31706,0.74718),rgb=(0.26652,0.32768,0.76412),rgb=(0.26816,0.33825,0.78050),rgb=(0.26967,0.34878,0.79631),rgb=(0.27103,0.35926,0.81156),rgb=(0.27226,0.36970,0.82624),rgb=(0.27334,0.38008,0.84037),rgb=(0.27429,0.39043,0.85393),rgb=(0.27509,0.40072,0.86692),rgb=(0.27576,0.41097,0.87936),rgb=(0.27628,0.42118,0.89123),rgb=(0.27667,0.43134,0.90254),rgb=(0.27691,0.44145,0.91328),rgb=(0.27701,0.45152,0.92347),rgb=(0.27698,0.46153,0.93309),rgb=(0.27680,0.47151,0.94214),rgb=(0.27648,0.48144,0.95064),rgb=(0.27603,0.49132,0.95857),rgb=(0.27543,0.50115,0.96594),rgb=(0.27469,0.51094,0.97275),rgb=(0.27381,0.52069,0.97899),rgb=(0.27273,0.53040,0.98461),rgb=(0.27106,0.54015,0.98930),rgb=(0.26878,0.54995,0.99303),rgb=(0.26592,0.55979,0.99583),rgb=(0.26252,0.56967,0.99773),rgb=(0.25862,0.57958,0.99876),rgb=(0.25425,0.58950,0.99896),rgb=(0.24946,0.59943,0.99835),rgb=(0.24427,0.60937,0.99697),rgb=(0.23874,0.61931,0.99485),rgb=(0.23288,0.62923,0.99202),rgb=(0.22676,0.63913,0.98851),rgb=(0.22039,0.64901,0.98436),rgb=(0.21382,0.65886,0.97959),rgb=(0.20708,0.66866,0.97423),rgb=(0.20021,0.67842,0.96833),rgb=(0.19326,0.68812,0.96190),rgb=(0.18625,0.69775,0.95498),rgb=(0.17923,0.70732,0.94761),rgb=(0.17223,0.71680,0.93981),rgb=(0.16529,0.72620,0.93161),rgb=(0.15844,0.73551,0.92305),rgb=(0.15173,0.74472,0.91416),rgb=(0.14519,0.75381,0.90496),rgb=(0.13886,0.76279,0.89550),rgb=(0.13278,0.77165,0.88580),rgb=(0.12698,0.78037,0.87590),rgb=(0.12151,0.78896,0.86581),rgb=(0.11639,0.79740,0.85559),rgb=(0.11167,0.80569,0.84525),rgb=(0.10738,0.81381,0.83484),rgb=(0.10357,0.82177,0.82437),rgb=(0.10026,0.82955,0.81389),rgb=(0.09750,0.83714,0.80342),rgb=(0.09532,0.84455,0.79299),rgb=(0.09377,0.85175,0.78264),rgb=(0.09287,0.85875,0.77240),rgb=(0.09267,0.86554,0.76230),rgb=(0.09320,0.87211,0.75237),rgb=(0.09451,0.87844,0.74265),rgb=(0.09662,0.88454,0.73316),rgb=(0.09958,0.89040,0.72393),rgb=(0.10342,0.89600,0.71500),rgb=(0.10815,0.90142,0.70599),rgb=(0.11374,0.90673,0.69651),rgb=(0.12014,0.91193,0.68660),rgb=(0.12733,0.91701,0.67627),rgb=(0.13526,0.92197,0.66556),rgb=(0.14391,0.92680,0.65448),rgb=(0.15323,0.93151,0.64308),rgb=(0.16319,0.93609,0.63137),rgb=(0.17377,0.94053,0.61938),rgb=(0.18491,0.94484,0.60713),rgb=(0.19659,0.94901,0.59466),rgb=(0.20877,0.95304,0.58199),rgb=(0.22142,0.95692,0.56914),rgb=(0.23449,0.96065,0.55614),rgb=(0.24797,0.96423,0.54303),rgb=(0.26180,0.96765,0.52981),rgb=(0.27597,0.97092,0.51653),rgb=(0.29042,0.97403,0.50321),rgb=(0.30513,0.97697,0.48987),rgb=(0.32006,0.97974,0.47654),rgb=(0.33517,0.98234,0.46325),rgb=(0.35043,0.98477,0.45002),rgb=(0.36581,0.98702,0.43688),rgb=(0.38127,0.98909,0.42386),rgb=(0.39678,0.99098,0.41098),rgb=(0.41229,0.99268,0.39826),rgb=(0.42778,0.99419,0.38575),rgb=(0.44321,0.99551,0.37345),rgb=(0.45854,0.99663,0.36140),rgb=(0.47375,0.99755,0.34963),rgb=(0.48879,0.99828,0.33816),rgb=(0.50362,0.99879,0.32701),rgb=(0.51822,0.99910,0.31622),rgb=(0.53255,0.99919,0.30581),rgb=(0.54658,0.99907,0.29581),rgb=(0.56026,0.99873,0.28623),rgb=(0.57357,0.99817,0.27712),rgb=(0.58646,0.99739,0.26849),rgb=(0.59891,0.99638,0.26038),rgb=(0.61088,0.99514,0.25280),rgb=(0.62233,0.99366,0.24579),rgb=(0.63323,0.99195,0.23937),rgb=(0.64362,0.98999,0.23356),rgb=(0.65394,0.98775,0.22835),rgb=(0.66428,0.98524,0.22370),rgb=(0.67462,0.98246,0.21960),rgb=(0.68494,0.97941,0.21602),rgb=(0.69525,0.97610,0.21294),rgb=(0.70553,0.97255,0.21032),rgb=(0.71577,0.96875,0.20815),rgb=(0.72596,0.96470,0.20640),rgb=(0.73610,0.96043,0.20504),rgb=(0.74617,0.95593,0.20406),rgb=(0.75617,0.95121,0.20343),rgb=(0.76608,0.94627,0.20311),rgb=(0.77591,0.94113,0.20310),rgb=(0.78563,0.93579,0.20336),rgb=(0.79524,0.93025,0.20386),rgb=(0.80473,0.92452,0.20459),rgb=(0.81410,0.91861,0.20552),rgb=(0.82333,0.91253,0.20663),rgb=(0.83241,0.90627,0.20788),rgb=(0.84133,0.89986,0.20926),rgb=(0.85010,0.89328,0.21074),rgb=(0.85868,0.88655,0.21230),rgb=(0.86709,0.87968,0.21391),rgb=(0.87530,0.87267,0.21555),rgb=(0.88331,0.86553,0.21719),rgb=(0.89112,0.85826,0.21880),rgb=(0.89870,0.85087,0.22038),rgb=(0.90605,0.84337,0.22188),rgb=(0.91317,0.83576,0.22328),rgb=(0.92004,0.82806,0.22456),rgb=(0.92666,0.82025,0.22570),rgb=(0.93301,0.81236,0.22667),rgb=(0.93909,0.80439,0.22744),rgb=(0.94489,0.79634,0.22800),rgb=(0.95039,0.78823,0.22831),rgb=(0.95560,0.78005,0.22836),rgb=(0.96049,0.77181,0.22811),rgb=(0.96507,0.76352,0.22754),rgb=(0.96931,0.75519,0.22663),rgb=(0.97323,0.74682,0.22536),rgb=(0.97679,0.73842,0.22369),rgb=(0.98000,0.73000,0.22161),rgb=(0.98289,0.72140,0.21918),rgb=(0.98549,0.71250,0.21650),rgb=(0.98781,0.70330,0.21358),rgb=(0.98986,0.69382,0.21043),rgb=(0.99163,0.68408,0.20706),rgb=(0.99314,0.67408,0.20348),rgb=(0.99438,0.66386,0.19971),rgb=(0.99535,0.65341,0.19577),rgb=(0.99607,0.64277,0.19165),rgb=(0.99654,0.63193,0.18738),rgb=(0.99675,0.62093,0.18297),rgb=(0.99672,0.60977,0.17842),rgb=(0.99644,0.59846,0.17376),rgb=(0.99593,0.58703,0.16899),rgb=(0.99517,0.57549,0.16412),rgb=(0.99419,0.56386,0.15918),rgb=(0.99297,0.55214,0.15417),rgb=(0.99153,0.54036,0.14910),rgb=(0.98987,0.52854,0.14398),rgb=(0.98799,0.51667,0.13883),rgb=(0.98590,0.50479,0.13367),rgb=(0.98360,0.49291,0.12849),rgb=(0.98108,0.48104,0.12332),rgb=(0.97837,0.46920,0.11817),rgb=(0.97545,0.45740,0.11305),rgb=(0.97234,0.44565,0.10797),rgb=(0.96904,0.43399,0.10294),rgb=(0.96555,0.42241,0.09798),rgb=(0.96187,0.41093,0.09310),rgb=(0.95801,0.39958,0.08831),rgb=(0.95398,0.38836,0.08362),rgb=(0.94977,0.37729,0.07905),rgb=(0.94538,0.36638,0.07461),rgb=(0.94084,0.35566,0.07031),rgb=(0.93612,0.34513,0.06616),rgb=(0.93125,0.33482,0.06218),rgb=(0.92623,0.32473,0.05837),rgb=(0.92105,0.31489,0.05475),rgb=(0.91572,0.30530,0.05134),rgb=(0.91024,0.29599,0.04814),rgb=(0.90463,0.28696,0.04516),rgb=(0.89888,0.27824,0.04243),rgb=(0.89298,0.26981,0.03993),rgb=(0.88691,0.26152,0.03753),rgb=(0.88066,0.25334,0.03521),rgb=(0.87422,0.24526,0.03297),rgb=(0.86760,0.23730,0.03082),rgb=(0.86079,0.22945,0.02875),rgb=(0.85380,0.22170,0.02677),rgb=(0.84662,0.21407,0.02487),rgb=(0.83926,0.20654,0.02305),rgb=(0.83172,0.19912,0.02131),rgb=(0.82399,0.19182,0.01966),rgb=(0.81608,0.18462,0.01809),rgb=(0.80799,0.17753,0.01660),rgb=(0.79971,0.17055,0.01520),rgb=(0.79125,0.16368,0.01387),rgb=(0.78260,0.15693,0.01264),rgb=(0.77377,0.15028,0.01148),rgb=(0.76476,0.14374,0.01041),rgb=(0.75556,0.13731,0.00942),rgb=(0.74617,0.13098,0.00851),rgb=(0.73661,0.12477,0.00769),rgb=(0.72686,0.11867,0.00695),rgb=(0.71692,0.11268,0.00629),rgb=(0.70680,0.10680,0.00571),rgb=(0.69650,0.10102,0.00522),rgb=(0.68602,0.09536,0.00481),rgb=(0.67535,0.08980,0.00449),rgb=(0.66449,0.08436,0.00424),rgb=(0.65345,0.07902,0.00408),rgb=(0.64223,0.07380,0.00401),rgb=(0.63082,0.06868,0.00401),rgb=(0.61923,0.06367,0.00410),rgb=(0.60746,0.05878,0.00427),rgb=(0.59550,0.05399,0.00453),rgb=(0.58336,0.04931,0.00486),rgb=(0.57103,0.04474,0.00529),rgb=(0.55852,0.04028,0.00579),rgb=(0.54583,0.03593,0.00638),rgb=(0.53295,0.03169,0.00705),rgb=(0.51989,0.02756,0.00780),rgb=(0.50664,0.02354,0.00863),rgb=(0.49321,0.01963,0.00955),rgb=(0.47960,0.01583,0.01055),
	},	
}

% Defines the macro "discard if not" that lets us filter data files
% (see http://tex.stackexchange.com/a/106092/34032)
\pgfplotsset{
	discard if not/.style 2 args={
		x filter/.code={
			\edef\tempa{\thisrow{#1}}
			\edef\tempb{#2}
			\ifx\tempa\tempb
			\else
			\def\pgfmathresult{inf}
			\fi
		}
	}
}

\usepackage{placeins} %% Allows use of FloatBarrier
\usepackage{import}
\usepackage{subcaption}
\captionsetup[subfigure]{skip=-5pt}
\graphicspath{{./figures/}}
\graphicspath{{./plots/}}
%
%\usetikzlibrary{external}
%
%\tikzexternalize
%\tikzsetexternalprefix{figures/}
%
\usetikzlibrary{external}
\tikzexternalize[prefix=tikz/]
%\tikzsetfigurename{output} %% Necesary for Overleaf
%\tikzset{external/force remake} %up-to-date checks of all following figures (see: 50.4.3 Remaking Figures or Skipping Figures in pgf manual)
%%%%%%%%%
%Cleveref
%%%%%%%%%
\usepackage{hyperref}
\usepackage[
nameinlink,
capitalize,
noabbrev	
]{cleveref}

%%%%%%%%%
% Gkeichungsnummern, Labels, etc. anzeigen
%%%%%%%%%
%\usepackage[inline]{showlabels}
%\renewcommand{\showlabelfont}{\footnotesize\ttfamily\color{magenta}}

%%%%%%%%%%%%%%%%%%%
%Paketvorschläge Tabellen
%%%%%%%%%%%%%%%%%%%
%\usepackage{array}     % Basispaket für Tabellenkonfiguration, wird von den folgenden automatisch geladen
\usepackage{tabularx}   % Tabellen, die sich automatisch der Breite anpassen
%\usepackage{longtable} % Mehrseitige Tabellen
%\usepackage{xltabular} % Mehrseitige Tabellen mit anpassarer Breite
\usepackage{booktabs}   % Verbesserte Möglichkeiten für Tabellenlayout über horizontale Linien

\crefalias{subequation}{equation}
\newcommand{\mycomment}[1]{}
\begin{document}

%\listoftodos

\frontmatter

\title{A Discontinuous Galerkin Method for Diffusion Flames}
\subtitle{Embedded in a low-Mach solver framework}
\author[J. Gutiérrez-Jorquera]{Juan Francisco Gutiérrez Jorquera}%optionales Argument ist die Signatur,
\birthplace{Santiago, Chile}%Geburtsort, bei Dissertationen zwingend notwendig
%\reviewer{Gutachter 1 \and Gutachter 2 \and noch einer \and falls das immernoch nicht reicht}%Gutachter
%Falls die Bezeichner entsprechend der Promotionsordnung angepasst werden sollen:
\reviewer*[Erstreferent, Koreferent]{Prof. Dr.-Ing Martin Oberlack \and Gutachter 2}
\publishers{Darmstadt}% Feld für die Ortsangabe oder einen Verlag. Dies ist mit Darmstadt -- D17 vorbelegt, s.u. jedoch wurde die Anforderung für diese Vorgabe reduziert, daher genügt auch die Ortsangabe.


%%Sofern keine passende Option verfügbar ist
%\drtext{}

%Diese Felder werden untereinander auf der Titelseite platziert.
%\department ist eine notwendige Angabe, siehe auch dem Abschnitt `Abweichung von den Vorgaben für die Titelseite'
\department{mb} %Kürzel werden entsprechend der Liste in diesem Dokument ersetzt.
\institute{Fachgebiet für Strömungsdynamik}
%\group{Arbeitsgruppe}

\submissiondate{\today}
\examdate{\today}

% Hinweis zur Lizenz:
% TUDa-CI verwendet momentan die Lizenz CC BY-NC-ND 2.0 DE als Voreinstellung.
% Die TU Darmstadt hat jedoch die Empfehlung von dieser auf die liberalere
% CC BY 4.0 geändert. Diese erlaubt eine Verwendung bearbeiteter Versionen und
% die kommerzielle Nutzung.
% TUDa-CI wird im nächsten größeren Release ebenfalls diese Anpassung vornehmen.
% Aus diesem Grund wird empfohlen die Lizenz manuell auszuwählen.
\tuprints{urn=1234,printid=12345,doi=10.25534/tuprints-1234,license=cc-by-4.0}
% To see further information on the license option in English, remove the license= key and pay attention to the warning & help message.

\dedication{\textit{Dedication}}

\selectlanguage{ngerman}
\maketitle
\selectlanguage{USenglish}

\affidavit

\cleardoublepage
\phantomsection
\hypertarget{abstr}{}
\pdfbookmark[0]{Abstract}{abstr}
\begin{abstract}
	%\glsresetall
Fully coupled.
High order.
Low Mach flow numbers with moderate to high temperature differences. 
One step chemical model with variable kinetic properties.
Globalized newton algorithm.
Implicit timesteping using BDF-schemes. 


Steady state combustion systems are initialized by flame-sheet estimates, which correspond to the solution of a system with infinitely fast chemistry. 
Homotopy methods for highly non-linear systems.
Adaptive mesh refinement.



\end{abstract}

\cleardoublepage
\phantomsection
\hypertarget{zsfg}{}
\pdfbookmark[0]{Zusammenfassung}{zsfg}
\begin{abstract}[ngerman]
	\glsresetall
In dieser Arbeit wird ein vollständig gekoppelter numerischer Löser hoher Ordnung vorgestellt, der auf der \Gls{DG}-Methode zur Simulation von reaktiven Strömungen basiert. Die diskretisierten Gleichungen der Kontinuität, des Impulses, der Energie und der chemischen Spezies werden auf vollständig gekoppelte Weise unter Verwendung eines globalisierten Newton-Algorithmus gelöst. Das Hauptziel des Lösers ist die Erstellung eines Frameworks für die Untersuchung von Diffusionsflammen mit der \Gls{DG}-Methode. Dazu wird die low-Mach-Approximation der Navier--Stokes-Gleichungen verwendet. Die chemische Reaktion wird mit einem einstufigen Verbrennungsmodell mit variablen kinetischen Parametern modelliert, das speziell auf die Verbrennung von Kohlenwasserstoffen zugeschnitten ist. Die Temperatur- und Konzentrationsabhängigkeit der Dichte-, Wärmekapazitäts- und Transportparameter wird bei der Formulierung berücksichtigt. 

Eine detaillierte Darstellung der in dieser Studie verwendeten Gleichungen wird zusammen mit einer umfassenden Diskussion ihrer Herleitung und der damit verbundenen Annahmen präsentiert.
Das allgemeine Verfahren für die zeitliche und räumliche Diskretisierung mit der DG-Methode wird anhand einer allgemeinen Transportgleichung erläutert. Anschließend wird die DG-Diskretisierung der Gleichungen für reaktive Strömungen vorgestellt, und die verwendeten numerischen Flüsse werden beschrieben. 

Die entwickelten Berechnungsmethoden zur Lösung der herrschenden Gleichungen werden im Detail vorgestellt. Insbesondere wird die Strategie zur Lösung des nichtlinearen Problems mit Hilfe der globalisierten Dogleg-Newton-Methode erläutert, zusammen mit einer effizienten Methode zur Berechnung der Jacobimatrix. Darüber hinaus werden verschiedene Strategien vorgestellt, die die Konvergenzeigenschaften des Algorithmus verbessern. Dazu gehören eine vollautomatisierte Homotopie-Fortsetzungsmethode für die Lösung von stark nichtlinearen Systemen, eine adaptive Netzverfeinerungsstrategie, die für adäquate Netze in kritischen Bereichen der Simulation verwendet wird, und eine Solver-Safeguard zur Vermeidung von unphysikalischen Lösungen während der Berechnung. 

Für stationär reaktive Strömungen wird eine zusätzliche Strategie verwendet, die es ermöglicht, geeignete Anfangsschätzungen zu finden, die für die Simulation einer Diffusionsflamme verwendet werden können. Dieser Ansatz erfordert die Lösung eines vereinfachten Satzes von Gleichungen, die unter der Annahme einer unendlich schnellen chemischen Reaktion aufgestellt werden, und ist eine robuste Methode zur Lösung von Verbrennungssystemen.

Eine gründliche Validierung des Lösers anhand mehrerer Testfälle wird gezeigt, wodurch auch zentrale Vorteile der DG-Methode und der in dieser Arbeit vorgestellten Algorithmen hervorgehoben werden können. Die Testfälle ermöglichen es, den Löser gegen verschiedene Benchmark-Lösungen zu validieren, wobei Ergebnisse in sehr guter Übereinstimmung mit der Literatur erhalten werden. Darüber hinaus wird die Genauigkeit der Methode in verschiedenen Strömungssituationen bewertet, wobei für alle die erwarteten hohen Konvergenzraten der \Gls{DG}-Methode erzielt werden.  Stabilitätsprobleme werden jedoch bei instationären Simulationen von low-Mach-Strömungen beobachtet, bei denen die Dichte große Schwankungen aufweist.

\end{abstract}
\cleardoublepage
\phantomsection
\hypertarget{ackn}{}
\pdfbookmark[0]{Acknowledgements}{ackn}
\chapter*{Acknowledgements}
\glsresetall
Mate + Faso + Merkelwave

\cleardoublepage
\phantomsection
\pdfbookmark{\contentsname}{Contents}
\tableofcontents

%%%%%%\addtocontents{toc}{\protect\pagebreak}	%Seitenumbruch im Inhaltsverzeichnis einfügen

\cleardoublepage
\phantomsection
\addcontentsline{toc}{chapter}{\listfigurename}
\listoffigures

\cleardoublepage
\phantomsection
\addcontentsline{toc}{chapter}{\listtablename}
\listoftables

\cleardoublepage
\phantomsection
\printglossary[type=acronym, title=List of Abbreviations, style=myacronymstyle]

\cleardoublepage
\phantomsection
\printglossary[type=symbols, style=mysymbolstyle]


%%%%%%%%%%
%Main part
%%%%%%%%%%
\mainmatter
%
\cleardoublepage
\chapter{Introduction}	\label{ch:introduction}
%\glsresetall

\section{Introduction and state of the art.}
\subsection{Combustion}
\subsubsection{Premixed Flames(???)}
\subsubsection{Non-Premixed Flames}
\subsubsection{Droplet(?)}

%%%%%%%%%%%%%%%%%%%%%%%%%%%%%%%%%%%%%%%%%%%%%%%%%%%%%%%%%%%%%%%%%%%%%%%%%%%%%%%%%%%%%%%%%%%%%%%%%%%%%%%%%%%%%%%%%%%%%%%%%%%%%%%%%%%%%%%%%%%%%%%%%%%%%%%%%%%%


High order discretization methods is a topic which has been gaining increasing attention in the last decades. An important exponent of them is the Discontinuous Galerkin (DG) method \cite{cockburnDevelopmentDiscontinuousGalerkin2000}. The DG method was initially developed and utilized for solving hyperbolic conservation laws, especially in the field of computational fluid dynamics (CFD), and has recently gained increased attention for incompressible CFD problems for structured and unstructured grids. Two main advantages stand out when compared with traditional methods such as the Finite Volume Method (FVM) or the Finite Difference Method (FDM): First, DG offers an arbitrary order of error convergence due to the polynomial local approximation of the solution field. A polynomial approximation of degree $p$ provides a numerical discretization error of the order  $\mathcal{O}(h^{p+1})$ for sufficiently smooth solutions, where $h$ is a characteristic grid length.
Secondly, regardless of the desired order of accuracy, any given cell of the grid only requires information from its immediate neighbours, allowing for efficient parallelization with minimal communication overhead. In contrast, more traditional schemes, such as the FVM, are usually limited to $\mathcal{O}(h^N)$ accuracy, with $N \leq 2$ for unstructured grids. Even for structured grids $N$ is practically limited to low values due to the increasing stencil size for increasing $N$. Advantageously the DG method offers the locality of low-order schemes and the accuracy per degree of freedom of spectral schemes.

In the context of CFD solvers, an important distinction is that between pressure-based and density-based solvers. Historically, pressure-based solvers have been used for incompressible flows, i.e. flows with divergence-free velocity fields. On the other hand, density-based (also called fully compressible) solvers should, in theory, be able to solve flows in all Mach number ranges. However, in practice, as the Mach number approaches zero, density-based solvers experience efficiency and accuracy problems. These issues are mainly attributed to the acoustic effects in the flow, which tends to generate very stiff systems. None of these approaches are directly applicable to flows with varying density in the low-Mach limit. \cite{henninkPressurebasedSolverLowMach2021} It is possible, however, to extend existing incompressible and compressible codes so that they are capable of dealing with low Mach numbers. \cite{keshtibanCompressibleFlowSolvers2003} In this work a extension from a incompressible solver is presented.

There are numerous works in which the DG method has been used within the context of incompressible flows. \cite{shahbaziHighorderDiscontinuousGalerkin2007,kummerBoSSSDiscontinuousGalerkin2012,kleinSIMPLEBasedDiscontinuous2013,rhebergenSpaceTimeDiscontinuous2013}  However, there are not many publications in which the flow problem at low Mach numbers is addressed using a pressure-based solver. In the work by Klein et al. \cite{kleinHighorderDiscontinuousGalerkin2016} the low-Mach equations are solved in a DG Framework making use of a SIMPLE type scheme. The solution of a time-step involves an iterative process that requires multiple matrix assemblies and solutions. The obtained systems of equations are solved by means of fixed-point iterations, where relaxation factors are necessary to obtain convergence of the computations.
In the work of Hennink et al. \cite{henninkPressurebasedSolverLowMach2021} a pressure-based solver for low-mach flows is presented. They solve  the mass flux instead of the velocity as the primitive variable.

High order methods are very attractive for complex reacting fluid dynamical systems, where usually a high numerical resolution is necessary. Particularly for combustion problems, where large amounts of heat are released in rather small zones within the flow, the high number of elements required to resolve the resulting steep gradients could result in prohibitive calculation times even for simple problems. In particular the study of so-called diffusion flames -- also known as non-premixed flames -- requires special consideration. In a diffusion flame, the reactants are initially spatially separated. For this kind of system, mixing plays a crucial role because reactants need to be brought together to the flame zone in order to maintain combustion. Many practical applications of diffusion flames consider deflagration flames, \cite{poinsotTheoreticalNumericalCombustion2005} which are characterized by a small characteristic velocity compared to the speed of sound. The low-Mach approximation of the Navier--Stokes equations is often chosen for describing this kind of system. This approximation allows for the calculation of non-constant density flows (such as temperature dependent density), while neglecting acoustic effects, thus dramatically reducing the required temporal resolution. \cite{mullerLowMachNumberAsymptoticsNavierStokes1998}

In addition to the compressibility effects mentioned above, the need to accurately and efficiently represent the chemical reactions governing the combustion problem poses a major challenge. Generally speaking, to study the combustion process a detailed chemistry description is preferable. However, this is often impractical as it can be very intensive computationally speaking. Stauch et al. investigated systems with detailed mechanism for methanol combustion, where 23 chemical species and 166 elementary reactions are involved, \cite{stauchDetailedNumericalSimulation2006} and for n-heptane with 62 chemical species and 572 elementary reactions, and with detailed transport processes\cite{stauchAutoignitionSingleNheptane2007}. Because of their high complexity the mentioned works are restricted to simple one- or two-dimensional configurations, and to a small number of grid elements. If one is interested in more complex geometries or more complicated flow systems, the use of detailed kinetics can be prohibitive. Simplified kinetic models have been developed to overcome this difficulty. In the work of Westbrook et al. \cite{westbrookSimplifiedReactionMechanisms1981} a one-step kinetic model is presented, where combustion is expressed as a single chemical reaction with a reaction rate given by an Arrhenius-type expression with constant parameters. Multi-step chemical reaction models have also been developed, such as the four-step mechanism for methane combustion by Peters, \cite{petersNumericalAsymptoticAnalysis1985} or the three-step mechanism by Peters and Williams. \cite{petersAsymptoticStructureStoichiometric1987}
Furthermore, extensions of one-step models exist, such as the one presented by Fernandez-Tarrazo et al. \cite{fernandez-tarrazoSimpleOnestepChemistry2006} for hydrocarbon combustion with air, where kinetic parameters are correlated to the equivalence ratio in order to better describe characteristic flame properties for premixed and non-premixed flames.

In the last decades several numerical investigations related to diffusion flames have been carried out. Burke and Schumann were the first ones to investigate the structure of diffusion flames by studying the flame jet problem. \cite{burkeDiffusionFlames1928} By assuming an infinitely fast chemical reaction they managed to predict flame properties fairly accurately. In the work of Smooke et al. \cite{smookeNumericalSolutionTwoDimensional1986} a numerical simulation for a two-dimensional axisymmetric laminar diffusion jet  with detailed chemistry was conducted and solved with Newton-type methods. In order to obtain adequate initial estimates for this problem, the solution of the problem for an infinitely fast chemistry was solved first. This idea was used in several works such as that by Keyes and Smooke \cite{keyesFlameSheetStarting1987} for a counter diffusion flame, by Smooke \cite{smookeNumericalModelingAxisymmetric1992} for a Tsuji-counterflow configuration and in the work by Dobbins et al. \cite{dobbinsFullyImplicitCompact2010} for an axisymmetric laminar jet diffusion flame with time dependent boundary conditions. In the work of Paxion \cite{paxionDevelopmentParallelUnstructured2001} unstructured multigrid solver for laminar flames with detailed chemistry is presented. A Krylov-Newton method was used for solving several flame configurations. A two-dimensional counter diffusion flame was calculated, and its results were compared with the one-dimensional self-similar solution of the equations.

The DG method has also been used for simulations of combustion, mainly within a fully compressible framework. In the work from Johnson et al. \cite{johnsonConservativeDiscontinuousGalerkin2020} the compressible Navier--Stokes equations are solved using a nodal DG scheme for combustion with complex chemistry and transport parameters. An hp-adaptive method is also presented, and shown to be useful for solving the ordinary differential equations used for describing the unsteady behaviour of the system.  Similarly in the work from Lv and Ihme \cite{lvHighorderDiscontinuousGalerkin2017} a DG solver for multi-component chemically reacting flows, which solves the fully compressible Navier-Stokes equation, is presented. A hybrid-flux formulation is used, where a conservative Riemann solver is used for shock treatment, and a double-flux formulation is used in smooth regions. They show its applicability for non-reacting and reacting flows, particularly for systems characterized by high Mach numbers. On the other hand, the solution of combustion problems using the DG method in an incompressible framework is a topic that has not received much attention in the literature. This fact is the motivation of the present paper.

The system of equations obtained from the discretization of highly nonlinear systems  can be very difficult to solve. Fixed-point iteration schemes have been used as a linearization strategy, as done by Klein et al. \cite{kleinHighorderDiscontinuousGalerkin2016} A major drawback of this approach is the highly problem dependent choice of under-relaxation factors. A much more robust strategy is the use of Newton type methods. \cite{shadidInexactNewtonMethod1997,pawlowskiGlobalizationTechniquesNewton2006} Here, a problem-dependent factor is not needed. There is however a trade-off, because the Jacobian matrix has to be calculated, which can be computationally expensive. This approach has been used for combustion problems in numerous works. Karaa et al. \cite{karaaPreconditionedMultigridSimulation2003} studied the axisymmetric laminar jet diffusion flame and investigated the behaviour of a multi-grid solver when using different pre-conditioners with a damped Newton algorithm. Shen et al. \cite{shenNewtonMethodSteady2006a} investigated the use  and efficiency of a Newton method coupled with the Bi-CGSTAB method for an axisymmetric laminar jet flame. They concluded that, in terms of computational cost, the steady-state solution  is more efficiently obtained by directly solving the steady formulation of the equations, than by solving the transient Navier-Stokes equations until the steady state is reached.

It is well known that the Newton method shows the property of having quadratic convergence sufficiently close to the solution\cite{deuflhardNewtonMethodsNonlinear2011}. However, if the initial solution guess is not adequate, the method converge to a solution slowly, or may not even converge at all. For some highly nonlinear problems this is a significant issue. The so-called globalized Newton methods are used to overcome this problem by effectively increasing the area of convergence of the method. A popular globalized Newton method is the Newton-Dogleg method, \cite{pawlowskiInexactNewtonDogleg2008} which is based on a trust-region technique.

In the present work, a steady low-Mach pressure-based solver for the simulation of temperature dependent non-reactive and reactive flows using the Discontinuous Galerkin method is presented.
To the best of our knowledge, this is the first time that a pressure-based solver is used together with the DG method for solving the reactive low-Mach equations using the Newton-Dogleg method in a fully coupled manner. We focus in this study on two-dimensional configurations, but the ideas presented could be extended to three-dimensional systems. The one-step combustion model presented by Fernandez-Tarrazo et al. \cite{fernandez-tarrazoSimpleOnestepChemistry2006} is used for describing the chemical reactions. In the present work we consider only methane combustion, but the one-step model could be used for other hydrocarbons as well. We choose this chemical model to obtain physically correct results for a wide range of applications, while avoiding the use of complex chemistry models. The discrete system of equations is solved by a globalized Newton method, by means of the Dogleg approach. In addition to the Newton-Dogleg method, a homotopy strategy is presented, which was found to be useful for obtaining solutions of steady state calculations of highly nonlinear problems. In order to find appropriate initial values for Newton's method in combustion applications, the concept of flame sheet estimates (i.e. the solution for infinitely fast chemistry) is used. Several benchmark cases are presented that allow us to validate our implementation. First we solve the differentially heated cavity problem, with which we intend to validate our implementation of the low-Mach solver for non-constant density flows using the fully coupled solver. Later two flame configurations are calculated, namely the counter-diffusion flame and the chambered diffusion flame. \citep{matalonDiffusionFlamesChamber1980}

In the following, we consider a two-dimensional system. However, the methods shown in this work could be also used for three-dimensional problems.
%%%%%%%%%%%%%%%%%%%%%%%%%%%%%%%%%%%%%%%%%%%%%%%%%%%%%%%%%%%%%%%%%%%%%%%%%%%%%%%%%%%%%%%%%%%%%%%%%%%%%%%%%%%%%%%%%%%%%%%%%%%%%%%%%%%%%%%%%%%%%%%%%%%%%%%%%%%%%%%%%%%%%%%%%%%%%%%%%%%%%%%%%%%%%%%%%%%%

% hablar sobre la derivacion (historica) de las ecuaciones de low mach. Porque son utiles? 
They were first derived by Rehm and Baum \citep{rehmEquationsMotionThermally1978}. A rigorous extension to combustion problems was done by Majda and Sethian \citep{majdaDerivationNumericalSolution1985}.
\cleardoublepage
\chapter{Governing equations}	\label{ch:gov_eqs}
\glsresetall
In this work we present the methodology used to simulate reactive fluids in the low-Mach regime using the DG method. In this chapter we intend to show the governing equations, which will be discretized in the following chapters. First, in \cref{sec:GovEqLowMach} the equations for a flow in the low-Mach regime with finite reaction velocity are shown. A brief derivation of the set low-Mach equations is presented, which should also serve as a way of showing and justifying the assumptions made in this work. We restricted ourselves to a one-step chemical model, but the algorithm presented could also be used for more complex chemistry models.
Subsequently, in \cref{sec:FlameSheet} the governing equations are shown for the case of assuming an irreversible chemical model with an infinite reaction rate. This case is often called the Burke–Schumann limit. These equations are used as part of the algorithm to solve the finite reaction rate case.%
%

\section{The low-Mach number equations for reactive flows} \label{sec:GovEqLowMach}
Combustion processes can be modeled by a system of nonlinear partial differential equations, namely the balance equations for total mass, momentum, energy, and mass of individual species (usually expressed in terms of mass fractions). This system needs to be solved together with an equation of state and expressions for the transport properties as well as for the chemical reaction rates. 
The derivation of the governing equations for a reacting flow system can be found in the literature. (see, for example, \citep{keeChemicallyReactingFlow2003}, \citep{poinsotTheoreticalNumericalCombustion2005}). In the following pages, the main ideas regarding the derivation of equations are presented. For a more detailed explanation, we refer to the cited works and the references therein. 
\subsection{The reactive Navier--Stokes equations}\label{ssect:lowMachModel}
Throughout this work, variables with a hat sign, e.g. $\hat{\rho}$ represent dimensional variables, while those without it are nondimensional.  We start the derivation of the low-Mach number equations with the Navier--Stokes equations, the energy equation (in its temperature form), and the species transport equations.
Let us consider a reacting fluid mixture composed of $\gls{TotalNumberSpecies}$ species. Let $\hat{\vec{x}} =(\hat x, \hat y, \hat z)$ and $\hat t$ be the spatial vector and time. The primitive variables are the velocity field $\hat{\mathbf{u}} =(\hat u, \hat v, \hat w)$, the pressure $\hat p$, the temperature $\hat T$, and the mass fractions $Y_k$ of the $\gls{TotalNumberSpecies}$ total species. The set of governing equations to be solved is
\begin{subequations}
	\begin{align}
	%%%%%%%%%%%%%%%%%%%%%%%%%%%
	%%% Continuity
	%%%%%%%%%%%%%%%%%%%%%%%%%%%
	\label{eq:NS_Conti}
	&\pfrac{\hat\rho}{\hat t} +\hat\nabla \cdot (\hat \rho \hat{\vec{u}})   = 0, \\%
	%%%%%%%%%%%%%%%%%%%%%%%%%%%
	%%% Momentum
	%%%%%%%%%%%%%%%%%%%%%%%%%%%
	\label{eq:NS_Momentum}
	&\pfrac{\hat \rho \hat{\vec{u}}}{\hat t} +\hat\nabla \cdot (\hat \rho \hat{\vec{u}} \otimes  \hat{\vec{u}})   = - \hat\nabla \hat p - \hat \nabla \cdot \glsHat{ViscTensor}   -\hat\rho\hat{\vec{g}}, \\
	%%%%%%%%%%%%%%%%%%%%%%%%%%%
	%%% Energy (temperature)
	%%%%%%%%%%%%%%%%%%%%%%%%%%%
	\label{eq:NS_energy}
	%	\glsHat{dens}\glsHat{heatCapacity} \left(\pfrac{\glsHat{temp}}{\hat{t}} +\glsHat{velVec}\cdot \hat \nabla \hat{\gls{temp}}\right)
	&\glsHat{dens}\glsHat{heatCapacity} \pfrac{\glsHat{temp}}{\hat{t}} +\glsHat{dens}\glsHat{heatCapacity} \glsHat{velVec}\cdot \hat \nabla \hat{\gls{temp}}
	= \Dfrac{\glsHat{TotalPress}} - \hat{\nabla}\cdot \hat{\vec{q}}
	%-\left(\hat \rho \sum_{k = 1}^{\gls{TotalNumberSpecies}} c_{p,k} Y_k \glsHat{velDiffusionvec}_k\right) \cdot \hat \nabla \glsHat{temp} %% Shouldi include this term?
	+\glsHat{ViscTensor}\colon\hat \nabla \glsHat{velVec} + \hat \omega_T  \\
	%%%%%%%%%%%%%%%%%%%%%%%%%%%
	%%% Mass fraction k
	%%%%%%%%%%%%%%%%%%%%%%%%%%%
	\label{eq:NS_SpeciesBalance}
	&%	\pfrac{\hat \rho  Y_k }{\hat t} +	\hat\nabla \cdot (\hat \rho \hat{\vec{u}} Y_k)  & = \hat\nabla \cdot(\hat\rho\hat D_k \hat\nabla Y_k)+ 	  \hat \omega_k	  \quad (k = 1, \dots,~N - 1) \label{eq:GasLowMachMassBalance} 
	\pfrac{\hat \rho  Y_k }{\hat t} +	\hat\nabla \cdot (\hat \rho \hat{\vec{u}} Y_k)  
	=  -	\hat\nabla \cdot \hat{\vec{j}_k}	 +  \hat{\gls{reactionRate}}  \quad (k = 1, \dots,~\gls{TotalNumberSpecies})  
	\end{align}
	\label{eq:NS-eq}
\end{subequations}
This results in (in the general case) $\gls{TotalNumberSpecies} + 5$ differential equations to be solved. In these equations, $\hat \rho$ is the density of the mixture and $\hat{\vec{g}}$ is the acceleration of gravity. $\glsHat{ViscTensor}$, $\hat{\vec{q}}$ and $\hat{\vec{j}_k}$ are the viscous tensor, the heat flux vector and the molecular mass flux vector of species $k$, respectively. Additionally, $\hat \omega_T$ is the heat release due to combustion and $\hat \omega_k$ is the reaction rate of species $k$. To close the system, expressions must be defined to link these variables with the primitive variables. They will be briefly shown and commented upon in the following paragraphs.
\subsubsection{Equation of state}
Assuming that the fluid behaves ideally, the density can be calculated as
\begin{equation}
\hat \rho = \frac{\hat{p}  \hat{W}_{\text{avg}}}{\mathcal{R} \hat{T} }. \label{eq:IdealGassDimensional}
\end{equation}
Here, $\hat{\gls{GasConstant}}$ is the universal gas constant of gases, and $\hat W_{\text{avg}}$ is the average molecular weight of the fluid, defined as
\begin{equation}
\hat W_{\text{avg}} = \left( \SumOvAllns \frac{Y_k}{\hat W_k} \right)^{-1}
\end{equation}
with $\hat{\gls{MolecularWeight}}$ as the molecular weight of species $k$.  For an ideal mixture, the specific heat capacity can be calculated as weighted averages of the specific heats of the species, 
\begin{equation}
\hat c_p = \SumOvAllns Y_k \hat{c}_{p,k},%(\hat{T}),
\end{equation}
where $\hat{c}_{p,k}$ corresponds to the specific heat capacity of the component $k$. The temperature dependence of $\hat{c}_{p,k}$ can be accounted for by using NASA polynomials \citep{mcbrideNASAThermodynamicData1993}.
\begin{equation}
\hat{c}_{p,k} = \left(\hat a_1 + \hat  a_2\glsHat{temp} +\hat  a_3\glsHat{temp}^2 +\hat  a_4\glsHat{temp}^3+\hat a_5\glsHat{temp}^4\right)\frac{\hat{\gls{GasConstant}}}{\hat{\gls{MolecularWeight}}}
\end{equation}
where $\hat a_1$, $\hat a_2$, $\hat a_3$, $\hat a_4$ and $\hat a_5$ are numerical coefficients supplied by the NASA database.
\subsubsection{Transport models}
The viscous tensor $\glsHat{ViscTensor}$ is defined for a Newtonian fluid as
\begin{equation}
\glsHat{ViscTensor}= -\hat{\gls{visc}}\left( \hat\nabla \hat{\gls{velVec}} +(\hat\nabla\hat{\gls{velVec}})^T\right)  + \left(\frac{2}{3}\hat{\gls{visc}} - \hat \kappa \right) (\hat\nabla\cdot \hat{\gls{velVec}})\mytensor{I} 
\end{equation}
Here, $\hat{\gls{visc}}$ is the dynamic viscosity of the fluid, which is generally specific to the fluids and depends on its temperature and pressure. Furthermore, $\hat \kappa$ corresponds to the bulk viscosity, which is usually negligible for fluids at low pressures \citep{birdTransportPhenomena1960}. $\hat \kappa$ will be taken equal to zero in the rest of this work.\\

The heat flux vector $\hat{\vec{q}}$ is given by Fourier's law of heat conduction, 
\begin{equation}
\hat{\vec{q}} = \hat \lambda \hat{\nabla} \glsHat{temp}.
\label{eq:fourierLaw}
\end{equation}
Here $\hat \lambda$ corresponds to thermal conductivity, which, similar to the viscosity, is dependent on the particular fluid under study and its temperature and pressure.

The molecular mass flux vector $\hat{\vec{j}}_k$ of species $k$ is defined as $ \hat{\vec{j}}_k = \hat \rho \glsHat{velDiffusionvec}_k$. Here $\glsHat{velDiffusionvec}_k$ is the diffusion velocity of the component $k$. In general, $\glsHat{velDiffusionvec}_k$ can be obtained by solving the Maxwell-Stefan equations.
\begin{equation}
\nabla X_p = \sum_{k=1}^{N}\frac{X_p X_k}{D_{pk}}(\vec{U}_k-\vec{U}_p) \qquad p = 1,\dots,N. \label{MaxwellStefan}
\end{equation}
Here $D_{pk} = D_{kp}$ is the binary mass diffusion coefficient of species $p$ in species $k$. $X_p$ is the mole fraction of species $k$ and is related to the mass fraction of $k$ as $ X_k = Y_k \hat W /\hat W_k$.
The solution of the system \eqref{MaxwellStefan} is often a difficult and costly task \citep{williamsCombustionTheoryFundamental2000,poinsotTheoreticalNumericalCombustion2005}, and often simplifications are made. It can be shown that for binary mixtures ($N = 2$) and for mixtures containing multiple species ($N>2$) where all diffusion coefficients are equal, the system \eqref{MaxwellStefan} reduces exactly to the well-known Fick's law.
\begin{equation}
\hat{\vec{j}}_k = -\rho \hat D_k \nabla Y_k \
\end{equation}\label{eq:FickLaw}
This expression is only exact in the cases mentioned above. For a system where the heat capacities of each species are different, this expression is inconsistent. The variable $\hat D_k$ corresponds in this case to the diffusion coefficient of species $k$ in the mixture. An issue regarding the global mass conservation can be pointed out here. Recall that by definition the sum of the the mass fractions must always be one ($ \sum_{k}^{N}Y_k = 1$). This is only true  for the solution of \cref{eq:NS-eq} if exact expressions for the diffusion velocities are used \citep{poinsotTheoreticalNumericalCombustion2005}. If some inexact expression is used (as for example Fick's law), the constraint for the sum of the mass fractions will not be fulfilled. This problem can be surpassed by solving the global mass conservation (continuity) equation \cref{eq:NS_Conti} and only the equations for the first $\gls{TotalNumberSpecies} - 1$ species \cref{eq:NS_SpeciesBalance}.
Therefore, all inconsistencies that originated from not using an exact species diffusion model are absorbed by $Y_{\gls{TotalNumberSpecies}}$. As pointed out in \cite{poinsotTheoreticalNumericalCombustion2005}, this simplification should only be used if all $\gls{TotalNumberSpecies}-1$ species are strongly diluted in species $\gls{TotalNumberSpecies}$, such as the case of a flame in air, where the mass fraction of nitrogen is large. It can also be noted that this approach reduces in one the number of differential equations needed to be solved. 

The temperature dependence of viscosity can be modeled by Sutherland's law \citep{sutherlandLIIViscosityGases1893}
\begin{equation}\label{eq:DimSutherland}
\hat{\mu}(\hat{T}) = \hat{\mu}_{\text{suth}}\left(\frac{\hat{T}}{\hat{T}_{\text{suth}}}\right)^{1.5}\frac{\hat{T}_{\text{suth}} + \hat{S}}{\hat{T}+\hat{S}}.
\end{equation}
Here $\hat{\mu}_{\text{suth}}$ is the viscosity evaluated at a reference temperature $\hat{T}_{\text{suth}}$, and $\hat S$ is a material dependent parameter. In all calculations in this work the value of $\hat{S}$ for air is used, $\hat{S} = $ \SI{110.5}{\kelvin}. Expressions for determining the thermal conductivity and diffusion coefficients as function of temperature can be obtained using similar expressions, as will be shown later in \cref{ssec:NonDimLowMachEquations}
%The enthalpy transport term due to diffusive fluxes have usually a small influence in the solution,\cite{smokeFormulationPremixedNonpremixed1991, goeyModelingSmallScale1995,paxionDevelopmentParallelUnstructured2001} and is neglected in the energy equation. 
%by assuming constant values for the Prandtl and Lewis numbers (to be defined later), we obtain expression for the temperature dependency of the other transport parameters as $\hat{k}/\hat{c}_p(\hat T) = \hat \mu/\text{Pr}$ and $\hat \rho \hat D_k(\hat T) = \hat \mu/\text{Pr}\text{Le}_k$.

\subsubsection{Chemical model} 
Consider a system composed of $N$ species where $M$ chemical reactions take place. Chemical reactions can be written in generalized form as follows.
\begin{equation}\label{eq:allChemEq}
\sum_{k=1}^{N} \nu'_{kj}\mathcal{M}_k  \rightleftharpoons \sum_{k=1}^{N} \nu''_{kj}\mathcal{M}_k  \qquad \text{for}\qquad j=1,\dots,M 
\end{equation}
where $\nu'_{jk}$ and $\nu''_{jk}$ are the molar stoichiometric coefficients of species $k$ in the chemical reaction $j$, and $\mathcal{M}_j$ represents the chemical component $k$. \newline

%%%%%%%%
The reaction rate of species $k$ is $\hat \omega_k$, which accounts for the total amount of species $k$ that appear or disappear due to $M$ chemical reactions (c.f. Equations \eqref{eq:allChemEq}). Its given by
\begin{equation} \label{eq:reacRateDef}
\hat \omega_k  = \hat W_k \sum_{j=1}^{M}\nu_{jk}\hat{\mathcal{Q}}_j.
\end{equation}
Here $\nu_{kj} = \nu''_{kj} -\nu'_{kj}$, and $\mathcal{Q}_j$ is the rate of progress of the reaction $j$. They are usually modeled using Arrhenius-type expressions, as will be shown later. 
\newline

%%%%%%%
The heat release $\hat \omega_T$ that appears in the energy equation is related to the reaction rates according to
\begin{equation}\label{eq:heatRelDef}
\hat \omega _T = - \sum_{k=1}^{N} \hat h_k\hat\omega_k = - \sum_{k=1}^{N}   \Delta \hat h_k^0 \hat\omega_k   - \sum_{k=1}^{N} \hat h_{ks} \hat\omega_k 
\end{equation}
Here, the specific enthalpy of $k$-th species $\hat h_k$  is written in terms of its formation enthalpy $\hat h_k^0$ and a sensible enthalpy $\hat h_{ks} =\int_{0}^{\hat{T}} \hat c_{p,k} \text{d}\hat{T} $. The second term on the right-hand side of \cref{eq:heatRelDef} is usually small and is exactly zero for a mixture where all the heat capacities of each component are equal \citep{poinsotTheoreticalNumericalCombustion2005}. It will be neglected in the rest of the analysis.
\newline

In this work, we use the one-step kinetic model for the combustion of hydrocarbons presented in \cite{fernandez-tarrazoSimpleOnestepChemistry2006}. The chemical reaction is represented by a single ($M = 1$)  exothermic global irreversible expression as
\begin{equation}
\ch{C}_n \ch{H}_m + \left(n+\frac{m}{4}\right) \ch{O}_2\ch{ -> } n \ch{CO2} + \frac{m}{2} \ch{H2O}
\end{equation}%TODO \todo{correct boldletter}
The rate of progress of the global reaction is modeled by an Arrhenius-type expression
\begin{align}
\hat{\mathcal{Q}}= \hat B e^{-\hat T_a/\hat T} \left(\frac{\hat \rho Y_F}{\hat W_F}\right)^a \left(\frac{\hat \rho Y_O}{\hat W_O}\right)^b . \label{eq:DimArr}
\end{align}%
Here, the subscripts $F$ and $O$ refer to fuel and oxidizer, respectively. The parameter $\hat{B}$ corresponds to the pre-exponential factor, $\hat T_a$ is the activation temperature, and $a$ and $b$ are reaction orders. For a one-step reaction model, the reaction rate of the $k$-th component (\cref{eq:reacRateDef}) is 
\begin{equation}
\hat \omega_k  =  \nu_{k} \hat W_k\hat{\mathcal{Q}}.
\end{equation}
With this definition, the heat release $\hat \omega_T$ (\cref{eq:heatRelDef}) is
\begin{equation}
\hat \omega_T = - \hat W_F \hat \omega_F\hat Q^m.
\end{equation}
Here is $\hat Q^m$ the molar heat of reaction of the one-step reaction. $\hat W_F$ and $ \hat \omega_F$ are the molar mass of the fuel species and the reaction rate of the fuel species. 

\begin{table}[!btp] 
	\centering
	\begin{tabular}{
			@{}
			S[table-format=4.2]
			S[table-format=7.2]
			S[table-format=1]
			S[table-format=1]
			S[table-format=2.2]
			@{}			
		}
		\toprule		
		{$\hat B$} &
		{$\hat{T}_{a0}$}&
		{$\hat{\heatRelease}_0$}  &
		{$a$} &
		{$b$}\\		
		{$(\si{\centi\metre\cubed\per(\mole\, \second)})$} &
		{$(\si{\kelvin})$} &
		{$(\si{\mega \joule \per \kilo \mole})$} &
		{} &
		{}\\
		\midrule
		${6.9\times 10^{14}}$&
		15900 & 
		802.4 &
		1  &
		1 \\
		\bottomrule
	\end{tabular}	
	\caption{Base parameters used in the one-step combustion model by \cite{fernandez-tarrazoSimpleOnestepChemistry2006}}\label{Tab:OneStepParameters}	
\end{table}%

Within the model from \citep{fernandez-tarrazoSimpleOnestepChemistry2006}, several parameters are adjusted to represent characteristic features of premixed flames and diffusion flames. The parameters $\hat T_a$ and $\hat \heatRelease$ are defined as functions of the local equivalence ratio $\phi$, which is, in turn, defined in terms of the local mass fractions of fuel $Y_F$ and oxidizer $Y_O$ as
\begin{equation}\label{eq:equivalenceRatio}
\phi = \frac{s Y_F^0}{Y_O^0}\frac{s Y_F-Y_O+Y_O^0}{s(Y_F^0-Y_F) + Y_O},
\end{equation}
where $Y_F^0$ and $Y_O^0$ are the mass fractions of the fuel and oxidizer flows in their corresponding feed streams, and $s$ is the mass stoichiometric ratio, defined as $s =\nu_O \hat W_O/\nu_F \hat W_F$. 
The activation energy and molar heat release depend on $\phi$ as 
\begin{equation}
\hat{T}_a(\phi)= 
\begin{cases} 
(1 + 8.250(\phi-0.64)^2) \hat{T}_{a0} &\text{if}~ 	\phi \leq 0.64,   \\
\hat{T}_{a0}  &\text{if}~ 	0.64 \leq \phi \leq 1.07, \\
(1 + 4.443(\phi-1.07)^2)\hat{T}_{a0} &\text{if}~\phi \geq 1.07,
\end{cases}       
\end{equation}

\begin{equation}
\hat{\heatRelease}(\phi)=
\begin{cases}
\hat{\heatRelease}_0 &\text{if}~ \phi \leq 1\\
(1 - \alpha(\phi -1))\hat{\heatRelease}_0&\text{if}~\phi > 1,
\end{cases}  \label{eq:heatReleaseOneStep}     
\end{equation}

The parameter $\alpha$ is a constant that depends on the hydrocarbon being considered. In particular $\alpha = 0.21$ for the combustion of methane.
%%%%%%%%%%%%%%%%%%%%%%%%%%%%%%%%%%%%%%%%%%%%%%%%%%%%%%
%%%%%%%%%%%%%%%%%%%%%%%%%%%%%%%%%%%%%%%%%%%%%%%%%%%%%%

\subsection{The unsteady non-dimensional low-Mach equations} \label{ssec:NonDimLowMachEquations}

In the present work, we use the low-Mach number equation approximation of the governing equations. The derivation of the equations can be found in \citep{rehmEquationsMotionThermally1978, majdaDerivationNumericalSolution1985,mullerLowMachNumberAsymptoticsNavierStokes1998a}. We refer to these references for a detailed explanation on how the set of equations is derived. In the following, we restrict ourselves to show and comment on the main consequences of the low-Mach limit.

The low-Mach number limit approximation of the governing equations is used for flows where the Mach number (defined as $\text{Ma} = \RefVal{u}/\hat c$, where $\RefVal{u}$ is a characteristic flow velocity and $\hat{c}$ the speed of sound) is small, which is usually the case in typical laminar combustion systems. \citep{dobbinsFullyImplicitCompact2010}. 
The low-Mach equations are obtained by using standard asymptotic methods. One of the main results of the analysis is that for flows with a small Mach number, the pressure can be decomposed as 
\begin{equation}
    \hat p(\hat {\vec{x}}, \hat t) = \hat p_0(\hat t) + \hat p_2(\hat{\vec{x}},\hat t). 
\end{equation}
The spatially uniform term $\hat p_0(\hat t)$ is called thermodynamic pressure, and only appears in the equation of state. It is constant in space, but can change in time.  For an open system, the thermodynamic pressure is constant and equal to the ambient pressure, while for a closed system (e.g. a system completely bounded by walls) it changes in order to ensure mass conservation.%; cf. \cref{ss:DHC}. 
On the other hand, the perturbational term $\hat p_2(\hat{\vec{x}},\hat t)$ appears only in the momentum equations and plays a role similar to that of the pressure in the classical incompressible formulation. This perturbational term satisfies $\hat p_2/\hat p \sim \mathcal{O}(\text{Ma})^2$, \citep{dobbinsFullyImplicitCompact2010,nonakaConservativeThermodynamicallyConsistent2018} showing that under these assumptions, the equation of state is satisfied only to $\mathcal{O}(\text{Ma}^2)$ (cf.  \cref{eq:IdealGassDimensional}) 

Effectively, the low-Mach limit of the Navier-Stokes equations allows for the calculation of systems where big density variations due to temperature differences are present, thus not restricting ourselves to approximations such as the Boussinesq approximation for bouyancy-driven flow. In addition, this approximation truncates the mechanism of pressure wave propagation which is a natural feature of the compressible Navier-Stokes equations. In doing this, we are no longer restricted to choosing small timesteps to be able to resolve wave phenomena, and the maximum allowed timestep is greatly increased.  
From now on we will drop the sub-index of the hydrodynamic pressure $\hat p_2$ and we will refer to it simply as $\hat p$, further emphasizing the similarity in its role to the pressure of the incomprehensible formula.

In this work we use a non-dimensional formulation of the governing equations. We define the non-dimensional quantities.
\begin{align*}
&\rho = \frac{\hat \rho}{\RefVal{\rho}}, \quad 
p = \frac{\hat p}{\RefVal{p}}, \quad 
\vec{u}= \frac{\hat{\vec{u}}}{\RefVal{u}}, \quad 
T = \frac{\hat T}{\RefVal{T}},  \quad 
c_p = \frac{\hat c_p}{\RefValS{c}{p}}, \quad
W_k = \frac{\hat{W}_k}{\RefVal{W}}
\\
\mu = &\frac{\hat \mu}{\RefVal{\mu}},\quad
D_k = \frac{\hat D_k}{\RefValS{D}{k}}, \quad
k = \frac{\hat k}{\RefVal{k}}\quad
\nabla = \frac{\hat \nabla}{\RefVal{L}}, \quad
t = \frac{\hat{t}}{\RefVal{t}},\quad 
\vec{g} = \frac{\hat{\vec{g}}}{\RefVal{g}},\quad
Q = \frac{\hat Q}{\hat{Q}_0}
\end{align*}
Here $\RefVal{u}, \RefVal{L}$, $\RefVal{p}$, $\RefVal{t}$, and $\RefVal{T}$ are the reference velocity, length, pressure, time, and temperature, respectively, and are equal to some characteristic value for the particular configuration studied. Furthermore, $\RefVal{g}$ is the magnitude of the gravitational acceleration and $\RefVal{W}$ is the reference molecular weight. The reference transport properties $\RefVal{\mu}$, $\RefVal{k}$, $\RefValS{D}{k}$ and the reference heat capacity of the mixture and $\RefValS{c}{p}$ are evaluated at the reference temperature $\RefVal{T}$. Similarly, the reference density must satisfy the equation of state, thus $\RefVal{\rho} = \RefVal{p}/(\hat{\mathcal{R}}\RefVal{T}\RefVal{W})$.  By introducing these definitions into the governing equations (\cref{eq:NS_Conti,eq:NS_Momentum,eq:NS_energy,eq:NS_SpeciesBalance}) the reactive set of non-dimensional low-Mach number equations is obtained. The system of differential equations to be solved reads as follows. 
\begin{subequations}
	\begin{align}
&\pfrac{\rho}{t} + \nabla \cdot (\rho \vec{u})   = 0, \label{eq:LowMach_Conti}\\
&\pfrac{\rho \vec{u}}{t} + \nabla \cdot (\rho \vec{u} \otimes \vec{u})   = - \nabla p + \frac{1}{\Rey}\nabla \cdot \mu\left( \nabla \vec{u} +\nabla \vec{u}^T  - \frac{2}{3}(\nabla\cdot \vec{u})\mytensor{I} \right)  - \frac{1}{\Fr^2}\rho\vec{g}, \label{eq:LowMach_Momentum}\\
&\pfrac{\rho T}{t} + \nabla \cdot (\rho \vec{u} T)  = \frac{1}{\text{Re}~\text{Pr}}\nabla \cdot\left(\frac{k}{c_p} \nabla T\right)+ \text{H}~\Da~\frac{\heatRelease~\mathcal{Q}}{c_p}, \label{eq:LowMachEnergy}\\ 
&\pfrac{\rho Y_k}{t} +	\nabla \cdot (\rho  \vec{u}Y_k)   = \frac{1}{\text{Re}~\text{Pr}~\text{Le}_k}\nabla \cdot(\rho D \nabla Y_k)+  \Da~\stoicCoef_k W_k \mathcal{Q}. \quad (k = 1, \dots,~N - 1) \label{eq:LowMachMassBalance} 
	\end{align}
	\label{eq:all-eq}
\end{subequations}
Note that we assumed that the spatial gradients of the mixture heat capacity are small, allowing us to introduce it in the derivative of the diffusive term of \cref{eq:LowMachEnergy}. Furthermore, using the fact that the sum of the mass fractions must always be one, the mass fraction of the last species $\gls{TotalNumberSpecies}$ can be calculated with
\begin{align} \label{eq:MassFractionConstraint}
Y_{\gls{TotalNumberSpecies}} = 1 - \sum_{k = 1}^{\gls{TotalNumberSpecies}-1}Y_k.
\end{align}
The $N$-th component mass fraction $Y_{N}$ is calculated using \cref{eq:MassFractionConstraint}. This system is solved for the primitive variables velocity $\vec{u} = (u_x, u_y)$, pressure $p$, temperature $T$ and mass fractions ${\mathbf{Y} = (Y_1,\dots,Y_{N})}$. We note that the form of the low-Mach equations is very similar to the Navier-Stokes equations. The greatest difference is in the decomposition of the pressure, as mentioned above. This similarity is beneficial, as it allows the use of similar techniques to solve the PDE system to those used for the completely incompressible case \citep{keshtibanCompressibleFlowSolvers2003}.

Six non-dimensional factors arise from the non-dimensionalization process:
\begin{gather*}
\text{Re} = \frac{\RefVal{\rho} \RefVal{u}  \RefVal{L}}{\RefVal{\Viscosity}}, \quad
\text{Fr} = \frac{\RefVal{u}}{\sqrt{\RefVal{g}\RefVal{L}}}, \quad
\text{Pr} = \frac{\RefValS{c}{p} \RefVal{\Viscosity}}{\RefVal{k}},\\%, \quad
% \text{Ma} = \frac{\RefVal{u}}{\sqrt{\gamma \RefVal{T} \hat{\gls{GasConstant}} / \RefVal{W} }},\\
\text{Le}_k = \frac{\RefVal{k}}{\RefVal{\rho} \RefValS{D}{k} \RefValS{c}{p}}, \quad
\Da = \frac{\hat B \RefVal{L} \RefVal{\rho}}{\RefVal{M}\RefVal{u}}, \quad \label{eq:Dahmkoeler}
\text{H} = \frac{\hat \heatRelease_0}{\RefValS{c}{p} \RefVal{T}}%%%%%%%%%%%%%%%%%%%%%%% CHECK THIS,  not sure if \RefVal{M} should be here
%\text{H} = \frac{\hat \heatRelease_0}{\RefVal{M} \RefValS{c}{p} \RefVal{T}}%%%%%%%%%%%%%%%%%%%%%%% CHECK THISONEEEEEEEEEEEEEEEEEEEEE
\end{gather*} 
The first three equations define the Reynolds, Froude and Prandtl number respectively. $\text{Le}_k$ is the Lewis number of species $k$. Finally $\text{Da}$ and H are the Damköhler number and the non-dimensional heat release respectively. The non-dimensional progress of the global reaction reads as follows.
\begin{align}
\mathcal{Q}(T, \vec{Y})  = \left(\frac{\rho Y_F}{M_F}\right) \left(\frac{\rho Y_O}{M_O}\right)\text{exp}\left(\frac{-T_a}{T}\right), \label{eq:NonDimArr}
\end{align}
where $T_a = \hat{T}_a / \RefVal{T}$. Furthermore, the non-dimensional heat release is
\begin{equation}
\heatRelease(\phi)=
\begin{cases}
1 &\text{if}~ \phi \leq 1\\
(1 - \alpha(\phi -1))&\text{if}~\phi > 1,
\end{cases}  \label{eq:heatReleaseOneStepNonDim}     
\end{equation}
with $\phi$ evaluated according to \cref{eq:equivalenceRatio}. In the low-Mach limit, the ideal gas equation depends on the thermodynamic pressure, temperature and mass fractions. It reads in its non-dimensional form 
\begin{align} \label{eq:ideal_gas}
\rho(p_0,T, \vec{Y}) = \frac{p_0}{T \SumOvAllns \frac{Y_k}{W_k}}.
\end{align}
As mentioned above, the thermodynamic pressure of a closed system is a parameter that has to be determined (for an open system is equal to the atmospheric pressure). Defining the initial mass of the fluid inside the closed system as $m_0$, and by integrating \cref{eq:ideal_gas} on the whole domain $\Omega$, one obtains 
\begin{equation}
p_0(T, \vec{Y}) == \frac{m_0}{\int_\Omega \frac{W_{\text{avg}}}{T}\text{d}V}, \label{eq:p0Condition}
\end{equation}
Similarly, the non-dimensional specific heat capacity of the mixture $c_p$ is calculated as
\begin{equation}\label{eq:nondim_cpmixture}
c_p(T,\mathbf{Y}) = \SumOvAllns Y_k c_{p,\alpha}(T),
\end{equation}
and the non-dimensional viscosity as
\begin{equation} \label{eq:nondim_sutherland}
\mu(T) =  T^{\frac{3}{2}}\frac{1+\hat{S} }{\RefVal{T}T+\hat{S}}.
\end{equation} 
As mentioned before, the model for the transport parameters can be simplified by assuming constant values for the Prandtl and Lewis numbers. \citep{smokeFormulationPremixedNonpremixed1991} and we can write $\mu(T) = k/c_p(T) = \rho D_k(T)$.
\section{The flame sheet approximation} \label{sec:FlameSheet}
Here we introduce the concept of the flame sheet approximation, which is used in our solution algorithm for solving the finite reaction rate system given by \cref{eq:NS_Conti,eq:NS_Momentum,eq:LowMachEnergy,eq:LowMachMassBalance}. We follow the ideas proposed in the work from \cite{keyesFlameSheetStarting1987}.
Assuming that all species have the same constant heat capacity $c_p$ and mass diffusion coefficient $D_k=D$, that the Lewis number is unity for all species, and that combustion can be described by a single-step chemical reaction, it is possible to obtain an equation for a scalar without source terms, taking a linear combination of the energy \cref{eq:LowMachEnergy} and mass fraction \cref{eq:LowMachMassBalance}. A commonly used scalar is the mixture fraction $z$, which per definition is equal to unity in the fuel feed stream, and equal to zero in the oxidizer feed stream. Thus, the system of \cref{eq:NS_Conti,eq:NS_Momentum,eq:LowMachEnergy,eq:LowMachMassBalance} can be simplified to solving the low-Mach Navier-Stokes equations together with an equation for the passive scalar $z$:
\begin{subequations}
	\begin{align}
\pfrac{\rho}{t}+	\nabla \cdot (\rho \vec{u})  & = 0, \label{eq:MixtFracConti2}\\ 
\pfrac{\rho \vec{u}}{t} +	\nabla \cdot (\rho \vec{u} \otimes \vec{u})  & = - \nabla p + \frac{1}{\Rey}\nabla \cdot \mu\left( \nabla \vec{u} +\nabla \vec{u}^T  - \frac{2}{3}(\nabla\cdot \vec{u})\mytensor{I} \right)  - \frac{1}{\Fr^2}\rho\vec{g},\label{eq:MixtFracMom} \\
\pfrac{\rho z}{t}+	\nabla \cdot (\rho \vec{u}z) & = \frac{1}{\text{Re Pr}}\nabla \cdot\left(\rho D \nabla z\right), \label{eq:MixtFracMF}
	\end{align}
	\label{eq:all-eq-mixfrac}
\end{subequations}
which are solved together with the equation of state and expressions for the transport parameters. Note that the system is not closed, because $\rho$, $\mu$ and $\rho D$ are still functions of temperature and mass fractions. These fields can be related to the mixture fraction using the concept of the Burke-Schumann flame structure. \citep{burkeDiffusionFlames1928} In the case of an infinitely fast chemical reaction, fuel and oxidizer cannot co-exist. On one side of this sheet only oxidizer is found, and on the other side only fuel. The exact position of the flame sheet can be determined by finding the location of points where both reactant mass fractions $Y_F$ and $Y_O$  meet in stoichiometric proportions, that is, the points where the mixture fraction $z = z_{st}$, with
\begin{equation}
z_{st} = \frac{Y_O^0}{Y_O^0+sY_F^0},
\end{equation}
where $Y_O^0$ is the mass fraction of oxidizer in the oxidizer inlet stream, and $Y_F^0$ is the mass fraction of fuel in the fuel inlet stream. 
The Burke-Schumann solution provides analytical expressions for temperature and mass fraction fields on either side of the flame sheet as function of the mixture fraction $z$ (see for example the textbook from \cite{poinsotTheoreticalNumericalCombustion2005} or the work from \cite{keyesFlameSheetStarting1987}).


%%%%%%%%%%%%%%%%%%%%%



\begin{equation}\label{eq:BS-T}
T(z) =
\begin{dcases}
z T_F^0 + (1-z)T_O^0 + \frac{Q Y_F^0}{c_p}z_{st}\frac{1 - z}{1-z_{st}} & \text{if}\quad z \geq z_{st}\\
z T_F^0 + (1-z)T_O^0 + \frac{Q Y_F^0}{c_p}z & \text{if}\quad z < z_{st}
\end{dcases}
\end{equation}
The mass fraction field of fuel and oxidizer species at either side of the flame are given by:
\begin{equation}\label{eq:BS-YF}
Y_F(z) =
\begin{dcases}
Y_F^0\frac{z - z_{st}}{1-z_{st}} & \text{if}\quad z \geq z_{st}\\
0 & \text{if} \quad z < z_{st}
\end{dcases}
\end{equation}
\begin{equation}\label{eq:BS-YO}
Y_O(z) =
\begin{dcases}
0 & \text{if} \quad z \geq z_{st}\\
Y_O^0 \left( 1- \frac{z}{z_{st}}  \right) & \text{if} \quad z < z_{st}
\end{dcases}
\end{equation}
and finally, the mass fraction field of product species $P$ is:
\begin{equation}\label{eq:BS-YP}
Y_P(z) =
\begin{dcases}
Y_O^0\frac{W_P\nu_P}{W_O\nu_O}(1-z) & \text{if} \quad z \geq z_{st}\\
Y_F^0\frac{W_P\nu_P}{W_F\nu_F}z & \text{if} \quad z < z_{st}
\end{dcases}
\end{equation}

 Once the mixture fraction field $z$ is obtained, the temperature and mass fraction fields are uniquely defined by \cref{eq:BS-T,eq:BS-YF,eq:BS-YO,eq:BS-YP}, which are used to evaluate the density and the transport properties. This coupling between variables and the associated nonlinear system leads to the need for an iterative solution scheme.
 
 The main idea of introducing this additional system (\cref{eq:MixtFracConti2,eq:MixtFracMom,eq:MixtFracMF}) is to find an approximate solution to the system where a finite reaction-rate is used (\cref{eq:LowMach_Conti,eq:LowMach_Momentum,eq:LowMachEnergy,eq:LowMachMassBalance}). In \cref{fig:MixtureFraction_finiteRateComparison} temperature and fuel mass fraction fields across the center-line of a counterflow flame configuration is shown. Clearly, both solutions are very similar, differing only in the area near the flame. However, this similarity is only valid under the assumptions made to derive \cref{eq:MixtFracMF}. In the case that the Lewis number is not equal to one, or that the heat capacities are not equal for each species, the finite-rate solution will differ slightly from that obtained with the infinite-reaction rate. 
\newpage
\begin{figure}[tb]
	\pgfplotsset{
		group/xticklabels at=edge bottom,
		%		legend style = {
		%			at ={ (1.0,1.0), anchor= north west}
		%		},
		unit code/.code={\si{#1}}
	}
	\centering
	\begin{tikzpicture} [spy using outlines={magnification=2.5,  width=2cm, height=1cm, connect spies}]
	\begin{axis}[
	axis on top, 	
	width= 0.42\textwidth ,
	height= 0.3\textwidth ,
	xmin = 0,xmax=1,
	xlabel = x,
	ylabel = Temperature,
	legend pos=north east,
	]
	\addplot [no markers, green]table {data/MF_FULL_COMPARISON/TMF.txt}; \addlegendentry{Infinite rate}
	\addplot [no markers]table {data/MF_FULL_COMPARISON/TFull.txt}; \addlegendentry{Finite rate}
	\coordinate (spypoint) at (axis cs:0.45,6.3);
	\coordinate (magnifyglass) at (axis cs:0.72,3.5);
	\begin{scope}
	\spy [size=2cm] on (spypoint)
	in node[fill=white] at (magnifyglass);
	\end{scope}
	\end{axis}
	\end{tikzpicture}%
	\begin{tikzpicture} [spy using outlines={chamfered rectangle, magnification=6,  width=2cm, height=1cm, connect spies}]
	\begin{axis}[
	axis on top,
	width= 0.42\textwidth ,
	height= 0.3\textwidth,
	xmin = 0,xmax=1,
	legend pos=north east,	
	xlabel = x,
	ylabel = Fuel mass fraction,
	]
	\addplot [no markers,green]table {data/MF_FULL_COMPARISON/Y0MF.txt}; \addlegendentry{Infinite rate}
	\addplot [no markers]table {data/MF_FULL_COMPARISON/Y0Full.txt};  \addlegendentry{Finite rate}
	%		\addplot [no markers,dashed]table {data/MF_FULL_COMPARISON/Y1MF.txt};
	%		\addplot [no markers]table {data/MF_FULL_COMPARISON/Y1Full.txt};
	\coordinate (spypoint) at (axis cs:0.455,0.0055);
	\coordinate (magnifyglass) at (axis cs:0.7,0.07);
	\begin{scope}
	\spy [size=2cm] on (spypoint)
	in node[fill=white] at (magnifyglass);
	\end{scope}
	\end{axis}
	\end{tikzpicture}
	\caption{ Temperature and fuel mass fraction profiles calculated in the center-line of a counter-flow flame configuration using finite chemistry (black) and the flame sheet approximation (green). }
	\label{fig:MixtureFraction_finiteRateComparison}
\end{figure}
\newpage
\section{Boundary conditions}
The following boundary conditions are imposed for the resolution of the finite reaction rate system (\cref{eq:LowMach_Conti,eq:LowMach_Momentum,eq:LowMachEnergy,eq:LowMachMassBalance}) and for the flame sheet problem (\cref{eq:MixtFracConti2,eq:MixtFracMom,eq:MixtFracMF}).
\begin{subequations} 	
	\begin{align}
	\Gamma_{\text{D}}:\quad 
	&	\vec{u} = \vec{u}_{\text{D}},
	\,\,\,
	T = T_{\text{D}},
	\,\,\,
	Y_{k}, =Y_{k,{\text{D}}},
	\,\,\,
	z = z_{\text{D}},
	\label{eq:bc_d}\\
	\Gamma_{{\text{DW}}}:\quad
	&	\vec{u} = \vec{u}_{\text{D}},
	\,\,\,
	\nabla T \cdot \vec{n}_{\partial \Omega} = 0,
	\,\,\,
	\nabla Y_{k} \cdot  \vec{n}_{\partial \Omega}= 0,
	\,\,\, 		
	\nabla z \cdot \vec{n}_{\partial \Omega} = 0,
	\label{eq:bc_dn}\\			
	\Gamma_{\text{N}}: \quad
	&	\left( -p \mathbf{I}	+ \left(\frac{\mu}{\Reynolds} \left(\nabla \vec{u} + (\nabla \vec{u})^T\right) - \frac{2}{3}\mu(\nabla \cdot \vec{u})\mathbf{I}\right)\right)\cdot  \vec{n}_{\partial \Omega} 	= 0, 
	\,\,\,\notag\\
	&
	\nabla T \cdot \vec{n}_{\partial \Omega} = 0,
	\,\,\,
	\nabla  Y_{k} \cdot \vec{n}_{\partial \Omega}= 0,
	\,\,\, 		
	\nabla z \cdot \vec{n}_{\partial \Omega} = 0,
	\label{eq:bc_O}\\
	\Gamma_{\text{ND}}:\quad
	&	\left( -p \mathbf{I}	+ \left(\frac{\mu}{\Reynolds} \left(\nabla \vec{u} + (\nabla \vec{u})^T\right) - \frac{2}{3}\mu(\nabla \cdot \vec{u})\mathbf{I}\right)\right)\cdot  \vec{n}_{\partial \Omega} 	= 0, 
	\,\,\,\notag\\
	&T = T_{\text{D}},
	\,\,\,
	Y_{k} =  Y_{k,{\text{D}}},
	\,\,\,
	z = z_{\text{D}} 
	\label{eq:bc_OD}\\	
	\Gamma_{P}:\quad
	&	\vec{u}(\vec{x}) = \vec{u}(\vec{x}'),
	\,\,\,
	T(\vec{x}) = T(\vec{x}'),
	\,\,\,
	Y_{k}(\vec{x}) =  Y_{k}(\vec{x}'),
	\,\,\,
	z(\vec{x}) = z(\vec{x}'),
	\label{eq:bc_P}	
	\end{align}	
\end{subequations} 

where $k$ denotes the index from mass fractions $k = (1,\dots,N-1)$. The boundary $\Gamma_\text{D}$ represents conditions for inlets and walls, with velocity, temperature, mass fractions and mixture fraction defined as Dirichlet boundary conditions. Boundaries $\Gamma_\text{DW}$  are used to represent adiabatic walls, where velocity is given as a Dirichlet boundary condition, but with the gradients perpendicular to the wall of the transported scalars set to zero. The boundary $\Gamma_\text{N}$ represent an outflow of the domain with homogeneous Neumann condition for all scalars. The boundary $\Gamma_{\text{ND}}$ also  represents an outlet boundary condition, but with Dirichlet boundary conditions for the scalars. Finally the boundaries $\Gamma_P$ are periodic boundaries, where $\vec{x}$ and $\vec{x}'$ are periodic pairs in the domain.

\mycomment{
%TODO cosas por poner en algun lado
}
\cleardoublepage
\chapter{The Discontinous Galerkin method}	\label{ch:NumericalMethods}
This chapter aims to give an overview of the basic ideas of the DG method, as well as to present the spatial and temporal discretization of the equations presented earlier. Parts of this chapter are based on the work presented at \parencite{kummerExtendedDiscontinuousGalerkin2017,kikkerFullyCoupledHighorder, smudamartinDirectNumericalSimulation2021}, in order to maintain a consistent nomenclature with previous work where the BoSSS code is used. Additionally, the reader interested in a more in-depth description of the DG method is referred to the works of  \parencite{cockburnDevelopmentDiscontinuousGalerkin2000,hesthavenNodalDiscontinuousGalerkin2008,dipietroMathematicalAspectsDiscontinuous2012}

\section{State of the art}

\section{The Discontinous Galerkin method}

\subsection{Definitions for the discretization} \label{ssec:SpatDiscretization}
First some standard definitions and notation are introduced in the context of DG methods. 
A computational domain $\Omega \subset \mathbb{R}^2$ with a polygonal and simply connected boundary $\partial \Omega$ is defined. The numerical grid is then formed by the set of nonoverlapping elements $\gls{grid} = \{K_1, ..., K_J\}$ with a characteristic mesh size $h$, so that $\Omega$ is the union of all elements, i.e. $\Omega = \bigcup_{i=1}^J K_i$. 

Define $\Gamma = \bigcup_j \partial K_j$ as the union of all edges (internal edges and boundary edges) and $\Gamma_I = \Gamma \setminus \partial \Omega$ as the union of all interior edges.
For each edge of $\Gamma$ a normal field $\myvector{n}_{\Gamma}$ is defined. Particularly on $\partial \Omega$ is defined as an outer normal and $\vec{n}_\Gamma = \vec{n}_{\partial\Omega}$.
For each field ${u} \in C^0\left(\Omega\setminus \Gamma_I\right)$, ${u}^-$  and ${u}^+$ is defined, which describe the values of the variables on the interior and exterior sides of the cell:
\begin{align}
	{u}^- & = \lim_{\xi \searrow 0} {u}\left(\myvector{x} - \xi \myvector{n}_{\Gamma}\right) \quad \text{for } \myvector{x}\in \Gamma   \\
	{u}^+ & = \lim_{\xi \searrow 0} {u}\left(\myvector{x} + \xi \myvector{n}_{\Gamma}\right) \quad \text{for } \myvector{x}\in \Gamma_I
\end{align}
The jump and mean values of ${u}$ on the inner edges $\Gamma_I$ are defined as
\begin{align}
	\llbracket {u} \rrbracket & = {u}^+-{u}^-                           \\
	\{{u}\}                   & = \frac{1}{2} \left({u}^-+{u}^+\right).
\end{align}
while the jump and mean values on the boundary edges $\partial \Omega$ are:
\begin{align}
	\llbracket {u} \rrbracket & = {u}^-  \\
	\{{u}\}                   & = {u}^-.
\end{align}
Furthermore, the broken polynomial space of a total degree $k$ is defined as
\begin{equation}
	\mathbb{P}_k(\gls{grid} )= \{f \in L^2\left(\Omega\right); \forall K \in \gls{grid} : f\vert_{K} \text{ is polynomial and deg}\left(f\vert_{K}\right)\leq k \}.
	\label{Eq:PolSpace}
\end{equation}
Additionally, for $u \in \mathcal{C}^1(\Omega \setminus \Gamma)$ the broken gradient $\nabla_h u$ is defined as:
\begin{equation}
	\nabla_h u
	= \begin{cases}
		0
		 & \text{on }\Gamma  \\
		\nabla u
		 & \text{elsewhere }
	\end{cases}
\end{equation}
The broken divergence $\nabla_h \cdot u$ is defined analogously. Furthermore, the function space for test and trial functions for $D_v$ dependent variables is defined as
\begin{equation}
	\mathbb{V}_\myvector{k} = \prod_{i=1}^{D_v} \mathbb{P}_{k_i}(K_h)
	\label{Eq:Vspace}
\end{equation}
where $\myvector{k} = \left(k_1,...,k_{D_v}\right)$. 
Additionally, for a cell $K$ we define for $u_K$, $v_K\in \mathbb{V}_\myvector{k}$ a local inner product and a local $L^2$-norm as
\begin{equation}
(u_K, v_K)_K \coloneqq \int_K u_K v_K\text{d}x, \qquad \norm{u_K}^2_K \coloneqq(u_K,u_K)_K
\end{equation}
Similarly, for $u_h$, $v_h \in \mathbb{V}_\myvector{k}$ a global inner product and global broken norm are defined as
\begin{equation}
	(u_h, v_h)_{\Omega_h} \coloneqq \sum_{i = 1}^N (u_h, v_h)_K, \qquad \norm{u_h}^2_{\Omega_h} \coloneqq(u_h,u_h)_{\Omega_h}
\end{equation}
\subsection{Discretization using the DG Method} \label{sec:DiscWithDG}
In this subsection, the discretization by the DG method of a simple problem will be shown to demonstrate how the discretization method works and some of its specific characteristics. For this purpose, the discretization of a general conservation law for a scalar quantity $u = u(\vec{x},t)$ governed by a nonlinear flux function $\vec{f}(u)$ will be considered. In addition, suitable Dirichlet boundary conditions on $\partial \Omega = \partial \Omega_D$ and initial conditions $u_0$ are defined. The problem reads
\begin{subequations}
\begin{align}
&\pfrac{u}{t} + \nabla \cdot \vec{f}(u) = 0, \qquad\qquad\qquad &\vec{x} \in \Omega,\label{eq:consEqDG}\\
&u = u_D, \qquad \qquad \qquad  &\vec{x} \in \partial \Omega_D,\\
&u(\vec{x},0) = u_0(\vec{x}), \qquad\qquad\qquad &\vec{x}\in\Omega.
\end{align}\label{eqs:DGTransportExample}
\end{subequations}
The DG-method allows finding an approximate solution $u_h = u_h(\vec{x},t)$ for the problem \cref{eqs:DGTransportExample} by taking a linear combination of polynomial functions in each cell. 
%The error can be defined as: 
%\begin{equation}
%\gls{dgerror}(\vec{x}) = \nuM{\gls{DGVar}} - \gls{DGVar} \label{DGerror}
%\end{equation}
This approximate solution sought in the DG method is the best approximation of $\gls{DGVar} \in L^2(\gls{domain})$, which gives a minimum global error in the approximation space $\gls{DGVar} \in 	\mathbb{P}_k(\gls{domain} )$. 
\begin{align}
	\int_{\domain} (\underbrace{ \nuM{\gls{DGVar}}(x)-\gls{DGVar}(x) }_{=: \gls{dgerror}(x)})^2 \dV
	= ||\nuM{\gls{DGVar}} - \gls{DGVar}||_2^2 \rightarrow \text{min}
\end{align}
Here $\gls{dgerror}$ is the error of the discretization. Minimization is equivalent to requiring
\begin{equation}
	\label{eq:L2projection}
	\scp{r(x)}{\basis{}_m} = \scp{f_h-f}{\basis{}_m} \stackrel{!}{=} 0 \qquad \forall\basis{}_m
\end{equation}
This means that the error is orthogonal to every polynomial function $ \basis{}_m$ in the approximation space. The Bramble-Hilbert lemma \parencite{brambleEstimationLinearFunctionals1970} says, that for a $p$-times differentiable variable $\gls{DGVar}$, the error is of the order $\mathcal{O}(h^{p+1})$, which is one of the major motivations for the use of high order methods. The differentiability assumption is essential. For non smooth $\gls{DGVar}$ the well known Gibbs phenomenon occurs.

The discretization procedure starts by the approximation of the domain $\gls{domain}$ with a numerical grid $\gls{grid}$. In each cell $K_j$ of the numerical grid a set of polynomial basis $\vectr{\phi}_j = (\tilde{\phi}_{j,l})_{l=1,\dots,N_k} \in \mathbb{P}_k(\mathcal{K}_h)$ with a local cell support $\text{supp}(\phi_j) = \overline{\gls{cell}}_j$ is defined. This allows to represent the local solution for each cell $K_j$ as
\begin{equation}
u_j(\vec{x},t) = \sum_{l=1}^{N_k}\tilde{u}_{j,l}(t)\phi_{j,l}(\vec{x}) = \tilde{\vec{u}}_j(t) \cdot \vectr{\phi}_j (\vec{x})\label{eq:DGAnsatz}
\end{equation}
The coefficients $\tilde{\vectr{u}}_j = (\tilde{u}_{j,l})_{l=1,\dots,N_k}$ are the degrees of freedom (DOF) of the local solution in the cell $K_j$, which are the unknowns of the problem. Note the time dependence of the coefficients $\tilde{\vec{u}}_j$, as well as the dependence of the basis functions $\vectr{\phi}_j (\vec{x})$ on the vector $\vec{x}$. 

There are two general approaches used for the representation of the solution with the basis functions: modal and nodal. Each one of them present some advantages and disadvantages \parencite{hesthavenNodalDiscontinuousGalerkin2008}. In this work, a modal polynomial representation is used. The basis functions are chosen such that they are orthogonal to each other 
\begin{equation}
	\int_{K_j} \phi_{j,m}\phi_{j,n} \text{d}V= \delta_{mn}
\end{equation}
where $\delta_{mn}$ is the Kronecker delta. In the present work Legendre polynomials are used, since they present the orthogonality property. This property implies that matrix equals the identity matrix (at least for constant density flows).

By inserting the approximate solution in the conservation \cref{eq:consEqDG}, a local residual $R_j$ can be defined
\begin{equation}\label{eq:DGResidualeq}
R_j(\vec{x},t) = \pfrac{u_j}{t} + \nabla \cdot \vec{f}(u_j), \quad \vec{x} \in K_j
\end{equation}
Minimization of this local residual is done by multiplying \cref{eq:DGResidualeq} by the so called tests functions. In the Galerkin approach, these test functions are required to be from the same space as the trial functions, i.e. $\gls{testF}_{j,l} = \gls{basis}_{j,l}$. Thus, by multiplying \cref{eq:DGResidualeq} by a trial function and integrating over the cell $K_j$, one obtains
\begin{equation}
	\int_{\gls{cell}_j} \gls{DGres}_j \gls{testF}_{j,l} \d{V} = \int_{\gls{cell}_j}  \pDeriv{{\gls{DGVar}}_j}{t} \gls{basis}_{j,l} + \div{\gls{flux}({\gls{DGVar}}_j)} \gls{basis}_{j,l} \d{V} \stackrel{!}{=} 0, \quad \forall \vectr{\gls{basis}}_{j,l}.
	\label{eq:DGminimization}
\end{equation}
Where the minimization comes from requiring the equality to zero. Note that until this point only a cell-local discretization has been adressed. The next step for obtaining a global DG formulation is to use integration by parts for rewriting the spatial term in \cref{eq:DGminimization}, in order to explicitly make the boundary edge integrals appear. 
This is done to make the boundary edge integrals explicitly appear in the formulation, which are used to couple neighbouring cells. The partial integration process results in
 \begin{equation}
	\int_{\gls{cell}_j}  \pDeriv{{\gls{DGVar}}_j}{t} \gls{basis}_{j,l} \d{V} + \oint_{\partial{\gls{cell}}_j} \left( \gls{flux}({\gls{DGVar}}_j)\cdot{\gls{normal}}_j \right) \gls{basis}_{j,l} \d{S} - \int_{\gls{cell}_j} \gls{flux}({\gls{DGVar}}_j) \cdot \gradH{\gls{basis}}_{j,l} \d{V}  = 0, \quad \forall \gls{basis}_{j,l},
	\label{eq:DGfluxFormulation}
\end{equation} 
Note that inserting the ansatz \cref{eq:DGAnsatz} into \cref{eq:DGfluxFormulation} is problematic, since $\partial{\gls{cell}}_j$ is shared by other cells, and in the DG method continuity of a variable is not enforced across cell boundaries. This means that in general the inner value $u^{-}_j$ and the outer value $u^{+}_j$ are not equal. This problem is solved by introducing the concept of a numerical flux function, denoted here with $\gls{numflux}$
\begin{equation}
	\gls{numflux}(\inn{{\gls{DGVar}}_j}, \out{{\gls{DGVar}}_j}, \gls{normalGam}) \approx \gls{flux}({\gls{DGVar}}_j) \cdot {\gls{normal}}_j.
\end{equation}
This expression defines an unique value for the flux of a given cell boundary, enforcing flux continuity. The numerical flux $\gls{numflux}$ couples the DOFs of neighbouring cells, and should satisfy certain mathematical and physical properties which will be discussed later. Many different numerical fluxes have been developed, and it is an active area of investigation. They differ mainly in computational cost, stability and dissipation of the scheme. By introducing the numerical flux in \cref{eq:DGfluxFormulation} the problem now reads
\begin{equation}
	\int_{\gls{cell}_j}  \pDeriv{{\gls{DGVar}}_j}{t} \gls{basis}_{j,l} \d{V} + \oint_{\partial{\gls{cell}}_j} \left( \gls{numflux}(\inn{{\gls{DGVar}}_j}, \out{{\gls{DGVar}}_j}, \gls{normalGam})    \right) \gls{basis}_{j,l} \d{S} - \int_{\gls{cell}_j} \gls{flux}({\gls{DGVar}}_j) \cdot \gradH{\gls{basis}}_{j,l} \d{V}  = 0, \quad \forall \gls{basis}_{j,l},
	\label{eq:DGNumericalfluxFormulation}
\end{equation} 
\subsubsection{The global formulation}
Note that \cref{eq:DGNumericalfluxFormulation} is still a local formulation. A global solution $u(\vec{x},t)$ can be defined by a piecewise polynomial approximation according to 
\begin{equation}
	u(\vectr{x},t) \approx  u_h(\vectr{x},t) = \bigoplus\limits_{j=1}^{J} {\gls{DGVar}}_j(\vectr{x},t) = \sum_{j=1}^{J} \sum_{l=1}^{\gls{NoDOFloc}} \tilde{\gls{DGVar}}_{j,l}(t) \gls{basis}_{j,l}(\vectr{x}) \in \gls{brknPspacek}(\gls{grid})
	\label{eq:globalApprox}
\end{equation}
which corresponds to the direct sum of the $J$ local solutions $u_j$. A vector $\tilde{\vec{u}} = \tilde{\gls{DGVar}}_{1,1},\allowbreak \tilde{\gls{DGVar}}_{1,2},\allowbreak\dots, \allowbreak \tilde{\gls{DGVar}}_{j,l},\allowbreak\dots,\allowbreak\tilde{\gls{DGVar}}_{J,\gls{NoDOFloc}}\allowbreak$ which comprises the DOFs of the global approximation $\gls{DGVar}_h$ is defined, and is of length $\gls{NoDOF} = J\cdot\gls{NoDOFloc}$.%

Finally, the global formulation is obtained by inserting the ansatz \cref{eq:DGAnsatz} into \cref{eq:DGNumericalfluxFormulation}, summing over all cells $K_j$ and making use of \cref{eq:globalApprox}. The problem reads: Find $\nuM{\gls{DGVar}} \in 	\mathbb{P}_k(\gls{grid})$, such that $\forall \phi \in 	\mathbb{P}_k(\gls{grid})$
\begin{equation}
	\int_{\gls{domain}}  \pDeriv{\nuM{\gls{DGVar}}}{t} \gls{basis} \d{V}  + \oint_{\gls{edge}} \gls{numflux}(\inn{\nuM{\gls{DGVar}}}, \out{\nuM{\gls{DGVar}}}, \gls{normalGam}) \jump{\gls{basis}} \d{S} - \int_{\gls{domain}} \gls{flux}(\nuM{\gls{DGVar}}) \cdot \gradH{\gls{basis}} \d{V} = 0,
	\label{eq:semiDiscWeakForm}
\end{equation}
The solution of this system requires finding the DOFs $\tilde{\vec{u}}$ of the global approximation $\gls{DGVar}_h$. Dirichlet boundary conditions are included in the formulation by defining at $\gls{edge}_D$ the outer value $\out{\gls{DGVar}}_h = \gls{DGVar}_D$.

Note that \cref{eq:semiDiscWeakForm} is semi-discrete, meaning that the system of equations has been discretized in space, but not in time. Time discretization will be treated in \cref{ssec:TemporalDiscretization}.

Finally, after selecting suitable numerical fluxes for the various terms of the governing equations, a system is obtained that in general has the form
\begin{equation}
	 \deriv{ }{t} \left( {\gls{massM}}(\tilde{\vectr{\gls{DGVar}}})\tilde{\vectr{\gls{DGVar}}} \right) + \gls{OpM}(\tilde{\vectr{\gls{DGVar}}}) = \vectr{b},
	\label{eq:discretMatrixForm}
\end{equation}
Where $\gls{massM}$ is the mass matrix, and $\gls{OpM}$ is the operator matrix.  The vector $b$ contains the Dirichlet boundary condition. The operator matrix is defined locally by
\begin{equation}
	(\gls{OpM}_j)_{m,n} =  \oint_{\partial{\gls{cell}}_j} \gls{numflux}(\tilde{{\gls{DGVar}}}_{j,n}, \tilde{{\gls{DGVar}}}_{j^{\ast},n},\gls{normalI}) \gls{basis}_{j,m} \d{S} - \int_{\gls{cell}_j} \gls{flux}(\tilde{{\gls{DGVar}}}_{j,n} {\gls{basis}}_{j,n}) \cdot \gradH{\gls{basis}}_{j,m} \d{V} ,
	\label{eq:genOpMatrix}
\end{equation}
with $j^{\ast}$ denoting the index of a neighbour cell. The matrix $\gls{OpM}$ has block-diagonal structure, but also including extra diagonals which relate the DOFs of the cell with the neighbouring cells.

The global mass matrix is
\begin{equation}
	{\gls{massM}} =
	\left[ 
	\begin{array}{cccc}
		{\gls{massM}}_{1} & 0 & \cdots & 0 \\
		0 & {\gls{massM}}_{2} & \cdots & 0 \\
		\vdots & \vdots & \ddots & \vdots \\
		0 & 0 & \cdots & {\gls{massM}}_{J}
	\end{array}
	\right],
\end{equation} 
which is block-diagonal, since  ${\gls{massM}}_{j}$ does not depend on neighbouring cells. 
\begin{equation}
	({\gls{massM}}_j)_{m,n} = \int_{\gls{cell}_j} \gls{basis}_{j,m} \gls{basis}_{j,n} \d{V}	\label{eq:massMatrix}
\end{equation} 
The mass matrix of a cell ${\gls{massM}}_{j} := \matrixDG{M}_{(j,-) \ (j,-)} $ only depends in this case on the cell-local basis functions. 
\todo[inline]{How is the mass matrix defined for variable density flows?}

\subsubsection{Note on the numerical fluxes}
As mentioned before, the numerical fluxes $\gls{numflux}$ have to fulfil certain physical and mathematical properties for obtaining a stable and convergent method. One of the requirement for proving the stability of the scheme is the monotonicity. First, the energy estimate is defined as
\begin{equation}
	\norm{\gls{DGVar}(\vectr{x},t)}_{\gls{domain}}^2 \leq \norm{\gls{DGVar}(\vectr{x},0)}_{\gls{domain}}^2, \quad \forall t \geq 0,
	\label{eq:energyEstimate}
\end{equation}
assuming homogeneous Dirichlet boundary conditions. This means that the system is stable if the energy norm $\norm{\gls{DGVar}(\vectr{x},t)}_{\gls{domain}}^2 $ is strictly decreasing in the absence of inflow. 

For a monotonic numerical flux is possible to show that the discrete problem \cref{eq:semiDiscWeakForm} satisfies the discrete equivalent of \cref{eq:energyEstimate}. In addition to the monotonicity of the numerical flux, it is necessary that $\gls{numflux}$ is Lipschitz continuous.%, which means
%\begin{equation}
%	\label{eq:flux_lipschitz_first}
%	\exists C_a \in \mathbb{R}: \abs{\gls{numflux}(a_1, b, \vec{n}) - \hat{f}(a_2, b, \vec{n})} \leq C_a \abs{a_1 - a_2} 
%	\qquad \forall a_1, a_2 \in \mathbb{R}
%\end{equation}
%and
%\begin{equation}
%	\label{eq:flux_lipschitz_second}
%	\exists C_b \in \mathbb{R}: \abs{\gls{numflux}(a, b_1, \vec{n}) - \hat{f}(a, b_2, \vec{n})} \leq C_b \abs{b_1 - b_2} 
%	\qquad \forall b_1, b_2 \in \mathbb{R}
%\end{equation}
For the complete proof the interested reader is referred to the book from \textcite{dipietroMathematicalAspectsDiscontinuous2012}.

Two more requirements are needed for the numerical flux. The first is the consistency of the flux, which can be written as
\begin{equation}
	\gls{numflux}(a,a,\gls{normal}) = \gls{flux}(a) \cdot \gls{normal}, \quad \forall a \in \mathbb{R}. 
	\label{eq:consistency}
\end{equation}
showing that a numerical flux function should deliver the same approximate solution that the original flux function in case of a continuous variable across the interface. A direct consequence of the consistency of the numerical flux is that the weak formulation \cref{eq:semiDiscWeakForm} is automatically fulfilled by $\nuM{\gls{DGVar}} = \gls{DGVar}$.

Finally the last requirement is that the numerical flux should be conservative, which means that the total amount of $\gls{DGVar}$ can only change due to fluxes across the domain boundary. This can be written as
\begin{equation}
	\gls{numflux}(a,b,\gls{normal}) = 	-\gls{numflux}(b,a,-\gls{normal}), \quad \forall a,b \in \mathbb{R}. 
	\label{eq:conservativity}
\end{equation}
All numerical fluxes used in this work for the spatial discretization of \cref{eq:all-eqs} fulfil these requirements and they will be shown in the next section.
\subsection{Temporal discretization}\label{ssec:TemporalDiscretization}
This section will give a brief introduction to the most popular time-stepping techniques and then show the time-stepping method used in this work. This section is mainly based on \textcite{levequeFiniteVolumeMethods2002,ferzigerComputationalMethodsFluid2002}.

In the previous chapter the spatial discretization using a DG method was shown, resulting in the semi-discrete formulation given by \cref{eq:semiDiscWeakForm}. The time discretization of this semi-discrete system leads to the so called method of lines, which is the name for a method first discretized in space, and later in time.  

An alternative to the method of lines is the so called Rothe's method, where time is first discretized then the space. This can be advantageous in some cases, such as problems with a moving domain. Another alternative is the space-time approach, where basically the temporal coordinate is treated as another spatial dimension. Again, the method can be very attractive for some cases. However, the discretized schemes lead often to prohibitively large systems. These approaches are ignored in the present work, and the method of lines is adopted.

First the time discretization for a system with a constant mass matrix will be discussed. The process of discretization results often in a system of ordinary differential equations (ODEs) of the form
\begin{equation}
	\frac{\text{d}\vec{u}}{\text{d}t} = - \gls{massM}^{-1}\vec{F}(t,\vec{u}(t))\qquad \text{for}\qquad t \in (0,T).\label{eq:timestep1}
\end{equation}
with a initial condition given by $\vec{u}(t=0) = \vec{u}^0$. The main idea of a time-stepper algorithm is to discretize the time coordinate, and advance gradually the solution $\vec{u}(t^n)$ in time from values at previous time levels $\vec{u}(t^{n-1})$, $\vec{u}(t^{n-2})$, $\dots$,  until a certain final time $t = T$ is reached.
By integrating \cref{eq:timestep1} in time, one obtains
\begin{equation}
\vec{u}(t^{n+1})-\vec{u}(t^{n}) = - \gls{massM}^{-1}\int_{t^n}^{t^{n+1}}\vec{F}(t,\vec{u}(t)) \text{d}t \label{eq:timestep2}
\end{equation}
This equation is the starting point for different class of time stepping techniques. Two kind of methods can be distinguished, depending on how the integral in \cref{eq:timestep2} is evaluated: explicit methods and explicit methods. Explicit methods are obtained when the approximation of the integral is done only by using information from old time steps, while for implicit methods the information from the actual timestep is also considered, making necessary the solution of a system of equations.

The simplest example of a explicit time-stepping method is the Explicit Euler Method: $\vec{u}(t^{n+1})= \vec{u}(t^{n})  - \Delta t \gls{massM}^{-1}\vec{F}(t^n,\vec{u}(t^n))$, which is first order accurate in time. Other methods exists with better properties than the Explicit Euler Method, typically using information from multiple known time levels or a interpolation of them. Adams-Bashforth methods are a good exponent of them. 

Due to the local nature of the approach, explicit methods present themselves specially attractive for DG methods, particularly for hyperbolic equations. Explicit methods are relatively easy to implement, and need considerably less storage compared to implicit methods. However, explicit methods experience the disadvantage that, in order to obtain a stable algorithm, they are limited by a maximal timestep size $\Delta t$. The timestep typically scales with the grid size $h$ and polynomial degree $k$ by $\Delta \sim h/k$ for hyperbolic and $\Delta \sim (h/k)^2$ for parabolic problems \parencite{gassnerContributionConstructionDiffusion2007}. For many problems of interest this limitation could be highly restrictive, as too little timesteps need to be chosen in order to obtain a stable method. 

Implicit methods are specially well suited for stiff problems, as they don't suffer from the timestep limitation of explicit methods, even not being restricted at all under certain conditions, which allows using considerably bigger timesteps. They present however the inconvenience that they require the solution of a system of equations, which for large problems is not a trivial task and specialized methods could be needed.

The simplest implicit method is the Implicit Euler Method ${\vec{u}(t^{n+1})= \vec{u}(t^{n})  - \Delta t \gls{massM}^{-1}\vec{F}(t^n,\vec{u}(t^n))}$. The Implicit Euler Methods is the first method of a family of backward differentiation formulas (BDF), and presents the property of being unconditionally stable, meaning that the algorithm allows an arbitrarily large timestep. This property allows the calculation of steady state solutions just by choosing a very big $\Delta t$ value.
\todo[inline]{why?}
\begin{table}[t]
	\centering
	\begin{tabular}{lllllll}
		\hline
		$s$                   & $\gamma$ & $\beta_0$ & $\beta_1$ & $\beta_2$ & $\beta_3$ & $\beta_4$ \\ \hline
		Implicit Euler (BDF1) & 1        & 1         & -1        &           &           &           \\
		BDF2                  & 2        & 3         & -4        & 1         &           &           \\
		BDF3                  & 6        & 11        & -18       & 9         & -2        &           \\
		BDF4                  & 12       & 25        & -48       & 36        & -16       & 3         \\ \hline
	\end{tabular}
	\caption{Coefficients of the BDF schemes.}
	\label{tab:BDFCoeff}
\end{table}
In the present work BDF methods are used. In case of a non-constant mass matrix $\gls{massM}$, they have a general formula given by
\begin{equation}
	\frac{\beta_0}{\gamma\Delta t}\gls{massM}(\vec{u}(t^n))\vec{u}(t^n)- \vec{F}(\vec{u}(t^n)) = - \sum_{i=1}^s \frac{\beta_i}{\gamma \Delta t}\gls{massM}(\vec{u}(t^{n-i}))\vec{u}(t^{n-i}).
\end{equation}\label{eq:BDFDiscretization}
where $s$ is the order of the BDF-scheme. The coefficients of each schema are shown in \cref{tab:BDFCoeff}. The solution of these kind of nonlinear problems will be treated in \cref{sec:CompMethodology}.



%%%%%%%%%%%%%%%%
%. It is well known that the inclusion of the $\partial\rho /\partial t$ term of the continuity equation in the source term, as done in the present work, is a source of numerical instability. In the work of \textcite{nicoudNumericalStudyChannel} it is reported that obtaining solutions for density ratios greater than three  becomes difficult.In \textcite{rauwoensConservativeDiscreteCompatibilityconstraint2009} a similar destabilization effect is also reported for high density ratio
\textcite{nicoudConservativeHighOrderFiniteDifference2000}
Special attention should be put into the discretization of the temporal derivative appearing in the continuity equation, \cref{eq:LowMach_Conti}. It has been observed that the treating this term as a source term, for cases where high density ratios are present, is problematic \parencite{cookDirectNumericalSimulation1996,nicoudConservativeHighOrderFiniteDifference2000}. \textcite{cookDirectNumericalSimulation1996} reported for a pressure projection method, and using a third-order Adams-Bashforth scheme, that the discretization of the $\partial \rho /\partial t$ is a source of instabilities. They used a second-order explicit approximation
\begin{equation}
	\left(\frac{\partial \rho}{\partial t} \right)^n= \frac{1}{2\Delta t}\left(3\rho^n-2\rho^{n-1}+\rho^{n-2}\right)
\end{equation}
which is reported to be much more stable than a third-order approximation, by arguing that the extra dissipation from even-ordered schemas, compared to the dispersive effects of odd-ordered schemas, is helpful for the stability of the algorithm. However, they found that even with even-ordered time approximations of the density time derivative, the algorithm is stable only to maximum density variations up to a factor of three. 

This approximation of the time derivative is used in this work. Some comments in this will be done later in \cref{ch:results}
\todo[inline]{Dos preguntas: Porque tengo que incluir la densidad como un source term?}
\todo[inline]{Es valido comprar en terminos de estabilidad mi approach (implicito si se ignora eltermino drhodt) con un approach como el del pressure projection method?}
% Even-ordered finite difference approximations to this derivative
%were found to be more stable but density ratios larger than 3 are difficult to
%compute. Sandoval (reported in [6]) found that by decreasing the Reynolds
%number, larger variations in density could be achieved. Larger density ratios
%seem computable by using a predictor-corrector time-stepping algorithm in
%which the predictor uses a second-order Adams-Bashforth time integration
%4
%scheme and the corrector relies on a quasi-Crank-Nicolson integration with
%the inversion of a pressure Poisson equation at each step [7, 8]
\todo[inline]{Escribir el problema como Ax = b}
The methods for solving nonlinear problems such as the one given by \cref{eq:BDFDiscretization} be shown in \cref{sec:CompMethodology}.
\section{Discontinuous Galerkin discretization of the low-Mach equations}
The democratization methodology shown in last section is used for finding a discrete formulation of the governing equations for low-Mach reactive flows. In the next sections the discretization for the fully coupled problem with finite reaction rate, and for the flame sheet problem are shown. The chosen numerical fluxes are also shown, and some of their particularities are discussed. 

In order to ensure the validity  of the  Ladyzenskaja-Babu\u{s}ka-Brezzi (or inf-sup) condition, (see \textcite{babuskaFiniteElementMethod1973})  a mixed order formulation is used in all calculations, where polynomials of order $k$ for velocity, temperature, mass fractions and mixture fractions, and of degree $k' = k-1$ for pressure are used. This is a required compatibility condition for obtaining a well posed problem. 
\subsection{Discontinuous Galerkin discretization of the finite reaction rate problem}
Here the DG discretization of the finite reaction system defined by \crefrange{eq:LowMach_Conti}{eq:LowMachMassBalance} is presented. 
First, the vector $\vec{Y}' = \left(Y_1,\dots,Y_{N-1}\right)$ is defined as the vector containing the first $(\gls{TotalNumberSpecies}-1)$ mass fractions and $\vec{s} = \left(s_1, \dots, s_{N-1} \right)$ as the vector containing the test functions for the first $(\gls{TotalNumberSpecies}-1)$  mass fraction equations. 

The discretized form of \crefrange{eq:LowMach_Conti}{eq:LowMachMassBalance} is obtained in a similar fashion to the methodology shown in \cref{sec:DiscWithDG}. This means, each equation is multiplied by a test function, integrated it over an element $K$, applying integration by parts, using an adequate numerical flux for each term and summing over all cells in order to obtain a global formulation. Note that the convective and diffusive terms of the temperature scalars $T$, mass fraction $Y_\alpha$ and mixture fraction $z$ have the same form, so they share the same expression in their discretized form.

Finally, the discretized problem can be written as: find the numerical solution $(p_h,\vec{u}_h, T_h, \MFVecPrima_h) \in \mathbb{V}_\myvector{k}$ such that for all test functions $(q_h,\vec{v}_h, r_h, \mathbf{s}_h) \in \mathbb{V}_\myvector{k}$:
\begin{subequations}
	\begin{flalign}%
		%% Continuity
		&\gls{BCcont}(q_h)= \ContDis\ + \mathcal{T}(\partial_t \rho|_{t^{n+1}},q_h ) ,& \label{DiscretizedConti}\\[1ex]
		%% Momentum
		&\gls{BCmom}(\vec{v}_h) =	\MomConv + \MomPres + \MomDiff & \notag\\
		& \quad\quad\quad + \MomSource,& \label{DiscretizedMomentum}\\[1ex]
		%% Energy
		&\gls{BCenergy}(r_h)
		 = \mathcal{S}^C\left(\vec{u}_h,T_h,r_h, \rhoh\right) + \mathcal{S}^{D,E}\left(T_h,r_h,k/c_p(T_h)\right)&  \notag\\
		& \quad\quad\quad + \mathcal{S}^S\left(r_h, \heatRelease(T_h,\vec{Y}_h), \rateReac(T_h,\vec{Y}_h),\cph\right),& \label{DiscretizedEnergy}   \\[1ex]
	%% MassFractions
		&\gls{BCMass}(s_{\alpha h})= \mathcal{S}^C\left(\vec{u}_h,\Yi, s_{\alpha h}, \rhoh\right) + \mathcal{S}^{D,M}\left(\Yi,s_{\alpha h},\rhodh\right)&  \notag \\
		& \quad\quad\quad\quad + \mathcal{M}^S_\alpha\left(s_{\alpha h},\rateReac(T_h,\vec{Y}_h )\right).& \label{DiscretizedMassFractions}
	\end{flalign}\label{eqs:variatProblemFull}
\end{subequations}
\todo[inline]{Add the temporal terms}
where the index $\alpha$ takes values $\alpha = 1, \dots,~(\gls{TotalNumberSpecies} - 1)$. Each one of the forms introduced here are shown later in \cref{ssec:nonLinearforms}.
\subsection{Discontinuous Galerkin discretization of the flame sheet problem}
Discretizing the flame sheet problem given by \crefrange{eq:MixtFracConti2}{eq:MixtFracMF} proceeds in a similar way. Due to the similarity of the mass fraction equation and the mixture fraction equation the discretization is analogous.

The resulting problem reads: find the numerical solution $(p_h,\vec{u}_h,z_h)\in \mathbb{V}_\myvector{k}$ such that for all test functions $(q_h,\vec{v}_h,r_h)\in \mathbb{V}_\myvector{k}$ we have:
\begin{subequations}
	\begin{flalign}
		%% Continuity
		&\mathcal{B}^1(q_h)=\mathcal{C}\left(\vec{u}_h,q_h, \rho(z_h)\right),& \label{DiscretizedConti2}\\
		%% Momentum
		&\mathcal{B}^2(\vec{v}_h) =\mathcal{U}^C\left(\vec{w}_h,\vec{u}_h,\vec{v}_h, \rho(z_h)\right)+ 	\mathcal{U}^P\left(p_h,\vec{v}_h\right) +\mathcal{U}^D\left(\vec{u}_h,\vec{v}_h,\mu(z_h)\right) +\mathcal{U}^S\left(\rho(z), \vec{v}_h\right),& \label{DiscretizedMomentum2}\\
		%% Mixture Fraction	 
		&\mathcal{B}^3(r_h) = {S}^C\left(\vec{u}_h,z_h,r_h, \rho(z_h)\right) + \mathcal{S}^{D,E}\left(z_h,r_h,\rho D(z_h)\right).& \label{DiscretizedEnergy2}
	\end{flalign}\label{eqs:variatFS}
\end{subequations}
Note that density and transport parameters are dependent on the mixture fraction $z$. The evaluation of those parameters is done as mentioned in \cref{sec:FlameSheet} and solved iteratively using a Newton-Dogleg type method as shown later in \cref{sec:newton}
\subsection{Definitions of nonlinear forms}\label{ssec:nonLinearforms}
In the following the nonlinear forms used in this work are shown. Regarding the choice of fluxes, the "best practices" known in literature for the incompressible Navier-Stokes equation are followed. These fluxes proved to be well suited for all the problems discussed in this thesis, providing stability to the algorithm, while maintaining the accuracy of the solver.

It is well known \parencite{pietroMathematicalAspectsDiscontinuous2012,giraultDiscontinuousGalerkinMethod2004} that central difference fluxes for the pressure gradient and velocity divergence, combined with a coercive form for the viscous terms, e.g. symmetric interior penalty, gives a stable discretization for the Stokes equation. Furthermore, it is known that for all kinds of convective terms, a numerical flux which transports information in characteristic direction, e.g. Upwind, Lax-Friedrichs or Local-Lax-Friedrichs, must be used. We opted for the last one in the present implementation, as it offers a good compromise between accuracy and stability.
\paragraph{Continuity equation}
A central difference flux for the discretization of the continuity equation is used:
\begin{equation}
	\mathcal{C}(\vec{u},q, \rho)  =  \oint_{\GammaI \cup\GammaN\cup \GammaND\cup \GammaP}{\mean{\rho\vec{u} }\cdot \vec{n}_\Gamma\jump{q} \dS} - \int_{\Omega}{\rho \vec{u}\cdot \nabla_h q} \dV.  \label{eq:Conti}
\end{equation}
The density in \cref{eq:Conti} is evaluated as a function of the temperature and mass fractions using the equation of state (\cref{eq:ideal_gas}). The term $\mathcal{B}^1$ on the left hand sides of \cref{DiscretizedConti} and \cref{DiscretizedConti2}  contains the Dirichlet boundary conditions:
\begin{equation}
	\mathcal{B}^1(q) =  -\oint_{\GammaD\cup \GammaDW}{q(\rho_{\text{D}} \vec{u}_{\text{D}} \cdot \normalBoundary). \dS}
\end{equation}
The density at the boundary  $\rho_{\text{D}}$ is evaluated with \cref{eq:ideal_gas} using the corresponding Dirichlet values of temperature and mass fractions.
\begin{equation}
\mathcal{T}() =   \int_{\Omega}{  ? q} \dV.
\end{equation}
\todo[inline]{Add the source term for the continuity equation}
\paragraph{Momentum equations}
The convective term of the momentum equations is discretized using a Lax-Friedrichs flux
\begin{equation}
	\mathcal{U}^C(\vec{w},\vec{u},\vec{v}, \rho)=  \oint_{\Gamma}{\left( \mean{\rho\vec{u}\otimes\vec{w} }\vec{n}_\Gamma + \frac{\gamma_1}{2}\jump{\vec{u}}\right)\cdot\jump{\vec{v}} \dS}
	-\int_{\Omega}({\rho\vec{u}\otimes\vec{w}}):\nabla_h\vec{v} d\text{V}.
	\label{eq:Mom_convective}\\
\end{equation}
The Lax-Friedrichs parameter $\gamma_1$ is calculated as \textcite{kleinHighorderDiscontinuousGalerkin2016}
\begin{equation}
	\gamma_1  = \max \left\{2 \overline{\rho^+} |\overline{\vec{u}^+} \cdot \vec{n}^+|,2 \overline{\rho^-} |\overline{\vec{u}^-} \cdot \vec{n}^-|\right\},
	\label{eq:vardens_lambda}
\end{equation}
\todo[inline]{How is this actually implemented?}
where $\overline{\rho_{h}^\pm}$ and $\overline{\vec{u}^\pm}$ are the mean values of $\rho^\pm$ and $\vec{u}^\pm$ in $K^\pm$, respectively.\\
The pressure term is discretized by using a central difference flux
\begin{equation}
	\mathcal{U}^P(p,\vec{v})=  \oint_{\Gamma \setminus \Gamma_{\text{N}}\setminus \Gamma_{\text{ND}}}{ \mean{p}(\jump{\vec{v}}\cdot \vec{n}_\Gamma  )\dS}
	- \int_{\Omega}{p \nabla_h \cdot \vec{v} \dV}. \label{eq:Mom_pressure}
\end{equation}
The diffusive term of the momentum equations is discretized using an Symmetric Interior Penalty (SIP)  formulation \parencite{shahbaziExplicitExpressionPenalty2005}
\begin{equation}
	\begin{aligned}
		\tilde{\mathcal{U}}^D(\vec{u},\vec{v},\mu) =
		  & \int_{\Omega}{\left(\mu\left((\nabla_h \vec{u}) + (\nabla_h \vec{u})^T - \frac{2}{3}(\nabla_h\cdot \vec{u})\mytensor{I} \right)\right)\colon \nabla_h\vec{v}} \dV \\
		- & \oint_{\Gamma \setminus \Gamma_{\text{N}}\setminus \Gamma_{\text{ND}}}
		\left(\mean{\mu(\nabla_h\vec{u} + \nabla_h\vec{u}^T - \frac{2}{3}(\nabla_h\cdot \vec{u})\mytensor{I})}\normalBoundary\right)\cdot\jump{\vec{v}}\dS                    \\
		- & \oint_{\Gamma \setminus \Gamma_{\text{N}}\setminus \Gamma_{\text{ND}}}
		\left(\mean{\mu(\nabla_h\vec{v} + \nabla_h\vec{v}^T - \frac{2}{3}(\nabla_h\cdot \vec{v})\mytensor{I})}\normalBoundary\right)\cdot\jump{\vec{u}} \dS                   \\
		+ & \oint_{\Gamma \setminus \Gamma_{\text{N}}\setminus \Gamma_{\text{ND}}} \eta \mu_{\text{max}} \jump{\vec{u}} \jump{\vec{v}}\dS.
		\label{eq:Mom_diffusive}
	\end{aligned}
\end{equation}
The viscosity $\mu$ is evaluated as a function of temperature according to \cref{eq:nondim_sutherland} and $\mu_{\text{max}} = \text{max}(\mu^{+}, \mu^{-})$.  Additionally  $\eta$ is the penalty term of the SIP formulation, which has to be chosen big enough to ensure coercivity of the form, but also as small as possible in order to not increase the condition number of the problem. The estimation of the penalty term is based on an expression of the form
\begin{equation}
	\eta = \eta_0 \frac{A(\partial K)}{V(K)},
\end{equation}\label{eq:PenaltyFactor}
where for a two-dimensional problem $A$ is the perimeter and $V$ the area of the element. Parameter $\eta_0$ is a safety factor. If not stated otherwise, the value  $\eta_0 = 4$ is  set in all calculations. Further information on the determination of the penalty term of the SIP formulation $\eta$ and the penalty term of the Lax-Friedrichs $\gamma_1 $ can be found in  the works from \textcite{hesthavenNodalDiscontinuousGalerkin2008} and \textcite{hillewaertDevelopmentDiscontinuousGalerkin2013}.

Note that the diffusive term of the momentum equations is scaled by the Reynolds number, obtaining finally
\begin{equation}
		\mathcal{U}^D(\vec{u},\vec{v},\mu) = 	\frac{1}{\Reynolds}\tilde{\mathcal{U}}^D(\vec{u},\vec{v},\mu)
\end{equation}

The source term arising due to body forces is:
\begin{equation}
	\mathcal{U}^S\left(\rho, \vec{v}\right) =  \smash{\frac{1}{\Froude^2}}\int_{\Omega}{  \rho \frac{\vec{g}}{\Vert \vec{g} \Vert}\cdot \vec{v}} \dV.  \label{eq:Mom_source}
\end{equation}
Finally, the right hand sides of \cref{DiscretizedMomentum} and \cref{DiscretizedMomentum2} contain the information from Dirichlet boundary conditions:
\begin{equation}
	\mathcal{B}^2(\vec{v}) =
	-\oint_{\Gamma_{\text{D}}}{ \left( (\rho\vec{u}_{\text{D}}\otimes\vec{u}_{\text{D}} )\normalBoundary + \frac{\gamma_1}{2}\vec{u}_{\text{D}}\right)\cdot\vec{v} \dS}  +
	\oint_{\Gamma_{\text{D}}}{\mu_{\text{D}}\vec{u}_{\text{D}}\cdot(\nabla_h\vec{v} \normalBoundary + \nabla_h\vec{v}^T \normalBoundary- \eta \vec{v} )\dS.}
\end{equation}
The Dirichlet viscosity value $\mu_{\text{D}}$ is calculated from \cref{eq:nondim_sutherland} using the Dirichlet values of the temperature at the boundary.
\paragraph{Scalar equations}
Since the convective and diffusive term for the temperature, mass fractions and mixture fraction share a similar form,  here their discretized expressions are summarized in terms of an arbitrary scalar $X$ (corresponding to $T$ in the energy equation, $Y_\alpha$ in the equation for species $\alpha$ and $z$ for the mixture fraction equation) and transport parameter $\xi$ (i.e. $k/c_p$ in the energy equation, and $(\rho D)$ for the mass fraction and mixture fraction equations). The convective term of the scalars is discretized using a Lax-Friedrichs flux
\begin{equation}
	\mathcal{S}^C(\vec{u},X,r, \rho) =  \oint_{\Gamma}{\left( \mean{\rho\vec{u}X }\cdot \vec{n} + \frac{\gamma_2}{2}\jump{X}\right)\jump{r} \dS}
	- \int_{\Omega}({\rho \vec{u} X \cdot \nabla_h r) d\text{V}}. \label{eq:scalar_convective}
\end{equation}
The Lax-Friedrichs parameter $\gamma_2$ is calculated as 
\begin{equation}
	\gamma_2  = \max \left\{\overline{\rho^+} |\overline{\vec{u}^+} \cdot \vec{n}^+|,\overline{\rho^-} |\overline{\vec{u}^-} \cdot \vec{n}^-|\right\}.
	\label{eq:vardens_lambda2}
\end{equation}
The diffusion term of scalars is discretized again with a SIP formulation:
\begin{align}
	\mathcal{S}^D(X,r,\xi)= & \int_{\Omega}{ \left(\xi \nabla_h X \cdot\nabla_h r\right) }\dV \notag         \\
	                        & -\oint_{\Gamma \setminus \Gamma_{\text{N}}\setminus \Gamma_{\text{ND}}}{\left(
		\mean{\xi \nabla_h X}\cdot \vec{n}\jump{r} +
		\mean{\xi \nabla_h r}\cdot \vec{n}\jump{X} -
		\eta \xi_{\text{max}} \jump{X} \jump{r}
		\right) \dS.
	} \label{eq:Temp_diffusive}
\end{align}
The transport parameter $\xi$ is calculated as a function of temperature using \cref{eq:nondim_sutherland} and $\xi_{\text{max}} = \text{max}(\xi^{+}, \xi^{-})$.
The diffusive term for the temperature equation  and mixture fraction equation is scaled by the Reynolds and Prandtl number as
\begin{equation}
\mathcal{S}^{D,E}\left(T,r,k/c_p\right) = \frac{1}{\Reynolds~\Prandtl} \mathcal{S}^D(T,r,k/c_p)
\end{equation}
Simmilarly the diffusive term for the mass fraction equations is
\begin{equation}
\mathcal{S}^{D,M}\left(\Yi,s_{\alpha h},\rho D_\alpha\right) = \frac{1}{\Reynolds~\Prandtl~\Lewis_\alpha} \mathcal{S}^D\left(\Yi,s_{\alpha h},\rho D_\alpha\right)
\end{equation}
The boundary condition term of the corresponding scalar equation is:
\begin{equation}
	\mathcal{B}^3(r) =  -
	\oint_{\GammaD\cup \GammaND}{ \left( (\rho_D\vec{u}_D X_D)\cdot \normalBoundary + \frac{\gamma_2}{2}X_D\right)r \dS}
	+\oint_{\GammaD\cup \GammaND} \xi_D X_D(\nabla_h r \cdot \normalBoundary - \eta r )\dS.
\end{equation}
Here, $X_D$ is the Dirichlet value of the scalar $X$ on boundaries and $\xi_D$ is the corresponding transport parameter calculated,which is calculated with \cref{eq:nondim_sutherland} using the Dirichlet values of the temperature at the boundary.
Finally, the volumetric source term of the energy and mass fraction equations are defined as follows:
\begin{gather}
	\mathcal{S}^S(r,\heatRelease, \rateReac,c_p) =  \text{H}~\Da ~ \int_{\Omega}{ \frac{\heatRelease \rateReac}{c_p} r} \dV, \\
	\mathcal{M}^S_\alpha(s_\alpha,\rateReac ) =  \Da \int_{\Omega}{  \stoicCoef_\alpha M_\alpha \rateReac s_\alpha} \dV.
\end{gather}
The heat release $\heatRelease$ is calculated with \cref{eq:heatReleaseOneStepNonDim}, the reaction rate $\rateReac$ is evaluated using \cref{eq:NonDimArr} and the mixture heat capacity with \cref{eq:nondim_cpmixture}.


\cleardoublepage
\chapter{Results}	\label{ch:results}
\glsresetall
The following sections present a comprehensive solver validation using various test cases. In \cref{sec:SingleCompIsotCase} the applicability of the solver is analyzed for isothermal single-component systems. Later in \cref{sec:SinCompNonIsothermCase} several single-component non-isothermal configurations are studied. Finally, in \cref{sec:MultCompNonIsothermCase} test cases for multicomponent non-isothermal systems are presented, with a particular emphasis on systems where combustion is present.
%Additionally the convergence properties of the DG-method will be analyzed for some of the systems.

All calculations shown here were performed on an AMD EPYC 7543 32-core Processor, DDR4 %TODO \todo[inline]{which specifications of the cluster should i include?}
Unless otherwise stated, all calculations use the termination criteria presented in \cref{ssec:TerminationCriterion}.
%%%%%%%%
%%%%%%%%
\section{Single-component isothermal cases}\label{sec:SingleCompIsotCase}
%%%%%%%%
%%%%%%%%
The solver presented in the previous chapter is initially validated for single-component isothermal cases. In these cases, only the continuity and momentum equations are used and solved. The energy equation and the species concentration equations are replaced by the conditions $T = 1.0$ and $Y_0 = 1.0$ in the whole domain. This means that the physical properties of the flow (density and viscosity) are constant and that the flow is fully incompressible because the density shows no thermodynamic or hydrodynamic dependence.
\subsection{Lid-driven cavity flow}

\begin{figure}[t]
	\begin{center}
		\def\svgwidth{0.3\textwidth}
		\import{./plots/}{LidDrivvenGeometry.pdf_tex}
		\caption{Schematic representation of the Lid-Driven cavity flow.}
		\label{fig:LidDrivenCavity}
	\end{center}
\end{figure}

The lid-driven cavity flow is a classic test problem used for the validation of Navier-Stokes solvers. The system configuration is shown in \cref{fig:LidDrivenCavity}. It consists simply of a two-dimensional square cavity enclosed by walls whose upper boundary moves at constant velocity, causing the fluid to move. Benchmark results can be found widely in the literature for different Reynolds numbers. In this section, the results obtained with the XNSEC solver are compared with those published by \textcite{botellaBenchmarkSpectralResults1998}. 

The problem is defined in the domain $\Omega = [0,1]\times [0,1]$. The system is solved for the velocity vector $\gls{velVec} = (u,v)$ and the pressure $p$. All boundary conditions are Dirichlet-type, particularly with $\gls{velVec} = (-1,0)$ for the boundary at $y = 1$ and $\gls{velVec} = (0,0)$ for all other sides. The gravity vector is set to $\gls{gravityVec} = (0,0)$.
A Cartesian mesh with extra refinement at both upper corners is used, and is shown in \cref{fig:LiddrivenMesh}. The refinement was done to better represent the complex effects that take place in the corners. The streamline plot presented in \cref{fig:LiddrivenMesh} shows the different vortex structures typical of this kind of system, where in addition to the main vortex of the cavity, smaller structures appear in the corners.

\begin{figure}[b]
	\centering
	\pgfplotsset{width=0.35 \textwidth, compat=1.3}
	\inputtikz{LiddrivenMesh1}
	\inputtikz{LiddrivenMesh2}
	\caption{Mesh and streamlines of the lid-driven cavity flow with $\gls{Reynolds} = 1000$} \label{fig:LiddrivenMesh}
\end{figure}

The lid-driven cavity was calculated for a Reynolds number $\gls{Reynolds} = 1000$. For the calculations presented here, the polynomial degree is set to four for both velocity components and three for the pressure. A regular Cartesian mesh with $16\times16$ elements is used with extra refinement in the corners. In \cref{fig:LidVelocities} a comparison of the calculated velocity with the DG-Solver and the velocities provided by the benchmark is shown. Clearly, very good agreement is obtained, even by using a relatively coarse mesh (the benchmark result uses a grid with $160\times160$ elements).

A more rigorous comparison of results is presented in \cref{tab:LidCavityExtrema}, where the extreme values of the velocity components calculated through the centerline of the cavity are compared with the results presented by \textcite{botellaBenchmarkSpectralResults1998}. Different mesh resolutions were used for this comparison, particularly meshes with $16\times16$, $32\times32$, $64\times64$, $128\times128$ and $256\times256$ elements, each with extra refinement at the corners. It can be clearly seen how for the finest mesh the results obtained with the DG-solver are extremely close to the reference. A difference is only appreciated at the fifth digit after the decimal point for the velocity components. In case of the position of the extremal values no difference is observed. It can also be seen that the results obtained with the coarser meshes are still very close to those of the reference %. It is worth mentioning that this comparison only considering number of elements could be considered unfair, since the reference uses another method for discretizing and solving the governing equations. One of the advantages of the DG method is that choosing higher-order polynomials allows more information to be packed into each cell. %A better comparison would be achieved by comparing results based on the number of degrees of freedom (DOF) used in the simulation, as will be done later in section %TODO XXX.

\begin{figure}[tb]
	\pgfplotsset{
		group/xticklabels at=edge bottom,
		legend style = {
				at ={ (0.09,0.3), anchor= north east}
			},
	}
	\inputtikz{LidVelocities1}
	\pgfplotsset{
		group/xticklabels at=edge bottom,
		legend style = {
				at ={ (0.59,0.3), anchor= north east}
			},
	}
	\inputtikz{LidVelocities2}
	\caption{Calculated velocities along the centerlines of the cavity and reference values. Left plot shows the x-velocity for $x = 0.5$. Right plot shows the y-velocity for $y = 0.5$  }
	\label{fig:LidVelocities}
\end{figure}

%	\begin{figure}[tb]
%		\pgfplotsset{
%			group/xticklabels at=edge bottom,
%			legend style = {
%				at ={ (0.09,0.2), anchor= north west}
%			},
%		}
%		\begin{tikzpicture}
%			\begin{axis}[
%				width= 0.4\textwidth ,
%				height= 0.3\textwidth ,
%				xlabel = $y$,
%				ylabel= $\omega$, 
%				]
%				\addplot+[color=black, only marks] table {data/LidDrivenCavity/omegaRe1000_x_ref.txt}; \addlegendentry{Reference}
%				\addplot[color=black, no marks] table {data/LidDrivenCavity/omegaRe1000_x.txt}; \addlegendentry{BoSSS}
%			\end{axis}
%		\end{tikzpicture}
%		\pgfplotsset{
%			group/xticklabels at=edge bottom,
%			legend style = {
%				at ={ (0.59,1.0), anchor= north east}
%			},
%		}
%		\begin{tikzpicture}
%			\begin{axis}[
%				width= 0.4\textwidth ,
%				height= 0.3\textwidth ,
%				xlabel = $x$,
%				ylabel= $\omega$, 
%				]
%				\addplot+[color=black, only marks] table {data/LidDrivenCavity/omegaRe1000_y_ref.txt}; \addlegendentry{Reference}
%				\addplot[color=black, no marks] table {data/LidDrivenCavity/omegaRe1000_y.txt}; \addlegendentry{BoSSS}
%			\end{axis}
%		\end{tikzpicture}
%		\caption{Calculated vorticity along the centerlines of the cavity and reference values. Left plot shows the vorticity for $x = 0.5$. Right plot shows the vorticity for $y = 0.5$  }
%		\label{fig:LidVorticities}
%	\end{figure}

\begin{table}[]
	\centering
	\begin{tabular}{lllllrr}
		\hline
		Mesh           & $u_{\text{max}}$ & $y_{\text{max}}$ & $v_{\text{max}}$ & $x_{\text{max}}$ & \multicolumn{1}{l}{$v_{\text{min}}$} & \multicolumn{1}{l}{$x_{\text{min}}$} \\ \hline
		$16\times16$   & 0.3852327        & 0.1820           & 0.3737295        & 0.8221           & -0.5056627                           & 0.0941                               \\
		$32\times32$   & 0.3872588        & 0.1821           & 0.3760675        & 0.8227           & -0.5080496                           & 0.0943                               \\
		$64\times64$   & 0.3897104        & 0.1748           & 0.3774796        & 0.8408           & -0.5248360                           & 0.0937                               \\
		$128\times128$ & 0.3886452        & 0.1720           & 0.3770127        & 0.8422           & -0.5271487                           & 0.0907                               \\
		$256\times256$ & 0.3885661        & 0.1717           & 0.3769403        & 0.8422           & -0.5270653                           & 0.0907                               \\\hline
		Reference      & 0.3885698        & 0.1717           & 0.3769447        & 0.8422           & \multicolumn{1}{l}{-0.5270771}       & \multicolumn{1}{l}{0.0908}           \\ \hline
	\end{tabular}
	\caption{Extrema of velocity components through the centerlines of the lid-driven cavity for $\gls{Reynolds} = 1000$. Reference values obtained from \textcite{botellaBenchmarkSpectralResults1998} }
	\label{tab:LidCavityExtrema}
\end{table}
\FloatBarrier
% \newpage

\subsection{Backward-facing step}\label{ssec:BackwardFacingStep}
The backward-facing step problem is another classical configuration widely used for validation of incompressible CFD codes. It has been widely studied theoretically, experimentally, and numerically by many authors in the last decades (see, for example, \cite{armalyExperimentalTheoreticalInvestigation1983,barkleyThreedimensionalInstabilityFlow2000,biswasBackwardFacingStepFlows2004} ).  In \cref{BFSsketch} a schematic representation of the problem is shown. It consists of a channel flow (usually considered fully developed) that is subjected to a sudden change in geometry that causes separation and reattachment phenomena. For these reasons, this case can be considered more challenging than the one presented in the previous section, since special care of the mesh used has to be taken in order to capture accurately all complex phenomena taking place.

Although the backward-facing step problem is known to be inherently three-dimensional, it has been shown that it can be studied as a two-dimensional configuration along the symmetry plane for moderate Reynolds numbers. For the range of Reynolds numbers used in the calculations presented here, the two-dimensional assumption is justified \citep{barkleyThreedimensionalInstabilityFlow2000, biswasBackwardFacingStepFlows2004}.  The origin of the coordinate system is set in the bottom part of the step. The step height \gls{StepHeigth} and channel height \gls{ChannelHeight} characterize the system. Results in the literature are often reported as a function of the expansion ratio, defined as $\gls{ExpansionRatio} = (\gls{ChannelHeight}+\gls{StepHeigth})/\gls{ChannelHeight}$.

A series of simulations were performed with the objective of reproducing the results reported by \cite{biswasBackwardFacingStepFlows2004}, where the backward-facing step was calculated for Reynolds numbers up to $400$ and for an expansion ratio of $1.9423$. In particular, the reported lengths of deattachment and reattachment are used as a means of comparison with the results from the XNSEC-solver.

The Reynolds number for the backward-facing step configuration is defined in the literature in many forms. Here, the definition based on the step height $\glsHat{StepHeigth}$ and the mean inlet velocity $\hat U_{\text{mean}}$ is adopted as the reference length and velocity, resulting in
\begin{equation}
	\gls{Reynolds}= \frac{\glsHat{StepHeigth}\hat U_{\text{mean}}}{\glsHat{kinVisc}}.
\end{equation}
The boundary at $x = - L_0$ is an inlet boundary condition, where a parabolic profile is defined with %, with a kinematic viscosity  $\nu = \SI{15.52e-6 }{\meter \squared \per \second}$
\begin{equation}
	u(y) = -6\frac{( y- S)( y-( h+ S))}{h^2} %= \frac{\hat u(y)}{\hat U_{\text{mean}}} 
\end{equation}
The system is isothermal, and the fluid is assumed to be air. The step length is set $S=1$ and $h = 1.061$. To minimize the effects of the outlet boundary condition on the part of interest in the system, the length $L$ of the domain is set to $L = 70 \gls{StepHeigth}$. All other boundaries are fixed walls. From prior calculations, the effect of the domain length before the step was found to have almost no impact on the results and is set to $L_0 = \gls{StepHeigth}$. Preliminary calculations showed that the calculated reattachment and detachment lengths are highly sensitive to the mesh resolution. For all calculations in this section, a structured grid with 88400 elements is useda. To better resolve the complex structures that occur in this configuration, smaller elements are used in the vicinity of the step, as seen in \cref{bfsmesh}.  A polynomial degree of three was chosen for both velocity components and two for pressure.


\begin{figure}[tb]
	\begin{center}
		\def\svgwidth{0.9\textwidth}
		\import{./plots/}{BFS_sketch.pdf_tex}
		\caption[Schematic representation of the backward-facing step.]{Schematic representation (not to scale) of the backward-facing step. Both primary and secondary vortices are shown.}
		\label{BFSsketch}
	\end{center}
\end{figure}

\begin{figure}[tb]
	\begin{center}
		\def\svgwidth{0.8\textwidth}
		\import{./plots/}{HBFS_MESH.pdf_tex}
		\caption{Mesh used for the backward-facing step configuration.}
		\label{bfsmesh}
	\end{center}
\end{figure}

\begin{figure}[bt]
	\centering
	\pgfplotsset{
		group/xticklabels at=edge bottom,
		%		legend style = {
		%			at ={ (1.0,1.0), anchor= north east}
		%		},
		unit code/.code={\si{#1}}
	}
	\inputtikz{uvelBFS}
	\caption[Distribution of x-component of velocity in the backward-facing step configuration for a Reynolds number of 400.]{Distribution of x-component of velocity in the backward-facing step configuration for a Reynolds number of 400. Solid lines correspond to results obtained with the XNSEC solver.}
	\label{fig:uvelBFS}
\end{figure}



\begin{figure}[tb]
	\pgfplotsset{
		group/xticklabels at=edge bottom,
		legend style = {
				at ={ (0.05,0.9), anchor= north west}
			},
		unit code/.code={\si{#1}}
	}
	\centering
	\inputtikz{Re_De_Attachmentlengths}
	\caption[Detachment and reattachment lengths of the primary and secondary recirculation zones after the backward-facing step compared to the reference solution]{ Detachment and reattachment lengths of the primary (left figure) and secondary (right figure) recirculation zones after the backward-facing step compared to the reference solution \citep{biswasBackwardFacingStepFlows2004}.}
	\label{fig:Re_De_Attachmentlengths}
\end{figure}
The backward-facing step configuration exhibits varying behavior as the number of Reynolds changes. For small Reynolds numbers, a single vortex, usually called the primary vortex, appears in the vicinity of the step. Furthermore, as the Reynolds number increases, a second vortex eventually appears on the top wall, as shown schematically in \cref{BFSsketch}.
The detachment and reattachment lengths of the vortices are values that are usually reported in the literature. It is possible to determine the detachment position by finding the point along the wall where the velocity gradient normal to the wall acquires a value equal to zero. 

\cref{fig:Re_De_Attachmentlengths} shows the detachment and reattachment lengths of the primary and secondary vortices obtained with the XNSEC solver for different Reynolds numbers, which are also compared with the results presented in the reference paper from \cite{biswasBackwardFacingStepFlows2004}. Cubic splines have been used to accurately locate this point. It can be seen that the results for the detachment lengths of the primary vortex $R_1$ are in very good agreement with those of the reference. In the case of the secondary vortex, it is possible to see a very minimal deviation for the lengths of the reattachment $R_3$, hinting at a possible spatial underresolution far away from the step. It is interesting to note that, despite the fact that the reference does not report the existence of a secondary vortex for $\gls{Reynolds} = 200$, it was possible to observe it with the XNSEC-solver. The results allow us to conclude that it is possible to study flows with complex behavior for low- to moderate Reynolds numbers, at least in the isothermal case. In the next section, a non-isothermal case of this configuration will be studied.

It is worth mentioning that the evaluation of the global order of accuracy of the solver using the two incompressible test cases presented in this section is problematic due to the presence of singularities, specifically at the corners at the coordinates $ \vec{x} = (0,1)$ and $\vec{x} =(1,1)$ of the Lid-driven cavity (where the pressure is not finite according to \cite{botellaBenchmarkSpectralResults1998}), and at the corner of the step $\vec{x} = (0,S)$ of the backward-facing step. The accuracy of the solver will be assessed later in  \cref{ssec:CouetteFlowTempDiff} making use of a analytical solution and \cref{ssec:ConvStudyHeatedCavity} using a solution obtained with a high spatial resolution.

%%%%%%%%
%%%%%%%%
\section{Single-component non-isothermal cases} \label{sec:SinCompNonIsothermCase}
%%%%%%%%
%%%%%%%%
For the test cases presented in this section, the equations for continuity, momentum and energy are solved. All systems are assumed to be single-component, thus $N = 1$ and $Y_0 = 1.0$. We start by showing in \cref{ssec:HeatedBackwardFacingStep} an extension of the backward-facing step configuration presented in the last section, considering now a non-isothermal system. Later in \cref{ssec:CouetteFlowTempDiff} a Couette flow configuration presenting a temperature gradient in the vertical direction is studied. Finally, in \cref{ss:DHC} a heated square cavity configuration is studied in order to assess the capability of the solver for variable density flows in closed systems.

\subsection{Heated backward-facing step}\label{ssec:HeatedBackwardFacingStep}
\begin{figure}[tb]
	\begin{center}
		\includegraphics[width=\linewidth]{../plots/HBFS_TemperatureRe700_2.pdf}
		\caption{Temperature profile (top) and streamlines (bottom) corresponding to the backward-facing Step configuration for $\gls{Reynolds} = 400$ and an expansion ratio of two.}
		\label{BFS_Streamlines}
	\end{center}
\end{figure}

As an extension to the previous case, we seek to reproduce the results presented by \cite{xieFluidFlowHeat2016}, where the same backward-facing step configuration as presented in \cref{ssec:BackwardFacingStep} is studied, but with the particularity that in this case the bottom wall is heated to a constant temperature higher than the inlet temperature, thus featuring a non-isothermal system. The fluid entering the system has a temperature equal to $\hat T_0 = \SI{283}{\kelvin}$ and the bottom wall is set to a constant temperature of $\hat T_1 =\SI{313}{\kelvin}$
In the work of \cite{xieFluidFlowHeat2016} results are reported for the local Nusselt numbers and local friction coefficients $f_d$  along the bottom wall ($y = 0$) for different expansion ratios and Reynolds numbers. By putting together the definition of the Nusselt number ( $\gls{Nusselt} = \frac{\gls{HeatTransCoef}\hat{L}}{\gls{HeatConductivity}}$), Newton's law of cooling ($\hat{\vec{q}} = \hat{h} (\hat{T}_0 - \hat{T}_W )$), and Fourier's law of heat conduction ($\hat{\vec{q}} = \hat \lambda \hat{\nabla} \glsHat{temp}$) a expression for the local Nusselt number is obtained.
\begin{equation}
	\gls{NusseltLoc} = \frac{\hat L}{\hat T_0-\hat T_W}\hat \nabla \hat T \cdot \hat {\vec{n}}
\end{equation}
where $\hat L$ is a reference length. We choose $ \hat L = \hat S$ to be consistent with the definition of the Reynolds number of the reference. Furthermore, recognizing that the wall shear stress along the bottom wall $\tau_{\text{w}} = -\mu \nabla u \cdot \vec{n}$, the local friction factor can be written as
\begin{equation}
	f_d = \frac{8\nu} { (U_{\text{mean}})^2}  \nabla u \cdot \vec{n}
\end{equation}
It should be noted that for the range of temperature differences involved in this case, the variation of physical parameters such as density, viscosity, and thermal conductivity with respect to temperature has no appreciable influence on the calculated flow fields.

Simulations were conducted for different Reynolds numbers and expansion ratios. In \cref{BFS_Streamlines} the temperature field and streamlines corresponding to a calculation with $\gls{Reynolds} = 700$ are shown. Here, the apparition of the secondary vortex is seen in the top wall.

It must be noted here that the results obtained by us are substantially different from those reported by \cite{xieFluidFlowHeat2016}, and will not be shown here. However, in the work of \cite{henninkLowMachNumberFlow2022} the same is also noted, saying that with his method it was not possible to reproduce the results presented by \cite{xieFluidFlowHeat2016}. Comparing our results with those of Hennink we can observe that the same results are obtained. as demonstrated in \cref{fig:fd_Nu_plot}.% We can conclude from these results that the solver is able to deal with non-isothermal flows adequately.

\begin{figure}[tb]
	\pgfplotsset{
		group/xticklabels at=edge bottom,
		legend style = {
				at ={ (0.59,1.0), anchor= north east}
			},
		unit code/.code={\si{#1}}
	}
	\inputtikz{fd_Nu_plot1}
	\inputtikz{fd_Nu_plot2}
	\caption{Local friction factor and local Nusselt number along the bottom wall for $\gls{Reynolds} = 700$ and an expansion ratio of two. The solid lines corresponds to our solution and the marks to the reference \citep{henninkLowMachNumberFlow2022}}
	\label{fig:fd_Nu_plot}
\end{figure}



\FloatBarrier


\subsection{Couette flow with vertical temperature gradient} \label{ssec:CouetteFlowTempDiff}
\begin{figure}[tb]
	\begin{center}
		\def\svgwidth{0.5\textwidth}
		\import{./plots}{HeatedCouetteSketch.pdf_tex}
		\caption{Schematic representation of the Couette flow with temperature difference test case.}
		\label{fig:CouetteTempDiff_scheme}
	\end{center}
\end{figure}
As a next test case for the low-Mach solver, we study a Couette flow with a vertical temperature gradient. This configuration was already studied in \citep{kleinHighorderDiscontinuousGalerkin2016}, where the SIMPLE algorithm in a DG framework was used. 
In this section, we intend to reproduce the results from \citep{kleinHighorderDiscontinuousGalerkin2016} by using the fully coupled solver presented in \cref{sec:discretDGmethod}. Additionally, it will be shown how the implemented solver performs in relation to the SIMPLE based solver in terms of runtime.

In \cref{fig:CouetteTempDiff_scheme} a schematic representation of the test case is shown. The top wall corresponds to a moving wall ($u = 1$) with a fixed temperature $T=T_h$. The bottom wall is static ($u = 0$), and has a constant temperature $T = T_c$.
The domain is chosen as $\Omega = [0,1]\times[0,1]$, and Dirichlet boundary conditions are used for all boundaries. Additionally, the system is subjected to a gravitational field, where the gravity vector only has a component in the $y$ direction. Under these conditions, the x-component of velocity, pressure and temperature are only dependent on the $y$ coordinate, i.e. $u = u(y)$, $T = T(y)$ and $p = p(y)$. The governing equations (\cref{eq:NS-eq}) reduce to 
\begin{align}
	 & \frac{1}{\gls{Reynolds}} \pfrac{ }{y}\left(\mu\pfrac{u}{y}\right) = 0,                                  \\
	 & \pfrac{p}{y} = -\frac{\gls{dens}}{\gls{Froude}^2},                                                      \\
	 & \frac{1}{\gls{Reynolds}~\gls{Prandtl}} \pfrac{ }{y}\left(\gls{HeatConductivity}\pfrac{T}{y}\right) = 0.
\end{align}

By assuming a temperature dependence of the transport properties according to a Power Law ($\mu = \lambda = T^{2/3}$) it is possible to find an analytical solution for this problem.
\begin{subequations}
	\begin{align}
		u(y) & = C_1 + C_2\left(y + \frac{T_c^{5/3}}{T_h^{5/3}-T_c^{5/3}} \right)^{3/5},\label{eq:CouetteU}                                                                \\
		p(y) & = -\frac{5p_0}{2\gls{Froude}^2}\frac{\left(y\left(T_h^{5/3}-T_c^{5/3}\right)+T_c^{5/3}\right)^{2/5}}{\left(T_h^{5/3}-T_c^{5/3}\right)}+C,\label{eq:Couettep} \\
		T(y) & = \left(C_3 - \frac{5}{3}C_4 y\right)^{3/5}\label{eq:CouetteT}.
	\end{align}
\end{subequations}
Where the constants $C_1$, $C_2$, $C_3$ and $C_4$ are determined using the boundary conditions on the top and bottom walls, and are given by
\begin{align}
	C_1 & = \frac{\left(\frac{T_c^{5/3}}{T_h^{5/3}-T_c^{5/3}}\right)^{3/5}}{\left(\frac{T_c^{5/3}}{T_h^{5/3}-T_c^{5/3}}\right)^{3/5}-\left(\frac{T_h^{5/3}}{T_h^{5/3}-T_c^{5/3}}\right)^{3/5}} \\
	C_2 & = \frac{1}{\left(\frac{T_h^{5/3}}{T_h^{5/3}-T_c^{5/3}}\right)^{3/5}-\left(\frac{T_c^{5/3}}{T_h^{5/3}-T_c^{5/3}}\right)^{3/5}}                                                        \\
	C_3 & = T_c^{5/3},                                                                                                                                                                         \\
	C_4 & = \frac{3}{5}\left(T_c^{5/3}-T_h^{5/3}\right)
\end{align}
and $C$ is a real-valued arbitrary constant for the pressure $p$.%
\begin{center}
	\begin{figure}[tb]
		\pgfplotsset{
			group/xticklabels at=edge bottom,
		}
		\inputtikz{CouetteSolution1}
		\inputtikz{CouetteSolution2}
		\inputtikz{CouetteSolution3}
		\caption{Solution of the Couette flow with vertical temperature gradient. Viscosity and heat conductivity are calculated with a Power-Law.}
		\label{fig:CouetteSolution}
	\end{figure}
\end{center}
\FloatBarrier
For all calculations of this configuration shown, the dimensionless parameters are set as $\gls{Reynolds} = 10$ and $\gls{Prandtl} =0.71$, $T_h = 1.6$ and $T_c = 0.4$. Since we are dealing with an open system, we set $p_0 =1.0$. The Froude number is calculated as
\begin{equation}
	\text{Fr} = \left( \frac{2\text{Pr}(T_h-T_c)}{(T_h+T_c)}\right)^{1/2}
\end{equation}
In \cref{fig:CouetteSolution} the solution for the velocity, pressure and temperature are shown. The results are for a mesh with $26\times26$ elements and a polynomial degree of three for $u$ and $T$, and a polynomial degree of two for $p$.
\subsubsection{h-convergence study}
The convergence properties of the DG method for this non-isothermal system was studied using the analytical solution described before. The domain is discretized and solved in uniform Cartesian meshes with $16\times16$, $32\times32$, $64\times64$ and $128\times128$ elements. The polynomial degrees for the velocity and temperature are changed from 1 to 4 and for the pressure from 0 to 3. The convergence criteria described in \cref{ssec:TerminationCriterion} was used for all calculations. The  analytical solution given by \cref{eq:CouetteU,eq:Couettep,eq:CouetteT} are used as Dirichlet boundary conditions on all boundaries of the domain. The error is calculated against the analytical solution using the $L^2$ norm. %TODO \todo[inline]{Comment more on the calculation of the l2norm}.
In \cref{fig:ConvergenceDHC} the results of the h-convergence study are shown. We observe how the expected convergence rates are reached for all variables, namely a slope of the order $k+1$ for both velocity components and the temperature, and a slope of $k'+1$ for the pressure.
\begin{figure}[t!]
	\centering
	\pgfplotsset{width=0.34\textwidth, compat=1.3}
	\inputtikz{ConvergenceCFTD}
	\caption{Convergence study of the Couette-flow with temperature difference. A power-law is used for the transport parameters.}\label{fig:ConvergenceCFTD}
\end{figure}

\subsubsection{Comparison with SIMPLE}
As mentioned before, a solver for solving low-Mach flows based on the SIMPLE algorithm (presented in \cite{kleinHighorderDiscontinuousGalerkin2016}) has been already developed and implemented within the BoSSS framework.
Though the solver was validated and shown to be useful for a wide variety of test cases, there were also disadvantages inherently associated with the SIMPLE algorithm. Within the solution algorithm, Picard-type iterations are used to search for a solution. This requires some prior knowledge from the user in order to select suitable relaxation factor values that provide stability to the algorithm, but at the same time do not slow down the computation substantially.
It was also observed that the calculation times were prohibitively high for some test cases. This point motivated the development of the solver presented in the present work, where the system is solved in a monolytic way and  and the Newton method globalized with a Dogleg-type method is used to solve the system.
%TODO \todo[inline]{what are exactly the advantages of the present solver? multigrid? Ortogonalization?} 
We intend to show in this section a comparison of runtimes of the calculation of the Couette flow with vertical temperature gradient between the DG-SIMPLE algorithm \citep{kleinHighorderDiscontinuousGalerkin2016} and the present solver (denoted here as XNSEC). Calculations were performed on uniform Cartesian meshes with $16\times16$, $32\times32$, $64\times64$ and $128\times128$ elements, and with varying polynomial degrees between 1 and 3 for the velocity and temperature, and between 0 y 2 for the pressure. All calculations where initialized with a zero velocity and pressure field, and with a temperature equal to one in the whole domain. Are calculations were performed single-core and the convergence criteria is set to $10^{-8}$ for both solvers. The under-relaxation factors for the SIMPLE algorithm were set for all calculations to 0.8, 0.5 and 1.0 for the velocity, pressure and temperature, respectively.

In figure \cref{fig:RuntimeComparison} a comparison of the runtimes from both solvers is shown. It is clearly  appreciated how the runtimes of the SIMPLE algorithm are higher for almost all of the cases studied. Obviously the under-relaxation parameters of the SIMPLE algorithm have an influence on the calculation times and an appropiate selection of them could decrease the runtimes. This is a clear disadvantage because the selection of adequate factors is highly problem dependent and requires some previous expertise from the user. On the other hand, the globalized Newton method used by the XNSEC avoid this problem by using a more sophisticated method and heuristics in order to find a better path to the solution.

%TODO \todo[inline]{I have to somehow highlight even more the positive points of using this solver} 

\begin{center}
	\begin{table}[tb!]
		\begin{tabular}{ccc}
			\inputtikz{RuntimeComparison1}
			 &
			\inputtikz{RuntimeComparison2}
			 &
			\inputtikz{RuntimeComparison3}
		\end{tabular}%
		\caption{Runtime comparison of the DG-SIMPLE algorithm \citep{kleinHighorderDiscontinuousGalerkin2016} and the present solver (XNSEC) for the Couette flow with vertical temperature gradient configuration}
		\label{fig:RuntimeComparison}
	\end{table}
\end{center}


\FloatBarrier

\subsection{Differentially heated cavity problem}\label{ss:DHC}
\begin{figure}[bt]
	\begin{center}
		\def\svgwidth{0.53\textwidth}
		\import{./plots/}{diffheatedCavityGeometry.pdf_tex}
		\caption{Schematic representation of the differentially heated cavity problem.}
		\label{DHCGeom}
	\end{center}
\end{figure}
The differentially heated cavity problem is a classical benchmark case that is often used to assess the ability of numerical codes to simulate variable density flows \citep{paillereComparisonLowMach2000,vierendeelsBenchmarkSolutionsNatural2003,tyliszczakProjectionMethodHighorder2014}.
The test case has the particularity that deals with a closed system, where the thermodynamic pressure $p_0$ is a parameter that must be adjusted so that the mass is conserved.The thermodynamic pressure $p_0$ determines the density field, which in turn appears in the momentum equation and the energy equation, making it necessary to use an adequate algorithm to solve the system. This point presents a special difficulty for the solution, since the calculation of $p_0$ requires knowledge of the temperature field on the whole computational domain, inducing a global coupling of the variables. 

The system is a fully enclosed two-dimensional square cavity filled with fluid.  A sketch of the problem is shown in \cref{DHCGeom}. The left and right walls of the cavity have constant temperatures $\hat{T}_h$ and $\hat{T}_c$, respectively, with $\hat{T}_h >\hat{T}_c$, and the top and bottom walls are adiabatic. A gravity field induces fluid movement because of density differences caused by the difference in temperature between the hot and cold walls.
The natural convection phenomenon is characterized by the Rayleigh number, defined as
\begin{equation}\label{eq:Rayleigh}
	\text{Ra} = \Prandtl \frac{\hat g \RefVal{\rho}^2(\hat T_h-\hat T_c) \RefVal{L}^3}{\RefVal{T}\RefVal{\mu}^2},
\end{equation}
For small values of $\text{Ra}$, conduction dominates the heat transfer process, and a boundary layer covers the entire domain. On the other hand, large values of $\text{Ra}$ represent a flow dominated by convection. When the number $\text{Ra}$ increases, a thinner boundary layer is formed.
Following \cite{vierendeelsBenchmarkSolutionsNatural2003}, a reference velocity for buoyancy-driven flows can be defined as
\begin{equation}
	\RefVal{u} = \frac{\sqrt{\text{Ra}} \RefVal{\mu}}{\RefVal{\rho}\RefVal{L}}.
\end{equation}
The Rayleigh number is then related to the Reynolds number according to
\begin{equation}
	\text{Re} = \sqrt{\text{Ra}}.
\end{equation}
Thus, it is sufficient to select a $\Reynolds$ number in the simulation, fixing the value of the $\text{Ra}$ number. The driving temperature difference $(\hat T_h - \hat T_c)$ appearing in \cref{eq:Rayleigh} can be represented as a non-dimensional parameter:
\begin{equation}\label{eq:nondimensionalTemperature}
	\varepsilon = \frac{\hat T_h - \hat T_c}{2\RefVal{T}}.
\end{equation}
Using these definitions, the Froude number can be calculated as
\begin{equation}
	\Froude = \sqrt{\Prandtl 2 \varepsilon}.
\end{equation}
The results of the XNSEC solver are compared with those of the reference solution for $\RefVal{T} = 600\si{K}$  and $\varepsilon = 0.6$. All calculations assume a constant Prandtl number equal to 0.71. The dependence of viscosity and heat conductivity on temperature is calculated using Sutherland's law (\cref{eq:nondim_sutherland}). The non-dimensional length of the cavity is $L=1$. The non-dimensional temperatures $T_h$ and $T_c$ are set to 1.6 and 0.4, respectively. The non-dimensional equation of state (\cref{eq:ideal_gas}) depends only on the temperature and reduces to
\begin{equation}
	\rho = \frac{p_0}{T}.
\end{equation}
The thermodynamic pressure $p_0$ in a closed system must be adjusted to ensure mass conservation. For a closed system is given by
\begin{equation}
	p_0 =\frac{\int_\Omega \rho_0\text{d}V}{\int_\Omega \frac{1}{T}\text{d}V}= \frac{m_0}{\int_\Omega \frac{1}{T}\text{d}V}, \label{eq:p0Condition}
\end{equation}
where $\Omega$ represents the complete closed domain. The initial mass of the system $m_0$ is constant and is set $m_0 = 1.0$. Note that the thermodynamic pressure is a parameter with a dependence on the temperature of the entire domain. This makes necessary the use of an iterative solution algorithm, so that the solution obtained respects the conservation of mass. Within the solution algorithm of the XNSEC-solver, \cref{eq:p0Condition} is used to update the value of the thermodynamic pressure after each Newton iteration.
The average Nusselt number is defined for a given wall $\Gamma$  as
\begin{equation}\label{eq:Nusselt}
	\text{Nu}_\Gamma = \frac{1}{T_h - T_c}\int_{\Gamma} k \pfrac{T}{x}\text{d}y.
\end{equation}
%%%%%%%%%%%%%%%
%%% Result comparison
%%%%%%%%%%%%%%%
\subsubsection{Comparison of results with the benchmark solution}

Here a comparison of the results obtained with the XNSEC solver and the results presented in the work of \textcite{vierendeelsBenchmarkSolutionsNatural2003} is made. They solved the fully compressible Navier-Stokes equations on a stretched grid with $1024\times1024$ using a finite-volume method with quadratic convergence, providing very accurate results that can be used as reference.
The benchmark results are presented for $\text{Ra} = \{10^2,10^3,10^4,10^5,10^6,10^7\}$. In this range of Rayleigh numbers, the problem has a steady-state solution. 
The cavity is represented by the domain $[0,1]\times[0,1]$. For all calculations in this subsection, the simulations are done with a polynomial degree of four for both velocitiy components and temperature and three for the pressure. The mesh is in an equidistant $128\times128$ mesh.

Preliminar calculations showed that for cases up to $\text{Ra} = 10^5$ the solution of the system using Newton's method presented in \cref{sec:newton} is possible without further modifications, while for higher values the algorithm couldnt find a solution and stagnates after certain number of iterations. The homotopy strategy mentioned in \cref{sec:CompMethodology} is used to overcome this problem and obtain solutions for higher Rayleigh numbers. Here, the Reynolds number is selected as the homotopy parameter and continuously increased until the desired value is reached.

In \cref{fig:TempProfile,fig:VelocityXProfile,fig:VelocityYProfile} the temperature and velocity profiles across the cavity for different Rayleigh numbers are shown. The profiles calculated with the XNSEC solver agree closely to the benchmark solution. As expected, an increase of the acceleration of the fluid in the vicinity of the walls for increasing Rayleigh numbers is observed.
\begin{figure}[h]
	\centering
	\pgfplotsset{width=0.3 \textwidth, compat=1.3}
	\inputtikz{HSCStreamlines}
	\caption{Streamlines of the heated cavity configuration with $\epsilon = 0.6$ for different Reynold numbers.}\label{fig:HSCStreamlines}
\end{figure}


\begin{figure}[h]
	\centering
	\pgfplotsset{width=0.22\textwidth, compat=1.3} 
	\inputtikz{TempProfile}
	\caption[Temperature profiles for the differentially heated square cavity along different vertical levels.]{Temperature profiles for the differentially heated square cavity along different vertical levels. Solid lines represent the XNSEC solver solution and marks the benchmark solution.}
	\label{fig:TempProfile}
\end{figure}
%
\begin{figure}[h]
	\centering
	\pgfplotsset{width=0.22\textwidth, compat=1.3}
	\inputtikz{VelocityXProfile}
	\caption[Profiles of the x-velocity component for the differentially heated square cavity along the vertical line $x=0.5$.]{Profiles of the x-velocity component for the differentially heated square cavity along the vertical line $x=0.5$. Solid lines represent the XNSEC solver solution and marks the benchmark solution.}
	\label{fig:VelocityXProfile}
\end{figure}
%
\begin{figure}[h]
	\centering
	\pgfplotsset{width=0.22\textwidth, compat=1.3}
	\inputtikz{VelocityYProfile}
	\caption[Profiles of the y-velocity component for the differentially heated square cavity along the horizontal line $y=0.5$.]{Profiles of the y-velocity component for the differentially heated square cavity along the horizontal line $y=0.5$. Solid lines represents the XNSEC solver solution and marks the benchmark solution.}
	\label{fig:VelocityYProfile}
\end{figure}
\FloatBarrier
A comparison of the thermodynamic pressure and the Nusselt numbers to the benchmark solution was also made. The results are shown in \cref{tab:p0_Nu_Results}.  The thermodynamic pressure is obtained from \cref{eq:p0Condition}, and the average Nusselt number is calculated with \cref{eq:Nusselt}. The results obtained with the XNSEC solver agree very well with the reference results, as can be seen for the thermodynamic pressure, which differs at most in the fourth decimal place. Note that the average Nusselt number of the heated wall $(\text{Nu}_\text{h})$ and the Nusselt number of the cold wall $(\text{Nu}_\text{c})$ are different. As the Rayleigh number grows, this discrepancy becomes larger, hinting that, at such Rayleigh numbers, the mesh used is not refined enough to adequately represent the thin boundary layer and more complex flow structures appearing in high-Rayleigh number cases. While for an energy conservative system $\text{Nu}_\text{h}$ and $\text{Nu}_c$ should be equal, for the DG-formulation this is not the case, since conservation is only ensured locally and the global values can differ. This discrepancy can be seen as a measure of the discretization error of the DG formulation and should decrease as the mesh resolution increases. This point will be discussed in the next section.
\begin{table}[t!]
	\begin{center}
		\begin{tabular}{cccccc}
			\hline
			Rayleigh                           & $p_0$  & $p_{0,\text{ref}}$ & $\text{Nu}_{h}$ & $\text{Nu}_{c}$ & $\text{Nu}_{\text{ref}}$ \\ \hline
			\parbox[0pt][13pt][c]{0pt}{}$10^2$ & 0.9574 & 0.9573             & 0.9787          & 0.9787          & 0.9787                   \\
			$10^3$                             & 0.9381 & 0.9381             & 1.1077          & 1.1077          & 1.1077                   \\
			$10^4$                             & 0.9146 & 0.9146             & 2.2180          & 2.2174          & 2.2180                   \\
			$10^5$                             & 0.9220 & 0.9220             & 4.4801          & 4.4796          & 4.4800                   \\
			$10^6$                             & 0.9245 & 0.9245             & 8.6866          & 8.6791          & 8.6870                   \\
			$10^7$                             & 0.9225 & 0.9226             & 16.2411         & 16.1700         & 16.2400                  \\ \hline
		\end{tabular}
	\end{center}
	\caption[Differentially heated cavity: Results of Nusselt number and Thermodynamic pressure]{Comparison of calculated Nusselt numbers of the hot and cold wall and Thermodynamic pressure $p_0$ reported values by \textcite{vierendeelsBenchmarkSolutionsNatural2003} for the differentially heated cavity.}
	\label{tab:p0_Nu_Results}
\end{table}
%%%%%%%%%%%%%%%
%%% Convergence study
%%%%%%%%%%%%%%%


\subsubsection{Convergence study}\label{ssec:ConvStudyHeatedCavity}
An $h-$convergence study of the XNSEC solver was conducted using the heated cavity configuration. Calculations were performed for polynomial degrees $k = {1,2,3,4}$ and equidistant regular meshes with, respectively, $8\times8$, $16\times16$, $32\times32$, $64\times64$, $128\times128$ and $256\times256$ elements.  The $L^2$ -Norm was used to calculate errors against the solution in the finest mesh. The results of the $h$-convergence study for varying polynomial orders $k$ are shown in \cref{fig:ConvergenceDHC}. It is observed how the convergence rates scale approximately as $k+1$. Interestingly, for $k=2$ the rates are higher than expected. On the other hand, some degeneration is observed in convergence rates for $k = 4$. This strange behavior can be explained if one considers that the heated cavity presents a singular behavior at the corners %TODO. am i sure of this?
-similar to the problem previously exposed for the lid-driven cavity -, which causes global pollution in the convergence behavior of the algorithm. %Figure XXX shows in an elevated plane the difference obtained by subtracting the pressure field for a simulation with $k=3$ and the meshes of $64times64$ and $128times128$ elements,  which can be understood as a measure of the error in the simulation. It is clearly seen that the corners of the system present a very high error, which could be explained by the inconsistency between boundary conditions in that area. 
 
As discussed in the previous section, the difference in the average values of the Nusselt number on the hot wall $\text{Nu}_\text{h}$  and the cold wall $\text{Nu}_\text{c}$ is a direct consequence of the spatial discretization error and should decrease for finer meshes. In \cref{fig:NusseltStudy} the convergence behavior of the Nusselt number is presented for different polynomial degrees $k$, different number of elements and for two different Ra numbers. As expected, it can be observed that this discrepancy is smaller when a larger number of elements is used. It can also be seen that  $\text{Nu}_\text{h}$ reaches the expected solution of cells for a much smaller number of elements. This can be explained if one thinks that more complex phenomena take place near the cold wall (see \cref{fig:HSCStreamlines}), which makes necessary a finer mesh resolution in that area.

The results presented in this section allow us to conclude that the implemented solver is capable of dealing with flows with variable densities, and in particular in closed spaces. Additionally, it was observed that even for this complex test, convergence properties close to those expected from the DG-method are obtained. In this section only systems with a steady state solution were treated. In the next section the ability of the solver to compute flows with varying densities in non-steady state will be shown.

\begin{figure}[tb]
	\centering
	\pgfplotsset{width=0.34\textwidth, compat=1.3}
	\inputtikz{ConvergenceDHC}
	\caption{Convergence study of the differentially heated cavity problem for $\text{Ra} = 10^3$.}\label{fig:ConvergenceDHC}
\end{figure}
\begin{figure}[tb]
	\centering
	\inputtikz{NusseltStudy}
	\caption{Nusselt numbers calculated with \cref{eq:Nusselt} at the hot wall ($\text{Nu}_h$) and the cold wall ($\text{Nu}_c$) for different number of cells and polynomial order $k$. The reference values from \cite{vierendeelsBenchmarkSolutionsNatural2003} are shown with dashed lines.}\label{fig:NusseltStudy}
\end{figure}

\FloatBarrier
\subsection{Poiseuille–Rayleigh–Bénard instability in a channel}
\blindtext[5]
\subsection{Flow around a heated cylinder}
\subsubsection{Square cylinder}
\blindtext[5]
\subsubsection{Circle? cylinder}
\blindtext[5]
%%%%%%%%
%%%%%%%%
\section{Multi-component non-isothermal cases}\label{sec:MultCompNonIsothermCase}
%%%%%%%%
%%%%%%%%
Later in in \cref{ss:CDF} and \cref{ss:CoFlowFlame} two different configurations for reactive flows are presented.
In the following, we show results of the simulations of two test cases with combustion, namely the counterflow diffusion flame and the chambered diffusion flame. For both cases, the solution of the flame sheet problem described in \cref{ssec:FlameSheet} is calculated first. This solution is used subsequently as initial estimates for the solution of the finite chemistry rate problem (c.f. \cref{ssec:NonDimLowMachEquations}). In all test cases presented in this section, a smoothing parameter $\sigma = 50$ was used (c.f. \cref{ssec:FlameSheet}). For all test cases methane combustion according to the one-step model shown in \cref{sec:ChemModel} is considered. The mass fraction transport \cref{eq:LowMachMassBalance} is solved for the species \ch{CH4}, \ch{O2}, \ch{CO2} and \ch{H2O}, thus $\vec{Y}' = \left(Y_{\ch{CH4}},Y_{\ch{O2}},Y_{\ch{CO2}},Y_{\ch{H2O}}\right)$. The nitrogen mass fraction $Y_{\ch{N2}}$ is calculated according to \cref{eq:MassFractionConstraint}.



\subsection{Co-flow laminar diffusion flame}\label{ssec:coflowFlame}
%C:\Users\jfgj8\AppData\Local\BoSSS\plots\sessions\CoFlowFinal_Mult3__Full_CoFlowFlamerP4K12smoothfactor0velMult3LF1__5b10173f-65b1-484a-bfd9-1dc089c3571e
The co-flowing flame configuration is used as a first test to assess the behavior of the solver for reactive flows applications. It basically consists of two concentric ducts that emit fuel and oxidant into the system, creating a flame. This configuration has been widely studied. In the seminal work of \cite{burkeDiffusionFlames1928} analytical expressions for the flame height and flame shape are obtained by studying a very simplified problem (constant density and velocity field, infinitely fast chemistry, among others). Later, \cite{smookeNumericalModelingAxisymmetric1992} and later works solved this configuration using a 2D-axisymmetric system and also used the flame sheet estimates to find adequate initial conditions for their Newton algorithm. 
\begin{figure}[t]
	\centering
	\def\svgwidth{0.38\textwidth}
	\subcaptionbox{Sketch\label{fig:CoFlowSketch}}{
		\import{./plots/}{CoFlowSketch_withBC.pdf_tex}\vspace{0.5cm}
	}
	\qquad\quad
	\def\svgwidth{0.35\textwidth}
	\subcaptionbox{Refined mesh \label{fig:CoFlowMesh}}{
		\vspace{1.2cm}
		\import{./plots/}{CoFlowMesh.pdf_tex}
	}
	\caption{Geometry of a coflowing flame configuration (not to scale).} \label{fig:CoFlowGeometry}
\end{figure}
It should be noted that the solution of the axisymmetric system of equations presents numerical difficulties that are not the main concern of the present work. For this reason, it was decided to solve a system with similar characteristics but which is possible to represent in Cartesian coordinates. This is possible if an infinitely long slot burner is considered.
\subsubsection{Set-up of the slot coflowing flame configuration}
A schematic diagram of the configuration can be seen in \cref{fig:CoFlowSketch}, and consists of a fuel inlet with two oxygen inlets on its sides. Note that the system also includes the tips that separate both oxidizer and fuel inlets. The inclusion of this separation was seen to be necesary to be able to obtain a converged solution. Altought the system is clearly symmetric around the axis $x = 0$, no symmetry assumption is made and the whole domain is considered for the simulation. The lengths  depicted on \cref{fig:CoFlowSketch} are set as $r_1 = \SI{0.635}{\centi \meter}$, $r_2 = \SI{0.762}{\centi \meter}$, and $r_3 = \SI{7.747}{\centi \meter}$. Aditionally $h = \SI{2.54}{\centi \meter}$ and $H = \SI{40}{\centi \meter}$.  The lengths $r_3$ and $L$ are set with arbitrarily large values in order to avoid influence on the outer boundary conditions on the solution of the flame zone. Setting a higher value of $r_3$ or $L$ did not significantly affect the results.  %The thickness of the tips didnt seem to have an impact on the convergence behaviour of the algorithm.   sure?
The inlet boundary conditions are set as: 
\begin{itemize}
	\item Oxidizer inlet: $\{\forall (x,y): y = -R \land x \in [-X_3,-X_2]\cup[X_2,X_3]\}$\\
	\begin{equation*}
		u = 0,\qquad v= v_O, \qquad T = T^O, \qquad \vec{Y}' = (0,Y^O_{\ch{O2}},0,0)
	\end{equation*}
\item Fuel Inlet: $\{\forall (x,y): y = -R \land x \in [-X_1,X_1]\} $ \\
\begin{equation*}
	u = 0,\qquad v= v_F(x), \qquad T = T^F, \qquad \vec{Y}' = (Y^F_{\ch{CH4}},0,0,0)
\end{equation*}
%\item Pressure outlet  $\forall (x,y): y = -R \land x \in [-X_3,-X_2]\cup[X_2,X_3]$\\
%\item Adiabatic wall:
\end{itemize}
The oxidizer enters the system as a plug flow with a constant velocity of $v_O = \SI{4.1}{\centi \meter \per \second}$. The inlet velocity of the fluel stream $v_F$ is a a parabolic profile given by
\begin{equation}
	v_F(x) = \left[1-\left(\frac{x}{X_1}\right)^2\right]v_m^F
\end{equation}
with $v^F_m = \SI{2.427}{\centi \meter \per \second}$.  The inlet temperatures of both streams is  $T^F = T^O = \SI{300}{K}$. Combustion of diluted methane on air is considered, with $Y^F_{\ch{CH4}} = 0.2$, $Y^F_{\ch{N2}} = 0.8$, $Y^O_{\ch{O2}} = 0.23$ and $Y^O_{\ch{N2}} = 0.77$. The superindexes $F$ and $O$ represent the fuel and oxidizer inlet respectively. The pressure outlet boundary condition is the same as \cref{eq:bc_O}. Finally, the boundary conditions at the tips correspond to adiabatic walls, which are defined as in \cref{eq:bc_dn}, with $\vec{u}_{\text{D}} = (0,0)$.             

The variables are nondimensionalized in the usual way. The reference length $\RefVal{L} = \hat{r}_1$ and the reference velocity $\RefVal{u} =v^F_m$. The reference temperature is $\RefVal{T} = T^F$.  All derived variables are nondimensionalized using the air stream as a reference condition, i.e. $\RefVal{\rho} = \SI{1.17}{\kilo \gram \per \cubic \meter}$, $\RefVal{\mu} = \SI{1.85e-5}{\kilo \gram \per \meter \per \second}$,$\RefVal{W} = \SI{28.82}{\kilo \gram \per \kilo\mole}$. This gives the non-dimensional numbers $\Reynolds = 16.5$ $\text{Da} = 4.3\cdot 10^9$. The Prandtl number is assumed to be constant with $\Prandtl = 0.71$. The reference heat capacity is set $\hat{c}_{p,\text{ref}}= \SI{1.3}{\kilo \joule \per \kilo \gram \per \kelvin}$. Obviously the solution of a nondimensionalized system of equations should be independent of the reference values selected. Nevertheless, the choice of this value for the heat capacity is important because it gives a solution of the flame-sheet which is similar to the actual solution of the full problem. For this calculation, all Lewis numbers are set to unity. Gravity effects are not taken into account. The transport parameters are calculated using Sutherland law with $\hat{S} = \SI{110.5}{\kelvin}$. The mixture heat capacity $c_p$ is calculated with \cref{eq:nondim_cpmixture} and using NASA polynomials for the heat capacity of each component.

\subsubsection{Numerical results}
Numerical experiments using the XNSEC-solver showed that the solution of the problem is highly mesh-dependent. The presence of very high gradients in some areas of the problem requires a higher density of cells around these regions to obtain a well-resolved solution. For the solution of the coflow flame, a stretched base mesh is used, with smaller elements in the vicinity of the inlets and larger elements farther away from the inlets. It was observed that the complex mixing phenomena that occur in the vicinity of the inlets have a critical effect on the convergence of the solution. For this reason, a special refinement of the base mesh was necessary in the vicinity of the tips. 
Furthermore, an adequate mesh resolution on the areas where the flame is located is also critical. To avoid over-solving in zones where actually no reaction is taking place, an adaptive mesh refinement strategy in a pseudo-time-stepping framework was used (see \cref{ssec:MeshRefinement}). After obtaining a steady state solution, the mesh is refined and the calculation is started again. In particular, for this case, several pseudo-timesteps were performed for the flame-sheet calculation. After each of these, the mesh is refined in the vicinity of the flame sheet, i.e., in the cells where $z = z_{\text{st}}$. In \cref{fig:CoFlowMesh} the actual mesh used for the solution of the full problem is shown. Clearly, the mentioned refinement strategy works for the cases where the Lewis number corresponds to unity and the $c_p$ of the mixture has a constant value (assumptions made for the flame-sheet estimate). In case these conditions are not met, the solution obtained for the case with finite reaction rate will be slightly different from the flame-sheet solution.  However, experiments with the XNSEC-solver have shown that this refinement strategy is still beneficial for the convergence of the complete problem, but the selection of a good representative constant value for the heat capacity is important. 
% C:\Users\jfgj8\AppData\Local\BoSSS\plots\sessions\CoFlowFinal_Mult3__Full_CoFlowFlamerP4K12smoothfactor0velMult3LF1__5b10173f-65b1-484a-bfd9-1dc089c3571e
\begin{figure}[t!]
	\centering
	\pgfplotsset{
		compat=1.3,
		tick align = outside,
		yticklabel style={/pgf/number format/fixed},
	}
	\inputtikz{CoFlow_ConvergenceStory}
	\caption{Typical convergence history of a diffusion flame in the coflowing flame configuration.A mesh refinement was done at iteration 21. }
	\label{fig:CoFlow_ConvergenceStory}
\end{figure}

In \cref{fig:CoFlow_ConvergenceStory} the convergence history using the newton algorithm presented in \cref{sec:newton} is shown. The flame-sheet calculation requires 20 Newton iterations to find a solution. It is clearly seen how the residuals $\| \mathcal{A}(\myvector{U}_{n}) \|_2 $  for about the first 14 iterations decrease very slowly, while the $delta$ parameter of the globalized Newton method is adapted to find an optimal value to reduce the residuals. Around iteration 13 the algorithm starts to increase $delta$, obtaining a faster reduction of the residuals and the solution to the problem is found in iteration number 20. Then mesh refinement is used, and 6 iterations are required to find a converged solution. Finally, in iteration number 27 the flame-sheet solution is used as the initial estimate for the full problem, which requires only 11 iterations to find a solution. 

In \cref{fig:CoFlowFlameFig} solution fields of the coflow flame configuration are shown. Since for the selected inlet velocities the flame corresponds to a overventilated one, the typical jet form is observed. It can also clearly be seen from \cref{fig:CoFlowCH4} that all the fuel entering the system reacts with the oxygen. The exothermic reaction produces an acceleration on the flow field, as shown in  \cref{fig:CoFlowVelMag}
The maximum temperature reached corresponds to X (X Kelvin). %TODO

From this test case it is concluded that the strategy of using the flame-sheet as an initial condition presents an efficient way to obtain steady state solutions of combustion problems. Since the flame-sheet is only an estimate, it is possible to perform the calculations on relatively coarse grids, and use low order polynomial degrees. The obtained solution can be used as an estimate for calculations with higher polynomial degrees to find a more accurate solution of the full problem in a few iterations. A disadvantage, as already mentioned, is the choice of the $c_p$ parameter, since an inappropriate choice of it gives a solution of the flame-sheet problem far away from the solution of the full problem. However, the user's experience and access to experimental information allows estimating this value relatively easily. It is worth mentioning that the value chosen in this section ($\hat{c}_{p,\text{ref}}= \SI{1.3}{\kilo \joule \per \kilo \gram \per \kelvin}$)  served as an adequate estimate for all the simulations presented in this thesis.  


\begin{figure}[t]
	\centering
	\pgfplotsset{width=0.6\textwidth, compat=1.3}
	\inputtikz{CoFlowFlameFigTemperature}	
	\inputtikz{CoFlowFlameFigVelMag}	
	\par\bigskip
	\inputtikz{CoFlowFlameFigMF0}	
	\inputtikz{CoFlowFlameFigMF1}	
	\caption{Solution field of a coflow flame configuration.} \label{fig:CoFlowFlameFig}
\end{figure}

\FloatBarrier

\subsection[Counterflow diffusion flame]{Counterflow diffusion flame \footnotemark}\label{ss:CDF}
\footnotetext{Modified version from \cite{gutierrez-jorqueraFullyCoupledHighorder2022}}
\begin{figure}[h!]
	\begin{center}
		\def\svgwidth{0.8\textwidth}
		\import{./plots/}{CounterDiffusionFlame_sketch_rotated2.pdf_tex}
		\caption{Schematic representation (not to scale) of the counterflow diffusion flame configuration.}
		\label{fig:CDFScheme}
	\end{center}
\end{figure}

%This test case is the main prototype flame for diffusion regimes \cite{poinsotTheoreticalNumericalCombustion2005}.
The counterflow diffusion flame is a canonical configuration used to study the structure of nonpremixed flames. This simple configuration has been a subject of study for decades because it provides a simple way of creating a strained diffusion flame, which proves to be useful when studying the flame structure, extinction limits or production of pollutants of flames \citep{pandyaStructureFlatCounterFlow1964,spaldingTheoryMixingChemical1961,keyesFlameSheetStarting1987, leeTwodimensionalDirectNumerical2000}. 

The counterflow diffusion flame consists of two oppositely situated jets. The fuel (possibly mixed with some inert component, such as nitrogen) is fed into the system by one of the jets, while the other jet feeds oxidyzer to the system, thereby establishing a stagnation point flow. On contact and after ignition, the reactants produce a flame that is located in the vicinity of the stagnation plane. A diagram of the setup can be seen in \cref{fig:CDFScheme}. In this section, the solution of a steady two-dimensional flame formed in an infinitely long slot burner will be treated. Simirlarly to the coflow configuration treated before, the infintely long slot burner configuration can be calculated naturally using cartesian coordinates.

First, as a means of verificating the solver for combustion applications, the results obtained with the XNSEC-solver for steady two-dimensional counterflow diffusion flame are compared with the solution of a simplified system of equations for a steady and quasi one-dimensional flame. Later, the influence of the inlet velocities on the maxmimum temperature is studied and finally some remarks concerning the convergence behaviour of the case counter diffusion flame are given. 
\subsubsection{The one-dimensional diffusion flame}
By assuming an infinite injector diameter, a self-similar solution and by neglecting the radial gradients of the scalar variables along the axis of symmetry, it is possible to reduce the three-dimensional governing equations to a one-dimensional formulation along the stagnation streamline $y = 0$ (see the textbook from \cite{keeChemicallyReactingFlow2003} for the derivation).  The governing equations for a steady planar stagnation flow reduce to
\begin{subequations}
\begin{gather}
	%%%%%%%%%%
	\frac{\partial\hat\rho \hat v}{\partial \hat x} +  \hat \rho \hat U = 0\label{eq:OneDimCont}\\ %
	%%%%%%%%%%
	\hat \rho \hat v \frac{\partial \hat U}{\partial \hat x} + \hat \rho \hat U^2 =
	- \hat \Lambda
	+ \frac{\partial}{\partial \hat x}\left(\hat \mu \frac{\partial \hat U}{\partial \hat x}\right)\label{eq:OneDimMom}\\ %
	%%%%%%%%%%
	\hat\rho \hat c_p \hat v \frac{\partial \hat T}{\partial\hat x} =
	\frac{\partial}{\partial \hat x}\left( \hat \lambda \frac{\partial \hat T}{\partial \hat x}\right)
	+\hat\heatRelease~\hat{\mathcal{Q}}\label{eq:OneDimTemp}\\
	%%%%%%%%%%%
	\hat\rho \hat v \frac{\partial Y_k}{\partial \hat x} = 
	\frac{\partial}{\partial \hat x}\left(\hat \rho \hat D \frac{\partial Y_k}{\partial \hat x}\right)
	+ \hat W_k \stoicCoef_k \hat{\mathcal{Q}} \quad (k = 1, \dots,~N - 1) \label{eq:OneDimMF}
	%%%%%%%%%%%
\end{gather}\label{eqs:OneDimEquations}%
\end{subequations}
\cref{eqs:OneDimEquations}.
where $\hat U$ is the scaled velocity and $\hat \Lambda$ is the radial pressure curvature, which is an eigenvalue independant of $x$. Again, the hat sign represent dimensional variables. The equations are written assuming a valid Fick's law and an one-step combustion model. The system of equations needs to be solved for $\hat v$, $\hat U$, $\hat T$ and for $Y_k$, $ (k = 1, \dots,~N - 1)$.  Aditionally an equation of state and expressions for the heat capacity $\hat c_p$ and transport parameters $\hat \mu, \hat \lambda, (\hat \rho \hat D)$ are needed. This formulation is very well known and often used for analysis of flame structure, determination of extintion points, to mention a few.

In order to assess the ability of the XNSEC-solver to simulate such a system, the solution obtained for a two dimentional configuration is compared with the solution of the quasi one-dimensional equations solved with \lstinline|BVP4|, a fourth order finite difference boundary value problem solver provided by \lstinline|MATLAB| \citep{kierzenkaBVPSolverBased2001}. The \lstinline|BVP4| solver provides automatic meshing and error control based on the residuals of the solution, allowing to create with relative easyness a code that solves the one-dimensional equations.


It is important to mention some points regarding the solution of these equations using the \lstinline|BVP4| solver. Analogous to the problem mentioned in \cref{ssec:MethodCombustion}, the solution of the system of  \crefrange{eq:OneDimCont}{eq:OneDimMF} presents the particularity that it is possible to find multiple solutions. One of them is clearly the cold solution and another solution is the burning one. The same idea mentioned in  \cref{ssec:MethodCombustion} is also valid for the quasi one-dimensional configuration. In particular this means, that as a first step for finding a converged solution of \crefrange{eq:OneDimCont}{eq:OneDimMF} is to solve the system
\begin{subequations}
\begin{gather}
	%%%%%%%%%%
\frac{\partial \hat \rho \hat v}{\partial \hat x} +  \hat \rho \hat U = 0\\ \label{eq:OneDimCont2}%
%%%%%%%%%%
\hat \rho \hat v \frac{\partial \hat U}{\partial \hat x} + \hat \rho \hat U^2 =
- \hat \Lambda
+ \frac{\partial}{\partial \hat x}\left(\hat \mu \frac{\partial \hat U}{\partial \hat x}\right)\\ \label{eq:OneDimMom2}%
%%%%%%%%%%
\hat \rho \hat v \frac{\partial Z}{\partial \hat x} = 
\frac{\partial}{\partial \hat x}\left(\hat \rho \hat D \frac{\partial Z}{\partial \hat x}\right)
%%%%%%%%%%%
\end{gather}\label{eqs:OneDimEquationsMixtureFraction}
\end{subequations}
together with the equation of state \cref{eq:ideal_gas} and expressions for the transport parameters. The dependency of the temperature and mass fractions on the mixture fraction $Z$ is given by the Burke-Schuhmann limit. This solution can be used as a initial estimate for the solution of \crefrange{eq:OneDimCont}{eq:OneDimMF}.
An inconvenience is that an initial estimate for the $\hat c_p$ has to be chosen. It was observed that if the value of $\hat c_p$ was chosen too big, the solver delivered solutions without a flame. For the calculations treated here $\hat c_p =\SI{1.3}{\kilo \joule\per\kilogram \kelvin}$ was an adequate value that delivered the ignited solution. 

It was however observed that this flamesheet solution wasnt directly useful as an initial estimation for the solution of the full system of equations. In order to help  the \lstinline|BVP4| solver to find a coverged solution, an intermediate step was necesary. First, the flamesheet solution was used as an initial estimate for the solution of system \cref{eqs:OneDimEquations} and the equation of state and equation for transport parameters but assuming a constant heat capacity. Once the algorithm has found a solution, it can be used for solving the same system but with a variable heat capacity according to \cref{eq:nondim_cpmixture}.


\subsubsection{Set-up of the two-dimensional counterflow diffusion flame}

The combustion of diluted methane with air in a two-dimensional infinitely long slot burner configuration is studied in this part. The solution is obtained by solving the system \cref{eq:all-eq}, making use of the flame-sheet solution as initial estimates. The transport parameters are calculated using Sutherland law with $\hat{S} = \SI{110.5}{\kelvin}$. Gravity effects are not taken into account.  The mixture heat capacity $c_p$ is calculated with \cref{eq:nondim_cpmixture} and using NASA polynomials for the heat capacity of each component.
\begin{table}[b]
	\centering
	\begin{tabular}{lccccc}
		\hline
		& \multicolumn{1}{l}{$\hat v^F_m$ ($\si{\centi \meter \per \second}$)} & \multicolumn{1}{l}{$\hat v^O_m$ ($\si{\centi \meter \per \second}$)} & $a$($\si{\per\second})$ & \multicolumn{1}{l}{$\hat T^F$($\si{\kelvin}$)} & \multicolumn{1}{l}{$\hat T^O$($\si{\kelvin}$)} \\ \hline
		case(a) & 4.85                                                                 & 12.29                                                                & 34                     & 300                                            & 300                                           \\
		case(b) & 12.13                                                                & 30.73                                                                & 76                     & 300                                            & 300                                           \\
		case(c) & 26.69                                                                & 67.62                                                                & 155                    & 300                                            & 300                                           \\ \hline
	\end{tabular}
	\caption{Maximum inlet velocity, strain and temperatures used for the counterflow diffusion flame calculations.}
	\label{tab:cdf_velocities}
\end{table}

For the comparison with the quasi one-dimensional model, three pairs of inlet velocities are considered. They are shown in \cref{tab:cdf_velocities}. Here, $v_m^F$ and $v_m^O$ are the maximal velocity of a parabolic profile for the fuel and air inlets, respectively. Both streams enter at a temperature $\hat T^O = \hat T^F = \SI{300}{\kelvin}$. The mass composition of the fuel inlet is assumed to be  $Y^F_{\ch{CH4}} = 0.2$ and $Y^F_{\ch{N2}} = 0.8$, and the oxidizer inlet is air with  $Y^O_{\ch{O2}} = 0.23$ and $Y^O_{\ch{N2}} = 0.77$. 
Counterflow diffusion flames are usually characterized by the strain rate $a$. Many different definitions for it can be found in the literature \citep{fialaNonpremixedCounterflowFlames2014}. In this work the definition of the strain rate the maximum axial velocity gradient is used. The strains for the three cases mentioned above are $\SI{34}{\per\second}$, $\SI{76}{\per\second}$ and $\SI{155}{\per\second}$, respectively. 

The lengths described in \cref{fig:CDFScheme} are $\hat D = \SI{2}{\centi\meter}$, $\hat H = \SI{2}{\centi\meter}$ and $\hat L = \SI{12}{\centi\meter}$. The variables are non-dimensionalized using $\RefVal{L} = \SI{2}{\centi\meter}$, $\RefVal{T} = \SI{300}{\kelvin}$ and $\RefVal{p} = \SI{101325}{\pascal}$.  For each case, the reference velocity is set to $\RefVal{u} = \hat v^O$.  Again, all derived variables are nondimensionalized using the air stream as a reference condition, i.e. $\RefVal{\rho} = \SI{1.17}{\kilo \gram \per \cubic \meter}$, $\RefVal{\mu} = \SI{1.85e-5}{\kilo \gram \per \meter \per \second}$ and $\RefVal{W} = \SI{28.82}{\kilo \gram \per \kilo\mole}$. The reference heat capacity is set $\hat{c}_{p,\text{ref}}= \SI{1.3}{\kilo \joule \per \kilo \gram \per \kelvin}$. 

Under this conditions, the Reynolds numbers are $\Rey = 156$, $\Rey = 390$ and $\Rey = 858$, for the low, medium and high inlet velocities respectively. The Dahmkoler numbers are $\Da = 4.6\cdot10^9$, $\Da = 1.8\cdot10^9$ and $\Da = 8.3\cdot10^8$. The Prandtl number is assumed to be constant with $\Prandtl = 0.71$. A non-unity but constant Lewis number formulation is used, with $\Lewis_{\ch{CH4}} =  0.97 $ , $\Lewis_{\ch{O2}} = 1.11 $, $\Lewis_{\ch{H2O}} = 0.83 $ and $\Lewis_{\ch{CO2}} = 1.39 $ \citep{smookePremixedNonpremixedTest1991}. The system is considered open, then the thermodynamic pressure is constant and set to  $ p_0 = 1$. 

%%%%%%%%%%%%%%%%%%%%%%%%%%%
%% BoundaryConditions
%%%%%%%%%%%%%%%%%%%%%%%%%%%
The boundary condition of the inlets are,
\begin{itemize}
	\item Oxidizer inlet: $\{\forall (x,y): y = 0 \land x \in [-D/2, D/2]\}$\\
	\begin{equation*}
		u = 0,\qquad v= v^O(y), \qquad T = 1.0, \qquad \vec{Y}' = (0,Y^O_{\ch{O2}},0,0)
	\end{equation*}
	\item Fuel Inlet: $\{\forall (x,y): y = H \land x \in [-D/2, D/2]\} $ \\
	\begin{equation*}
		u = 0,\qquad v= v^F(y), \qquad T = 1.0, \qquad \vec{Y}' = (Y^F_{\ch{CH4}},0,0,0)
	\end{equation*}
\end{itemize}
 
\begin{figure}[p]
	\centering
	\pgfplotsset{width=0.73\textwidth, compat=1.3}
	\inputtikz{CounterFlowFlameMesh}	
	\inputtikz{CounterFlowFlameStreamlines}
	\inputtikz{CounterFlowFlameTemperature}
	\inputtikz{CounterFlowFlameDensity}	
	\inputtikz{CounterFlowFlamekReact}
	\inputtikz{CounterFlowFlamePressure}	
	\caption{Nondimensional solution and derived fields of the counterflow flame configuration for case (a).} \label{fig:CoFlowFlameFig1}
\end{figure}
\begin{figure}[p]
	\ContinuedFloat
	\centering
	\pgfplotsset{width=0.73\textwidth, compat=1.3}		
	\inputtikz{CounterFlowFlameCpMixture}	
	\inputtikz{CounterFlowFlameMF0}
	\inputtikz{CounterFlowFlameMF1}
	\inputtikz{CounterFlowFlameMF2}
	\inputtikz{CounterFlowFlameMF3}
	%	\inputtikz{CounterFlowFlameDensity}	
	\caption{Nondimensional solution and derived fields of the counterflow flame configuration for case (a) (continued).}% \label{fig:CoFlowFlameFig1}
\end{figure} 


The pressure outlet boundary condition is the same as \cref{eq:bc_O}. The pressure outlet boundaries are placed far away of the center of the domain, in order to decrease the effect on the centerline. Placing the boundary further away did not change appreciably the results. Finally the bundary conditions at the walls are defined as in \cref{eq:bc_dn}, with $\vec{u}_{\text{D}} = (0,0)$ and a constant temperature $T = 1.0$.          
 
In \cref{fig:CoFlowFlameFig1} the solution profiles for the case (a) are shown. The used mesh was obtained by a process of mesh refinement. The base mesh is initially created with a larger concentration of elements in the center of the domain. The points of intersection from the velocity inlet and wall boundary conditions are also refined, which was observed to improve the robustness of the algoritm. Similarly to the coflowing flame (\cref{ssec:coflowFlame}), during the solution algorithm of the flame sheet problem, mesh is aditionally refined around the flame sheet making use of a pseudo-timestepping setting.
As expected, a stagnation flow develops and a flame forms close to it. For this strain rate, a maximum temperature of $T = 6.05$ is obtained (1815 $\si{\kelvin}$). This big increase in the temperature is also reflected in a big decrease in the density field, where a decrease of almost six times on the density values is apreciated. This change of density also provokes the acceleration of fluid, as observed in \cref{fig:CounterFlowStreamlines}. 

In \cref{fig:CounterFlowReactionRate} the reaction rate given by \cref{eq:NonDimArr} is ploted. It is interesting to see that the actual reacting zone is very small, which clearly demonstrates why adequate meshing is necessary to capture the steep gradients resulting from the strong and highly localized heat sources. Finally, and as expected, the fuel and oxidizers field seem to only be found on either side of the flame. Altought it can not be seen here, some reactant leaking occurs, meaning that there exists a small zone where both especies coexist. This point will be adressed later. 


\subsubsection{Comparison of two-dimensional and the quasi one-dimensional counterflow flames}
In this section a comparison of the results obtained with the XNSEC-solver for a two dimensional counterflow diffusion flame, and the results obtained with the \lstinline|BVP4| solver for a quasi one-dimensional flame is done. The comparison of both set of results is done along the centerline of the domain (see \cref{fig:CDFScheme}). In this section, only dimensional variables will be considered. The transport parameters, chemical model and the equation of state are exactly the same for both formulations. For all calculations in this section, a polynomial degree of four is used for the velocity components, temperature and mass fractions. A polynomial degree of three is used for the pressure. This resulted on systems with approximately 439000 degrees of freedom. 
 
\begin{figure}[t!]
	\centering
	\inputtikz{CounterFlowFlame_DifferentBoundaryConditions}
	\caption{Velocity profiles of the counterflow diffusion flame for parabolic and plug inlet boundary conditions.}\label{fig:CounterFlowFlame_DifferentBoundaryConditions}
\end{figure}
%TODO Still calculating => same plot but with vel mult 11 '\\hpccluster\hpccluster-scratch\gutierrez\CounterFlowFlame_BCComparison6'.
The choice of the type of velocity boundary conditions for the inlets requires some attention. Different possiblities exist to describe the velocity profiles. One posibility is to characterize the velocity boundary conditions by assuming a Hiemenz potential flow, where a single parameter defines the flow field. Other posibilities are also a constant velocity value (plug flow) or a parabolic profile, allowing to define different velocity values for each jet inlet.
The effect of boundary conditions on the flame structure has been estudied by \cite{chelliahExperimentalTheoreticalInvestigation1991} and \cite{johnsonAxisymmetricCounterflowFlame2015}, where it was concluded that both plug and potential are able to adequately describe experimental data. 


The question whether a plug or parabolic flow profile allows a better representation of the quasi one-dimensional equations was treated in the work from \cite{frouzakisTwodimensionalDirectNumerical1998}. There is stated that the one and two dimensional formulations deliver very similar results, provided that the inlets of the two-dimensional configurations are uniform. Furthermore, preliminary calculations with the XNSEC-solver showed that the selection of a plug flow or parabolic have an influence on the solution, as shown in \cref{fig:CounterFlowFlame_DifferentBoundaryConditions}. Based on these results, the plug flow boundary condition is adopted for all following test cases.

\begin{figure}[t]
	\pgfplotsset{
		width=0.95\textwidth,
		group/xticklabels at=edge bottom,
		legend style = {
			at ={ (0.49,1), anchor= north east}
		},
		unit code/.code={\si{#1}}
	}
	\centering
	\inputtikz{BoSSS_1D_Comparison_velocity}
	\caption{Comparison of the axial velocity calculated with the XNSEC-solver and the one-dimensional approximation.}
	\label{fig:BoSSS_1D_Comparison_velocity}
\end{figure}
 In \cref{fig:BoSSS_1D_Comparison_velocity} a comparison of the axial velocities calculated with the XNSEC-solver and the one-dimensional solution is shown. While for the high strain case the results agree closely, for lower strains a discrepancy is observed. Recall that the derivation of the one-dimensional approximation assumes a constant velocity field incoming to the flame zone in order to obtain a self-similar solution. In the case of the two-dimensional configuration presented here, the border effects do have an influence on the centerline, which disrupts the self-similarity. This effect is more pronounced for low velocities, which explains the discrepancy between curves.
 
 
 \tikzexternaldisable
 \begin{figure}[p]
 	\centering
 	\pgfplotsset{
 		width=0.85\textwidth,
 		height = 0.33\textwidth,
 		compat=1.3,
 		tick align = outside,
 		yticklabel style={/pgf/number format/fixed},
 	}
 	\inputtikz{BoSSS_1D_Comparison1}
 	\inputtikz{BoSSS_1D_Comparison2}
 	\inputtikz{BoSSS_1D_Comparison3}
 	\caption[Comparison of temperature and mass fraction fields obtained with the XNSEC-solver and the one-dimensional approximation.]{Comparison of temperature and mass fraction fields obtained with the XNSEC-solver (solid lines) and the one-dimensional approximation (dashed lines).}
 	\label{fig:BoSSS_1D_Comparison}
 \end{figure}
 \tikzexternalenable
Similarly, In \cref{fig:BoSSS_1D_Comparison} the temperature and mass fraction fields are presented. Again, a discrepancy is observed for low strains, but results show a good agreement for higher inlet velocities. It can also be observed how, as expected, at higher strains a significant leakage of oxygen across the flame is present. This is a typical behaviour of a flame that is getting closer to its extintion point \cite{fernandez-tarrazoSimpleOnestepChemistry2006}.  

This is a drawback from usual one-step models with constant activation temperature, because they tend to overpredict fuel leakage. This behaviour is not appreciated in the one-step model with variable activation temperature used here.  In \cref{fig:VarParams} the comparison is shown for the configuration (c) between the mass fractions fields obtained using a chemical model with variable kinetic parameters given by \cref{eq:ActivationTemperatureOneStep,eq:heatReleaseOneStep} and other with constant kinetic parameters using $\hat T_a = \hat T_{a0}$ and  $\hat Q = \hat Q_{0}$.  The oxygen leakage obtained by using the chemical model with variable parameters is evident, confirming 
 
 
 \begin{figure}[h]
 	\centering
	\inputtikz{MassFractionDifferentChemModel}
		\inputtikz{MassFractionDifferentChemModelZoomed}
 	\caption[Fuel and oxidizer mass fraction profiles using constant kinetic parameters and variable kinetic parameters]{Oxygen leakage in the counterflow diffusion flame configuration.Fuel and oxidizer mass fraction profiles  using variable kinetic parameters (VK) and constant kinetic parameters (CK) are shown. Right picture is zoomed in near the flame zone.} \label{fig:VarParams}
 \end{figure}

 
In \cref{fig:TemperatureStrainPlot} the maximum temperature obtained in the centerline for different strain rates is ploted. Qualitatively speaking, the solution obtained with the XNSEC-solver agrees with the expectations. As the strain rate increases, the maximum temperature decreases (see \cref{fig:Sshaped}). On the other hand, the comparison of values obtained with the XNSEC-solver and those of the quasi one-dimensional approximation clearly shows a discrepancy in the results. For low strain rates this discrepancy is small, being for $a = \SI{20}{\per\second}$ only 10K, approximately a difference of 0.5\%. As the strain rate increases so does the discrepancy. For $a = \SI{200}{\per\second}$ the difference is almost 50K, which is a 9\% disagreement. A similar behaviour is also reported in \cite{frouzakisTwodimensionalDirectNumerical1998}, where a difference of 50K was reported. 

It is worth noting that the XNSEC-solver wasnt able to find a converged solution for $a > \SI{202}{\per\second}$, and the newton algorithm stagnates. This is most probably a sign of underresolution of the mesh, and that the used refinement strategy did not help for such high strain rates. A better mesh refinement strategy is necesary for calculating the flame at conditions near the extintion point. 
Moreover, for high strain rates the flame will be far from the thermochemical equilibrium, and it is likely that the solution obtained for the flame sheet will be far away from the solution with finite rates. A posibility would be to use one of the well-known continuation methods (see for example \cite{nishiokaFlamecontrollingContinuationMethod1996}) to progressively move in the direction of the extinction point. Obviously, the Homotopy Methodology presented in \cref{sec:HomotopyMethod} can be visualized as one of those methods, and would be useful when looking for solutions of systems that are close to the extinction point. A complexity that arises then is how to create a dynamical mesh that is suitable for obtaining the intermediate solutions while searching for the final result in a robust way. This issue is beyond the scope of this thesis, and may be the subject of future research.


The difference between the results obtained for the two-dimensional configuration and the quasi-one dimensional approximation could be explained by some condition within the 1D system assumptions being violated in the 2D configuration. It is known that in addition to the boundary conditions, the ratio between the slot width and the separation between the two slots ($D/H$) also has an influence on the solution, and that a high ratio is desirable \citep{frouzakisTwodimensionalDirectNumerical1998}. Another point that was not addressed here is whether the boundary conditions chosen for cold walls have an effect on the solution along the centerline. Other posibilities could have been using outlet boundary conditions or an adiabatic wall. It is nevertheless expected that its effect on the centerline would not be big. The points mentioned here should be adressed in future work.

\begin{figure}[t]
	\centering
	\inputtikz{TemperatureStrainPlot}
	\caption{Maximum centerline temperature of a counterflow flame for different strains.}
	\label{fig:TemperatureStrainPlot}
\end{figure}
\subsubsection{Temperature convergence study}
\begin{figure}[h]
	\centering
	\inputtikz{TemperatureConvergenceDiffFlame}
	\caption{Convergence study of the maximum value of the temperature for the counterflow diffusion flame configuration.}
	\label{fig:TemperatureConvergenceDiffFlame}
\end{figure}
Similar to problems presented in earlier sections, the presence of singularities caused by non-consistent boundary conditions causes a degenerative effect on the global error values, making a global convergence study for this configuration is problematic. However, it is still possible to study the behaviour of some characteristic point value under different conditions to prove the mesh independence of the solution.

In \cref{fig:TemperatureConvergenceDiffFlame} it is shown how the maximum temperature along the centerline obtained for configuration (b)  behaves under different mesh resolutions and polynomial degrees. Values for $k=1$ are not shown, because for this range of cell elements the maximum temperature value was of the order of 60K higher than the ones depicted here. The temperature tends to a limit value, and its posible to observe how this value is reached already for coarse meshes when using a polynomial degree of three or four. For $k=2$ the temperature also tends to a limit value, but at a slower rate when compared to $k =3$ or $k = 4$. 

In the next section a simplified one-dimensional flame configuration will be used in order to be able to realize a $h$-convergence study of the whole sytem operator.


\subsection{Chambered diffusion flame}\label{ss:UDF}
\begin{figure}[h]
	\begin{center}
		\def\svgwidth{0.4\textwidth}
		\import{./plots/}{UnstrainedFlameConfig.pdf_tex}
		\caption{Schematic representation of the chambered diffusion flame configuration. }
		\label{fig:chamberedDifFlame}
	\end{center}
\end{figure}
In this chapter an $h$-convergence study for a quasi-one-dimensional configuration is shown. This is done by using a planar unstrained diffusion flame in the so-called chambered diffusion flame. This configuration has served as a model for many theoretical studies related to diffusion flames \parencite{matalonDiffusionFlamesChamber1980,rameauNumericalBifurcationChambered1985,matalonEffectThermalExpansion2010}. 

A sketch of the configuration can be seen in \cref{fig:chamberedDifFlame}. Fuel is injected at a constant rate into the bottom of a small insulated chamber, while oxidant diffuses into the system against the direction of the flow. Constant conditions at the outlet of the chamber are achieved in an experimental setting by a rapid renewal of the flow of the oxidant. Under these conditions, an unstrained planar flame is formed.

The fuel inlet into the chamber is modeled with a constant velocity inlet boundary condition \cref{eq:bc_d}, while the flow outlet at the top is considered an outlet as given by \cref{eq:bc_OD}. Since the interest is in the flame far away from the container walls, it is sufficient to set the remaining boundary conditions as periodic boundaries. This effectively transforms the problem into a pseudo-two-dimensional configuration.

\begin{figure}[t!]
	\centering
	\pgfplotsset{width=0.34\textwidth, compat=1.3}
	\inputtikz{ConvergenceDiffFlame}
	\caption{Convergence study for the chambered diffusion flame configuration.}
	\label{ConvergenceDiffFlame}
\end{figure}
The inlet velocity of the fuel jet is set to $\SI{2.5}{\centi\meter \per \second}$ and its mass composition is $Y^0_{\ch{CH4}} = 0.2$ and $Y^0_{\ch{N2}} = 0.8$ while air has a composition $Y^0_{\ch{O2}} = 0.23$ and $Y^0_{\ch{N2}} = 0.77$. The temperature of the fuel and air feed streams is $\SI{300}{\kelvin}$. The length of the system $L$ is equal to $\SI{0.015}{\meter}$. The Reynolds number is $\Reynolds = 2$.

For this configuration, an $h$-convergence study is conducted, where uniform Cartesian meshes with $5\times2^6$, $5\times2^7$, $5\times2^8$,  $5\times2^9$ and $5\times2^{10}$  cells are used. The polynomial degrees are varied from one to four for velocity, temperature and mass fractions, and from zero to three for pressure.  Errors are calculated using the finest mesh as a reference solution.  

The results are shown in \cref{ConvergenceDiffFlame} for variables $u$, $T$, $Y_{\ch{CH4}}$ and $p$. The convergence results for other variables are similar and not shown here. The expected convergence rates characteristic for the DG method are observed. For low polynomial degrees the orders of convergence are very close to the theoretical values. However for higher polynomial degrees  a slight deterioration of the convergence rate is observed.



\FloatBarrier

\cleardoublepage
\chapter{Conclusion}	\label{ch:conclusion}
\glsresetall
\begin{figure}[tb]
	\begin{center}
		\def\svgwidth{0.9\textwidth}
	\import{./plots/}{BFS_sketch.pdf_tex}
		\caption{Schematic representation of the backward-facing step. Both primary and secondary vortices are shown. Sketch is not to scale.}
		\label{BFSsketch}
	\end{center}	
\end{figure} 


%%%%%%%%%%%%%%%%%%%%%%%%%%%%%%%%%%%%%%%%%%
% En la conclusion puedo hablar de que cosas serian necesarias en futuro trabajo
% The governing equations treated in this work were based on some rather strong assumptions. In particular, the one-step model chemical model 




%C:\Users\jfgj8\AppData\Local\BoSSS\plots\sessions\CounterFlowFlame_MF_FullComparison_ForThesis3__Full_CounterDifFlameP3K10mult2_c0__90434a33-abe4-45a2-8e8e-2383a9c72d31

%%%%%%%%%%%%%TODOOOOOOOOOOOOOOOOOO debo establecer de forma consistente si me refiero a las coordenadas como (x,y) o (x1,x2), similarmente si la velocidad es (u,v)o (u1,u2) 
%%%%%%%%%%%%%%%%%%% TODO: Verificar si realmente todos los indices de las mass fractions ahora están referidos como k, y el numero total como N. Hacer lo mismo con M y W para los pesos moleculares.
%
%TODO still i need to check all the consistency between dimensionless and normal variables
% TODO express in a formal way what actually the subscripts F, O, P, N mean in the context of combustion.
%TODO check for consistency between WE and other formulations
% Cambiar en todos lados donde se hace referencia a mi solver, y referirse a el consistentemente como XNSEC
%TODO: acortar usando \caption[]{} los nombres de las figuras que apareceran en la lista de imagenes/tablas
%TODO: pregunta abierta: basta con definir el poly deg k, y con eso esta ya definido k'?
%TODO: agregar puta imagen con los errores de P en la convergencestudy seccion
%TODO: corregir los limites de los colorbar de tal forma que aparezcan los minimos/maximos y que tengan los formatos correctos
\cleardoublepage
\printbibliography[heading=bibintoc]
\label{bibref}
%
% \cleardoublepage
% \appendix
% \chapter{Appendix}	\label{ch:appendix}
\glsresetall
%By substituting the expansions into the non-dimensionalized equations XX and the non-dimensionalized ideal gas equation XX, and collecting the lowest orders for $\epsilon$, the low-Mach number equations are obtained.
	\section{NS and energy equations}
	Equations with dimensions.\\
	Conti:
	\begin{equation}
	\frac{\partial \rho^*}{\partial t^*}+ \nabla ^* \cdot (\cd{\rho}\cd{\vec{u}}) = 0
	\end{equation}
	Momentum
	\begin{gather}
	\dtpartcd{\rho^*\vec{u}^* } + \nabla ^* \cdot (\rho^*\vec{u}^*\vec{u}^*) + \nabla^*p^* = \nabla^* \cdot \tau^* + \rho^* \vec{g}^*\\
	\tau^* = \mu^*(\nabla^*\vec{u}^* + (\nabla^*\vec{u}^*)^T) - \frac{2}{3}\mu^*\nabla^*\cdot \vec{u}^*\mathbf{I}
	\end{gather}
	energy
	\begin{gather}
	\dtpartcd{\rho^*E^* } + \nabla ^* \cdot (\rho^*H^*\vec{u}^*) = Q^*\\
	Q^* = \nabla^* \cdot (\tau^*\cdot\vec{u}^*) + \rho^* \vec{g}^* \cdot \vec{u}^* + \nabla^*\cdot (k^*\nabla^*T^*)+\rho^*q^*
	\end{gather}
	And with the equations of state 
	\begin{align}
	p^* &= \rho ^* R^* T^*\\
	e^* &= c_v^*T^*
	\end{align}
	\section{Low Mach number asymptotics}
	\subsection{nondimensionalization}
	The equations are non-dimensionalized (?) by using reference quantities denoted with the subscript $\infty$ and a reference length scale $L^*$
	
	\begin{align}
	\nondimA{\rho},\quad
	\nondimA{p},\quad
	\nondimA{\vec{u}},\quad
	\nondimA{T},\quad
	\nondimA{\mu},\quad
	\nondimA{\kappa},\quad \\
	\mathbf{x} = \frac{\mathbf{x}^*}{L^*},\quad
	t = \frac{t^*}{L^*/u^*_\infty},\quad
	\nondimB{e},\quad
	\nondimB{E},\quad
	\nondimB{H}
	\end{align}
	The reference quantities are chosen such taht the nondimensional flow quantities remain of order O(1) for any low-reference-Mach number. The Mach number is defined as
	\begin{equation}
	M_\infty = \frac{u^*_\infty}{\sqrt{\gamma p^*_\infty/\rho^*_\infty}}
	\end{equation}
	To avoid the dependence on $\gamma$ we work with $\tilde{M}$ 
	\begin{equation}
	\tilde{M} = \frac{u^*_\infty}{\sqrt{p^*_\infty/\rho^*_\infty}} = \sqrt{\gamma}M_\infty
	\end{equation}
	Using the aforementioned reference quantities the nondimensionalized Navier-Stokes read\\
	Continuity equation:
	\begin{equation}
	\frac{\partial \rho}{\partial t} + \nabla \cdot (\rho \vec{u}) = 0
	\end{equation}
	Momentum equations
	\begin{equation}
	\frac{\partial (\rho \vec{u})}{\partial t} + \nabla   \cdot (\rho \vec{u} \vec{u} ) + \frac{1}{\tilde{M}^2}\nabla p  = \frac{1}{\text{Re}_\infty}\nabla  \cdot \tau  + \frac{1}{\text{Fr}_\infty^2}\rho (-\mathbf{e}_r)
	\end{equation}
	Energy equations
	\begin{gather}
	\frac{\partial (\rho E)}{\partial t} + \nabla   \cdot (\rho H \vec{u} )   = Q \\
	Q  =\frac{\tilde{M}^2}{\text{Re}_\infty} \nabla  \cdot (\tau \cdot\vec{u} ) + 
	\frac{\tilde{M}^2}{\text{Fr}_\infty^2}\rho (-\mathbf{e}_r) \cdot \vec{u}  + 
	\frac{\gamma}{(\gamma-1)\text{Re}_\infty\text{Pr}_\infty}\nabla \cdot (k \nabla T )+
	\rho q 
	\end{gather}
	Where the Reynolds number, Froude number and Prandtl number are defined as
	\begin{equation}
	\text{Re}_\infty = \frac{\rho^*_\infty u^*_\infty L^*}{\mu^*_\infty}, \qquad \text{Fr}_\infty = \frac{u^*_\infty}{\sqrt{g^*L^*}}, \qquad \text{Pr}_\infty = \frac{c_p^* \mu^*_\infty}{\kappa^*_\infty}
	\end{equation}
	
	
	\subsection{Asymptotic analysis}
	
	We are interested in slow flow affected by acoustic effects in a confined gas over a long time. Therefore, we introduce the fast acoustic time scale
	\begin{equation}
	\tau = \frac{t^*}{L^* / \sqrt{\frac{p^*_\infty}{\rho^*_\infty}}} = \frac{t}{\tilde{M}}
	\end{equation}
	Note that the flow time scale ($t$) is determined by the time it takes the reference flow to travel one length scale, and the acoustic time scale corresponds to the time it takes to travel one length scale at the reference speed of sound divided by $\sqrt{\gamma}$\\
	
	In the two-time scale, single space scale low Mach number asymptotic analysis each flow variable is expanded as e.g. the pressure:
	\begin{equation}\label{eq:defExp}
	p(\mathbf{x},t,\tilde{M}) = p_0(\mathbf{x},t,\tau) + \tilde{M} p_1(\mathbf{x},t,\tau) + \tilde{M}^2 p_2(\mathbf{x},t,\tau) + \mathcal{O}(\tilde{M}^3)
	\end{equation}
	
	Note that the time derivative at constant $\mathbf{x}$ and $\tilde{M}$ involves the flow time derivative $\partial/\partial t$ and the acoustic time derivative $\partial/\partial\tau$
	\begin{equation}
	\left.\frac{\partial p}{\partial t}\right|_{\mathbf{x},\tilde{M}} = 
	\left( \frac{\partial}{\partial t} + \frac{1}{\tilde{M}}\frac{\partial}{\partial \tau}\right)[p_0 + \tilde{M} p_1 + \tilde{M}^2 p_2 + \mathcal{O}(\tilde{M}^3)]
	\end{equation}
	
	Expanding each variable ($\rho$, $\vec{u}$, $p$, $E$...) according to \ref{eq:defExp}, inserting them in the Navier-Stokes equations and energy equations and comparing the Mach number powers we obtain for the continuity equation:
	\begin{align}
	M^{-1}:&\qquad \frac{\partial \rho_0}{\partial \tau} = 0\\
	M^{0}:&\qquad  \frac{\partial \rho_1}{\partial \tau} + \frac{\partial \rho_0}{\partial t}  + \nabla \cdot (\rho \vec{u})_0 = 0\\
	M^{1}:&\qquad  \frac{\partial \rho_2}{\partial \tau} + \frac{\partial \rho_1}{\partial t}  + \nabla \cdot (\rho \vec{u})_1 = 0
	\end{align}
	Note that the first equation implies that $\rho_0$ does not depend on the acoustic time scale ($\rho_0 = \rho_0(\mathbf{x}, t)$)
	
	Momentum equations:
	\begin{align}
	M^{-2}:& \qquad \nabla p_0 = 0 \label{eq:LeadMom}\\
	M^{-1}:& \qquad \frac{\partial (\rho \vec{u})_0}{\partial \tau} + \nabla p_1 = 0\\
	M^{0}:&  \qquad \frac{\partial (\rho \vec{u})_1}{\partial \tau} + \frac{\partial (\rho \vec{u})_0}{\partial t} + \nabla \cdot (\rho \vec{u}\vec{u})_0 + \nabla p_2 = \mathbf{G}_0 \label{eq:secOrderMomEq}\\
	\text{with}& \qquad \mathbf{G}_0 = \frac{1}{\text{Re}_\infty}\nabla \cdot \tau_0 + \frac{1}{\text{Fr}_\infty^2}\rho_0(-\mathbf{e}_r)  
	\end{align}
	Again, note that the first equation implies that $p_0$ does not depend on $\mathbf{x}$.\\
	
	Energy Equations 
	\begin{align}
	M^{-1}:&\qquad \frac{\partial (\rho E)_0}{\partial \tau} = 0,\label{eq:LeadEn}\\
	M^{0}: &\qquad \frac{\partial (\rho E)_1}{\partial \tau} + \frac{\partial (\rho E)_0}{\partial t} + \nabla \cdot (\rho H \vec{u})_0= Q_0,\\
	M^{0}: &\qquad \frac{\partial (\rho E)_2}{\partial \tau} + \frac{\partial (\rho E)_1}{\partial t} + \nabla \cdot (\rho H \vec{u})_1= Q_1
	\end{align}
	With 
	\begin{gather}
	Q_0 = \frac{\gamma}{(\gamma - 1)\text{Re}_\infty \text{Pr}_\infty} \nabla \cdot (\kappa \nabla T)_0 + (\rho q)_0\\
	Q_1 = \frac{\gamma}{(\gamma - 1)\text{Re}_\infty \text{Pr}_\infty} \nabla \cdot (\kappa \nabla T)_1 + (\rho q)_1
	\end{gather}
	
	Similar treatment can be done for the equations of state
	\begin{align}
	p_0 &= (\gamma -1)(\rho E)_0,\label{eq:EqSt_p0}\\
	p_1 &= (\gamma -1)(\rho E)_1,\\
	p_2 &= (\gamma -1)[(\rho E)_2 - \frac{1}{2}\rho_0 u_0^2]\\
	T_0 &= \frac{p_0}{\rho_0},\\
	T_1 &= \frac{p_1-\rho_1 T_0}{\rho_0},\\
	T_2 &= \frac{p_2 - \rho_1 T_1 - \rho_2 T_0}{\rho_0}
	\end{align}
	Using equations \eqref{eq:LeadMom}, \eqref{eq:LeadEn} and \eqref{eq:EqSt_p0} we see that the zeroth order pressure $p_0$ and the total energy density $(\rho E )_0$ only depend on the flow time $t$
	\begin{equation}
	p_0(t) = (\gamma - 1) (\rho E)_0 (t)
	\end{equation}
	
	\section{Low Mach number equations}
	If the first order continuity and energy equations, and the second order momentum equations are averaged over an acoustic wave period, the averaged velocity tensor describes the net acoustic effect on the averaged flow field. 
	It can be shown that starting from the equations derived using asymptotic analysis in the last section, a simplified set of equations can be obtained. Removal of the acoustic effects leads to the low Mach number equations.
	It is assumed that acoustic waves have a period $T_a$, i.e. $(\rho_1,(\rho \vec{u})_1,(\rho E)_1)(\mathbf{x}, t, \tau) = (\rho_1,(\rho \vec{u})_1,(\rho E)_1)(\mathbf{x}, t, \tau+ T_a)$
	
	For example, integrating the second order momentum equation \eqref{eq:secOrderMomEq} over a acostuic wave period $T_a$ we get
	
	\begin{equation}
	\frac{\partial \rho_0 \overline{\vec{u}}_0 }{\partial t}  + \nabla \cdot (\rho_0  \overline{\vec{u}_0\vec{u}_0}) + \nabla \overline{p}_2 = \overline{\mathbf{G}}_0
	\end{equation}
	
	The averaged velocity tensor $\overline{\vec{u}_0\vec{u}_0} = \frac{1}{T_a}\int_{0}^{T_a}\vec{u}_0(\vec{x},t,\tau)\vec{u}_0(\vec{x},t,\tau)\text{d}\tau$ 
	describes the net acoustic effect on the averaged flow field.
	
	If the averaged velocity tensor  $\overline{\vec{u}_0\vec{u}_0}$ is approximated by the tensor  $\overline{\vec{u}}_0\overline{\vec{u}}_0$, the net acoustic effect is removed and we obtain the momentum equation in the zero Mach number limit.
	\begin{equation}
	\frac{\partial \rho_0 \overline{\vec{u}}_0 }{\partial t}  + \nabla \cdot (\rho_0  \overline{\vec{u}}_0\overline{\vec{u}}_0) + \nabla \overline{p}_2 = \overline{\mathbf{G}}_0
	\end{equation}
	
	Using a similar procedure the averaged first order continuity equation and energy equation become the low mach continuity and momentum equations
	\begin{equation}
	\frac{\partial \rho_0 }{\partial t}  + \nabla \cdot (\rho_0  \overline{\vec{u}_0}) = 0
	\end{equation}
	\begin{equation}
	\frac{\gamma}{\gamma -1} \rho_0 \left[\frac{\partial T_0}{\partial t}   + \overline{\vec{u}}_0 \cdot\nabla T_0 \right] - \frac{\text{d}p_0}{\text{d}t} = \overline{Q}_0
	\end{equation} 
	The low-Mach number equations are valid for arbitrary temperature and density changes governed by the equation of state $p_0(t) = \rho_0(\vec{x},t)T_0(\vec{x},t)$
	
	Since $p_0$ serves as the mean pressure in the energy equations, it represents the global thermodynamic pressure part. %As the second order pressure $p_2$ is determined by the momentum equation, 
	The low mach number momentunm equation is coupled to the low mach number energy eqaution via the density $\rho_0$, viscosity $\mu(T_0)$ and equation of state $p_0 = \rho_0T_0$
	
	If only small temperature and density changes are allowed and if $p_0$ is asumed to be constant, the low mach number equations simplify to the bousinnesq equations. 
	
	The euler and NS equations can be obtained by neglecting the bouyancy forec and thereby the coupling between the momentum and energy equations












\section{bla}\label{AP:DerivationOfLowMach}


 For a comprehensive derivation of the low-Mach equations we refer to other works\cite{majdaDerivationNumericalSolution1985} \cite{rauwoensConservativeDiscreteCompatibilityconstraint2009} \cite{mullerLowMachNumberAsymptoticsNavierStokes1998} \cite{kleinNumericalModellingHigh2002}.  
\section{The low-Mach number equations for reactive flows}


\subsection{The steady non-dimensional low-Mach equations} \label{ssec:NonDimLowMachEquations}
In the present work the low-Mach number limit approximation of the governing equations is used. This approximation is often used for flows where the Mach number (defined as $\text{Ma} = \RefVal{u}/\hat c$, where $\RefVal{u}$ is a characteristic flow velocity and $\hat{c}$ the speed of sound) is small, which is usually the case in typical laminar combustion systems. \cite{dobbinsFullyImplicitCompact2010} For a comprehensive derivation of the low-Mach equations we refer to other works\cite{majdaDerivationNumericalSolution1985} \cite{rauwoensConservativeDiscreteCompatibilityconstraint2009} \cite{mullerLowMachNumberAsymptoticsNavierStokes1998} \cite{kleinNumericalModellingHigh2002}.  
One of the main results of the analysis is that for flows with a small Mach number the pressure can be decomposed as $\hat p(\hat {\vec{x}}, \hat t) = \hat p_0(\hat t) + \hat p_2(\hat{\vec{x}},\hat t)$. The spatially uniform  term $\hat p_0(\hat t)$ is called thermodynamic pressure, and only appears in the equation of state. For an open system it is constant and equal to the ambient pressure, while for a closed system (e.g. a system completely bounded by walls) it changes in order to ensure mass conservation, c.f. \cref{ss:DHC}. The perturbational term  $\hat p_2(\hat{\vec{x}},\hat t)$  plays a similar role as the pressure in the classical incompressible formulation, and only appears in the momentum equations. Effectively this approximation allows density variations due to temperature differences, and decouples it from the hydrodynamic pressure. From now on we will drop the sub-index of the hydrodynamic pressure $\hat p_2$ and we will refer to it simply as $\hat p$.

We use in this work a non-dimensional formulation of the governing equations. We define the non-dimensional quantities
%By substituting the expansions into the non-dimensionalized equations XX and the non-dimensionalized ideal gas equation XX, and collecting the lowest orders for $\epsilon$, the low-Mach number equations are obtained.
%note that the form of the low-Mach equations is very similar to the original governing equations. La mayor diferencia proviene de la descomposicion de la presion en dos partes. blabla.
\begin{align*}
&\rho = \frac{\hat \rho}{\RefVal{\rho}}, \quad 
p = \frac{\hat p}{\RefVal{p}}, \quad 
\vec{u}= \frac{\hat{\vec{u}}}{\RefVal{u}}, \quad 
T = \frac{\hat T}{\RefVal{T}},  \quad 
c_p = \frac{\hat c_p}{\RefValS{c}{p}}, \quad
M_\alpha = \frac{\hat{M}_\alpha}{\RefVal{M}}
\\
\mu = &\frac{\hat \mu}{\RefVal{\mu}},\quad
D_\alpha = \frac{\hat D_\alpha}{\RefValS{D}{\alpha}}, \quad
k = \frac{\hat k}{\RefVal{k}}\quad
\nabla = \frac{\hat \nabla}{\RefVal{L}}, \quad
t = \frac{\hat{t}}{\RefVal{t}},\quad 
\vec{g} = \frac{\hat{\vec{g}}}{\RefVal{g}},\quad
Q = \frac{\hat Q}{\hat{Q}_0}
\end{align*}
Here $\RefVal{u}, \RefVal{L}$, $\RefVal{p}$, $\RefVal{t}$ and $\RefVal{T}$ are the reference velocity, length, pressure, time and temperature, respectively, and are equal to some characteristic value for the particular studied configuration. Additionally, $\RefVal{g}$ is the gravitational acceleration and $\RefVal{M}$ is a reference molecular weight.  The reference transport properties $\RefVal{\mu}$, $\RefVal{k}$, $\RefValS{D}{\alpha}$ and the reference heat capacity of the mixture and $\RefValS{c}{p}$ are evaluated at the reference temperature $\RefVal{T}$. Similarly, the reference density has to satisfy the equation of state, thus $\RefVal{\rho} = \RefVal{p}/(\mathcal{R}\RefVal{T}\RefVal{M})$.  By introducing these definitions in the governing equations \cref{eq:GasLowMachConti,eq:GasLowMachMassBalance} the non-dimensional reactive low-Mach number set of equations are obtained. Since we are interested in the steady solution of the governing equations, all temporal derivatives vanish. Finally, the system of differential equations to be solved reads 
\begin{subequations}
	\begin{align}
	\nabla \cdot (\rho \vec{u})   = 0, \label{eq:LowMachConti}\\\nabla \cdot (\rho \vec{u} \otimes \vec{u})  & = - \nabla p + \frac{1}{\Rey}\nabla \cdot \mu\left( \nabla \vec{u} +\nabla \vec{u}^T  - \frac{2}{3}(\nabla\cdot \vec{u})\mytensor{I} \right)  - \frac{1}{\Fr^2}\rho\frac{\vec{g}}{\Vert \vec{g} \Vert}, \label{eq:LowMachMomentum}\\\nabla \cdot (\rho \vec{u} T) & = \frac{1}{\text{Re}~\text{Pr}}\nabla \cdot\left(\frac{k}{c_p} \nabla T\right)+  
	\text{H}~\Da~\frac{\heatRelease~\rateReac}{c_p}, \label{eq:LowMachEnergy}\\ 
	\nabla \cdot (\rho  \vec{u}Y_\alpha)  & = \frac{1}{\text{Re}~\text{Pr}~\text{Le}_\alpha}\nabla \cdot(\rho D \nabla Y_\alpha)+  \Da~\stoicCoef_\alpha M_\alpha \rateReac. \quad (\alpha = 1, \dots,~n_s - 1) \label{eq:LowMachMassBalance} 
	\end{align}
	\label{eq:all-eq}
\end{subequations}

The $n_s$-th component mass fraction $Y_{n_s}$ is calculated using \cref{eq:MassFractionConstraint}. This system is solved for the primitive variables velocity $\vec{u} = (u_x, u_y)$, pressure $p$, temperature $T$ and mass fractions ${\mathbf{Y} = (Y_1,\dots,Y_{n_s})}$.
Six non-dimensional factors arise from the non-dimensionalization process:
\begin{gather*}
\text{Re} = \frac{\RefVal{\rho} \RefVal{u}  \RefVal{L}}{\RefVal{\Viscosity}}, \quad
\text{Fr} = \frac{\RefVal{u}}{\sqrt{\RefVal{g}\RefVal{L}}}, \quad
\text{Pr} = \frac{\RefValS{c}{p} \RefVal{\Viscosity}}{\RefVal{k}}, \quad
\text{Ma} = \frac{\RefVal{u}}{\sqrt{\gamma \RefVal{T} \hat{\gls{GasConstant}} / \RefVal{W} }},\\
\text{Le}_\alpha = \frac{\RefVal{k}}{\RefVal{\rho} \RefValS{D}{\alpha} \RefValS{c}{p}}, \quad
\Da = \frac{\hat B \RefVal{L} \RefVal{\rho}}{\RefVal{M}\RefVal{u}}, \quad \label{eq:Dahmkoeler}
\text{H} = \frac{\hat \heatRelease_0}{\RefVal{M} \RefValS{c}{p} \RefVal{T}}
\end{gather*} 
the first three are the Reynolds, Froude and Prandtl number respectively. $\text{Le}_\alpha$ is the Lewis number of species $\alpha$. Finally $\text{Da}$ and H are the Damköhler number and the non-dimensional heat release respectively. The non-dimensional reaction rate is 
\begin{align}
\rateReac(T, \vec{Y})  = \left(\frac{\rho Y_F}{M_F}\right) \left(\frac{\rho Y_O}{M_O}\right)\text{exp}\left(\frac{-T_a}{T}\right), \label{eq:NonDimArr}
\end{align}
where $T_a = \hat{T}_a / \RefVal{T}$. Furthermore, the non-dimensional heat release is
\begin{equation}
\heatRelease(\phi)=
\begin{cases}
1 &\text{if}~ \phi \leq 1\\
(1 - \alpha(\phi -1))&\text{if}~\phi > 1,
\end{cases}  \label{eq:heatReleaseOneStepNonDim}     
\end{equation}
with $\phi$ evaluated according to \cref{eq:equivalenceRatio}. In the low-Mach limit, the ideal gas equation depends on the thermodynamic pressure, temperature and mass fractions. It reads in its non-dimensional form 
\begin{align} \label{eq:ideal_gas}
\rho(p_0,T, \vec{Y}) = \frac{p_0}{T \SumOvAllns \frac{Y_\alpha}{M_\alpha}}.
\end{align}
Similarly, the non-dimensional specific heat capacity of the mixture $c_p$ is calculated as
\begin{equation}\label{eq:nondim_cpmixture}
c_p(T,\mathbf{Y}) = \SumOvAllns Y_\alpha c_{p,\alpha}(T),
\end{equation}
and the non-dimensional viscosity as
\begin{equation} \label{eq:nondim_sutherland}
\mu(T) =  T^{\frac{3}{2}}\frac{1+\hat{S} }{\RefVal{T}T+\hat{S}}.
\end{equation} 
As mentioned before, the model for the transport parameters can be simplified by assuming constant values for the Prandtl and Lewis numbers. \cite{smokeFormulationPremixedNonpremixed1991} and we can write $\mu(T) = k/c_p(T) = \rho D_\alpha(T)$.
In all calculations in this work the value of $\hat{S}$ for air is used, $\hat{S} = $ \SI{110.5}{\kelvin}.

\cleardoublepage
\phantomsection
\addcontentsline{toc}{chapter}{Curriculum vitae}
\chapter*{Curriculum vitae}
% Der Lebenslauf ist aus Datenschutzgründen in der Online-Version nicht enthalten.

% Comment this for online version
%\begin{center}
	{\Large Vor- und Nachname}
%\end{center}

\vspace{0.5cm}

\begin{tabbing}
\hspace{4.2cm}\=\hspace{4.2cm}\=\kill

{\bfseries Persönliche Daten} \\
\bigskip

Geburtsdatum:                \> -\\
Geburtsort:                  \> -\\
Staatsangehörigkeit:         \> -\\
{ }\\

{\bfseries Schulbildung}\\
\bigskip

Jahr - Jahr			\>	-\\

Jahr - Jahr			\>	-\\
\\

{\bfseries Studium}\\
\bigskip

Jahr - Jahr			\>	-\\
{ }\\\\

{\bfseries Wissenschaftliche}\\
{\bfseries Tätigkeit}\\
\bigskip

Jahr - Jahr			\>	Wissenschaftlicher Mitarbeiter am Fachgebiet für Strömungs-\\
					\>	dynamik im Fachbereich Maschinenbau der TU Darmstadt,\\
					\>	Promotion und Lehrtätigkeit\\													
\end{tabbing}


\end{document}

