\chapter{Conclusion}	\label{ch:conclusion}
%\glsresetall

In the present work, the discretization and implementation of a fully coupled implicit method for simulating steady-state diffusion flames using the DG method are shown. For this the governing equations in the low-Mach limit are used, which allows to account for expansion and compression effects for large variations in temperature, not restricting to models such as the Boussinesq approximation. The chemical model used corresponds to a one-step model with variable parameters that allows to capture fundamental characteristics of diffusion flames, but with the advantage that it requires much less computational power compared to complex chemical models.  Additionally, temperature-dependent expressions for the transport parameters according to Sutherland's law and variable heat capacities according to NASA polynomials are used. The presented formulation of the equations allows the simulation of open and closed systems. 

The discretization using DG-methods allows for a high-order formulation which offers a high accuracy with low computational costs. For DG-discretization, a mixed-order formulation is used for stability reasons, where the velocity, temperature, and mass fractions are represented by polynomials of degree $k$, and pressure by polynomials of degree $k-1$. The system obtained from the discretization is solved in a fully coupled manner by means of a Newton-Dogleg type method, which proved to be a very robust algorithm for the test cases presented, even for cases where an adequate initial estimate is not available. In addition, an efficient method for the calculation of the Jacobian matrix as part of Newton's algorithm is presented. Systems of linear equations are solved in two ways: Systems with up to approximately 500,000 \gls{DOFs} are solved using the direct solver \gls{PARDISO} and larger systems are solved using a multi-grid method. 

The algorithms presented here form a solid foundation for the solution using a fully coupled approach. The presented solver is used for the simulation of a wide range of test cases, which proves to be very efficient in the search for solutions and, most importantly, with little need of intervention from the user in terms of configuration.

The fully coupled method proved to be adequate and very time-efficient in finding solutions of all the test cases presented in this thesis. Tests were carried out to compare the fully implicit approach presented in this work with the SIMPLE-DG algorithm-based solver that is already present in the BoSSS framework. A clear and very large difference in the runtimes is appreciated, where the XNSEC solver presents computation times up to 20 times shorter than the SIMPLE-DG solver. It is important to note that this is by no means an indicator that the SIMPLE-DG method is in general less efficient in terms of computational time than the approach presented in this work, since the low performance of the SIMPLE-DG method could be explained by a poor choice of under-relaxation factors. However, the fully coupled approach presented in this work requires less user input, which also makes it more robust.

Additionally, as a convergence supporting strategy, an homotopy method is included in the structure of the non-linear solver, which allows for solving highly nonlinear systems in a fully automatic manner. This type of algorithm is useful to solve steady state systems where some parameter makes the solution of the system difficult, as is, for example, the simulation of the square heated cavity problem for high Rayleigh numbers. The homotopy algorithm shown here is shown to be a robust and automatic strategy that allows finding solutions to such problems without the need for user intervention.

For reactive test cases, the concept of the flame sheet estimate is demonstrated to be a useful and computationally inexpensive way to initialize steady-state calculation of combustion systems with finite reaction rate. Using this strategy avoids the need for ignition simulation, usually performed by means of timestepping or pseudo-timestepping techniques, which can be very time consuming.






%%%%%%%%%%%%%%%%%%%%%%%%%%%%%%%%%%%%%%%%%%
\section{Future work}

Although the algorithms presented in this paper were useful for a large number of cases, there are aspects that still need to be worked on. Some of them are mentioned below.

It is important to note that all the methods presented are capable of simulating three-dimensional flows. In the present work only two-dimensional systems are treated, basically for computational performance reasons, since the systems of equations to be solved in the three-dimensional case are too large for the linear-solvers that are part of the BoSSS-code. The development of iterative solvers that allow the solution of such problems is ongoing work in the BoSSS developing group, and the simulation of systems with three-dimensional combustion could be future work.

The governing equations treated in this work are based on some strong assumptions.
The diffusion model used is highly simplified, and it is expected that in certain systems with combustion, it may yield results with appreciable error, particularly in systems that are not significantly diluted. Implementing a more complex diffusion model, such as the Hirschfelder and Curtiss approximation, would be a simple and efficient way to solve this problem. 
Another important point is that the simulations dealt mainly with very dilute fuels. Simulation of pure fuel combustion would in theory require much finer meshing (which greatly increases the number of DOFs to be solved), or similarly, a more specialized refinement strategy. Future work could address these points, and the development of specialized linear solvers could help with computation times for systems with a large number of DOFs.

This work mainly dealt with steady-state combustion systems. The use of the flame sheet solution as an initial estimate proved to be an efficient way to find the burning solution. With this strategy, the need to simulate the flame initiation process is circumvented, since only the steady-state solution is of interest.  Simulation of the ignition process is an open topic that should be addressed in future work. 

Although the work showed that the fully coupled approach worked very well for a wide variety of problems, for more complex systems, such as non-stationary combustion processes, the computational times could be quite prohibitive. Further computational parallelization, in particular of linear-solvers, could speed up the calculations. 

In future work, the implemented solver is intended to be used in conjunction with a extended-DG solver \parencite{kummerExtendedDiscontinuousGalerkin2017,kummerBoSSSPackageMultigrid2021,krauseIncompressibleImmersedBoundary2017} in order to study multiphase reactive systems such as burning droplets. Some preliminary results have already been obtained and will be part of a future publication.

