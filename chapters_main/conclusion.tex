\chapter{Conclusion}	\label{ch:conclusion}
\glsresetall
% \section{Conclusions}\label{sec5}
In the present work we have shown the discretization using the DG method for the reactive low-Mach equations. A mixed order formulation has been used, where velocity, temperature and mass fractions are represented by polynomials of degree $k$, and pressure by polynomials of degree $k-1$. The system obtained from the discretization was solved by means of a Newton-Dogleg type method. Additionally, a homotopy strategy for solving highly nonlinear systems was introduced and used for some of the presented test cases . For the reactive test cases the concept of the flame sheet estimate was demonstrated to be a useful and computationally cheap way of initializing the finite reaction rate combustion calculations.
The solver was used to calculate three different benchmark configurations, one without combustion and two cases with combustion. The first of these is the differentially heated cavity. This benchmark case was solved for varying Rayleigh numbers, spanning from $\text{Ra} = 10^2$ to $\text{Ra} = 10^7$.  It was observed that for cases with Rayleigh number larger than $10^5$ the use of our homotopy strategy was necessary to ensure convergence. Velocity and temperature profiles as well as thermodynamic pressure and Nusselt number were compared with a reference solution, obtaining very satisfactory results, thus validating our implementation of the method for variable density systems. An $h$-convergence study was presented for the differentially heated cavity, obtaining the expected convergence rates up to $k = 4$, where some degeneration on the rates was observed. Furthermore the counterflow diffusion flame configuration was analyzed, with which we intended to test the implemented chemistry model in conjunction with our solver. The results obtained with \BoSSS were compared with results obtained by solving the self-similar one-dimensional representation of the counter diffusion flame configuration for varying strain rates. While for low strains some discrepancies between results were observed, the difference narrowed for high strains, which can be explained by the influence of the border effects on the centerline results for a two-dimensional configuration. Finally, the chambered diffusion flame configuration was used to study the convergence rates of our method. It was shown that, as expected, the convergence rates increase when using higher degree polynomials, but with some deterioration compared to the theoretical expected convergence. In future work the implemented solver is intended to be used in conjunction with our extended-DG solver \cite{kummerExtendedDiscontinuousGalerkin2017,kummerBoSSSPackageMultigrid2021,krauseIncompressibleImmersedBoundary2017}in order to study multiphase reactive systems such as droplets.

% Solver can find solutions for problems with a moderate temperature variation. For high variations the temporal discretization presented is not able to find solutions. This is a well known problem which requieres special attention and should be adressed in future work. 
%%%%%%%%%%%%%%%%%%%%%%%%%%%%%%%%%%%%%%%%%%
Future Work:
% En la conclusion puedo hablar de que cosas serian necesarias en futuro trabajo
The governing equations treated in this work were based on some strong assumptions. In particular, the one-step model chemical model (esto no es tan malo).
El modelo de difusion utilizado es altamente simplificado, y se espera que en ciertos sistemas con combustion pueda arrojar resultados con un error apreciable, en particular en sistemas que no se encuentren altamente diluidos. La implementacion de un modelo de difusion más complejo, como por ejemplo la Hirschfelder and Curtiss approximation sería una forma simple y eficiente de solucionar el problema. 
En este trabajo se trató principalmente con sistemas de combustion en estado estacionario. El uso de la flame sheet solution como estimado inicial probó ser una manera eficiente de encontrar la burning solution. Con esta estrategia, se circumvent la necesidad de simular el proceso de inición de la flama puesto  que solo se está interesado en el steady state solution.  La simulación del proceso de ignicion es un topico abierto que debe ser tratado en futuros trabajos. 
Si bien en el trabajo se demostró que el fully coupled approach funcionó muy bien para una gran variedad de problemas, para sistemas más complejos, como por ejemplo procesos de combustion no estacionarios, los tiempos de calculo podrían resultar prohibitivos. Una mayor paralelización computacional, en particular de los linear-solvers, podrían brindar aceleración a los calculos. Este es un topico de actual estudio en el departamento de mecanica de fluidos FDY, en donde distintos metodos para resolucion de sistemas de ecuaciones están siendo estudiados, y en particular con aplicaciones en la mecánica de fluidos. 

Las simulaciones con combustion ocupan combustible diluido. Simulaciones con combustion     