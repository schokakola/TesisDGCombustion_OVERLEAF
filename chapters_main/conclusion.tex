\chapter{Conclusion}	\label{ch:conclusion}
\glsresetall
% \section{Conclusions}\label{sec5}
In the present work we have shown the discretization using the DG method for the reactive low-Mach equations. A mixed order formulation has been used, where velocity, temperature and mass fractions are represented by polynomials of degree $k$, and pressure by polynomials of degree $k-1$. The system obtained from the discretization was solved by means of a Newton-Dogleg type method. Additionally, a homotopy strategy for solving highly nonlinear systems was introduced and used for some of the presented test cases . For the reactive test cases the concept of the flame sheet estimate was demonstrated to be a useful and computationally cheap way of initializing the finite reaction rate combustion calculations.
The solver was used to calculate three different benchmark configurations, one without combustion and two cases with combustion. The first of these is the differentially heated cavity. This benchmark case was solved for varying Rayleigh numbers, spanning from $\text{Ra} = 10^2$ to $\text{Ra} = 10^7$.  It was observed that for cases with Rayleigh number larger than $10^5$ the use of our homotopy strategy was necessary to ensure convergence. Velocity and temperature profiles as well as thermodynamic pressure and Nusselt number were compared with a reference solution, obtaining very satisfactory results, thus validating our implementation of the method for variable density systems. An $h$-convergence study was presented for the differentially heated cavity, obtaining the expected convergence rates up to $k = 4$, where some degeneration on the rates was observed. Furthermore the counterflow diffusion flame configuration was analyzed, with which we intended to test the implemented chemistry model in conjunction with our solver. The results obtained with \BoSSS were compared with results obtained by solving the self-similar one-dimensional representation of the counter diffusion flame configuration for varying strain rates. While for low strains some discrepancies between results were observed, the difference narrowed for high strains, which can be explained by the influence of the border effects on the centerline results for a two-dimensional configuration. Finally, the chambered diffusion flame configuration was used to study the convergence rates of our method. It was shown that, as expected, the convergence rates increase when using higher degree polynomials, but with some deterioration compared to the theoretical expected convergence. In future work the implemented solver is intended to be used in conjunction with our extended-DG solver \cite{kummerExtendedDiscontinuousGalerkin2017,kummerBoSSSPackageMultigrid2021,krauseIncompressibleImmersedBoundary2017}in order to study multiphase reactive systems such as droplets.
%%%%%%%%%%%%%%%%%%%%%%%%%%%%%%%%%%%%%%%%%%
% En la conclusion puedo hablar de que cosas serian necesarias en futuro trabajo
% The governing equations treated in this work were based on some rather strong assumptions. In particular, the one-step model chemical model 