\chapter{Conclusion}	\label{ch:conclusion}
%\glsresetall

In the present work the discretization and implementation of a fully coupled implicit method for simulating steady state diffusion flames using the DG-method is shown. For this the governing equations in the low-Mach limit are used, which allow to consider expansion and compression effects for large variations in temperature, not restricting to models such as the bousinesq approximation. The chemical model used corresponds to a one-step model with variable parameters that allows to capture fundamental characteristics of diffusion flames, but with the advantage that it requires much less computational power compared to complex chemical models.  Additionally, temperature dependent expressions for the transport parameters according to Sutherland's law, and variable heat capacities according to NASA-polynomials are used. The presented formulation of the equations allows the simulation of open and closed systems. 

The discretization using DG-methods allows for a high-order formulation which offers a high accuracy with low computational costs. For the DG-discretization a mixed order formulation is used for stability reasons, where velocity, temperature and mass fractions are represented by polynomials of degree $k$, and pressure by polynomials of degree $k-1$. The system obtained from the discretization is solved in a fully coupled manner by means of a Newton-Dogleg type method, which proved to be a very robust algorithm for the test cases presented, even for cases where a initial estimate was not available. Additionally, an efficient method for the calculation of the Jacobian matrix as part of the Newton's algorithm is presented. Systems of linear equations that must be solved as part of the newton algorithm are solved in two ways: Systems with up to approximately 500,000 \gls{DOFs} are solved using the direct solver \gls{PARDISO}. Larger systems are solved using a multigrid method. 

Additionally, as a convergence supporting strategy, an homotopy method is included in the structure of the non linear solver, which allows for solving highly nonlinear systems in a fully automatic manner. This type of algorithms was useful to solve steady state systems where some parameter makes difficult the solution of the system, as it is for example the simulation of the square heated cavity problem for high Rayleigh numbers. The homotopy algorithm shown here is demonstrated as a robust and automatic strategy that allows finding solutions to such problems without the need for user intervention.

For the reactive test cases the concept of the flame sheet estimate was demonstrated to be a useful and computationally cheap way of initializing steady state calculation of combustion systems with finite reaction rate. Using this strategy avoids the need for ignition simulation, which is usually performed by means of timestepping or pseudo-timestepping techniques.

Several benchmark configurations were solved with the present solver, and were shown in increasing level of complexity. First two classical incompressible benchmark cases were selected: the lid-driven cavity flow and the Backward-facing step. These cases were simulated and compared with benchmark solutions, obtaining very good agreement of the results. 

Subsequently, several testcases where the temperature plays a significant role were analysed. A heated backward-facing step was calculated and compared with benchmark results, obtaining again a very good result agreements. Later, a Couette flow configuration with a vertical temperature gradient was studied, which allowed to verify the implementation of the solver, since an analytical solution can be obtained for this problem. The analytical solution was used for determining the experimental order of convergence of the solver for single component non-isothermal systems, where the expected rates of the DG-method were observed. This test also served as a means to compare the fully implicit approach presented in this work with the SIMPLE algorithm based solver that is already present in the BoSSS framework. A clear and very large difference in the runtimes could be appreciated, where the XNSEC solver presents computation times up to 20 times shorter than the SIMPLE-DG solver. It is important to note that this is by no means an indicator that the SIMPLE-DG method is in general less efficient in terms of computational time than the approach presented in this work, since the low performance of the SIMPLE-DG method could be explained by a poor choice of under-relaxation factors. Nevertheless, the fully coupled approach presented in this work requires less user input, which makes it also more robust. Later the capability of the XNSEC solver for simulating buoyancy driven flows was tested by means of the heated square cavity problem. This testcase also served for proving the capability of solving flows in a closed system. A very thorough comparison with benchmark results was performed, obtaining very satisfactory results. The Newton-Dogleg method proved to be adequate for systems up to a Rayleigh number of $\text{Ra} = 10^5$. Larger Rayleigh number values required the use of the homotopy algorithm in order to find a converged solution. The convergence properties for the non-isothermal closed-system flow were also calculated, and the expected DG convergence rates were again obtained, only for $k=4$ a slight deterioration of the rates was observed.
Later a unsteady test case was shown, namely the flow over a heated circular cylinder. The unsteady behaviour of the solution obtained agrees very well with benchmark results, and the expected Kármán vortex street is observed. The behavior of the solver with respect to perturbations was then tested, in which the Rayleigh-Bénard convection problem was treated. The critical value of the Rayleigh number at which the system exhibits convective fluid motion was calculated with $0.009\%$ accuracy compared to theory based values. 
Finally the XNSEC solver was used for solving several classical diffusion flame configurations. First, a coflowing flame was simulated, which served to highlight the benefits of the strategy of using flame-sheet estimates, and also for showing the behaviour of the non-linear solver.
A throughout verification of the spatial discretization for the reactive case was done by means of the counterflow diffusion flame configuration. The results obtained using the XNSEC solver for this configuration at varying strain rates were compared with results obtained by using a in MATLAB implemented code for solving the equations for a quasi one-dimensional flame. Comparison of the results showed that for high strain rates the results agree very closely, while for low strain rates they differ slightly. This can be explained by the influence of the border effects on the centerline results for a two-dimensional configuration. Additionally the influence of different inlet boundary conditions types was studied, concluding that a plug flow is the most adequate for comparison with the one-dimensional equations. Finally a comparison of the maximum temperatures obtained for different strain rates showed discrepancies of up to $10\%$.
A pseudo one-dimensional flame configuration was used to study the convergence rates of the method for cases where combustion is present. And again the expected convergence rates where obtained, only observing a slight deterioration for higher polynomial orders $k$. Finally a unsteady test case was shown. It was observed that the temporal term of the continuity equation is a source of instability in cases with high temperature variations, and caused the algorithm to not converge. Nevertheless, simulations ignoring the term were performed, thus showing that the mesh refinement algorithm in a timestepping framework works as expected.

%%%%%%%%%%%%%%%%%%%%%%%%%%%%%%%%%%%%%%%%%%
\section{Future work}
% En la conclusion puedo hablar de que cosas serian necesarias en futuro trabajo
The governing equations treated in this work were based on some strong assumptions. In particular, the one-step model chemical model (esto no es tan malo).
El modelo de difusion utilizado es altamente simplificado, y se espera que en ciertos sistemas con combustion pueda arrojar resultados con un error apreciable, en particular en sistemas que no se encuentren altamente diluidos. La implementacion de un modelo de difusion más complejo, como por ejemplo la Hirschfelder and Curtiss approximation sería una forma simple y eficiente de solucionar el problema. 
En este trabajo se trató principalmente con sistemas de combustion en estado estacionario. El uso de la flame sheet solution como estimado inicial probó ser una manera eficiente de encontrar la burning solution. Con esta estrategia, se circumvent la necesidad de simular el proceso de inición de la flama puesto  que solo se está interesado en el steady state solution.  La simulación del proceso de ignicion es un topico abierto que debe ser tratado en futuros trabajos. 
Si bien en el trabajo se demostró que el fully coupled approach funcionó muy bien para una gran variedad de problemas, para sistemas más complejos, como por ejemplo procesos de combustion no estacionarios, los tiempos de calculo podrían resultar prohibitivos. Una mayor paralelización computacional, en particular de los linear-solvers, podrían brindar aceleración a los calculos. Este es un topico de actual estudio en el departamento de mecanica de fluidos FDY, en donde distintos metodos para resolucion de sistemas de ecuaciones están siendo estudiados, y en particular con aplicaciones en la mecánica de fluidos. 

Las simulaciones con combustion ocupan combustible diluido. Simulaciones con combustion     

It is important to mention that all the methods presented, as well as the implemented code, are capable of simulating three-dimensional flows. In the present work only two-dimensional systems were treated, basically for computational performance reasons, since the systems of equations to be solved in the three-dimensional case are too large for the linear-solvers that are part of the BoSSS-code. The development of iterative solvers that allow the solution of such problems is ongoing work in the BoSSS developing group, and the simulation of systems with three-dimensional combustion could be future work.

El fenomeno de ignicion no fue tratado en el presente trabajo y las soluciones para combustion usando el finite reaction rate solo fueron calculadas en estado estacionario. El uso los flame-sheet estimates permitió circumvent la simulación de la ignicion. En trabajo futuro este proceso de ignicion podria ser parte 
In future work the implemented solver is intended to be used in conjunction with our extended-DG solver \textcite{kummerExtendedDiscontinuousGalerkin2017,kummerBoSSSPackageMultigrid2021,krauseIncompressibleImmersedBoundary2017}in order to study multiphase reactive systems such as droplets.

