\chapter{The Discontinuous Galerkin method}	\label{ch:NumericalMethods}
This chapter aims to give an overview of the DG method, as well as to present the spatial and temporal discretization of the equations presented earlier.  Parts of this chapter are based on the works presented by \textcite{kummerExtendedDiscontinuousGalerkin2017,kikkerFullyCoupledHighorder, smudamartinDirectNumericalSimulation2021}.

\section{The Discontinuous Galerkin method}
The spatial and temporal discretization of a scalar transport equation using the DG method is demonstrated in this section. Initially, basic definitions are provided, followed by the presentation of the general procedure for obtaining a DG discretization. The works of \textcite{cockburnDevelopmentDiscontinuousGalerkin2000,hesthavenNodalDiscontinuousGalerkin2008,dipietroMathematicalAspectsDiscontinuous2012} are referred to for a more in-depth description of the DG method.
\subsection{Definitions for the discretization} \label{ssec:SpatDiscretization}
First some standard definitions and notation are introduced in the context of DG methods. 

Let $\Omega \subset \mathbb{R}^2$ be a computational domain with a polygonal and simply connected boundary $\partial \Omega$. The numerical grid is then formed by the set of non-overlapping elements $\gls{grid} = \{K_1, ..., K_J\}$ with a characteristic mesh size $h$, so that $\Omega$ is the union of all $J$ elements, i.e. $\Omega = \bigcup_{i=1}^J K_i$. 

Define $\Gamma = \bigcup_j \partial K_j$ as the union of all edges (internal edges and boundary edges) and $\Gamma_I = \Gamma \setminus \partial \Omega$ as the union of all interior edges.
For each edge of $\Gamma$ a normal field $\myvector{n}_{\Gamma}$ is defined. Particularly on $\partial \Omega$ the normal field is defined as an outer normal and $\vec{n}_\Gamma = \vec{n}_{\partial\Omega}$.
\begin{figure}[h]
	\begin{center}
		\def\svgwidth{0.5\textwidth}
		\import{./plots/}{Normals.pdf_tex}
		\caption{Schematic illustration of two cells. Normals of the cell $K_1$ are shown. }
		\label{fig:TwoCells}
	\end{center}
\end{figure}

For each field ${u} \in C^0\left(\Omega\setminus \Gamma_I\right)$, ${u}^-$  and ${u}^+$ is defined, which describe the values of the variables on the interior and exterior sides of the cell:
\begin{align}
	{u}^- & = \lim_{\xi \searrow 0} {u}\left(\myvector{x} - \xi \myvector{n}_{\Gamma}\right) \quad \text{for } \myvector{x}\in \Gamma   \\
	{u}^+ & = \lim_{\xi \searrow 0} {u}\left(\myvector{x} + \xi \myvector{n}_{\Gamma}\right) \quad \text{for } \myvector{x}\in \Gamma_I
\end{align}
The jump and mean values of ${u}$ on the inner edges $\Gamma_I$ are defined as
\begin{align}
	\llbracket {u} \rrbracket & = {u}^+-{u}^-,                           \\
	\{{u}\}                   & = \frac{1}{2} \left({u}^-+{u}^+\right).
\end{align}
while the jump and mean values on the boundary edges $\partial \Omega$ are given by
\begin{align}
	\llbracket {u} \rrbracket & = {u}^-,  \\
	\{{u}\}                   & = {u}^-.
\end{align}
Furthermore, the broken polynomial space of a total degree $k$ is defined as
\begin{equation}
	\mathbb{P}_k(\gls{grid} )= \{f \in L^2\left(\Omega\right); \forall K \in \gls{grid} : f\vert_{K} \text{ is polynomial and deg}\left(f\vert_{K}\right)\leq k \}.
	\label{Eq:PolSpace}
\end{equation}
Additionally, for $u \in \mathcal{C}^1(\Omega \setminus \Gamma)$ the broken gradient $\nabla_h u$ is defined as:
\begin{equation}
	\nabla_h u
	= \begin{cases}
		0
		 & \text{on }\Gamma  \\
		\nabla u
		 & \text{elsewhere }
	\end{cases}
\end{equation}
The broken divergence $\nabla_h \cdot u$ is defined analogously. Furthermore, the function space for test and trial functions for $D_v$ dependent variables is defined as
\begin{equation}
	\mathbb{V}_\myvector{k} = \prod_{i=1}^{D_v} \mathbb{P}_{k_i}(\gls{grid})
	\label{Eq:Vspace}
\end{equation}
where $\myvector{k} = \left(k_1,...,k_{D_v}\right)$ is the degree vector. 
Additionally, for a cell $K$  a local inner product and a local $L^2$-norm is defined for $u_K$, $v_K\in \mathbb{V}_\myvector{k}$ as
\begin{equation}
(u_K, v_K)_K \coloneqq \int_K u_K v_K\text{d}x, \qquad \norm{u_K}^2_K \coloneqq(u_K,u_K)_K
\end{equation}
Similarly, for $u_h$, $v_h \in \mathbb{V}_\myvector{k}$ a global inner product and global broken norm are defined as
\begin{equation}
	(u_h, v_h)_{\Omega_h} \coloneqq \sum_{i = 1}^N (u_h, v_h)_K, \qquad \norm{u_h}^2_{\Omega_h} \coloneqq(u_h,u_h)_{\Omega_h}
\end{equation}
\subsection{Discretization using the DG Method} \label{sec:DiscWithDG}
In this subsection the DG discretization of a simple problem will be shown in order to demonstrate the method and some of its specific characteristics. For this purpose, the discretization of a general conservation law for a scalar quantity $u = u(\vec{x},t)$ governed by a nonlinear flux function $\vec{f}(u)$ will be considered. In addition, suitable Dirichlet boundary conditions on $\partial \Omega = \partial \Omega_D$ and initial conditions $u_0$ are defined. The problem reads
\begin{subequations}
\begin{align}
&\pfrac{u}{t} + \nabla \cdot \vec{f}(u) = 0, \qquad\qquad\qquad &\vec{x} \in \Omega,\label{eq:consEqDG}\\
&u = u_D, \qquad \qquad \qquad  &\vec{x} \in \partial \Omega_D,\\
&u(\vec{x},0) = u_0(\vec{x}), \qquad\qquad\qquad &\vec{x}\in\Omega.
\end{align}\label{eqs:DGTransportExample}
\end{subequations}
The DG method allows finding an approximate solution $u_h = u_h(\vec{x},t)$ for the problem defined by \cref{eqs:DGTransportExample} by forming a linear combination of polynomial functions in each cell.  The discretization procedure starts by the approximation of the domain $\gls{domain}$ with a numerical grid $\gls{grid}$. In each cell $K_j$ of the numerical grid a set of polynomial basis $\vectr{\phi}_j = ({\phi}_{j,l})_{l=1,\dots,N_k} \in \mathbb{P}_k(\mathcal{K}_h)$ with a local cell support $\text{supp}(\phi_j) = \overline{\gls{cell}}_j$ is defined. This allows to represent the local solution for each cell $K_j$ as
\begin{equation}
u_j(\vec{x},t) = \sum_{l=1}^{N_k}\tilde{u}_{j,l}(t)\phi_{j,l}(\vec{x}) = \tilde{\vec{u}}_j(t) \cdot \vectr{\phi}_j (\vec{x})\label{eq:DGAnsatz}
\end{equation}
The coefficients $\tilde{\vectr{u}}_j = (\tilde{u}_{j,l})_{l=1,\dots,N_k}$ are the \gls{DOF} of the local solution in the cell $K_j$, which are the unknowns of the problem. Note the time dependence of the coefficients $\tilde{\vec{u}}_j$, as well as the spatial dependence of the basis functions $\vectr{\phi}_j$ on the vector $\vec{x}$. 

This approximate solution sought  is the best approximation of $\gls{DGVar} \in L^2(\gls{domain})$, which gives a minimum global error in the approximation space $\gls{DGVar} \in \mathbb{P}_k(\gls{domain} )$. 
\begin{align}
	\int_{\domain} (\underbrace{ \nuM{\gls{DGVar}}(x)-\gls{DGVar}(x) }_{=: \gls{dgerror}(x)})^2 \dV
	= ||\nuM{\gls{DGVar}} - \gls{DGVar}||_2^2 \rightarrow \text{min}
\end{align}
Here $\gls{dgerror}$ is the error of the discretization. Minimization is equivalent to the requirement 
\begin{equation}
	\label{eq:L2projection}
	\scp{r(x)}{\basis{}_m} = \scp{\gls{DGVar}_h-\gls{DGVar}}{\basis{}_m} \stackrel{!}{=} 0 \qquad \forall\basis{}_m.
\end{equation}
This means that the error is orthogonal to every polynomial function $ \basis{}_m$ in the approximation space. This requirement, together with the ansatz \cref{eq:DGAnsatz} define the projection operator $\pi_p$ as
\begin{equation}
	L^2(\domain) \ni \gls{DGVar} \mapsto \pi_p(\gls{DGVar}) = \gls{DGVar}_h \in  \poly_p( \NumericalGrid )
	\label{eq:ProjOperatorMapping}
\end{equation}
One of the major motivations for the use of high order methods stems from the so called Bramble-Hilbert lemma \parencite{brambleEstimationLinearFunctionals1970}, which states that for a $p$-times differentiable variable $\gls{DGVar}$, %the error $r$ is of the order $\mathcal{O}(h^{p+1})$, where $h$ is a characteristic cell length.
\begin{equation}
	\LTwoNorm{\gls{DGVar}(x) - \pi_p(\gls{DGVar}(x))} \leq C \cdot h^{p+1}
\end{equation}
where $h$ is a characteristic cell length and $C$ is a constant that depends on $\gls{DGVar}$ but not on $h$. This means, that for a sufficiently smooth $\gls{DGVar}$, the approximation error is of the order $\mathcal{O}(h^{p+1})$. Note that he differentiability assumption is essential: for non smooth $\gls{DGVar}$ the well known Gibbs phenomenon occurs.
%There are two general approaches used for the representation of the solution using basis functions: modal and nodal. Each one of them present some advantages and disadvantages \parencite{hesthavenNodalDiscontinuousGalerkin2008}. 
In this work a modal polynomial representation is used, using in particular Legendre polynomials. The basis functions are chosen such that they are orthogonal to each other 
\begin{equation}
	\int_{K_j} \phi_{j,m}\phi_{j,n} \text{d}V= \delta_{mn}
\end{equation}
where $\delta_{mn}$ is the Kronecker delta. This property implies that the mass matrix (to be defined later) equals the identity matrix, at least for constant density systems. This is advantageous particularly for incompressible flows, where the calculation of the mass matrix is dramatically simplified using this property. 
\subsubsection{The local formulation}
By inserting the approximate solution defined by \cref{eq:DGAnsatz} in the conservation \cref{eq:consEqDG}, a local residual $R_j$ can be defined
\begin{equation}\label{eq:DGResidualeq}
R_j(\vec{x},t) = \pfrac{u_j}{t} + \nabla \cdot \vec{f}(u_j), \quad \vec{x} \in K_j
\end{equation}
Minimization of this local residual is done by multiplying \cref{eq:DGResidualeq} by the so called test functions $\gls{testF}_{j,l}$. In the Galerkin approach, these test functions are required to be from the same space as the trial functions, i.e. $\gls{testF}_{j,l} = \gls{basis}_{j,l}$. Thus, by multiplying \cref{eq:DGResidualeq} by a trial function and integrating over the cell $K_j$, one obtains
\begin{equation}
	\int_{\gls{cell}_j} \gls{DGres}_j \gls{basis}_{j,l} \d{V} = \int_{\gls{cell}_j} \underbrace{ \pDeriv{{\gls{DGVar}}_j}{t} \gls{basis}_{j,l}}_{\text{Temporal}} + \underbrace{\div{\gls{flux}({\gls{DGVar}}_j)} \gls{basis}_{j,l} \d{V}}_{\text{Spatial}} \stackrel{!}{=} 0, \quad \forall \vectr{\gls{basis}}_{j,l}.
	\label{eq:DGminimization}
\end{equation}
Where the minimization comes from requiring the equality to zero. Note that until this point only a cell-local discretization has been addressed. The next step for obtaining a global DG formulation is to use integration by parts for rewriting the spatial term in \cref{eq:DGminimization}. This is done to make the boundary edge integrals explicitly appear in the formulation, which are used to couple cell $K_j$ with neighbouring cells. The partial integration process results in
 \begin{equation}
	\int_{\gls{cell}_j}  \pDeriv{{\gls{DGVar}}_j}{t} \gls{basis}_{j,l} \d{V} + \oint_{\partial{\gls{cell}}_j} \left( \gls{flux}({\gls{DGVar}}_j)\cdot{\gls{normal}}_j \right) \gls{basis}_{j,l} \d{S} - \int_{\gls{cell}_j} \gls{flux}({\gls{DGVar}}_j) \cdot \gradH{\gls{basis}}_{j,l} \d{V}  = 0, \quad \forall \gls{basis}_{j,l},
	\label{eq:DGfluxFormulation}
\end{equation} 
Note that inserting the ansatz \cref{eq:DGAnsatz} into \cref{eq:DGfluxFormulation} is problematic, since $\partial{\gls{cell}}_j$ is shared by other cells, and in the DG method, continuity of a variable is not enforced across cell boundaries. This means that in general the inner value $u^{-}_j$ and the outer value $u^{+}_j$ are not equal. This problem is solved by introducing the concept of a numerical flux function, denoted here with $\gls{numflux}$, which follows 
\begin{equation}
	\gls{numflux}(\inn{{\gls{DGVar}}_j}, \out{{\gls{DGVar}}_j}, \gls{normalGam}) \approx \gls{flux}({\gls{DGVar}}_j) \cdot {\gls{normal}}_j.
\end{equation}
This expression defines an unique value for the flux of a given cell boundary, enforcing flux continuity. The numerical flux $\gls{numflux}$ couples the \glspl{DOF} of neighbouring cells, and should satisfy certain mathematical and physical properties which will be discussed later. Many different numerical fluxes have been developed, and it is an active area of investigation. They differ mainly in computational cost, stability and dissipation of the scheme.

Finally by introducing the numerical flux in \cref{eq:DGfluxFormulation} the problem now reads
\begin{equation}
	\int_{\gls{cell}_j}  \pDeriv{{\gls{DGVar}}_j}{t} \gls{basis}_{j,l} \d{V} + \oint_{\partial{\gls{cell}}_j} \left( \gls{numflux}(\inn{{\gls{DGVar}}_j}, \out{{\gls{DGVar}}_j}, \gls{normalGam})    \right) \gls{basis}_{j,l} \d{S} - \int_{\gls{cell}_j} \gls{flux}({\gls{DGVar}}_j) \cdot \gradH{\gls{basis}}_{j,l} \d{V}  = 0, \quad \forall \gls{basis}_{j,l},
	\label{eq:DGNumericalfluxFormulation}
\end{equation} 
\subsubsection{The global formulation}
Note that \cref{eq:DGNumericalfluxFormulation} is still a local formulation. A global solution $u(\vec{x},t)$ can be defined by a piecewise polynomial approximation according to 
\begin{equation}
	u(\vectr{x},t) \approx  u_h(\vectr{x},t) = \bigoplus\limits_{j=1}^{J} {\gls{DGVar}}_j(\vectr{x},t) = \sum_{j=1}^{J} \sum_{l=1}^{\gls{NoDOFloc}} \tilde{\gls{DGVar}}_{j,l}(t) \gls{basis}_{j,l}(\vectr{x}) \in \gls{brknPspacek}(\gls{grid})
	\label{eq:globalApprox}
\end{equation}
which corresponds to the direct sum of the $J$ local solutions $u_j$. A vector $\tilde{\vec{u}} = \tilde{\gls{DGVar}}_{1,1},\allowbreak \tilde{\gls{DGVar}}_{1,2},\allowbreak\dots, \allowbreak \tilde{\gls{DGVar}}_{j,l},\allowbreak\dots,\allowbreak\tilde{\gls{DGVar}}_{J,\gls{NoDOFloc}}\allowbreak$ which comprises all the DOFs of the global approximation $\gls{DGVar}_h$ is defined, and is of length $\gls{NoDOF} = J\cdot\gls{NoDOFloc}$.%

Finally, the global formulation is obtained by inserting the ansatz (\cref{eq:DGAnsatz}) into \cref{eq:DGNumericalfluxFormulation}, summing over all cells $K_j$ and making use of \cref{eq:globalApprox}. The problem reads: Find $\nuM{\gls{DGVar}} \in 	\mathbb{P}_k(\gls{grid})$, such that $\forall \phi \in 	\mathbb{P}_k(\gls{grid})$
\begin{equation}
	\int_{\gls{domain}}  \pDeriv{\nuM{\gls{DGVar}}}{t} \gls{basis} \d{V}  + \oint_{\gls{edge}} \gls{numflux}(\inn{\nuM{\gls{DGVar}}}, \out{\nuM{\gls{DGVar}}}, \gls{normalGam}) \jump{\gls{basis}} \d{S} - \int_{\gls{domain}} \gls{flux}(\nuM{\gls{DGVar}}) \cdot \gradH{\gls{basis}} \d{V} = 0,
	\label{eq:semiDiscWeakForm}
\end{equation}
The solution of this system requires finding the DOFs $\tilde{\vec{u}}$ of the global approximation $\gls{DGVar}_h$. Dirichlet boundary conditions are included in the formulation by defining at $\gls{edge}_D$ the outer value $\out{\gls{DGVar}}_h = \gls{DGVar}_D$.

Note that \cref{eq:semiDiscWeakForm} is semi-discrete, meaning that the system of equations has been discretized in space, but not in time. Time discretization will be treated in \cref{ssec:TemporalDiscretization}.

Finally, after selecting suitable numerical fluxes for the various terms of the governing equations, a system is obtained that in general has the form
%\begin{equation}
%	 \deriv{ }{t} \left( {\gls{massM}}(\tilde{\vectr{\gls{DGVar}}})\tilde{\vectr{\gls{DGVar}}} \right) + \gls{OpM}(\tilde{\vectr{\gls{DGVar}}}) = \vectr{b},
%	\label{eq:discretMatrixForm}
%\end{equation}
\begin{equation}
	 \gls{massM}\deriv{ \tilde{\vectr{\gls{DGVar}}}}{t}+ \gls{OpM}(\tilde{\vectr{\gls{DGVar}}}) = \vectr{b},
	\label{eq:discretMatrixForm}
\end{equation}
Where $\gls{massM}$ is the mass matrix, and $\gls{OpM}$ is the operator matrix.  The vector $\vectr{b}$ contains the Dirichlet boundary condition. The operator matrix is defined locally by
\begin{equation}
	(\gls{OpM}_j)_{m,n} =  \oint_{\partial{\gls{cell}}_j} \gls{numflux}(\tilde{{\gls{DGVar}}}_{j,n}, \tilde{{\gls{DGVar}}}_{j^{\ast},n},\gls{normalI}) \gls{basis}_{j,m} \d{S} - \int_{\gls{cell}_j} \gls{flux}(\tilde{{\gls{DGVar}}}_{j,n} {\gls{basis}}_{j,n}) \cdot \gradH{\gls{basis}}_{j,m} \d{V} ,
	\label{eq:genOpMatrix}
\end{equation}
with $j^{\ast}$ denoting the index of a neighbour cell. The matrix $\gls{OpM}$ has block-diagonal structure, but also including extra diagonals which relate the DOFs of the cell with the neighbouring cells.

The global mass matrix is
\begin{equation}
	{\gls{massM}} =
	\left[ 
	\begin{array}{cccc}
		{\gls{massM}}_{1} & 0 & \cdots & 0 \\
		0 & {\gls{massM}}_{2} & \cdots & 0 \\
		\vdots & \vdots & \ddots & \vdots \\
		0 & 0 & \cdots & {\gls{massM}}_{J}
	\end{array}
	\right],
\end{equation} 
which is block-diagonal, since  ${\gls{massM}}_{j}$ does not depend on neighbouring cells. 
\begin{equation}
	({\gls{massM}}_j)_{m,n} = \int_{\gls{cell}_j} \gls{basis}_{j,m} \gls{basis}_{j,n} \d{V}	\label{eq:massMatrix}
\end{equation} 
The mass matrix of a cell ${\gls{massM}}_{j} := \matrixDG{M}_{(j,-) \ (j,-)} $ only depends in this case on the cell-local basis functions. Note also that for an orthonormal basis, $\gls{massM} = \mathbf{1}$, i.e. the identity matrix.

It is interesting to note regarding the DG discretization, that a \gls{FVM}-type  discretization can be recovered if the basis functions in this formulation are restricted to zero-degree polynomials. Similarly, a discretization similar to a \gls{FEM} formulation would be obtained if the allowance of discontinuities in the formulation of the DG method would have been ignored.
\subsubsection{Note on the numerical fluxes}
As mentioned before, the numerical fluxes $\gls{numflux}$ have to fulfil certain physical and mathematical properties for obtaining a stable and convergent method. A proof for the stability of the numerical flux is given in \textcite{dipietroMathematicalAspectsDiscontinuous2012}. One of the requirement for proving the stability is the Lipschitz continuity, meaning 
\begin{equation}
	\label{eq:flux_lipschitz_first}
	\exists C_a \in \mathbb{R}: \abs{\gls{numflux}(a_1, b, \vec{n}) - \hat{f}(a_2, b, \vec{n})} \leq C_a \abs{a_1 - a_2} 
	\qquad \forall a_1, a_2 \in \mathbb{R}
\end{equation}
and
\begin{equation}
	\label{eq:flux_lipschitz_second}
	\exists C_b \in \mathbb{R}: \abs{\gls{numflux}(a, b_1, \vec{n}) - \hat{f}(a, b_2, \vec{n})} \leq C_b \abs{b_1 - b_2} 
	\qquad \forall b_1, b_2 \in \mathbb{R}
\end{equation}
Additionally, the proof of the stability requires is the monotonicity of the flux:

\begin{equation}
	\label{eq:flux_monotonicity}
	\del{\hat{f}(a, b, \vec{n})}{a} \geq 0 \quad\land\quad \del{\hat{f}(a, b, \vec{n})}{b} \leq 0 \qquad \forall a, b, \vec{n}
\end{equation}

Two more requirements are needed for the numerical flux. The first is its consistency, which can be written as
\begin{equation}
	\gls{numflux}(a,a,\gls{normal}) = \gls{flux}(a) \cdot \gls{normal}, \quad \forall a \in \mathbb{R}. 
	\label{eq:consistency}
\end{equation}
imposing that a numerical flux function should deliver the same approximate solution as the original flux function in case of a continuous variable across the interface. A direct consequence of the consistency of the numerical flux is that the weak formulation \cref{eq:semiDiscWeakForm} is automatically fulfilled by $\nuM{\gls{DGVar}} = \gls{DGVar}$.

Finally the last requirement is that the numerical flux should be conservative, which means that the total amount of $\gls{DGVar}$ can only change due to fluxes across the domain boundary. This can be written as
\begin{equation}
	\gls{numflux}(a,b,\gls{normal}) = 	-\gls{numflux}(b,a,-\gls{normal}), \quad \forall a,b \in \mathbb{R}. 
	\label{eq:conservativity}
\end{equation}
All numerical fluxes used in this work for the spatial discretization of \cref{eq:all-eqs} fulfil these requirements.% and they will be shown in the next section.
\subsection{Temporal discretization}\label{ssec:TemporalDiscretization}
This section gives a brief introduction to the most used time-stepping techniques and then show the time-stepping method used in this work. It is mainly based on \textcite{levequeFiniteVolumeMethods2002,ferzigerComputationalMethodsFluid2002}.
%\subsubsection{}

In the previous section the spatial discretization of a transport equation using a DG method was shown, resulting in the semi-discrete formulation given by \cref{eq:semiDiscWeakForm}.
The time discretization of this semi-discrete system leads to the so called method of lines, which is the name for a method first discretized in space, and later in time.  

An alternative to the method of lines is the so called Rothe's method, where time is first discretized then the space. This can be advantageous in some cases, such as problems with a moving domain. Another alternative is the space-time approach, where basically the temporal coordinate is treated as another spatial dimension. Again, the method can be very attractive for some cases. However, the discretized schemes lead often to prohibitively large systems. These approaches are ignored in the present work, and the method of lines is adopted.

First the time discretization for a system with a constant mass matrix will be discussed. The process of discretization results often in a system of \gls{ODE} of the form
\begin{equation}
\gls{massM}\frac{\text{d}\vec{u}}{\text{d}t} = - \vec{F}(t,\vec{u}(t))\qquad \text{for}\qquad t \in (0,T).\label{eq:timestep1}
\end{equation}
with a initial condition given by $\vec{u}(t=0) = \vec{u}^0$. The main idea of a time-stepping algorithm is to discretize the time coordinate, and advance gradually the solution $\vec{u}(t^n)$ in time using the information  at previous time levels $\vec{u}(t^{n-1})$, $\vec{u}(t^{n-2})$, $\dots$,  until a certain final time $t = T$ is reached.
By integrating \cref{eq:timestep1} in time, one obtains
\begin{equation}
\gls{massM}(\vec{u}(t^{n+1})-\vec{u}(t^{n})) = - \int_{t^n}^{t^{n+1}}\vec{F}(t,\vec{u}(t)) \text{d}t \label{eq:timestep2}
\end{equation}
This equation is the starting point for different class of time stepping techniques. Two kind of methods can be distinguished, depending on how the integral in \cref{eq:timestep2} is evaluated: explicit methods and implicit methods. Explicit methods are obtained when the approximation of the integral is done only by using information from old time steps, while for implicit methods the information from the actual timestep is also considered, necessitating to solve a system of equations.

The simplest example of an explicit time-stepping method is the Explicit Euler Method:
\begin{equation}
\gls{massM}\vec{u}(t^{n+1})= \gls{massM}\vec{u}(t^{n})  - \Delta t \vec{F}(t^n,\vec{u}(t^n)),
\end{equation}
which is first order accurate in time. Other explicit methods exists with better properties than the Explicit Euler Method, typically using information from multiple known time levels or a interpolation of them. Adams-Bashforth methods are an example of them. 

Due to the local nature of the approach, explicit methods present themselves specially attractive for DG methods, particularly for hyperbolic equations. Explicit methods are relatively easy to implement, and need considerably less storage compared to implicit methods. However, explicit methods experience the disadvantage that the stability of the algorithm is heavily limited by a maximal timestep size $\Delta t$. The timestep typically scales with the grid size $h$ and polynomial degree $k$ by $\Delta \sim h/k$ for hyperbolic and $\Delta \sim (h/k)^2$ for parabolic problems \parencite{gassnerContributionConstructionDiffusion2007}. For many problems of interest, particularly stiff systems, this limitation is highly restrictive, since very little timesteps need to be chosen in order to obtain a stable method. 

Implicit methods  on the other hand are specially well suited for stiff problems, as they don't suffer from the restrictive timestep limitation of explicit methods, even not being restricted at all under certain conditions. This allows using considerably bigger timesteps, potentially reducing drastically calculation times. Implicit methods present however the inconvenience that they require the solution of a system of equations, which for large problems is not a trivial task and specialized methods are needed. The use of implicit methods is justified, particularly for stiff problems, as the extra computational overhead originating from solving of the system of equations is usually smaller than the time it would take to solve the same problem using explicit schemes with very small timesteps.

The simplest implicit method is the Implicit Euler Method,
\begin{equation}
\gls{massM}\vec{u}(t^{n+1}) + \Delta t\vec{F}(t^{n+1},\vec{u}(t^{n+1}))= \gls{massM}\vec{u}(t^{n}).
\end{equation}
Note that this is a nonlinear system of algebraic equations that has to be solved for $\vec{u}(t^{n+1})$. The Implicit Euler Methods is the first method of a family of \gls{BDF} methods, and under some conditions presents the property of being unconditionally stable, meaning that the algorithm allows an arbitrarily large timestep. This property allows the calculation of steady state solutions just by choosing a very large $\Delta t$ value. However, not all BDF schemes are unconditionally stable, as it will be shown next.
\subsubsection{Backward Differentiation Formula}
\begin{table}[h]
	\centering
	\begin{tabular}{lllllll}
		\hline
		$s$                   & $\gamma$ & $\beta_0$ & $\beta_1$ & $\beta_2$ & $\beta_3$ & $\beta_4$ \\ \hline
		Implicit Euler (BDF1) & 1        & 1         & -1        &           &           &           \\
		BDF2                  & 2        & 3         & -4        & 1         &           &           \\
		BDF3                  & 6        & 11        & -18       & 9         & -2        &           \\
		BDF4                  & 12       & 25        & -48       & 36        & -16       & 3         \\ \hline
	\end{tabular}
	\caption{Coefficients of the BDF schemes.}
	\label{tab:BDFCoeff}
\end{table}

In the present work BDF methods are used.  The main characteristic of this family of methods is that they are linear multistep methods, which means that they use information available from previous timesteps, increasing the accuracy of the scheme. In case of a non-constant mass matrix $\gls{massM}$, they have a general formula given by
\begin{equation}
	\frac{\beta_0}{\gamma\Delta t}\gls{massM}\bigl(\vec{u}(t^n)\bigr)\vec{u}(t^n)- \vec{F}\bigl(\vec{u}(t^n)\bigr) = - \sum_{i=1}^s \frac{\beta_i}{\gamma \Delta t}\gls{massM}\bigl(\vec{u}(t^{n-i})\bigr)\vec{u}(t^{n-i}).\label{eq:BDFDiscretization}
\end{equation}
where $s$ is the order of the BDF-scheme.  The coefficients of each schema are shown in \cref{tab:BDFCoeff}. The main advantage of BDF methods is their large stability regions, which make them suitable for solving stiff problems.  In \cref{Fig:AStability} the stability regions for the first four BDF schemes are shown. It is possible to observe that only the BDF-1 (implicit Euler) and BDF-2 schemes exhibit the property that they are A-stable, which means that the stability region contains the entire left complex plane \parencite{dahlquistSpecialStabilityProblem1963}. On the other hand, BDF schemes of order $s > 2$ are not A-stable. In this work however, BDF schemes up to order three have been used, as the unstable eigenvalues for $s=3$ are comparatively small \parencite{smudamartinDirectNumericalSimulation2021}.


\begin{figure}
	\centering
	\pgfplotsset{width=0.5\textwidth}
	\inputtikz{BDF_Stability}
	\caption[{A-stability regions of the BDF schemes for different order $s$. The areas shown in grey are unstable regions.} ]{A-stability regions of the BDF schemes for different $s$. The areas shown in grey are unstable regions. Figure taken and adapted from \parencite{kikkerHighOrderEXtendedDiscontinuous2020}} 
	%	\caption{A-stability regions of the \gls{BDF}\gls{l} schemes for $1\leq \gls{l}\leq 4$. The inner regions of the circles are unstable, the outer regions are stable. It can be seen that the schemes up to the \gls{BDF}2 are always stable for all $\mathfrak{R}(\gls{lambda} \Delta \gls{t})<0$. For the \gls{BDF}4 the opening angle of the A($\gls{open}$)-stability region is marked.}
	\label{Fig:AStability}
\end{figure}





%%%%%%%%%%%%%%%%
%. It is well known that the inclusion of the $\partial\rho /\partial t$ term of the continuity equation in the source term, as done in the present work, is a source of numerical instability. In the work of \textcite{nicoudNumericalStudyChannel} it is reported that obtaining solutions for density ratios greater than three  becomes difficult.In \textcite{rauwoensConservativeDiscreteCompatibilityconstraint2009} a similar destabilization effect is also reported for high density ratio
\subsubsection{Temporal discretization of the low-Mach equations}
Performing the same procedure as explained in the previous section for the low-Mach equations (\cref{eq:LowMach_Conti,eq:LowMach_Momentum,eq:LowMachEnergy,eq:LowMachMassBalance}), a semi-discrete formulation of the form
\begin{equation}
	\frac{\partial}{\partial t} \left(
	\underbrace{	\left[ 
		\begin{array}{cccc}
			\rho(T,\vec{Y})&0&0&0\\
			0&0&0&0\\
			0&0&	\rho(T,\vec{Y})&0\\
			0&0&0&	\rho(T,\vec{Y})
		\end{array}
		\right]}_{\gls{massM}(\vectr{U})}
	\left[
	\begin{array}{cccc}
		\vec{u}\\
		p\\
		T\\
		\vec{Y}'
	\end{array}
	\right]
	\right)
	+
	\underbrace{	\left[ \gls{OpM}	\left(
		\begin{array}{cccc}
			\vec{u}\\
			p\\
			T\\
			\vec{Y}'
		\end{array}
		\right)
		+
		\left(
		\begin{array}{cccc}
			0\\
			\partial_t 	\rho(T,\vec{Y})\\
			0\\
			0
		\end{array}
		\right)
		\right]}_{\vec{F}(\vectr{U})} = 0 \label{eq:SemiDiscreteLowMa}
\end{equation} 
is obtained. Here the spatial operator is denoted as $\gls{OpM}$.  \cref{eq:SemiDiscreteLowMa} has exactly the form that can be solved with a BDF formula as expressed by \cref{eq:BDFDiscretization}. Note that the time derivative $\partial \rho /\partial t$ from the continuity equation is written as an additional source term, since the density is not a primitive variable. A second-order approximation of the density time derivative is used for its discretization
\begin{equation}
	\left(\frac{\partial \rho}{\partial t} \right)^n= \frac{1}{2\Delta t}\left(3\rho^n-2\rho^{n-1}+\rho^{n-2}\right) \label{eq:DiscretizationDrhoDT}
\end{equation}
While this type of discretization works for a wide range of applications, as it will be seen later, it can lead to instabilities in systems where density variations are large. 
%The solution of problems of the form of \cref{eq:BDFDiscretization} will be treated in \cref{ch:CompMethodology}.
\section{Discontinuous Galerkin discretization of the low-Mach equations}
The discretization methodology shown in last section is used for finding a discrete formulation of the governing equations for low-Mach reactive flows. In the next sections the discretization for the fully coupled problem with finite reaction rate, and for the flame sheet problem are shown. The chosen numerical fluxes are also shown, and some of their particularities are discussed. 

Special care has to be taken to avoid spurious oscillations on the pressure field. This is done in the present work by using a mixed order formulation, where polynomials of order $k$ for velocity, temperature, mass fractions and mixture fractions, and of degree $k' = k-1$ for pressure are used. 
This is a required compatibility condition for obtaining a well posed problem and is used to ensure the validity of the Ladyzenskaja-Babu\u{s}ka-Brezzi (or inf-sup) condition \parencite{babuskaFiniteElementMethod1973}.  

\subsection{Discontinuous Galerkin discretization of the finite reaction rate problem}
Here the DG discretization of the finite reaction system defined by \crefrange{eq:LowMach_Conti}{eq:LowMachMassBalance} is presented. 
First, the vector $\vec{Y}' = \left(Y_1,\dots,Y_{N-1}\right)$ is defined as the vector containing the first $(\gls{TotalNumberSpecies}-1)$ mass fractions and $\vec{s} = \left(s_1, \dots, s_{N-1} \right)$ as the vector containing the test functions for the first $(\gls{TotalNumberSpecies}-1)$  mass fraction equations. 

The discretized form of \crefrange{eq:LowMach_Conti}{eq:LowMachMassBalance} is obtained in a similar fashion to the methodology shown in \cref{sec:DiscWithDG}. This means, (1) each equation is multiplied by a test function, (2) integrated over an element $K_j$, (3) applied integration by parts, (4) introduced an adequate numerical flux for each term and (5) summing over all cells in order to obtain a global formulation.

The discretized equations of the transient problem can be written as: find the numerical solution $(p^{n+1}_h,\vec{u}^{n+1}_h, T^{n+1}_h, \MFVecPrima^{n+1}_h) \in \mathbb{V}_\myvector{k}$ such that for all test functions $(q_h,\vec{v}_h, r_h, \mathbf{s}_h) \in \mathbb{V}_\myvector{k}$ , following equations are fulfilled:
\begin{subequations}
	\begin{flalign}%
		%% Continuity
		&\textbf{Continuity equation}& \notag \\
		&\gls{BCcont}(q_h)= \ContDis\ + \mathcal{T}(\partial_t \rho|_{t^{n+1}},q_h ) ,& \label{DiscretizedConti}\\[1ex]
		%% Momentum
		&\textbf{Momentum equations}& \notag \\
		&\gls{BCmom}(\vec{v}_h) =	\MomConv + \MomPres + \MomDiff & \notag\\
		& \quad\quad\quad + \MomSource + \mathcal{E}(\partial_t (\rho \vec{u})|_{t^{n+1}},\vec{v}_h ) ,& \label{DiscretizedMomentum}\\[1ex]
		%% Energy
		&\textbf{Energy equation}& \notag \\
		&\gls{BCenergy}(r_h)
		 = \mathcal{S}^C\left(\vec{u}^{n+1}_h,T^{n+1}_h,r_h, \rhoh\right) + \mathcal{S}^{D,E}\left(T^{n+1}_h,r_h,k/c_p(T^{n+1}_h)\right)&  \notag\\
		& \quad\quad\quad + \mathcal{S}^S\left(r_h, \heatRelease(T^{n+1}_h,\vec{Y}^{n+1}_h), \rateReac(T^{n+1}_h,\vec{Y}^{n+1}_h),\cph\right)&\notag\\
		&\quad\quad\quad+ \mathcal{T}(\partial_t (\rho T)|_{t^{n+1}},r_h ),& \label{DiscretizedEnergy}   \\[1ex]
	%% MassFractions
		&\textbf{Mass fraction of species $\alpha$ equation}& \notag \\
		&\gls{BCMass}(s_{\alpha h})= \mathcal{S}^C\left(\vec{u}^{n+1}_h,\Yi, s_{\alpha h}, \rhoh\right) + \mathcal{S}^{D,M}\left(\Yi,s_{\alpha h},\rhodh\right)&  \notag \\
		& \quad\quad\quad\quad + \mathcal{M}^S_\alpha\left(s_{\alpha h},\rateReac(T^{n+1}_h,\vec{Y}^{n+1}_h )\right) + \mathcal{T}(\partial_t (\rho Y_{\alpha,h})|_{t^{n+1}},s_h ).& \label{DiscretizedMassFractions}
	\end{flalign}\label{eqs:variatProblemFull}
\end{subequations}
where the index $\alpha$ of the mass fraction equations takes values $\alpha = 1, \dots,~(\gls{TotalNumberSpecies} - 1)$.
Here $\mathcal{C}$ is the discretized divergence form of the continuity equation. Furthermore $\mathcal{U}^C$ is the discretized convective form of the momentum equation and  $\mathcal{S}^C$ is the discretized convective form of the temperature and mass fraction equations. $\mathcal{U}^P$ is the discretized gradient form. Additionally, $\mathcal{U}^D$, $\mathcal{S}^{D,E}$ and $\mathcal{S}^{D,M}$ are the discretized diffusive form of the momentum, temperature and mass fractions equations, respectively. Furthermore $\mathcal{U}^S$, $\mathcal{S}^S$ and $\mathcal{M}^S$ are the discretized source terms of the momentum, temperature and mass fraction equations.  Finally $\mathcal{T}$ and $\mathcal{E}$ are the contributions of the temporal derivatives, and $\gls{BCcont}$, $\gls{BCmom}$, $\gls{BCenergy}$ and $\gls{BCMass}$ contain the information of the Dirichlet boundary conditions for the continuity, momentum, energy and mass fraction equations, respectively.

Note that the convective and diffusive terms of the temperature scalars $T$, mass fraction $Y_\alpha$ and mixture fraction $z$ have the same form, so they share the same expression in their discretized form. Each one of the forms introduced here are defined later in \cref{ssec:nonLinearforms}.
\subsection{Discontinuous Galerkin discretization of the flame sheet problem}
The discretization the flame sheet problem given by \crefrange{eq:MixtFracConti2}{eq:MixtFracMF} proceeds in a analogous way. 

The resulting problem reads: find the numerical solution $(p^{n+1}_h,\vec{u}^{n+1}_h,z^{n+1}_h)\in \mathbb{V}_\myvector{k}$ such that for all test functions $(q_h,\vec{v}_h,r_h)\in \mathbb{V}_\myvector{k}$ following equations are fulfilled:
\begin{subequations}
	\begin{flalign}
		%% Continuity
		&\textbf{Continuity equation}& \notag \\
		&\mathcal{B}^1(q_h)=\mathcal{C}\left(\vec{u}^{n+1}_h,q_h, \rho(z^{n+1}_h)\right)  + \mathcal{T}(\partial_t \rho|_{t^{n+1}},q_h ) ,& \label{DiscretizedConti2}\\
		%% Momentum
		&\textbf{Momentum equations}& \notag \\
		&\mathcal{B}^2(\vec{v}_h) =\mathcal{U}^C\left(\vec{u}^{n+1}_h,\vec{u}^{n+1}_h,\vec{v}_h, \rho(z^{n+1}_h)\right)+ 	\mathcal{U}^P\left(p^{n+1}_h,\vec{v}_h\right) +\mathcal{U}^D\left(\vec{u}^{n+1}_h,\vec{v}_h,\mu(z^{n+1}_h)\right)&\notag \\
		&\qquad\qquad	+\mathcal{U}^S\left(\rho(z^{n+1}_h), \vec{v}_h\right) + \mathcal{E}(\partial_t (\rho \vec{u})|_{t^{n+1}},\vec{v}_h ),& \label{DiscretizedMomentum2}\\ 
		%% Mixture Fraction	 
		&\textbf{Mixture fraction equation}& \notag \\
		&\mathcal{B}^3(r_h) = {S}^C\left(\vec{u}^{n+1}_h,z^{n+1}_h,r_h, \rho(z^{n+1}_h)\right) + \mathcal{S}^{D,E}\left(z^{n+1}_h,r_h,\rho D(z^{n+1}_h)\right)  \mathcal{T}(\partial_t (\rho z)|_{t^{n+1}},r_h ).& \label{DiscretizedEnergy2}
	\end{flalign}\label{eqs:variatFS}
\end{subequations}
Note that density and transport parameters depend on the mixture fraction $z$, which indirectly modifies the temperature and mass fraction fields.%Due to the similarity of the mass fraction equation and the mixture fraction equation the discretization is analogous.

%The evaluation of those parameters is done as mentioned in \cref{sec:FlameSheet} and solved iteratively using a Newton-Dogleg type method as shown later in \cref{sec:Newton}
\subsection{Definitions of nonlinear forms}\label{ssec:nonLinearforms}
In the following the nonlinear forms used in this work are shown. Regarding the choice of fluxes, the "best practices" known in literature for the incompressible Navier-Stokes equation are followed. These fluxes proved to be well suited for all the problems discussed in this thesis, providing stability to the algorithm, while maintaining the accuracy of the solver.

It is well known that central difference fluxes for the pressure gradient and velocity divergence, combined with a coercive form for the viscous terms, e.g. symmetric interior penalty, gives a stable discretization for the Stokes equation \parencite{pietroMathematicalAspectsDiscontinuous2012,giraultDiscontinuousGalerkinMethod2004}. Furthermore, it is known that for all kinds of convective terms, a numerical flux which transports information in characteristic direction, e.g. Upwind, Lax-Friedrichs or Local-Lax-Friedrichs, must be used. The last one was opted for in the present implementation, as it provides a good compromise between accuracy and stability.
\subsubsection{Continuity equation}
A central difference flux for the discretization of the continuity equation is used:
\begin{equation}
	\mathcal{C}(\vec{u},q, \rho)  =  \oint_{\GammaI \cup\GammaN\cup \GammaND\cup \GammaP}{\mean{\rho\vec{u} }\cdot \vec{n}_\Gamma\jump{q} \dS} - \int_{\Omega}{\rho \vec{u}\cdot \nabla_h q} \dV.  \label{eq:Conti}
\end{equation}
The density in \cref{eq:Conti} is evaluated as a function of the temperature and mass fractions using the equation of state (\cref{eq:ideal_gas}). The term $\mathcal{B}^1$ on the left hand sides of \cref{DiscretizedConti} and \cref{DiscretizedConti2}  contains the Dirichlet boundary conditions:
\begin{equation}
	\mathcal{B}^1(q) =  -\oint_{\GammaD\cup \GammaDW}{q(\rho_{\text{D}} \vec{u}_{\text{D}} \cdot \normalBoundary). \dS}
\end{equation}
The density at the boundary  $\rho_{\text{D}}$ is evaluated with \cref{eq:ideal_gas} using the corresponding Dirichlet values of temperature and mass fractions, i.e. $\rho_D = \rho (T_D,\vec{Y}_D)$. Finally the temporal contribution is
\begin{equation}
\mathcal{T}(u,v) =   \int_{\Omega}{uv} \dV.
\end{equation}
Note that in case of the temporal term of the continuity equation $\partial_t \rho$, it is discretized as mentioned in \cref{eq:DiscretizationDrhoDT}
\subsubsection{Momentum equations}
The convective term of the momentum equations is discretized using a Lax-Friedrichs flux
\begin{equation}
	\mathcal{U}^C(\vec{w},\vec{u},\vec{v}, \rho)=  \oint_{\Gamma}{\left( \mean{\rho\vec{u}\otimes\vec{w} }\vec{n}_\Gamma + \frac{\gamma_1}{2}\jump{\vec{u}}\right)\cdot\jump{\vec{v}} \dS}
	-\int_{\Omega}({\rho\vec{u}\otimes\vec{w}}):\nabla_h\vec{v} d\text{V}.
	\label{eq:Mom_convective}\\
\end{equation}
The Lax-Friedrichs parameter $\gamma_1$ is calculated as \textcite{kleinHighorderDiscontinuousGalerkin2016}
\begin{equation}
	\gamma_1  = \max \left\{2 \overline{\rho^+} |\overline{\vec{u}^+} \cdot \vec{n}^+|,2 \overline{\rho^-} |\overline{\vec{u}^-} \cdot \vec{n}^-|\right\},
	\label{eq:vardens_lambda}
\end{equation}
where $\overline{\rho^\pm}$ and $\overline{\vec{u}^\pm}$ are the mean values of $\rho^\pm$ and $\vec{u}^\pm$ in $K^\pm$, respectively. %It should be noted however, that preliminary tests using the mean values for the calculation of the penalty parameter\\
The pressure term is discretized by using a central difference flux
\begin{equation}
	\mathcal{U}^P(p,\vec{v})=  \oint_{\Gamma \setminus \Gamma_{\text{N}}\setminus \Gamma_{\text{ND}}}{ \mean{p}(\jump{\vec{v}}\cdot \vec{n}_\Gamma  )\dS}
	- \int_{\Omega}{p \nabla_h \cdot \vec{v} \dV}. \label{eq:Mom_pressure}
\end{equation}
The diffusive term of the momentum equations is discretized using a \gls{SIP} formulation \parencite{shahbaziExplicitExpressionPenalty2005}
\begin{equation}
	\begin{aligned}
		\tilde{\mathcal{U}}^D(\vec{u},\vec{v},\mu) =
		  & \int_{\Omega}{\left(\mu\left((\nabla_h \vec{u}) + (\nabla_h \vec{u})^T - \frac{2}{3}(\nabla_h\cdot \vec{u})\mytensor{I} \right)\right)\colon \nabla_h\vec{v}} \dV \\
		- & \oint_{\Gamma \setminus \Gamma_{\text{N}}\setminus \Gamma_{\text{ND}}}
		\left(\mean{\mu(\nabla_h\vec{u} + \nabla_h\vec{u}^T - \frac{2}{3}(\nabla_h\cdot \vec{u})\mytensor{I})}\normalBoundary\right)\cdot\jump{\vec{v}}\dS                    \\
		- & \oint_{\Gamma \setminus \Gamma_{\text{N}}\setminus \Gamma_{\text{ND}}}
		\left(\mean{\mu(\nabla_h\vec{v} + \nabla_h\vec{v}^T - \frac{2}{3}(\nabla_h\cdot \vec{v})\mytensor{I})}\normalBoundary\right)\cdot\jump{\vec{u}} \dS                   \\
		+ & \oint_{\Gamma \setminus \Gamma_{\text{N}}\setminus \Gamma_{\text{ND}}} \eta \mu_{\text{max}} \jump{\vec{u}} \jump{\vec{v}}\dS.
		\label{eq:Mom_diffusive}
	\end{aligned}
\end{equation}
The viscosity $\mu$ is evaluated as a function of temperature according to \cref{eq:nondim_sutherland} and $\mu_{\text{max}} = \text{max}(\mu^{+}, \mu^{-})$.  Additionally  $\eta$ is the penalty term of the \gls{SIP} formulation, which has to be chosen big enough to ensure coercivity of the form, but also as small as possible in order to not increase the condition number of the problem. The estimation of the penalty term is based on an expression of the form
\begin{equation}
	\eta = \eta_0 \frac{A(\partial K)}{V(K)},
\end{equation}\label{eq:PenaltyFactor}
where for a two-dimensional problem $A$ is the perimeter and $V$ the area of the element. The parameter $\eta_0$ is a safety factor. If not stated otherwise, the value  $\eta_0 = 4$ is  set in all calculations. Further information on the determination of the penalty term of the SIP formulation $\eta$ and the penalty term of the Lax-Friedrichs  flux $\gamma_1 $ can be found in  the works from \textcite{hesthavenNodalDiscontinuousGalerkin2008} and \textcite{hillewaertDevelopmentDiscontinuousGalerkin2013}.

Note that the diffusive term of the momentum equations is scaled by the Reynolds number, obtaining finally
\begin{equation}
		\mathcal{U}^D(\vec{u},\vec{v},\mu) = 	\frac{1}{\Reynolds}\tilde{\mathcal{U}}^D(\vec{u},\vec{v},\mu)
\end{equation}

The source term arising due to body forces is:
\begin{equation}
	\mathcal{U}^S\left(\rho, \vec{v}\right) =  \smash{\frac{1}{\Froude^2}}\int_{\Omega}{  \rho \frac{\vec{g}}{\Vert \vec{g} \Vert}\cdot \vec{v}} \dV.  \label{eq:Mom_source}
\end{equation}
Finally, the left hand sides of \cref{DiscretizedMomentum} and \cref{DiscretizedMomentum2} contain the information from Dirichlet boundary conditions:
\begin{equation}
	\mathcal{B}^2(\vec{v}) =
	-\oint_{\Gamma_{\text{D}}}{ \left( (\rho\vec{u}_{\text{D}}\otimes\vec{u}_{\text{D}} )\normalBoundary + \frac{\gamma_1}{2}\vec{u}_{\text{D}}\right)\cdot\vec{v} \dS}  +
	\oint_{\Gamma_{\text{D}}}{\mu_{\text{D}}\vec{u}_{\text{D}}\cdot(\nabla_h\vec{v} \normalBoundary + \nabla_h\vec{v}^T \normalBoundary- \eta \vec{v} )\dS.}
\end{equation}
The Dirichlet viscosity value $\mu_{\text{D}}$ is calculated from \cref{eq:nondim_sutherland} using the Dirichlet values of the temperature at the boundary.
Finally the temporal derivative of the momentum equation in its discretized form corresponds to
\begin{equation}
\mathcal{E} (\vec{u},\vec{v}) =   \int_{\Omega}{\vec{u}\cdot\vec{v}} \dV.
\end{equation}
\subsubsection{Scalar equations}
Since the convective and diffusive terms for the temperature, mass fractions and mixture fraction share a similar form,  here their discretized expressions are summarized in terms of an arbitrary scalar $X$ (corresponding to $T$ in the energy equation, $Y_\alpha$ in the equation for species $\alpha$ and $z$ for the mixture fraction equation) and transport parameter $\xi$ (i.e. $k/c_p$ in the energy equation, and $(\rho D)$ for the mass fraction and mixture fraction equations). The convective term of the scalars is discretized using a Lax-Friedrichs flux
\begin{equation}
	\mathcal{S}^C(\vec{u},X,r, \rho) =  \oint_{\Gamma}{\left( \mean{\rho\vec{u}X }\cdot \vec{n} + \frac{\gamma_2}{2}\jump{X}\right)\jump{r} \dS}
	- \int_{\Omega}({\rho \vec{u} X \cdot \nabla_h r) d\text{V}}. \label{eq:scalar_convective}
\end{equation}
The Lax-Friedrichs parameter $\gamma_2$ is calculated as 
\begin{equation}
	\gamma_2  = \max \left\{\overline{\rho^+} |\overline{\vec{u}^+} \cdot \vec{n}^+|,\overline{\rho^-} |\overline{\vec{u}^-} \cdot \vec{n}^-|\right\}.
	\label{eq:vardens_lambda2}
\end{equation}
The diffusion term of scalars is discretized again with a SIP formulation:
\begin{align}
	\mathcal{S}^D(X,r,\xi)= & \int_{\Omega}{ \left(\xi \nabla_h X \cdot\nabla_h r\right) }\dV \notag         \\
	                        & -\oint_{\Gamma \setminus \Gamma_{\text{N}}\setminus \Gamma_{\text{ND}}}{\left(
		\mean{\xi \nabla_h X}\cdot \vec{n}\jump{r} +
		\mean{\xi \nabla_h r}\cdot \vec{n}\jump{X} -
		\eta \xi_{\text{max}} \jump{X} \jump{r}
		\right) \dS.
	} \label{eq:Temp_diffusive}
\end{align}
The transport parameter $\xi$ is calculated as a function of temperature using \cref{eq:nondim_sutherland} and $\xi_{\text{max}} = \text{max}(\xi^{+}, \xi^{-})$.
The diffusive term for the temperature equation  and mixture fraction equation is scaled by the Reynolds and Prandtl number as
\begin{equation}
\mathcal{S}^{D,E}\left(T,r,k/c_p\right) = \frac{1}{\Reynolds~\Prandtl} \mathcal{S}^D(T,r,k/c_p)
\end{equation}
Similarly the diffusive term for the mass fraction equations is
\begin{equation}
\mathcal{S}^{D,M}\left(\Yi,s_{\alpha h},\rho D_\alpha\right) = \frac{1}{\Reynolds~\Prandtl~\Lewis_\alpha} \mathcal{S}^D\left(\Yi,s_{\alpha h},\rho D_\alpha\right)
\end{equation}
The boundary condition term of the corresponding scalar equation is:
\begin{equation}
	\mathcal{B}^3(r) =  -
	\oint_{\GammaD\cup \GammaND}{ \left( (\rho_D\vec{u}_D X_D)\cdot \normalBoundary + \frac{\gamma_2}{2}X_D\right)r \dS}
	+\oint_{\GammaD\cup \GammaND} \xi_D X_D(\nabla_h r \cdot \normalBoundary - \eta r )\dS.
\end{equation}
Here, $X_D$ is the Dirichlet value of the scalar $X$ on boundaries and $\xi_D$ is the corresponding transport parameter, which is calculated with \cref{eq:nondim_sutherland} using the Dirichlet values of the temperature at the boundary.
Finally, the volumetric source terms of the energy and mass fraction equations are defined as follows:
\begin{gather}
	\mathcal{S}^S(r,\heatRelease, \rateReac,c_p) =  \text{H}~\Da ~ \int_{\Omega}{ \frac{\heatRelease \rateReac}{c_p} r} \dV, \\
	\mathcal{M}^S_\alpha(s_\alpha,\rateReac ) =  \Da \int_{\Omega}{  \stoicCoef_\alpha M_\alpha \rateReac s_\alpha} \dV.
\end{gather}
The heat release $\heatRelease$ is calculated with \cref{eq:heatReleaseOneStepNonDim}, the reaction rate $\rateReac$ is evaluated using \cref{eq:NonDimArr} and the mixture heat capacity with \cref{eq:nondim_cpmixture}.


