\chapter{Results}	\label{ch:results}
\glsresetall
The following sections present a comprehensive solver validation using various test cases. In \cref{sec:SingleCompIsotCase} the applicability of the solver is analyzed for isothermal single-component systems. Later in \cref{sec:SinCompNonIsothermCase} several single-component non-isothermal configurations are studied. Finally, in \cref{sec:MultCompNonIsothermCase} test cases for multicomponent non-isothermal systems are presented, with a particular emphasis on systems where combustion is present.
%Additionally the convergence properties of the DG-method will be analyzed for some of the systems.

All calculations shown here were performed on an AMD EPYC 7543 32-core Processor, DDR4 %TODO \todo[inline]{which specifications of the cluster should i include?}
Unless otherwise stated, all calculations use the termination criteria presented in \cref{ssec:TerminationCriterion}.
%%%%%%%%
%%%%%%%%
\section{Single-component isothermal cases}\label{sec:SingleCompIsotCase}
%%%%%%%%
%%%%%%%%
The solver presented in the previous chapter is initially validated for single-component isothermal cases. In these cases, only the continuity and momentum equations are used and solved. The energy equation and the species concentration equations are replaced by the conditions $T = 1.0$ and $Y_0 = 1.0$ in the whole domain. This means that the physical properties of the flow (density and viscosity) are constant and that the flow is fully incompressible because the density shows no thermodynamic or hydrodynamic dependence.
\subsection{Lid-driven cavity flow}

\begin{figure}[t]
	\begin{center}
		\def\svgwidth{0.3\textwidth}
		\import{./plots/}{LidDrivvenGeometry.pdf_tex}
		\caption{Schematic representation of the Lid-Driven cavity flow.}
		\label{fig:LidDrivenCavity}
	\end{center}
\end{figure}

The lid-driven cavity flow is a classically used test problem for the validation of Navier-Stokes solvers. The system configuration is shown in \cref{fig:LidDrivenCavity}. It consists simply of a two-dimensional square cavity enclosed by walls whose upper boundary moves at constant velocity, causing the fluid to move. Benchmark results can be found widely in the literature for different Reynolds numbers. In this section, the results obtained with the XNSEC-Solver are compared with those published by \textcite{botellaBenchmarkSpectralResults1998}. 

The problem is defined in the domain $\Omega = [0,1]\times [0,1]$. The system is solved for the velocity vector $\gls{velVec} = (u,v)$ and the pressure $p$. All boundary conditions are Dirichlet-type, particularly with $\gls{velVec} = (-1,0)$ for the boundary at $y = 1$ and $\gls{velVec} = (0,0)$ for all other sides. The gravity vector is set to $\gls{gravityVec} = (0,0)$.
A Cartesian mesh with extra refinement at both upper corners is used, and is shown in \cref{fig:LiddrivenMesh}. The refinement was done to better represent the complex effects that take place in the corners. The streamline plot presented in \cref{fig:LiddrivenMesh} shows the different vortex structures typical of this kind of system, where in addition to the main vortex of the cavity, smaller structures appear in the corners.

\begin{figure}[b]
	\centering
	\pgfplotsset{width=0.35 \textwidth, compat=1.3}
	\inputtikz{LiddrivenMesh1}
	\inputtikz{LiddrivenMesh2}
	\caption{Mesh and streamlines of the lid-driven cavity flow with $\gls{Reynolds} = 1000$} \label{fig:LiddrivenMesh}
\end{figure}

The lid-driven cavity was calculated for a Reynolds number $\gls{Reynolds} = 1000$. For the calculations presented here, the polynomial degree is set to $k = 4$ and $k' = 3$. A regular Cartesian mesh with $16\times16$ elements is used with extra refinement in the corners. In \cref{fig:LidVelocities} a comparison of the calculated velocity with the DG-Solver and the velocities provided by the benchmark is shown. Clearly, very good agreement is obtained, even by using a relatively coarse mesh (the benchmark result uses a grid with $160\times160$ elements).

A more rigorous comparison of results is presented in \cref{tab:LidCavityExtrema}, where the extreme values of the velocity components calculated through the centerline of the cavity are compared with the results presented by \textcite{botellaBenchmarkSpectralResults1998}. Different mesh resolutions were used for this comparison, particularly meshes with $16\times16$, $32\times32$, $64\times64$, $128\times128$ and $256\times256$ elements, each with extra refinement at the corners. It can be clearly seen how for the finest mesh the results obtained with the DG-solver are extremely close to the reference, seeing a difference only at the fifth digit after the decimal point for the velocity components, and no difference for the spatial coordinates where the extrema are. It can also be appreciated that the results obtained with the coarser meshes are still very close to those of the reference. It is worth mentioning that this comparison only considering number of elements could be considered unfair, since the reference uses another method for solving the governing equations. One of the advantages of the DG method is that choosing higher-order polynomials allows more information to be packed into each cell. A better comparison would be achieved by comparing results based on the number of degrees of freedom(dofs) used in the simulation, as will be done later in section %TODO XXX.

\begin{figure}[tb]
	\pgfplotsset{
		group/xticklabels at=edge bottom,
		legend style = {
				at ={ (0.09,0.3), anchor= north east}
			},
	}
	\inputtikz{LidVelocities1}
	\pgfplotsset{
		group/xticklabels at=edge bottom,
		legend style = {
				at ={ (0.59,0.3), anchor= north east}
			},
	}
	\inputtikz{LidVelocities2}
	\caption{Calculated velocities along the centerlines of the cavity and reference values. Left plot shows the x-velocity for $x = 0.5$. Right plot shows the y-velocity for $y = 0.5$  }
	\label{fig:LidVelocities}
\end{figure}

%	\begin{figure}[tb]
%		\pgfplotsset{
%			group/xticklabels at=edge bottom,
%			legend style = {
%				at ={ (0.09,0.2), anchor= north west}
%			},
%		}
%		\begin{tikzpicture}
%			\begin{axis}[
%				width= 0.4\textwidth ,
%				height= 0.3\textwidth ,
%				xlabel = $y$,
%				ylabel= $\omega$, 
%				]
%				\addplot+[color=black, only marks] table {data/LidDrivenCavity/omegaRe1000_x_ref.txt}; \addlegendentry{Reference}
%				\addplot[color=black, no marks] table {data/LidDrivenCavity/omegaRe1000_x.txt}; \addlegendentry{BoSSS}
%			\end{axis}
%		\end{tikzpicture}
%		\pgfplotsset{
%			group/xticklabels at=edge bottom,
%			legend style = {
%				at ={ (0.59,1.0), anchor= north east}
%			},
%		}
%		\begin{tikzpicture}
%			\begin{axis}[
%				width= 0.4\textwidth ,
%				height= 0.3\textwidth ,
%				xlabel = $x$,
%				ylabel= $\omega$, 
%				]
%				\addplot+[color=black, only marks] table {data/LidDrivenCavity/omegaRe1000_y_ref.txt}; \addlegendentry{Reference}
%				\addplot[color=black, no marks] table {data/LidDrivenCavity/omegaRe1000_y.txt}; \addlegendentry{BoSSS}
%			\end{axis}
%		\end{tikzpicture}
%		\caption{Calculated vorticity along the centerlines of the cavity and reference values. Left plot shows the vorticity for $x = 0.5$. Right plot shows the vorticity for $y = 0.5$  }
%		\label{fig:LidVorticities}
%	\end{figure}

\begin{table}[]
	\centering
	\begin{tabular}{lllllrr}
		\hline
		Mesh           & $u_{\text{max}}$ & $y_{\text{max}}$ & $v_{\text{max}}$ & $x_{\text{max}}$ & \multicolumn{1}{l}{$v_{\text{min}}$} & \multicolumn{1}{l}{$x_{\text{min}}$} \\ \hline
		$16\times16$   & 0.3852327        & 0.1820           & 0.3737295        & 0.8221           & -0.5056627                           & 0.0941                               \\
		$32\times32$   & 0.3872588        & 0.1821           & 0.3760675        & 0.8227           & -0.5080496                           & 0.0943                               \\
		$64\times64$   & 0.3897104        & 0.1748           & 0.3774796        & 0.8408           & -0.5248360                           & 0.0937                               \\
		$128\times128$ & 0.3886452        & 0.1720           & 0.3770127        & 0.8422           & -0.5271487                           & 0.0907                               \\
		$256\times256$ & 0.3885661        & 0.1717           & 0.3769403        & 0.8422           & -0.5270653                           & 0.0907                               \\\hline
		Reference      & 0.3885698        & 0.1717           & 0.3769447        & 0.8422           & \multicolumn{1}{l}{-0.5270771}       & \multicolumn{1}{l}{0.0908}           \\ \hline
	\end{tabular}
	\caption{Extrema of velocity components through the centerlines of the lid-driven cavity for $\gls{Reynolds} = 1000$. Reference values obtained from \textcite{botellaBenchmarkSpectralResults1998} }
	\label{tab:LidCavityExtrema}
\end{table}
\FloatBarrier
% \newpage

\subsection{Backward-facing step}\label{ssec:BackwardFacingStep}
The backward-facing step problem is another classical configuration widely used for validation of incompressible CFD codes. It has been widely studied theoretically, experimentally, and numerically by many authors in the last decades (see, for example, \textcite{armalyExperimentalTheoreticalInvestigation1983,barkleyThreedimensionalInstabilityFlow2000,biswasBackwardFacingStepFlows2004} ).  In \cref{BFSsketch} a schematic representation of the problem is shown. It consists of a channel flow (usually considered fully developed) that is subjected to a sudden change in geometry that causes separation and reattachment phenomena. For these reasons, this case can be considered more challenging than the one presented in the previous section, since special care of the mesh used has to be taken in order to capture accurately all complex phenomena taking place.

Although the backward-facing step problem is known to be inherently three-dimensional, it has been shown that it can be studied as a two-dimensional configuration along the symmetry plane for moderate Reynolds numbers. For the range of Reynolds numbers used in the calculations presented here, the two-dimensional assumption is justified \parencite{barkleyThreedimensionalInstabilityFlow2000, biswasBackwardFacingStepFlows2004}.  The origin of the coordinate system is set in the bottom part of the step. The step height \gls{StepHeigth} and channel height \gls{ChannelHeight} characterize the system. Results in the literature are often reported as a function of the expansion ratio, defined as $\gls{ExpansionRatio} = (\gls{ChannelHeight}+\gls{StepHeigth})/\gls{ChannelHeight}$.

A series of simulations were performed with the objective of reproducing the results reported by \textcite{biswasBackwardFacingStepFlows2004}, where the backward-facing step was calculated for Reynolds numbers up to $400$ and for an expansion ratio of $\gls{ExpansionRatio} = 1.9423$. In particular, the reported lengths of detachment and reattachment are used as a means of comparison with the results from the XNSEC solver.

The Reynolds number for the backward-facing step configuration is defined in the literature in many forms. Here, the definition based on the step height $\glsHat{StepHeigth}$ and the mean inlet velocity $\hat U_{\text{mean}}$ is adopted as the reference length and velocity, resulting in
\begin{equation}
	\gls{Reynolds}= \frac{\glsHat{StepHeigth}\hat U_{\text{mean}}}{\glsHat{kinVisc}}.
\end{equation}
The boundary at $x = - L_0$ is an inlet boundary condition, where a parabolic profile is defined with %, with a kinematic viscosity  $\nu = \SI{15.52e-6 }{\meter \squared \per \second}$
\begin{equation}
	u(y) = -6\frac{( y- S)( y-( h+ S))}{h^2} %= \frac{\hat u(y)}{\hat U_{\text{mean}}} 
\end{equation}
The system is isothermal, and the fluid is assumed to be air. The step length is set $S=1$ and $h = 1.061$. To minimize the effects of the outlet boundary condition on the part of interest in the system, the length $L$ of the domain is set to $L = 70 \gls{StepHeigth}$. All other boundaries are fixed walls. From prior calculations, the effect of the domain length before the step was found to have almost no impact on the results and is set to $L_0 = \gls{StepHeigth}$. Preliminary studies showed that the calculated reattachment and detachment lengths are highly sensitive to the mesh resolution. For all simulations in this section, a structured grid with 88,400 elements is used. To better resolve the complex structures that occur in this configuration, smaller elements are used in the vicinity of the step, as seen in \cref{bfsmesh}.  A polynomial degree of three was chosen for both velocity components and two for pressure.

\begin{figure}[tb]
	\begin{center}
		\def\svgwidth{0.9\textwidth}
		\import{./plots/}{BFS_sketch.pdf_tex}
		\caption[Schematic representation of the backward-facing step.]{Schematic representation (not to scale) of the backward-facing step. Both primary and secondary vortices are shown.}
		\label{BFSsketch}
	\end{center}
\end{figure}

\begin{figure}[tb]
	\begin{center}
		\def\svgwidth{0.8\textwidth}
		\import{./plots/}{HBFS_MESH.pdf_tex}
		\caption{Mesh used for the backward-facing step configuration.}
		\label{bfsmesh}
	\end{center}
\end{figure}

\begin{figure}[bt]
	\centering
	\pgfplotsset{
		group/xticklabels at=edge bottom,
		%		legend style = {
		%			at ={ (1.0,1.0), anchor= north east}
		%		},
		unit code/.code={\si{#1}}
	}
	\inputtikz{uvelBFS}
	\caption[Distribution of x-component of velocity in the backward-facing step configuration for a Reynolds number of 400.]{Distribution of x-component of velocity in the backward-facing step configuration for a Reynolds number of 400. Solid lines correspond to results obtained with the XNSEC solver.}
	\label{fig:uvelBFS}
\end{figure}



\begin{figure}[tb]
	\pgfplotsset{
		group/xticklabels at=edge bottom,
		legend style = {
				at ={ (0.05,0.9), anchor= north west}
			},
		unit code/.code={\si{#1}}
	}
	\centering
	\inputtikz{Re_De_Attachmentlengths}
	\caption[Detachment and reattachment lengths of the primary and secondary recirculation zones after the backward-facing step compared to the reference solution]{ Detachment and reattachment lengths of the primary (left figure) and secondary (right figure) recirculation zones after the backward-facing step compared to the reference solution \parencite{biswasBackwardFacingStepFlows2004}.}
	\label{fig:Re_De_Attachmentlengths}
\end{figure}
The backward-facing step configuration exhibits varying behavior as the number of Reynolds changes. For small Reynolds numbers, a single vortex, usually called the primary vortex, appears in the vicinity of the step. Furthermore, as the Reynolds number increases, a second vortex eventually appears on the top wall, as shown schematically in \cref{BFSsketch}.
The detachment and reattachment lengths of the vortices are values that are usually reported in the literature. It is possible to determine the detachment position by finding the point along the wall where the velocity gradient normal to the wall acquires a value equal to zero. 

\cref{fig:Re_De_Attachmentlengths} shows the detachment and reattachment lengths of the primary and secondary vortices obtained with the XNSEC solver for different Reynolds numbers, which are also compared with the results presented in the reference paper from \textcite{biswasBackwardFacingStepFlows2004}. Cubic splines have been used to accurately locate this point. It can be seen that the results for the detachment lengths of the primary vortex $R_1$ are in very good agreement with those of the reference. In the case of the secondary vortex, it is possible to see a very minimal deviation for the lengths of the reattachment $R_3$, hinting at a possible spatial underresolution far away from the step. It is interesting to note that, despite the fact that the reference does not report the existence of a secondary vortex for $\gls{Reynolds} = 200$, it was possible to observe it with the XNSEC solver. The results allow us to conclude that it is possible to study flows with complex behavior for low- to moderate Reynolds numbers, at least in the isothermal case. In the next section, a non-isothermal case of this configuration will be studied.

It is worth mentioning that the evaluation of the global order of accuracy of the solver using the two incompressible test cases presented in this section is problematic due to the presence of singularities. Specifically the points at the corners at the coordinates $ \vec{x} = (0,1)$ and $\vec{x} =(1,1)$ of the Lid-driven cavity (where the pressure is not finite according to \textcite{botellaBenchmarkSpectralResults1998}), and at the corner of the step $\vec{x} = (0,S)$ of the backward-facing step are problematic. The accuracy of the solver will be assessed later in \cref{ssec:CouetteFlowTempDiff} making use of a analytical solution and in \cref{ssec:ConvStudyHeatedCavity} using a solution obtained with a high spatial resolution.

%%%%%%%%
%%%%%%%%
\section{Single-component non-isothermal cases} \label{sec:SinCompNonIsothermCase}
%%%%%%%%
%%%%%%%%
For the test cases presented in this section, the equations for continuity, momentum and energy are solved. All systems are assumed to be single-component, thus $N = 1$ and $Y_0 = 1.0$. We start by showing in \cref{ssec:HeatedBackwardFacingStep} an extension of the backward-facing step configuration presented in the last section, considering now a non-isothermal system. Later in \cref{ssec:CouetteFlowTempDiff} a Couette flow configuration presenting a temperature gradient in the vertical direction is studied. Finally, in \cref{ss:DHC} a heated square cavity configuration is studied in order to assess the capability of the solver for variable density flows in closed systems.

\subsection{Heated backward-facing step}\label{ssec:HeatedBackwardFacingStep}


As an extension to the previous casem the backward-facing step configuration in a non-isothermal configuration is studied, where the bottom wall is heated to a constant temperature. 


In this section the configuration for a heated backward-facing step proposed in \cite{xieFluidFlowHeat2016} is solved. 
The fluid entering the system has a temperature equal to $\hat T_0 = \SI{283}{\kelvin}$ and the bottom wall is set to a constant temperature of $\hat T_1 =\SI{313}{\kelvin}$. The inlet temperature is used as the reference temperature, obtaining $T_0 = 1.0$ and $T_1 = 1.106$.
In the work of \cite{xieFluidFlowHeat2016} results are reported for the local Nusselt numbers and the local friction coefficients $f_d$  along the bottom wall ($y = 0$) for different expansion ratios and Reynolds numbers.
By combining the definition of the Nusselt number ($\gls{Nusselt} = \gls{HeatTransCoef}\hat{L}/\glsHat{HeatConductivity}$), Newton's law of cooling ($\hat{\vec{q}} = \hat{h} (\hat{T}_0 - \hat{T}_1 )$), and Fourier's law of heat conduction ($\hat{\vec{q}} = \hat \lambda \hat{\nabla} \glsHat{temp}$) a expression for the local Nusselt number is obtained.
\begin{equation}
	\gls{NusseltLoc} = \frac{\hat L}{\hat T_0-\hat T_1}\hat \nabla \hat T \cdot \hat {\vec{n}}
\end{equation}
where $\hat L$ is the reference length. $ \hat L = \hat S$ is chosen to be consistent with the definition of the Reynolds number of the reference. Furthermore, the local friction factor can be written as % recognizing that the wall shear stress along the bottom wall $\tau_{\text{w}} = -\mu \nabla u \cdot \vec{n}$,
\begin{equation}
	f_d = \frac{8\hat \nu} { (\hat U_{\text{mean}})^2}  \hat \nabla \hat u \cdot \hat {\vec{n}}
\end{equation}

\begin{figure}[t]
	\centering
	\pgfplotsset{width=0.81\textwidth, compat=1.3}
	\inputtikz{BackwardFacingStepTemperatureField}	
	\inputtikz{BackwardFacingStepStreamlines}
	\caption{Temperature profile and streamlines corresponding to the backward-facing Step configuration for $\gls{Reynolds} = 400$ and an expansion ratio of two.} \label{fig:BFS_Temperature_Streamlines}
\end{figure}

 Simulations were conducted for different Reynolds numbers and expansion ratios. In \cref{fig:BFS_Temperature_Streamlines} the temperature field and the streamlines corresponding to a calculation with $\gls{Reynolds} = 700$ are shown. Here, the apparition of the secondary vortex is seen in the top wall. Note that only a small part of the computational domain is shown. Far away from the step, a lightly skewed parabolic velocity profile is obtained, which is influenced by the density variations on the vertical direction.
 
 For this range of temperature differences, the temperature profile is just influenced by conductive effects, since no appreciable natural convection phenomena appears. For larger temperature differences, Rayleigh-Bénard type instabilities would appear in the flow. This type of system will be treated later in \cref{ssec:RayBer}.
 
It should be noted here that the results obtained using the XNSEC-solver are substantially different from those reported by \cite{xieFluidFlowHeat2016}, and will not be shown here. However, in the work of \cite{henninkLowMachNumberFlow2022} the same is also reported, stating that with his method it was not possible to reproduce the results presented by \cite{xieFluidFlowHeat2016}. 

In \cref{fig:fd_Nu_plot} the local friction factor and local Nusselt number along the wall $y = 0$ are plotted for $\gls{Reynolds} = 700$ and ER $= 2$. Comparing the results from the XNSEC-solver with those reported in \cite{henninkLowMachNumberFlow2022} a very good agreement can be observed.
With this test it is possible to confirm that the XNSEC solver is able to deal with complex systems where heat transfer is present. However, for the range of temperature differences involved in this case, the variation of physical parameters such as density, viscosity and thermal conductivity with respect to temperature has no appreciable influence on the simulated flow fields. The next two test-cases will show how the XNSEC-solver is able to simulate low-Mach number flows with a considerable temperature difference.
\begin{figure}[tb]
	\pgfplotsset{
		group/xticklabels at=edge bottom,
		legend style = {
				at ={ (0.59,1.0), anchor= north east}
			},
		unit code/.code={\si{#1}}
	}
	\inputtikz{fd_Nu_plot1}
	\inputtikz{fd_Nu_plot2}
	\caption[Local friction factor and local Nusselt number along the bottom wall of the backward-facing step for $\gls{Reynolds} = 700$ and an expansion ratio of two.]{Local friction factor and local Nusselt number along the bottom wall of the backward-facing step for $\gls{Reynolds} = 700$ and an expansion ratio of two. The solid lines corresponds to our solution and the marks to the reference \citep{henninkLowMachNumberFlow2022}}
	\label{fig:fd_Nu_plot}
\end{figure}%
\FloatBarrier%
\subsection{Couette flow with vertical temperature gradient} \label{ssec:CouetteFlowTempDiff}
As a further test case for the low-Mach solver, a Couette flow with a vertical temperature gradient is considered. This configuration was already studied in \textcite{kleinHighorderDiscontinuousGalerkin2016}, where the SIMPLE algorithm was used in an DG framework for the solution of the governing equations. In this section, the results from said publication are reproduced by using the XNSEC solver, which features a fully coupled algorithm, in contrast to the SIMPLE solver, which solves the system in a segregated way. Additionally, it will be shown how the implemented solver performs in relation to the SIMPLE based solver in terms of runtime. %
\begin{figure}[tb]
	\begin{center}
		\def\svgwidth{0.5\textwidth}
		\import{./plots}{HeatedCouetteSketch.pdf_tex}
		\vspace{0.2cm}
		\caption{Schematic representation of the Couette flow with temperature difference test case.}\label{fig:CouetteTempDiff_scheme}
	\end{center}
\end{figure}%
%\subsubsection{Set-up}

In \cref{fig:CouetteTempDiff_scheme} a schematic representation of the test case is shown. The domain is chosen as $\Omega = [0,1]\times[0,1]$, and Dirichlet boundary conditions are used for all boundaries. The upper wall corresponds to a moving wall ($u = 1$) with a fixed temperature $T=T_h$. The bottom wall is static ($u = 0$) and has a constant temperature $T = T_c$.
Additionally, the system is subjected to a gravitational field, where the gravity vector only has a component in the $y$ direction. Under these conditions, the x-component of velocity, pressure, and temperature are only dependent on the $y$ coordinate, that is, $u = u(y)$, $T = T(y)$ and $p = p(y)$. The governing equations reduce to%
\begin{subequations}
    \begin{align}
    	 & \frac{1}{\gls{Reynolds}} \pfrac{ }{y}\left(\mu\pfrac{u}{y}\right) = 0,\\
    	 & \pfrac{p}{y} = -\frac{\gls{dens}}{\gls{Froude}^2},\\
    	 & \frac{1}{\gls{Reynolds}~\gls{Prandtl}} \pfrac{ }{y}\left(\gls{HeatConductivity}\pfrac{T}{y}\right) = 0.
    \end{align}\label{eq:AllCouetteEquations}
\end{subequations}
By assuming a temperature dependence of the transport properties according to a Power Law ($\mu = \lambda = T^{2/3}$) it is possible to find an analytical solution for this problem.
\begingroup
\allowdisplaybreaks
\begin{subequations}
    \begin{align}
    	u(y) & = C_1 + C_2\left(y + \frac{T_c^{5/3}}{T_h^{5/3}-T_c^{5/3}} \right)^{3/5},\\
    	p(y) & = -\frac{5p_0}{2\gls{Froude}^2}\frac{\left(y\left(T_h^{5/3}-T_c^{5/3}\right)+T_c^{5/3}\right)^{2/5}}{\left(T_h^{5/3}-T_c^{5/3}\right)}+C,\\
    	T(y) & = \left(C_3 - \frac{5}{3}C_4 y\right)^{3/5}.
    \end{align}\label{eq:AllCouetteSolutions}
\end{subequations}
\endgroup
Where the constants $C_1$, $C_2$, $C_3$ and $C_4$ are determined using the boundary conditions on the upper and lower walls and are given by
\begin{subequations}
\begin{align}
	C_1 & = \frac{\left(\frac{T_c^{5/3}}{T_h^{5/3}-T_c^{5/3}}\right)^{3/5}}{\left(\frac{T_c^{5/3}}{T_h^{5/3}-T_c^{5/3}}\right)^{3/5}-\left(\frac{T_h^{5/3}}{T_h^{5/3}-T_c^{5/3}}\right)^{3/5}} \\
	C_2 & = \frac{1}{\left(\frac{T_h^{5/3}}{T_h^{5/3}-T_c^{5/3}}\right)^{3/5}-\left(\frac{T_c^{5/3}}{T_h^{5/3}-T_c^{5/3}}\right)^{3/5}}                                                        \\
	C_3 & = T_c^{5/3},                                                                                                                                                                         \\
	C_4 & = \frac{3}{5}\left(T_c^{5/3}-T_h^{5/3}\right)
\end{align}
\end{subequations}
and $C$ is a real-valued constant determined by an arbitrary zero level for the pressure. The dimensionless parameters are set as $\gls{Reynolds} = 10$ and $\gls{Prandtl} =0.71$, $T_h = 1.6$, and $T_c = 0.4$ for all calculations. The system is considered open and the thermodynamic pressure is $p_0 =1.0$. The Froude number is calculated as
\begin{equation}
	\text{Fr} = \left( \frac{2\text{Pr}(T_h-T_c)}{(T_h+T_c)}\right)^{1/2}.\label{eq:FroudeNumber1}
\end{equation}%
\begin{center}
	\begin{figure}[bt]
		\pgfplotsset{
			group/xticklabels at=edge bottom,
		}
		\inputtikz{CouetteSolution1}
		\inputtikz{CouetteSolution2}
		\inputtikz{CouetteSolution3}
		\caption{Solution of the Couette flow with vertical temperature gradient using a Power-Law.}\label{fig:CouetteSolution}
	\end{figure}
\end{center}%
A derivation for \cref{eq:FroudeNumber1} will be given in \cref{ss:DHC}. In \cref{fig:CouetteSolution} the solutions for the velocity, pressure and temperature are shown. The results are for a mesh with $26\times26$ elements and a polynomial degree of three for $u$ and $T$, and a polynomial degree of two for $p$. The vertical velocity $v$ is zero everywhere. 
\subsubsection{h-convergence study}
The convergence properties of the DG method for this non-isothermal system were studied using the analytical solution described before. The domain is discretized and solved in uniform Cartesian meshes with $16\times16$, $32\times32$, $64\times64$ and $128\times128$ elements. The polynomial degrees for the velocity and temperature are changed from one to four and for the pressure from zero to three. The convergence criterion described in \cref{ssec:TerminationCriterion} was used for all calculations. The analytical solutions given by \cref{eq:AllCouetteSolutions} are used as Dirichlet boundary conditions on all the boundaries of the domain. The global error is calculated against the analytical solution using a $L^2$ norm. 
In \cref{fig:ConvergenceCFTD} the results of the h-convergence study are shown.  Recall that, for increasing polynomial order, the expected order of convergence is given by the slope of the line curve when cell length and errors are presented in a log-log plot. Due to the mixed-order formulation used, the slopes should be equal to $k$ for the pressure and equal to $k+1$ for all other variables, which is possible to observe for all variables.% It is observed that the expected convergence rates are reached for all variables. 
\begin{figure}[t!]
	\centering
	\pgfplotsset{width=0.34\textwidth, compat=1.3}
	\inputtikz{ConvergenceCFTD}
	\caption{Convergence study of the Couette-flow with temperature difference. A power-law is used for the transport parameters.}\label{fig:ConvergenceCFTD}
\end{figure}
\subsubsection{Comparison with SIMPLE}
As mentioned before, a solver for solving low-Mach number flows based on the SIMPLE algorithm presented in \textcite{kleinHighorderDiscontinuousGalerkin2016} has already been developed and implemented within the BoSSS framework.
Although the solver was validated and shown to be useful for a wide variety of test cases, there were also disadvantages inherent to the SIMPLE algorithm. For example,
within the solution algorithm, under-relaxed Picard-type iterations are used to search for a solution. This usually requires some prior knowledge from the user in order to select suitable relaxation factor values that provide stability to the algorithm, but at the same time do not slow down the computation substantially. 
The intention of this subsection is to show a comparison of runtimes of the calculation of the Couette flow with vertical temperature gradient between the DG-SIMPLE algorithm \parencite{kleinHighorderDiscontinuousGalerkin2016} and the XNSEC solver. Calculations were performed on uniform Cartesian meshes with $16\times16$, $32\times32$, $64\times64$ and $128\times128$ elements, and with varying polynomial degrees between one and three for the velocity and temperature, and between zero and two for the pressure. All calculations where initialized with a zero velocity and pressure field, and with a temperature equal to one in the whole domain. All calculations were performed in a single core. The convergence criteria of the nonlinear solver is set to $10^{-8}$ for both solvers. The under-relaxation factors for the SIMPLE algorithm are set for all calculations to 0.8, 0.5 and 1.0 for the velocity, pressure and temperature, respectively.

In \cref{fig:RuntimeComparison}, a comparison of the runtimes from both solvers is presented. It is evident that the runtimes of the SIMPLE algorithm are generally higher for almost all the cases studied. Only for systems with a small number of cells does the solver using the SIMPLE algorithm outperform the XNSEC solver. Additionally, it is observed that the runtimes are comparable only for low polynomial degrees. Furthermore, it is noticeable that the runtimes are similar only for low polynomial degrees. It can be observed  how the runtime of the simulations with the XNSEC code seems to scale linearly with the polynomial degree. In the case of the SIMPLE solver, the scaling is much more unfavourable, and the runtime increases dramatically as the polynomial degree increases. Obviously, the under-relaxation parameters used within the SIMPLE algorithm have an influence on the calculation times and an appropriate selection of them could decrease the runtimes. This is a clear disadvantage, since the adequate selection of under-relaxation factors is highly problem dependent and requires previous expertise from the user. On the other hand, the globalized Newton method used by the XNSEC solver avoid this problem by using a more sophisticated method and heuristics in order to find a better path to the solution, which does not require user-defined parameters.
\begin{figure}	
\centering
    \inputtikz{RuntimeComparisonCouetteFlow}
	\caption{Runtime comparison of the DG-SIMPLE solver and the XNSEC solver for the Couette flow with vertical temperature gradient for different polynomial degrees $k$ and number of cells.}
	\label{fig:RuntimeComparison}
\end{figure}
\subsection{Differentially heated cavity problem}\label{ss:DHC}
\begin{figure}[bt]
	\begin{center}
		\def\svgwidth{0.53\textwidth}
		\import{./plots/}{diffheatedCavityGeometry.pdf_tex}
		\caption{Schematic representation of the differentially heated cavity problem.}
		\label{DHCGeom}
	\end{center}
\end{figure}
The differentially heated cavity problem is a classical benchmark case that is often used to assess the ability of numerical codes to simulate variable density flows \parencite{paillereComparisonLowMach2000,vierendeelsBenchmarkSolutionsNatural2003,tyliszczakProjectionMethodHighorder2014}.
The test case has the particularity that deals with a closed system, where the thermodynamic pressure $p_0$ is a parameter that must be adjusted so that the mass is conserved. The thermodynamic pressure $p_0$ determines the density field, which in turn appears in the momentum equation and the energy equation, making it necessary to use an adequate algorithm to solve the system. This point presents a special difficulty for the solution, since the calculation of $p_0$ requires knowledge of the temperature field on the whole computational domain, inducing a global coupling of the variables. 

The system is a fully enclosed two-dimensional square cavity filled with fluid.  A sketch of the problem is shown in \cref{DHCGeom}. The left and right walls of the cavity have constant temperatures $\hat{T}_h$ and $\hat{T}_c$, respectively, with $\hat{T}_h >\hat{T}_c$, and the top and bottom walls are adiabatic. A gravity field induces fluid movement because of density differences caused by the difference in temperature between the hot and cold walls.
The natural convection phenomenon is characterized by the Rayleigh number, defined as
\begin{equation}\label{eq:Rayleigh}
	\text{Ra} = \Prandtl \frac{\hat g \RefVal{\rho}^2(\hat T_h-\hat T_c) \RefVal{L}^3}{\RefVal{T}\RefVal{\mu}^2},
\end{equation}
For small values of $\text{Ra}$, conduction dominates the heat transfer process, and a boundary layer covers the entire domain. On the other hand, large values of $\text{Ra}$ represent a flow dominated by convection. When the number $\text{Ra}$ increases, a thinner boundary layer is formed.
Following \textcite{vierendeelsBenchmarkSolutionsNatural2003}, a reference velocity for buoyancy-driven flows can be defined as
\begin{equation}
	\RefVal{u} = \frac{\sqrt{\text{Ra}} \RefVal{\mu}}{\RefVal{\rho}\RefVal{L}}.
\end{equation}
The Rayleigh number is then related to the Reynolds number according to
\begin{equation}
	\text{Re} = \sqrt{\text{Ra}}.
\end{equation}
Thus, it is sufficient to select a $\Reynolds$ number in the simulation, fixing the value of the $\text{Ra}$ number. The driving temperature difference $(\hat T_h - \hat T_c)$ appearing in \cref{eq:Rayleigh} can be represented as a nondimensional parameter:
\begin{equation}\label{eq:nondimensionalTemperature}
	\varepsilon = \frac{\hat T_h - \hat T_c}{2\RefVal{T}}.
\end{equation}
Using these definitions, the Froude number can be calculated as
\begin{equation}
	\Froude = \sqrt{\Prandtl 2 \varepsilon}.
\end{equation}
The results of the XNSEC solver are compared with those of the reference solution for $\RefVal{T} = 600\si{K}$ and $\varepsilon = 0.6$. All calculations assume a constant Prandtl number equal to 0.71. The dependence of viscosity and heat conductivity on temperature is calculated using Sutherland's law (\cref{eq:nondim_sutherland}). The nondimensional length of the cavity is $L=1$. The nondimensional temperatures $T_h$ and $T_c$ are set to 1.6 and 0.4, respectively. The nondimensional equation of state (\cref{eq:ideal_gas}) depends only on the temperature and reduces to
\begin{equation}
	\rho = \frac{p_0}{T}.
\end{equation}
The thermodynamic pressure $p_0$ in a closed system must be adjusted to ensure mass conservation. For a closed system is given by
\begin{equation}
	p_0 =\frac{\int_\Omega \rho_0\text{d}V}{\int_\Omega \frac{1}{T}\text{d}V}= \frac{m_0}{\int_\Omega \frac{1}{T}\text{d}V}, \label{eq:p0Condition}
\end{equation}
where $\Omega$ represents the complete closed domain. The initial mass of the system $m_0$ is constant and is set $m_0 = 1.0$. Note that the thermodynamic pressure is a parameter with a dependence on the temperature of the entire domain. This makes necessary the use of an iterative solution algorithm, so that the solution obtained respects the conservation of mass. Within the solution algorithm of the XNSEC solver, \cref{eq:p0Condition} is used to update the value of the thermodynamic pressure after each Newton iteration.
The average Nusselt number is defined for a given wall $\Gamma$  as
\begin{equation}\label{eq:Nusselt}
	\text{Nu}_\Gamma = \frac{1}{T_h - T_c}\int_{\Gamma} k \pfrac{T}{x}\text{d}y.
\end{equation}
%%%%%%%%%%%%%%%
%%% Result comparison
%%%%%%%%%%%%%%%
\subsubsection{Comparison of results with the benchmark solution}

Here a comparison of the results obtained with the XNSEC solver and the results presented in the work of \textcite{vierendeelsBenchmarkSolutionsNatural2003} is made. They solved the fully compressible Navier-Stokes equations on a stretched grid with $1024\times1024$ using a finite-volume method with quadratic convergence, providing very accurate results that can be used as reference.
The benchmark results are presented for $\text{Ra} = \{10^2,10^3,10^4,10^5,10^6,10^7\}$. In this range of Rayleigh numbers, the problem has a steady-state solution. 
The cavity is represented by the domain $[0,1]\times[0,1]$. For all calculations in this subsection, the simulations are done with a polynomial degree of four for both velocitiy components and temperature and three for the pressure. The mesh is in an equidistant $128\times128$ mesh.

Preliminar calculations showed that for cases up to $\text{Ra} = 10^5$ the solution of the system using Newton's method presented in \cref{sec:newton} is possible without further modifications, while for higher values the algorithm couldnt find a solution and stagnates after certain number of iterations. The homotopy strategy mentioned in \cref{sec:CompMethodology} is used to overcome this problem and obtain solutions for higher Rayleigh numbers. Here, the Reynolds number is selected as the homotopy parameter and continuously increased until the desired value is reached.

In \cref{fig:TempProfile,fig:VelocityXProfile,fig:VelocityYProfile} the temperature and velocity profiles across the cavity for different Rayleigh numbers are shown. The profiles calculated with the XNSEC solver agree closely to the benchmark solution. As expected, an increase of the acceleration of the fluid in the vicinity of the walls for increasing Rayleigh numbers is observed.
\begin{figure}[h]
	\centering
	\pgfplotsset{width=0.3 \textwidth, compat=1.3}
	\inputtikz{HSCStreamlines}
	\caption{Streamlines of the heated cavity configuration with $\epsilon = 0.6$ for different Reynold numbers.}\label{fig:HSCStreamlines}
\end{figure}


\begin{figure}[h]
	\centering
	\pgfplotsset{width=0.22\textwidth, compat=1.3} 
	\inputtikz{TempProfile}
	\caption[Temperature profiles for the differentially heated square cavity along different vertical levels.]{Temperature profiles for the differentially heated square cavity along different vertical levels. Solid lines represent the XNSEC solver solution and marks the benchmark solution.}
	\label{fig:TempProfile}
\end{figure}
%
\begin{figure}[h]
	\centering
	\pgfplotsset{width=0.22\textwidth, compat=1.3}
	\inputtikz{VelocityXProfile}
	\caption[Profiles of the x-velocity component for the differentially heated square cavity along the vertical line $x=0.5$.]{Profiles of the x-velocity component for the differentially heated square cavity along the vertical line $x=0.5$. Solid lines represent the XNSEC solver solution and marks the benchmark solution.}
	\label{fig:VelocityXProfile}
\end{figure}
%
\begin{figure}[h]
	\centering
	\pgfplotsset{width=0.22\textwidth, compat=1.3}
	\inputtikz{VelocityYProfile}
	\caption[Profiles of the y-velocity component for the differentially heated square cavity along the horizontal line $y=0.5$.]{Profiles of the y-velocity component for the differentially heated square cavity along the horizontal line $y=0.5$. Solid lines represents the XNSEC solver solution and marks the benchmark solution.}
	\label{fig:VelocityYProfile}
\end{figure}
\FloatBarrier
A comparison of the thermodynamic pressure and the Nusselt numbers to the benchmark solution was also made. The results are shown in \cref{tab:p0_Nu_Results}.  The thermodynamic pressure is obtained from \cref{eq:p0Condition}, and the average Nusselt number is calculated with \cref{eq:Nusselt}. The results obtained with the XNSEC solver agree very well with the reference results, as can be seen for the thermodynamic pressure, which differs at most in the fourth decimal place. Note that the average Nusselt number of the heated wall $(\text{Nu}_\text{h})$ and the Nusselt number of the cold wall $(\text{Nu}_\text{c})$ are different. As the Rayleigh number grows, this discrepancy becomes larger, hinting that, at such Rayleigh numbers, the mesh used is not refined enough to adequately represent the thin boundary layer and more complex flow structures appearing in high-Rayleigh number cases. While for an energy conservative system $\text{Nu}_\text{h}$ and $\text{Nu}_c$ should be equal, for the DG-formulation this is not the case, since conservation is only ensured locally and the global values can differ. This discrepancy can be seen as a measure of the discretization error of the DG formulation and should decrease as the mesh resolution increases. This point will be discussed in the next section.
\begin{table}[t!]
	\begin{center}
		\begin{tabular}{cccccc}
			\hline
			Rayleigh                           & $p_0$  & $p_{0,\text{ref}}$ & $\text{Nu}_{h}$ & $\text{Nu}_{c}$ & $\text{Nu}_{\text{ref}}$ \\ \hline
			\parbox[0pt][13pt][c]{0pt}{}$10^2$ & 0.9574 & 0.9573             & 0.9787          & 0.9787          & 0.9787                   \\
			$10^3$                             & 0.9381 & 0.9381             & 1.1077          & 1.1077          & 1.1077                   \\
			$10^4$                             & 0.9146 & 0.9146             & 2.2180          & 2.2174          & 2.2180                   \\
			$10^5$                             & 0.9220 & 0.9220             & 4.4801          & 4.4796          & 4.4800                   \\
			$10^6$                             & 0.9245 & 0.9245             & 8.6866          & 8.6791          & 8.6870                   \\
			$10^7$                             & 0.9225 & 0.9226             & 16.2411         & 16.1700         & 16.2400                  \\ \hline
		\end{tabular}
	\end{center}
	\caption[Differentially heated cavity: Results of Nusselt number and Thermodynamic pressure]{Comparison of calculated Nusselt numbers of the hot and cold wall and Thermodynamic pressure $p_0$ reported values by \textcite{vierendeelsBenchmarkSolutionsNatural2003} for the differentially heated cavity.}
	\label{tab:p0_Nu_Results}
\end{table}
%%%%%%%%%%%%%%%
%%% Convergence study
%%%%%%%%%%%%%%%


\subsubsection{Convergence study}\label{ssec:ConvStudyHeatedCavity}
An $h-$convergence study of the XNSEC solver was conducted using the heated cavity configuration. Calculations were performed for polynomial degrees $k = {1,2,3,4}$ and equidistant regular meshes with, respectively, $8\times8$, $16\times16$, $32\times32$, $64\times64$, $128\times128$ and $256\times256$ elements.  The $L^2$ -Norm was used to calculate errors against the solution in the finest mesh. The results of the $h$-convergence study for varying polynomial orders $k$ are shown in \cref{fig:ConvergenceDHC}. It is observed how the convergence rates scale approximately as $k+1$. Interestingly, for $k=2$ the rates are higher than expected. On the other hand, some degeneration is observed in convergence rates for $k = 4$. This strange behavior can be explained if one considers that the heated cavity presents a singular behavior at the corners (similar to the problem previously exposed for the lid-driven cavity), which causes global pollution in the convergence behavior of the algorithm. 
 
As discussed in the previous section, the difference in the average values of the Nusselt number on the hot wall $\text{Nu}_\text{h}$  and the cold wall $\text{Nu}_\text{c}$ is a direct consequence of the spatial discretization error and should decrease for finer meshes. In \cref{fig:NusseltStudy} the convergence behavior of the Nusselt number is presented for different polynomial degrees $k$, different number of elements and for two Rayleigh numbers. As expected, it can be observed that this discrepancy is smaller when a larger number of elements is used. It can also be seen that  $\text{Nu}_\text{h}$ reaches the expected solution of cells for a much smaller number of elements. This can be explained if one thinks that more complex phenomena take place near the cold wall (see \cref{fig:HSCStreamlines}), which makes necessary a finer mesh resolution in that area.



\begin{figure}[tb]
	\centering
	\pgfplotsset{width=0.34\textwidth, compat=1.3}
	\inputtikz{ConvergenceDHC}
	\caption{Convergence study of the differentially heated cavity problem for $\text{Ra} = 10^3$.}\label{fig:ConvergenceDHC}
\end{figure}
\begin{figure}[tb]
	\centering
	\inputtikz{NusseltStudy}
	\caption[Nusselt numbers of the differentially heated square cavity at the hot wall ($\text{Nu}_h$) and the cold wall ($\text{Nu}_c$) for different number of cells and polynomial order $k$.]{Nusselt numbers of the differentially heated square cavity at the hot wall ($\text{Nu}_h$) and the cold wall ($\text{Nu}_c$) for different number of cells and polynomial order $k$. The reference values from \textcite{vierendeelsBenchmarkSolutionsNatural2003} are shown with dashed lines.}\label{fig:NusseltStudy}
\end{figure}
\FloatBarrier
\subsubsection{Influence of the penalty factor}
\begin{table}[h]
\centering
\begin{tabular}{lllllll}
	\hline \vspace{0.1cm}
		$\eta_0$                  &  $k$ & DOFs& $\text{Nu}_c$ &  $\frac{\text{Nu}_c - \text{Nu}_{c,\text{ref}}} {\text{Nu}_{c,\text{ref}}}\times 10^2 $   & $p_0$    & $\frac{p_0 - p_{0,\textbf{ref}}} {p_{0,\textbf{ref} }}\times 10^4 $ \\ \hline
\multirow{3}{*}{0.01} & 2 & 6804 & 0.549483* & 50.39424 & 0.899757* & 408.7292 \\
                      & 3 & 7056 & 0.722593* & 34.76637 & 0.936085* & 21.48436 \\
                      & 4 & 6655 & -0.50954* & 146      & 1.016691* & 837.7683 \\ \hline
\multirow{3}{*}{1}    & 2 & 6804 & 1.090047 & 1.593667 & 0.938192 & 0.980674 \\
                      & 3 & 7056 & 1.102072 & 0.508037 & 0.938057 & 0.453674 \\
                      & 4 & 6655 & 1.105225 & 0.22348  & 0.938046 & 0.570494 \\ \hline
\multirow{3}{*}{4}    & 2 & 6804 & 1.089332 & 1.65817  & 0.93843  & 3.521926 \\
                      & 3 & 7056 & 1.102261 & 0.491027 & 0.938076 & 0.25384  \\
                      & 4 & 6655 & 1.105359 & 0.211372 & 0.938047 & 0.561709 \\ \hline
\multirow{3}{*}{16}   & 2 & 6804 & 1.08694  & 1.874166 & 0.939109 & 10.75641 \\
                      & 3 & 7056 & 1.102266 & 0.490563 & 0.938124 & 0.251627 \\
                      & 4 & 6655 & 1.105439 & 0.204153 & 0.93805  & 0.537265 \\ \hline
\end{tabular}
\caption[Thermodynamic pressure and cold-side Nusselt number for different penalty safety factors in a heated cavity with Ra $=10^3$.]{Thermodynamic pressure and cold-side Nusselt number for different penalty safety factors in a heated cavity with Ra $=10^3$. Values marked with an asterisk are from problems not converged after 100 iterations.} \label{fig:EtaInfluence}
\end{table}
A point not still discussed is the choice of the safety parameter $\eta_0$ of the penalty terms from the SIP discretization (see \cref{eq:PenaltyFactor}).  \Cref{fig:EtaInfluence} shows results obtained for Ra $=10^3$, for different polynomial degrees and penalty safety factor.  For the tests presented here, the penalty terms of the diffusive terms from the momentum and energy equations are considered equal. Furthermore, the number of elements in the mesh is selected in such a way that the number of degrees of freedom remains approximately constant for each simulation. 

It is possible to see that the penalty safety factor (and therefore the penalty term) can have a great influence on the solution. If the value chosen is very small, as in the case of the table for $\eta_0 = 0.01$, the algorithm is not able to find a solution. On the other hand, if the chosen value is too high, the error also increases. It can be concluded that an optimal value for the penalty factor exists.

It is also noticeable that, maintaining a constant penalty safety factor, increasing the polynomial degree for an approximately constant number of DOFs gives an improvement in the results compared to the literature. Although for this testcase the effect of the penalty factor on the solution is not very large, the effect could be considerable, especially when dealing with more complex geometries and coarser meshes. The value $\eta_0 = 4$ has shown to be a value that gives stability to the scheme and is used for all simulations in this thesis, as already has been done in many works \parencite{krauseIncompressibleImmersedBoundary2017,kummerExtendedDiscontinuousGalerkin2017,smudamartinDirectNumericalSimulation2021} and is used for all calculations in this work.
 
The results presented in this section allows to conclude that the implemented solver is capable of dealing with flows with variable densities, and in particular in closed spaces. Additionally, it was observed that even for this complex test, convergence properties close to those expected from the DG method are obtained. Until this point only systems with a steady state solution were treated. Later in \cref{ssec:FlowCircCyl} the ability of the solver to compute flows with varying densities in non-steady state will be shown.

\subsection{Poiseuille–Rayleigh–Bénard instability in a channel}
\blindtext[5]
\subsection{Flow around a heated cylinder}
\subsubsection{Square cylinder}
\blindtext[5]
\subsubsection{Circle? cylinder}
\blindtext[5]
%%%%%%%%
%%%%%%%%
\section{Multi-component non-isothermal cases}\label{sec:MultCompNonIsothermCase}
%%%%%%%%
%%%%%%%%
Later in in \cref{ss:CDF} and \cref{ss:CoFlowFlame} two different configurations for reactive flows are presented.
In the following, we show results of the simulations of two test cases with combustion, namely the counterflow diffusion flame and the chambered diffusion flame. For both cases, the solution of the flame sheet problem described in \cref{ssec:FlameSheet} is calculated first. This solution is used subsequently as initial estimates for the solution of the finite chemistry rate problem (c.f. \cref{ssec:NonDimLowMachEquations}). In all test cases presented in this section, a smoothing parameter $\sigma = 50$ was used (c.f. \cref{ssec:FlameSheet}). For all test cases methane combustion according to the one-step model shown in \cref{sec:ChemModel} is considered. The mass fraction transport \cref{eq:LowMachMassBalance} is solved for the species \ch{CH4}, \ch{O2}, \ch{CO2} and \ch{H2O}, thus $\vec{Y}' = \left(Y_{\ch{CH4}},Y_{\ch{O2}},Y_{\ch{CO2}},Y_{\ch{H2O}}\right)$. The nitrogen mass fraction $Y_{\ch{N2}}$ is calculated according to \cref{eq:MassFractionConstraint}.



\subsection{Co-flow laminar diffusion flame}\label{ssec:coflowFlame}
%C:\Users\jfgj8\AppData\Local\BoSSS\plots\sessions\CoFlowFlameIntentoParaTesis2__Full_CoFlowFlamerP3K12smoothfactor0velMult2__e1120753-addc-4671-a7d4-7443132981fb
The co-flowing flame configuration is used as a first test to assess the behavior of the solver for reactive flows applications. It basically consists of two concentric ducts that emit fuel and oxidant into the system, creating a flame. This configuration has been widely studied. In the seminal work of \cite{burkeDiffusionFlames1928} analytical expressions for the flame height and flame shape are obtained by studying a very simplified problem (constant density and velocity field, infinitely fast chemistry, among others). Later, \cite{smookeNumericalModelingAxisymmetric1992} and later works solved this configuration using a 2D-axisymmetric system and also used the flame sheet estimates to find adequate initial conditions for their Newton algorithm. 

It should be noted that the solution of the axisymmetric system of equations presents numerical difficulties that are not the main concern of the present work. For this reason, it was decided to solve a system with similar characteristics but which is possible to represent in Cartesian coordinates. This is possible if an infinitely long slot burner is considered. A schematic diagram of the configuration can be seen in \cref{fig:CoFlowSketch}. Note that the system also includes the tips that separate both oxidizer and fuel inlets. Altought the system is clearly symmetric around the axis $x = 0$, no symmetry assumption is made and the whole domain is considered.   The lengths are set as $X_1 = \SI{0.635}{\centi \meter}$, $X_2 = \SI{0.762}{\centi \meter}$, and $X_3 = \SI{7.747}{\centi \meter}$. The lengths $R$ and $L$ are set with arbitrarily large values,  $R = \SI{2.54}{\centi \meter}$ and $L = \SI{40}{\centi \meter}$. Setting a higher value of $L$ did not significantly change the calculations. 
The inlet boundary conditions are set as: 
\begin{itemize}
\item Fuel Inlet: $\{\forall (x,y): y = -R \land x \in [-X_1,X_1]\} $ \\
\begin{equation*}
	u = 0,\qquad v= v_F(x)), \qquad T = T^F, \qquad \vec{Y}' = (Y^F_{\ch{CH4}},0,0,0)
\end{equation*}
\item Oxidizer inlet: $\{\forall (x,y): y = -R \land x \in [-X_3,-X_2]\cup[X_2,X_3]\}$\\
\begin{equation*}
 	u = 0,\qquad v= v_O, \qquad T = T^O, \qquad \vec{Y}' = (0,Y^O_{\ch{O2}},0,0)
\end{equation*}
%\item Pressure outlet  $\forall (x,y): y = -R \land x \in [-X_3,-X_2]\cup[X_2,X_3]$\\
%\item Adiabatic wall:
\end{itemize}
 where $v_F$ is a a parabolic profile given by
\begin{equation}
	v_F(x) = \left[1-\left(\frac{x}{X_1}\right)^2\right]v_m
\end{equation}
with $v^F_m = \SI{2.427}{\centi \meter \per \second}$. The oxidizer enters the system as a plug flow with a constant velocity of $v_O = \SI{4.1}{\centi \meter \per \second}$. The inlet temperatures of both streams is  $T^F = T^O = \SI{300}{K}$. Combustion diluted methane on air is considered, with $Y^F_{\ch{CH4}} = 0.2$, $Y^F_{\ch{N2}} = 0.8$, $Y^O_{\ch{O2}} = 0.23$ and $Y^O_{\ch{N2}} = 0.77$. The superindexes $F$ and $O$ represent the fuel and oxidizer inlet respectively. The pressure outlet boundary condition is the same as \cref{eq:bc_O}. Finally, the boundary conditions at the tips correspond to adiabatic walls, which are defined as in \cref{eq:bc_dn}, with $\vec{u}_{\text{D}} = (0,0)$.             

The variables are nondimensionalized in the usual way. The reference length $\RefVal{L} = \hat{X}_1$ and the reference velocity $\RefVal{u} =v^F_m$. The reference temperature is $\RefVal{T} = T^F$, which gives to the nondimensional numbers $\Reynolds = 16.5$ $Da = 4.3\cdot 10^9$. The Prandtl number is assumed constant with $\Prandtl = 0.71$. For this calculation all lewis numbers are set to unity. Gravity effects are not taken into acount.
\begin{figure}[t]
	\centering
	\def\svgwidth{0.43\textwidth}
	\subcaptionbox{Sketch\label{fig:CoFlowSketch}}{
		\import{./plots/}{CoFlowSketch_withBC.pdf_tex}\vspace{0.5cm}
	}
	\qquad\quad
	\def\svgwidth{0.35\textwidth}
	\subcaptionbox{Refined mesh \label{fig:CoFlowMesh}}{
		\vspace{1.2cm}
		\import{./plots/}{CoFlowMesh.pdf_tex}
	}
	
	\caption{Geometry of a coflowing flame configuration (not to scale).} \label{fig:CoFlowGeometry}
\end{figure}
The area where the chemical reaction takes place is usually a thin region, whose thickness is defined by the availability of reactants. It is of critical importance for the numerical simulation to have an adequate mesh that allows to solve the flame appropriately. Not doing so can lead to non-physical effects and the apparition of unwanted numerical artifacts. 

To avoid over-resolving in zones where actually no reaction is taking place, an adaptive mesh refinement strategy (see \cref{ssec:MeshRefinement} ) in a pseudo-time-stepping framework was used.  Here, a suitable strategy for choosing cells to be refined. Two refinement strategies have been used for this simulation. First, prior to any type of calculation, the base mesh is refined in the vicinity of the tips.  Second, several pseudo-timesteps are performed for the flame-sheet calculation. After each of these is refined in the vicinity of the flame sheet, i.e., in the cells where $z = z_{\text{st}}$. 
 


For reactive flows, this strategy is based on the variable $\omega$.
Before each refinement step the values of $\omega$ are normalized by the highest value of the domain, and according to this normalized effects the cells are refined. %In Figure \ref{fig:AMR} the refined meshes can be seen. The legend of the pseudocolor plots is not shown, because for each plot the magnitude scales are different, and just the normalized value accounts for the AMR. 


\begin{figure}[t]
	\centering
	\pgfplotsset{width=0.6\textwidth, compat=1.3}
	\inputtikz{CoFlowFlameFigTemperature}	
	\inputtikz{CoFlowFlameFigVelMag}	
	\par\bigskip
	\inputtikz{CoFlowFlameFigMF0}	
	\inputtikz{CoFlowFlameFigMF1}	
	\caption{Solution field of a coflowing flame configuration.} \label{fig:CoFlowFlameFig}
\end{figure}
\begin{figure}[t!]
	\centering
	\pgfplotsset{
		compat=1.3,
		tick align = outside,
		yticklabel style={/pgf/number format/fixed},
	}
	\inputtikz{CoFlow_ConvergenceStory}
	\caption{Typical convergence history of a diffusion flame in the coflowing flame configuration with mesh refinement.}
	\label{fig:CoFlow_ConvergenceStory}
\end{figure}
\FloatBarrier

\subsection[Counterflow diffusion flame]{Counterflow diffusion flame \footnotemark}\label{ss:CDF}
\footnotetext{Modified version from \cite{gutierrez-jorqueraFullyCoupledHighorder2022}}
\begin{figure}[h!]
	\begin{center}
		\def\svgwidth{0.8\textwidth}
		\import{./plots/}{CounterDiffusionFlame_sketch_rotated2.pdf_tex}
		\caption{Schematic representation (not to scale) of the counterflow diffusion flame configuration.}
		\label{fig:CDFScheme}
	\end{center}
\end{figure}

%This test case is the main prototype flame for diffusion regimes \cite{poinsotTheoreticalNumericalCombustion2005}.
The counterflow diffusion flame is a canonical configuration used to study the structure of nonpremixed flames. This simple configuration has been a subject of study for decades because it provides a simple way of creating a strained diffusion flame, which proves to be useful when studying the flame structure, extinction limits or production of pollutants of flames \citep{pandyaStructureFlatCounterFlow1964,spaldingTheoryMixingChemical1961,keyesFlameSheetStarting1987, leeTwodimensionalDirectNumerical2000}. 

The counterflow diffusion flame consists of two oppositely situated jets. The fuel (possibly mixed with some inert component, such as nitrogen) is fed into the system by one of the jets, while the other jet feeds oxidyzer to the system, thereby establishing a stagnation point flow. On contact and after ignition, the reactants produce a flame that is located in the vicinity of the stagnation plane. A diagram of the setup can be seen in \cref{fig:CDFScheme}. In this section, the solution of a steady two-dimensional flame formed in an infinitely long slot burner will be treated. Simirlarly to the coflow configuration treated before, the infintely long slot burner configuration can be calculated naturally using cartesian coordinates.

First, as a means of verificating the solver for combustion applications, the results obtained with the XNSEC-solver for steady two-dimensional counterflow diffusion flame are compared with the solution of a simplified system of equations for a steady and quasi one-dimensional flame. Later, the influence of the inlet velocities on the maxmimum temperature is studied and finally some remarks concerning the convergence behaviour of the case counter diffusion flame are given. 
\subsubsection{The one-dimensional diffusion flame}
By assuming an infinite injector diameter, a self-similar solution and by neglecting the radial gradients of the scalar variables along the axis of symmetry, it is possible to reduce the three-dimensional governing equations to a one-dimensional formulation along the stagnation streamline $y = 0$ (see the textbook from \cite{keeChemicallyReactingFlow2003} for the derivation).  The governing equations for a steady planar stagnation flow reduce to
\begin{subequations}
\begin{gather}
	%%%%%%%%%%
	\frac{\partial\hat\rho \hat v}{\partial \hat x} +  \hat \rho \hat U = 0\label{eq:OneDimCont}\\ %
	%%%%%%%%%%
	\hat \rho \hat v \frac{\partial \hat U}{\partial \hat x} + \hat \rho \hat U^2 =
	- \hat \Lambda
	+ \frac{\partial}{\partial \hat x}\left(\hat \mu \frac{\partial \hat U}{\partial \hat x}\right)\label{eq:OneDimMom}\\ %
	%%%%%%%%%%
	\hat\rho \hat c_p \hat v \frac{\partial \hat T}{\partial\hat x} =
	\frac{\partial}{\partial \hat x}\left( \hat \lambda \frac{\partial \hat T}{\partial \hat x}\right)
	+\hat\heatRelease~\hat{\mathcal{Q}}\label{eq:OneDimTemp}\\
	%%%%%%%%%%%
	\hat\rho \hat v \frac{\partial Y_k}{\partial \hat x} = 
	\frac{\partial}{\partial \hat x}\left(\hat \rho \hat D \frac{\partial Y_k}{\partial \hat x}\right)
	+ \hat W_k \stoicCoef_k \hat{\mathcal{Q}} \quad (k = 1, \dots,~N - 1) \label{eq:OneDimMF}
	%%%%%%%%%%%
\end{gather}\label{eqs:OneDimEquations}%
\end{subequations}
\cref{eqs:OneDimEquations}.
where $\hat U$ is the scaled velocity and $\hat \Lambda$ is the radial pressure curvature, which is an eigenvalue independant of $x$. Again, the hat sign represent dimensional variables. The equations are written assuming a valid Fick's law and an one-step combustion model. The system of equations needs to be solved for $\hat v$, $\hat U$, $\hat T$ and for $Y_k$, $ (k = 1, \dots,~N - 1)$.  Aditionally an equation of state and expressions for the heat capacity $\hat c_p$ and transport parameters $\hat \mu, \hat \lambda, (\hat \rho \hat D)$ are needed. This formulation is very well known and often used for analysis of flame structure, determination of extintion points, to mention a few.

In order to assess the ability of the XNSEC-solver to simulate such a system, the solution obtained for a two dimentional configuration is compared with the solution of the quasi one-dimensional equations solved with \lstinline|BVP4|, a fourth order finite difference boundary value problem solver provided by \lstinline|MATLAB| \citep{kierzenkaBVPSolverBased2001}. The \lstinline|BVP4| solver provides automatic meshing and error control based on the residuals of the solution, allowing to create with relative easyness a code that solves the one-dimensional equations.


It is important to mention some points regarding the solution of these equations using the \lstinline|BVP4| solver. Analogous to the problem mentioned in \cref{ssec:MethodCombustion}, the solution of the system of  \crefrange{eq:OneDimCont}{eq:OneDimMF} presents the particularity that it is possible to find multiple solutions. One of them is clearly the cold solution and another solution is the burning one. The same idea mentioned in  \cref{ssec:MethodCombustion} is also valid for the quasi one-dimensional configuration. In particular this means, that as a first step for finding a converged solution of \crefrange{eq:OneDimCont}{eq:OneDimMF} is to solve the system
\begin{subequations}
\begin{gather}
	%%%%%%%%%%
\frac{\partial \hat \rho \hat v}{\partial \hat x} +  \hat \rho \hat U = 0\\ \label{eq:OneDimCont2}%
%%%%%%%%%%
\hat \rho \hat v \frac{\partial \hat U}{\partial \hat x} + \hat \rho \hat U^2 =
- \hat \Lambda
+ \frac{\partial}{\partial \hat x}\left(\hat \mu \frac{\partial \hat U}{\partial \hat x}\right)\\ \label{eq:OneDimMom2}%
%%%%%%%%%%
\hat \rho \hat v \frac{\partial Z}{\partial \hat x} = 
\frac{\partial}{\partial \hat x}\left(\hat \rho \hat D \frac{\partial Z}{\partial \hat x}\right)
%%%%%%%%%%%
\end{gather}\label{eqs:OneDimEquationsMixtureFraction}
\end{subequations}
together with the equation of state \cref{eq:ideal_gas} and expressions for the transport parameters. The dependency of the temperature and mass fractions on the mixture fraction $Z$ is given by the Burke-Schuhmann limit. This solution can be used as a initial estimate for the solution of \crefrange{eq:OneDimCont}{eq:OneDimMF}.
An inconvenience is that an initial estimate for the $\hat c_p$ has to be chosen. It was observed that if the value of $\hat c_p$ was chosen too big, the solver delivered solutions without a flame. For the calculations treated here $\hat c_p =\SI{1.3}{\kilo \joule\per\kilogram \kelvin}$ was an adequate value that delivered the ignited solution. 

It was however observed that this flamesheet solution wasnt directly useful as an initial estimation for the solution of the full system of equations. In order to help  the \lstinline|BVP4| solver to find a coverged solution, an intermediate step was necesary. First, the flamesheet solution was used as an initial estimate for the solution of system \cref{eqs:OneDimEquations} and the equation of state and equation for transport parameters but assuming a constant heat capacity. Once the algorithm has found a solution, it can be used for solving the same system but with a variable heat capacity according to \cref{eq:nondim_cpmixture}.


\subsubsection{Set-up of the two-dimensional counterflow diffusion flame}

The combustion of diluted methane with air in a two-dimensional infinitely long slot burner configuration is studied in this part. The solution is obtained by solving the system \cref{eq:all-eq}, making use of the flame-sheet solution as initial estimates. The transport parameters are calculated using Sutherland law with $\hat{S} = \SI{110.5}{\kelvin}$. Gravity effects are not taken into account.  The mixture heat capacity $c_p$ is calculated with \cref{eq:nondim_cpmixture} and using NASA polynomials for the heat capacity of each component.
\begin{table}[b]
	\centering
	\begin{tabular}{lccccc}
		\hline
		& \multicolumn{1}{l}{$\hat v^F_m$ ($\si{\centi \meter \per \second}$)} & \multicolumn{1}{l}{$\hat v^O_m$ ($\si{\centi \meter \per \second}$)} & $a$($\si{\per\second})$ & \multicolumn{1}{l}{$\hat T^F$($\si{\kelvin}$)} & \multicolumn{1}{l}{$\hat T^O$($\si{\kelvin}$)} \\ \hline
		case(a) & 4.85                                                                 & 12.29                                                                & 34                     & 300                                            & 300                                           \\
		case(b) & 12.13                                                                & 30.73                                                                & 76                     & 300                                            & 300                                           \\
		case(c) & 26.69                                                                & 67.62                                                                & 155                    & 300                                            & 300                                           \\ \hline
	\end{tabular}
	\caption{Maximum inlet velocity, strain and temperatures used for the counterflow diffusion flame calculations.}
	\label{tab:cdf_velocities}
\end{table}

For the comparison with the quasi one-dimensional model, three pairs of inlet velocities are considered. They are shown in \cref{tab:cdf_velocities}. Here, $v_m^F$ and $v_m^O$ are the maximal velocity of a parabolic profile for the fuel and air inlets, respectively. Both streams enter at a temperature $\hat T^O = \hat T^F = \SI{300}{\kelvin}$. The mass composition of the fuel inlet is assumed to be  $Y^F_{\ch{CH4}} = 0.2$ and $Y^F_{\ch{N2}} = 0.8$, and the oxidizer inlet is air with  $Y^O_{\ch{O2}} = 0.23$ and $Y^O_{\ch{N2}} = 0.77$. 
Counterflow diffusion flames are usually characterized by the strain rate $a$. Many different definitions for it can be found in the literature \citep{fialaNonpremixedCounterflowFlames2014}. In this work the definition of the strain rate the maximum axial velocity gradient is used. The strains for the three cases mentioned above are $\SI{34}{\per\second}$, $\SI{76}{\per\second}$ and $\SI{155}{\per\second}$, respectively. 

The lengths described in \cref{fig:CDFScheme} are $\hat D = \SI{2}{\centi\meter}$, $\hat H = \SI{2}{\centi\meter}$ and $\hat L = \SI{12}{\centi\meter}$. The variables are non-dimensionalized using $\RefVal{L} = \SI{2}{\centi\meter}$, $\RefVal{T} = \SI{300}{\kelvin}$ and $\RefVal{p} = \SI{101325}{\pascal}$.  For each case, the reference velocity is set to $\RefVal{u} = \hat v^O$.  Again, all derived variables are nondimensionalized using the air stream as a reference condition, i.e. $\RefVal{\rho} = \SI{1.17}{\kilo \gram \per \cubic \meter}$, $\RefVal{\mu} = \SI{1.85e-5}{\kilo \gram \per \meter \per \second}$ and $\RefVal{W} = \SI{28.82}{\kilo \gram \per \kilo\mole}$. The reference heat capacity is set $\hat{c}_{p,\text{ref}}= \SI{1.3}{\kilo \joule \per \kilo \gram \per \kelvin}$. 

Under this conditions, the Reynolds numbers are $\Rey = 156$, $\Rey = 390$ and $\Rey = 858$, for the low, medium and high inlet velocities respectively. The Dahmkoler numbers are $\Da = 4.6\cdot10^9$, $\Da = 1.8\cdot10^9$ and $\Da = 8.3\cdot10^8$. The Prandtl number is assumed to be constant with $\Prandtl = 0.71$. A non-unity but constant Lewis number formulation is used, with $\Lewis_{\ch{CH4}} =  0.97 $ , $\Lewis_{\ch{O2}} = 1.11 $, $\Lewis_{\ch{H2O}} = 0.83 $ and $\Lewis_{\ch{CO2}} = 1.39 $ \citep{smookePremixedNonpremixedTest1991}. The system is considered open, then the thermodynamic pressure is constant and set to  $ p_0 = 1$. 

%%%%%%%%%%%%%%%%%%%%%%%%%%%
%% BoundaryConditions
%%%%%%%%%%%%%%%%%%%%%%%%%%%
The boundary condition of the inlets are,
\begin{itemize}
	\item Oxidizer inlet: $\{\forall (x,y): y = 0 \land x \in [-D/2, D/2]\}$\\
	\begin{equation*}
		u = 0,\qquad v= v^O(y), \qquad T = 1.0, \qquad \vec{Y}' = (0,Y^O_{\ch{O2}},0,0)
	\end{equation*}
	\item Fuel Inlet: $\{\forall (x,y): y = H \land x \in [-D/2, D/2]\} $ \\
	\begin{equation*}
		u = 0,\qquad v= v^F(y), \qquad T = 1.0, \qquad \vec{Y}' = (Y^F_{\ch{CH4}},0,0,0)
	\end{equation*}
\end{itemize}
 
\begin{figure}[p]
	\centering
	\pgfplotsset{width=0.73\textwidth, compat=1.3}
	\inputtikz{CounterFlowFlameMesh}	
	\inputtikz{CounterFlowFlameStreamlines}
	\inputtikz{CounterFlowFlameTemperature}
	\inputtikz{CounterFlowFlameDensity}	
	\inputtikz{CounterFlowFlamekReact}
	\inputtikz{CounterFlowFlamePressure}	
	\caption{Nondimensional solution and derived fields of the counterflow flame configuration for case (a).} \label{fig:CoFlowFlameFig1}
\end{figure}
\begin{figure}[p]
	\ContinuedFloat
	\centering
	\pgfplotsset{width=0.73\textwidth, compat=1.3}		
	\inputtikz{CounterFlowFlameCpMixture}	
	\inputtikz{CounterFlowFlameMF0}
	\inputtikz{CounterFlowFlameMF1}
	\inputtikz{CounterFlowFlameMF2}
	\inputtikz{CounterFlowFlameMF3}
	%	\inputtikz{CounterFlowFlameDensity}	
	\caption{Nondimensional solution and derived fields of the counterflow flame configuration for case (a) (continued).}% \label{fig:CoFlowFlameFig1}
\end{figure} 


The pressure outlet boundary condition is the same as \cref{eq:bc_O}. The pressure outlet boundaries are placed far away of the center of the domain, in order to decrease the effect on the centerline. Placing the boundary further away did not change appreciably the results. Finally the bundary conditions at the walls are defined as in \cref{eq:bc_dn}, with $\vec{u}_{\text{D}} = (0,0)$ and a constant temperature $T = 1.0$.          
 
In \cref{fig:CoFlowFlameFig1} the solution profiles for the case (a) are shown. The used mesh was obtained by a process of mesh refinement. The base mesh is initially created with a larger concentration of elements in the center of the domain. The points of intersection from the velocity inlet and wall boundary conditions are also refined, which was observed to improve the robustness of the algoritm. Similarly to the coflowing flame (\cref{ssec:coflowFlame}), during the solution algorithm of the flame sheet problem, mesh is aditionally refined around the flame sheet making use of a pseudo-timestepping setting.
As expected, a stagnation flow develops and a flame forms close to it. For this strain rate, a maximum temperature of $T = 6.05$ is obtained (1815 $\si{\kelvin}$). This big increase in the temperature is also reflected in a big decrease in the density field, where a decrease of almost six times on the density values is apreciated. This change of density also provokes the acceleration of fluid, as observed in \cref{fig:CounterFlowStreamlines}. 

In \cref{fig:CounterFlowReactionRate} the reaction rate given by \cref{eq:NonDimArr} is ploted. It is interesting to see that the actual reacting zone is very small, which clearly demonstrates why adequate meshing is necessary to capture the steep gradients resulting from the strong and highly localized heat sources. Finally, and as expected, the fuel and oxidizers field seem to only be found on either side of the flame. Altought it can not be seen here, some reactant leaking occurs, meaning that there exists a small zone where both especies coexist. This point will be adressed later. 


\subsubsection{Comparison of two-dimensional and the quasi one-dimensional counterflow flames}
In this section a comparison of the results obtained with the XNSEC-solver for a two dimensional counterflow diffusion flame, and the results obtained with the \lstinline|BVP4| solver for a quasi one-dimensional flame is done. The comparison of both set of results is done along the centerline of the domain (see \cref{fig:CDFScheme}). In this section, only dimensional variables will be considered. The transport parameters, chemical model and the equation of state are exactly the same for both formulations. For all calculations in this section, a polynomial degree of four is used for the velocity components, temperature and mass fractions. A polynomial degree of three is used for the pressure. This resulted on systems with approximately 439000 degrees of freedom. 
 
\begin{figure}[t!]
	\centering
	\inputtikz{CounterFlowFlame_DifferentBoundaryConditions}
	\caption{Velocity profiles of the counterflow diffusion flame for parabolic and plug inlet boundary conditions.}\label{fig:CounterFlowFlame_DifferentBoundaryConditions}
\end{figure}
%TODO Still calculating => same plot but with vel mult 11 '\\hpccluster\hpccluster-scratch\gutierrez\CounterFlowFlame_BCComparison6'.
The choice of the type of velocity boundary conditions for the inlets requires some attention. Different possiblities exist to describe the velocity profiles. One posibility is to characterize the velocity boundary conditions by assuming a Hiemenz potential flow, where a single parameter defines the flow field. Other posibilities are also a constant velocity value (plug flow) or a parabolic profile, allowing to define different velocity values for each jet inlet.
The effect of boundary conditions on the flame structure has been estudied by \cite{chelliahExperimentalTheoreticalInvestigation1991} and \cite{johnsonAxisymmetricCounterflowFlame2015}, where it was concluded that both plug and potential are able to adequately describe experimental data. 


The question whether a plug or parabolic flow profile allows a better representation of the quasi one-dimensional equations was treated in the work from \cite{frouzakisTwodimensionalDirectNumerical1998}. There is stated that the one and two dimensional formulations deliver very similar results, provided that the inlets of the two-dimensional configurations are uniform. Furthermore, preliminary calculations with the XNSEC-solver showed that the selection of a plug flow or parabolic have an influence on the solution, as shown in \cref{fig:CounterFlowFlame_DifferentBoundaryConditions}. Based on these results, the plug flow boundary condition is adopted for all following test cases.

\begin{figure}[t]
	\pgfplotsset{
		width=0.95\textwidth,
		group/xticklabels at=edge bottom,
		legend style = {
			at ={ (0.49,1), anchor= north east}
		},
		unit code/.code={\si{#1}}
	}
	\centering
	\inputtikz{BoSSS_1D_Comparison_velocity}
	\caption{Comparison of the axial velocity calculated with the XNSEC-solver and the one-dimensional approximation.}
	\label{fig:BoSSS_1D_Comparison_velocity}
\end{figure}
 In \cref{fig:BoSSS_1D_Comparison_velocity} a comparison of the axial velocities calculated with the XNSEC-solver and the one-dimensional solution is shown. While for the high strain case the results agree closely, for lower strains a discrepancy is observed. Recall that the derivation of the one-dimensional approximation assumes a constant velocity field incoming to the flame zone in order to obtain a self-similar solution. In the case of the two-dimensional configuration presented here, the border effects do have an influence on the centerline, which disrupts the self-similarity. This effect is more pronounced for low velocities, which explains the discrepancy between curves.
 
 
 \tikzexternaldisable
 \begin{figure}[p]
 	\centering
 	\pgfplotsset{
 		width=0.85\textwidth,
 		height = 0.33\textwidth,
 		compat=1.3,
 		tick align = outside,
 		yticklabel style={/pgf/number format/fixed},
 	}
 	\inputtikz{BoSSS_1D_Comparison1}
 	\inputtikz{BoSSS_1D_Comparison2}
 	\inputtikz{BoSSS_1D_Comparison3}
 	\caption[Comparison of temperature and mass fraction fields obtained with the XNSEC-solver and the one-dimensional approximation.]{Comparison of temperature and mass fraction fields obtained with the XNSEC-solver (solid lines) and the one-dimensional approximation (dashed lines).}
 	\label{fig:BoSSS_1D_Comparison}
 \end{figure}
 \tikzexternalenable
Similarly, In \cref{fig:BoSSS_1D_Comparison} the temperature and mass fraction fields are presented. Again, a discrepancy is observed for low strains, but results show a good agreement for higher inlet velocities. It can also be observed how, as expected, at higher strains a significant leakage of oxygen across the flame is present. This is a typical behaviour of a flame that is getting closer to its extintion point \cite{fernandez-tarrazoSimpleOnestepChemistry2006}.  

This is a drawback from usual one-step models with constant activation temperature, because they tend to overpredict fuel leakage. This behaviour is not appreciated in the one-step model with variable activation temperature used here.  In \cref{fig:VarParams} the comparison is shown for the configuration (c) between the mass fractions fields obtained using a chemical model with variable kinetic parameters given by \cref{eq:ActivationTemperatureOneStep,eq:heatReleaseOneStep} and other with constant kinetic parameters using $\hat T_a = \hat T_{a0}$ and  $\hat Q = \hat Q_{0}$.  The oxygen leakage obtained by using the chemical model with variable parameters is evident, confirming 
 
 
 \begin{figure}[h]
 	\centering
 	\begin{tikzpicture} 
 		\pgfplotsset{width=0.35\textwidth, compat=1.3}
 		\begin{axis}[
 			name=plot1,%	
 			xlabel = {$x$ [cm]},
 			ylabel= {Mass fraction},		 
 			x filter/.code={\pgfmathparse{#1*100}\pgfmathresult},
 			%		y filter/.code={\pgfmathparse{#1*100}\pgfmathresult},
 			legend style={at={(0.95,0.65)}, anchor=north east},
 			]
 			\addplot+[no marks] table {data/StudyVariableParameters/11/FullMassFraction0_vp0.txt}; \addlegendentry{\ch{CH4}, CK}
 			\addplot+[no marks] table {data/StudyVariableParameters/11/FullMassFraction0_vp1.txt};\addlegendentry{\ch{CH4}, VK}
 			\addplot+[no marks] table {data/StudyVariableParameters/11/FullMassFraction1_vp0.txt};\addlegendentry{\ch{O2}, CK}
 			\addplot+[no marks] table {data/StudyVariableParameters/11/FullMassFraction1_vp1.txt};\addlegendentry{\ch{O2}, VK}	
 		\end{axis}
 	\end{tikzpicture}
 	\begin{tikzpicture} 
 		\pgfplotsset{width=0.35\textwidth, compat=1.3}
 		\begin{axis}[
 			name=plot1,%	
 			xlabel = {$x$[cm]},
 			ylabel= {Mass fraction},		 
 			x filter/.code={\pgfmathparse{#1*100}\pgfmathresult},
 			%		y filter/.code={\pgfmathparse{#1*100}\pgfmathresult},
 			%		legend style={at={(0.84,0.45)}, anchor=south west},
 			xmin = 0.4,xmax=0.80,
 			ymax = 0.05,
 			]
 			\addplot+[no marks] table {data/StudyVariableParameters/11/FullMassFraction0_vp0.txt}; 
 			\addplot+[no marks] table {data/StudyVariableParameters/11/FullMassFraction1_vp0.txt};
 			\addplot+[no marks] table {data/StudyVariableParameters/11/FullMassFraction0_vp1.txt};
 			\addplot+[no marks] table {data/StudyVariableParameters/11/FullMassFraction1_vp1.txt};
 		\end{axis}
 	\end{tikzpicture}
 	\caption[Comparison of fuel and oxidizer mass fraction profiles using constant kinetic parameters and variable kinetic parameters]{Mass fraction profiles of methane and oxygen along the centerline of a counterflow diffusion flame configuration  using variable kinetic parameters (VK) and constant kinetic parameters (CK). Right picture is zoomed in near the flame zone.} \label{fig:VarParams}
 \end{figure}

 
In \cref{fig:TemperatureStrainPlot} the maximum temperature obtained in the centerline for different strain rates is ploted. Qualitatively speaking, the solution obtained with the XNSEC-solver agrees with the expectations. As the strain rate increases, the maximum temperature decreases (see \cref{fig:Sshaped}). On the other hand, the comparison of values obtained with the XNSEC-solver and those of the quasi one-dimensional approximation clearly shows a discrepancy in the results. For low strain rates this discrepancy is small, being for $a = \SI{20}{\per\second}$ only 10K, approximately a difference of 0.5\%. As the strain rate increases so does the discrepancy. For $a = \SI{200}{\per\second}$ the difference is almost 50K, which is a 9\% disagreement. A similar behaviour is also reported in \cite{frouzakisTwodimensionalDirectNumerical1998}, where a difference of 50K was reported. 

It is worth noting that the XNSEC-solver wasnt able to find a converged solution for $a > \SI{202}{\per\second}$, and the newton algorithm stagnates. This is most probably a sign of underresolution of the mesh, and that the used refinement strategy did not help for such high strain rates. A better mesh refinement strategy is necesary for calculating the flame at conditions near the extintion point. 
Moreover, for high strain rates the flame will be far from the thermochemical equilibrium, and it is likely that the solution obtained for the flame sheet will be far away from the solution with finite rates. A posibility would be to use one of the well-known continuation methods (see for example \cite{nishiokaFlamecontrollingContinuationMethod1996}) to progressively move in the direction of the extinction point. Obviously, the Homotopy Methodology presented in \cref{sec:HomotopyMethod} can be visualized as one of those methods, and would be useful when looking for solutions of systems that are close to the extinction point. A complexity that arises then is how to create a dynamical mesh that is suitable for obtaining the intermediate solutions while searching for the final result in a robust way. This issue is beyond the scope of this thesis, and may be the subject of future research.


The difference between the results obtained for the two-dimensional configuration and the quasi-one dimensional approximation could be explained by some condition within the 1D system assumptions being violated in the 2D configuration. It is known that in addition to the boundary conditions, the ratio between the slot width and the separation between the two slots ($D/H$) also has an influence on the solution, and that a high ratio is desirable \citep{frouzakisTwodimensionalDirectNumerical1998}. Another point that was not addressed here is whether the boundary conditions chosen for cold walls have an effect on the solution along the centerline. Other posibilities could have been using outlet boundary conditions or an adiabatic wall. It is nevertheless expected that its effect on the centerline would not be big. The points mentioned here should be adressed in future work.

\begin{figure}[h]
	\centering
	\inputtikz{TemperatureStrainPlot}
	\caption{Maximum centerline temperature of a counterflow flame for different strains.}
	\label{fig:TemperatureStrainPlot}
\end{figure}
\subsubsection{Temperature convergence study}
\begin{figure}[h]
	\centering
	\inputtikz{TemperatureConvergenceDiffFlame}
	\caption{Convergence study of the maximum value of the temperature for the counterflow diffusion flame configuration.}
	\label{fig:TemperatureConvergenceDiffFlame}
\end{figure}
Similar to problems presented in earlier sections, the presence of singularities caused by non-consistent boundary conditions causes a degenerative effect on the global error values, making a global convergence study for this configuration is problematic. However, it is still possible to study the behaviour of some characteristic point value under different conditions to prove the mesh independence of the solution.

In \cref{fig:TemperatureConvergenceDiffFlame} it is shown how the maximum temperature along the centerline obtained for configuration (b)  behaves under different mesh resolutions and polynomial degrees. Values for $k=1$ are not shown, because for this range of cell elements the maximum temperature value was of the order of 60K higher than the ones depicted here. The temperature tends to a limit value, and its posible to observe how this value is reached already for coarse meshes when using a polynomial degree of three or four. For $k=2$ the temperature also tends to a limit value, but at a slower rate when compared to $k =3$ or $k = 4$. 

In the next section a simplified one-dimensional flame configuration will be used in order to be able to realize a $h$-convergence study of the whole sytem operator.


\subsection{Chambered diffusion flame}\label{ss:UDF}
In this chapter a h-convergence study for a quasi-one-dimensional configuration is shown. This is done by using a planar unstrained diffusion flame in the so-called chambered diffusion flame. This configuration has served as a model for many theoretical studies related to diffusion flames (\cite{matalonEffectThermalExpansion2010,rameauNumericalBifurcationChambered1985,matalonDiffusionFlamesChamber1980}). 

A sketch of the configuration can be seen in \cref{fig:chamberedDifFlame}. Fuel is injected at a constant rate into the bottom of a small insulated chamber, while oxidant diffuses into the system against the direction of flow. Constant conditions at the outlet of the chamber are achieved in an experimental setting by a rapid renewal of the flow of the oxidant. Under these conditions, an unstrained planar flame is formed.

The fuel inlet into the chamber is modeled with a constant velocity inlet boundary condition \cref{eq:bc_d}, while the flow outlet at the top is considered an outlet as given by \cref{eq:bc_OD}. Since the interest is in the flame far away from the container walls, it is sufficient to set the remaining boundary conditions as periodic boundaries. This effectively transforms the problem into a pseudo-two-dimensional configuration.

\begin{figure}[t!]
	\centering
	\pgfplotsset{width=0.34\textwidth, compat=1.3}
	\inputtikz{ConvergenceDiffFlame}
	\caption{Convergence study for the chambered diffusion flame configuration.}
	\label{ConvergenceDiffFlame}
\end{figure}
The inlet velocity of the fuel jet is set to $\SI{2.5}{\centi\meter \per \second}$ and its mass composition is $Y^0_{\ch{CH4}} = 0.2$ and $Y^0_{\ch{N2}} = 0.8$ while air has a composition $Y^0_{\ch{O2}} = 0.23$ and $Y^0_{\ch{N2}} = 0.77$. The temperature of the fuel and air feed streams is $\SI{300}{\kelvin}$. The length of the system $L$ is equal to $\SI{0.015}{\meter}$. The Reynolds number is $\Reynolds = 2$

For this configuration, a $h$ convergence study is conducted, where uniform Cartesian meshes with $5\times2^6$, $5\times2^7$, $5\times2^8$,  $5\times2^9$ and $5\times2^{10}$  cells are used. The polynomial degrees are varied from one to four for velocity, temperature and mass fractions, and from 0 to 3 for pressure.  Errors are calculated using the finest mesh as a reference solution.  

The results are shown in \cref{ConvergenceDiffFlame} for variables $u$, $T$, $Y_{\ch{CH4}}$ and $p$. The convergence results for other variables are similar and not shown here. The expected convergence rates characteristic from the DG-method are observed. For low polynomial degrees the orders of convergence are very close to the theoretical values. However for higher polynomial degrees  a slight deterioration of the convergence rate is observed.
\begin{figure}[h]
	\begin{center}
		\def\svgwidth{0.4\textwidth}
		\import{./plots/}{UnstrainedFlameConfig.pdf_tex}
		\caption{Schematic representation of the chambered diffusion flame configuration. }
		\label{fig:chamberedDifFlame}
	\end{center}
\end{figure}


\FloatBarrier
