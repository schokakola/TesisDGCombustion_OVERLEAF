\chapter{Results}	\label{ch:results}
\glsresetall
 The following sections will present a comprehensive validation of the solver using a variety of test cases. The chosen test cases will be introduced incrementally in complexity. In \cref{sec:SingleCompIsotCase} the applicability of the solver for isothermal single-component systems is analyzed. Later in \cref{sec:SinCompNonIsothermCase} several single-component non-isothermal configurations are studied. Finally in \cref{sec:MultCompNonIsothermCase} test-cases for multi-component non-isothermal systems are presented. 
%Additionally the convergence properties of the DG-method will be analyzed for some of the systems.



All calculations shown here where performed on a  AMD EPYC 7543 32-Core Processor, DDR4 %TODO \todo[inline]{which specifications of the cluster should i include?}
Unless otherwise stated, all calculations use the termination criteria presented in \cref{ssec:TerminationCriterion}

\section{Single-component isothermal cases}\label{sec:SingleCompIsotCase}
The solver presented in the previous sections is initially validated for single-component isothermal cases. In these cases only the continuity and momentum equations are solved. The energy and species concentration equations  are replaced by the condition $T = 1.0$ and $Y_0 = 1.0$ in the complete domain. This means that the physical properties of the flow (density and viscosity) are constants and the flow is completely incompressible.

%(Subsequently in section XX the temporal discretization is studied using the traditional Taylor-Green vortex.)
\subsection{Lid-driven cavity flow}
\begin{figure}[b]
	\begin{center}
		\def\svgwidth{0.3\textwidth}
		\import{./plots/}{LidDrivvenGeometry.pdf_tex}		
		\caption{Schematic representation of the Lid-Driven cavity flow.}
		\label{fig:LidDrivenCavity}
	\end{center}	
\end{figure} 

The lid-driven cavity flow is a test problem classically used for the validation of Navier-Stokes solvers. The system configuration is shown in \cref{fig:LidDrivenCavity}. It consists simply of a two-dimensional square cavity  enclosed by walls whose upper boundary moves at constant velocity, causing the fluid to move. This provides a simple test-case which can be used for the validation of incompressible Navier-Stokes solvers. Benchmarks results can be found widely in the literature for different Reynolds numbers. We compare in this section our results with the ones published by \cite{botellaBenchmarkSpectralResults1998} and focus in the case where $\gls{Reynolds} = 1000$. 

The problem is defined in the domain $\Omega = [0,1]\times [0,1]$. The system is solved for the velocity vector $\gls{velVec} = (u,v)$ and the pressure $p$. All boundary conditions are of the Dirichlet type, particularly with $\gls{velVec} = (-1,0)$ for the boundary at $y = 1$, and $\gls{velVec} = (0,0)$ for all other sides. The gravity vector is set to $\gls{gravityVec} = (0,0)$. 

 In \cref{fig:LiddrivenMesh} the used mesh is presented, which corresponds to a Cartesian mesh with extra refinement on both upper corners. This was done to be able to better represent the complex effects that take place there.  The streamlines plot presented in \cref{fig:LiddrivenMesh} show the different vortex structures typical to this kind of systems. Additional to the main vortex, smaller structures appear in the corners of the cavity. 
	
	\begin{figure}[t]
		\centering
		\pgfplotsset{width=0.35 \textwidth, compat=1.3}	
		\inputtikz{LiddrivenMesh1}
		\inputtikz{LiddrivenMesh2}
		\caption{Used mesh and streamlines obtained for the lid-driven cavity flow with $\gls{Reynolds} = 1000$} \label{fig:LiddrivenMesh}
	\end{figure}

In \cref{fig:LidVelocities} a comparison of the calculated velocity and the velocities provided by the benchmark are shown. For the calculations presented here, the polynomial degree is set to $k = 4$ and $k' = 3$. A regular cartesian mesh with $16\times16$ elements with extra refinement in the corners is used. Clearly a very good agreement is obtained, even by using a relatively coarse mesh (the benchmark result uses a grid with $160\times160$ elements). 

A more rigorous comparison of results is presented in \cref{tab:LidCavityExtrema}, where the extreme values of the velocity components calculated through the centerline of the cavity are compared with the results presented by \cite{botellaBenchmarkSpectralResults1998}. We use for this comparison different mesh resolutions. It can be clearly seen how for the $256\times 256$ mesh the results obtained with our solver are extremely close to the reference, seeing a difference only at the fifth digit after the decimal point for the velocity components, and no difference for the spatial coordinates where the extrema are. It can also be appreciated how the results obtained with the coarser meshes are still very close to those of the reference. 
%It is worth mentioning this comparison here could be considered unfair, since the simulations were performed with polynomial degrees of degree 4 for the velocity and three for the pressure, while the calculations of the reference use %TODO  This point will be further discussed in later sections. 

\newpage
	\begin{figure}[tb]
	\pgfplotsset{
		group/xticklabels at=edge bottom,
		legend style = {
			at ={ (0.09,0.3), anchor= north east}
		},
	}
	\inputtikz{LidVelocities1}
	\pgfplotsset{
		group/xticklabels at=edge bottom,
		legend style = {
			at ={ (0.59,0.3), anchor= north east}
		},
	}
	\inputtikz{LidVelocities2}
	\caption{Calculated velocities along the centerlines of the cavity and reference values. Left plot shows the x-velocity for $x = 0.5$. Right plot shows the y-velocity for $y = 0.5$  }
	\label{fig:LidVelocities}
\end{figure}

%	\begin{figure}[tb]
%		\pgfplotsset{
%			group/xticklabels at=edge bottom,
%			legend style = {
%				at ={ (0.09,0.2), anchor= north west}
%			},
%		}
%		\begin{tikzpicture}
%			\begin{axis}[
%				width= 0.4\textwidth ,
%				height= 0.3\textwidth ,
%				xlabel = $y$,
%				ylabel= $\omega$, 
%				]
%				\addplot+[color=black, only marks] table {data/LidDrivenCavity/omegaRe1000_x_ref.txt}; \addlegendentry{Reference}
%				\addplot[color=black, no marks] table {data/LidDrivenCavity/omegaRe1000_x.txt}; \addlegendentry{BoSSS}
%			\end{axis}
%		\end{tikzpicture}
%		\pgfplotsset{
%			group/xticklabels at=edge bottom,
%			legend style = {
%				at ={ (0.59,1.0), anchor= north east}
%			},
%		}
%		\begin{tikzpicture}
%			\begin{axis}[
%				width= 0.4\textwidth ,
%				height= 0.3\textwidth ,
%				xlabel = $x$,
%				ylabel= $\omega$, 
%				]
%				\addplot+[color=black, only marks] table {data/LidDrivenCavity/omegaRe1000_y_ref.txt}; \addlegendentry{Reference}
%				\addplot[color=black, no marks] table {data/LidDrivenCavity/omegaRe1000_y.txt}; \addlegendentry{BoSSS}
%			\end{axis}
%		\end{tikzpicture}
%		\caption{Calculated vorticity along the centerlines of the cavity and reference values. Left plot shows the vorticity for $x = 0.5$. Right plot shows the vorticity for $y = 0.5$  }
%		\label{fig:LidVorticities}
%	\end{figure}



%\subsection{ Taylor-Green vortex}
%\blindtext[5]
\begin{table}[]
	\centering
	\begin{tabular}{lllllrr}
		\hline
		Mesh     & $u_{\text{max}}$ & $y_{\text{max}}$ & $v_{\text{max}}$ & $x_{\text{max}}$ & \multicolumn{1}{l}{$v_{\text{min}}$} & \multicolumn{1}{l}{$x_{\text{min}}$} \\ \hline
		$16\times16$        & 0.3852327        & 0.1820           & 0.3737295        & 0.8221           & -0.5056627                           & 0.0941                               \\
		$32\times32$        & 0.3872588        & 0.1821           & 0.3760675        & 0.8227           & -0.5080496                           & 0.0943                               \\
		$64\times64$       & 0.3897104        & 0.1748           & 0.3774796        & 0.8408           & -0.5248360                           & 0.0937                               \\
		$128\times128$       & 0.3886452        & 0.1720           & 0.3770127        & 0.8422           & -0.5271487                           & 0.0907                               \\
		$256\times256$       & 0.3885661        & 0.1717           & 0.3769403        & 0.8422           & -0.5270653                           & 0.0907                               \\\hline
		Reference & 0.3885698        & 0.1717           & 0.3769447        & 0.8422           & \multicolumn{1}{l}{-0.5270771}       & \multicolumn{1}{l}{0.0908}           \\ \hline
	\end{tabular}
	\caption{Extrema of velocity components through the centerlines of the cavity for $\gls{Reynolds} = 1000$. Reference values obtainder from \cite{botellaBenchmarkSpectralResults1998} }
	\label{tab:LidCavityExtrema}
\end{table}
	\FloatBarrier
\newpage

\subsection{Backward-facing step}\label{ssec:BackwardFacingStep}
The backward-facing step problem is another classical configuration widely used for validation of incompressible CFD codes. It has been widely studied theoretically, experimentally and numerically by many authors in the last decades (see for example \cite{armalyExperimentalTheoreticalInvestigation1983,barkleyThreedimensionalInstabilityFlow2000,biswasBackwardFacingStepFlows2004} ).  In \cref{BFSsketch} a schematic representation of the problem is shown. It consists of a channel flow (usually considered fully developed) that is subjected to a sudden change in geometry, which causes separation and reattachment phenomena. For these reasons, this case can be considered more challenging than the one presented in the previous section, since special care of the used geometry has to be taken in order to capture accurately all complex phenomena taking place.

Although the backward-facing step problem is known to be inherently three-dimensional, it has been shown that it can be studied as a two-dimensional configuration along the symmetry plane for moderate Reynolds numbers \citep{barkleyThreedimensionalInstabilityFlow2000, biswasBackwardFacingStepFlows2004}. For the range of Reynolds numbers used in our calculations the two-dimensional assumption is justified.  The origin of the coordinate system is set at the bottom part of the step. The step height \gls{StepHeigth} and channel height \gls{ChannelHeight} characterize the system. The results on the literature are often reported as a function of the expansion ratio, defined as $\gls{ExpansionRatio} = (\gls{ChannelHeight}+\gls{StepHeigth})/\gls{ChannelHeight}$. 
%For all calculations shown here we set $L = 70 \gls{StepHeigth}$ and $L_0 = \gls{StepHeigth}$.
We conduct a series of simulations with the objective of reproducing the results reported in \cite{biswasBackwardFacingStepFlows2004}, where the backward-facing step was calculated for Reynolds numbers up to $400$ and for a expansion ratio of $1.9423$. 
The Reynolds number for the backward-facing step configuration is defined in the literature in many forms. We adopt here the definition based on the step height \gls{StepHeigth} and mean inlet velocity $U_{\text{mean}}$ as reference values, resulting in
\begin{equation}
\gls{Reynolds}= \frac{\gls{StepHeigth}U_{\text{mean}}}{\gls{kinVisc}}
\end{equation} 
The system is isothermal and the fluid is assumed to be air. The boundary at $ x = -L_0$ is a inlet boundary condition, where a parabolic profile is defined with %, with a kinematic viscosity  $\nu = \SI{15.52e-6 }{\meter \squared \per \second}$
\begin{equation}
u(y) = -6U_{\text{mean}}\frac{(y-S)(y-(h+S))}{h^2}
\end{equation}
In order to minimize the effects of the outlet  boundary condition in the system, the length $L$ of the domain was set to $L = 70 \gls{StepHeigth}$. All other boundaries are fixed walls. The effect of the domain length before the step was seen to have almost no impact on the calculation, and is just set to $L_0 = \gls{StepHeigth}$. For all calculations in this section a structured grid with 88400 elements is used. To better resolve the complex structures that occur in this configuration, smaller elements are used in the vicinity of the step, as seen in figure \cref{bfsmesh}.

%Preliminary calculations showed that the calculated reattachment and detachment lengths are highly sensitivy to an adequate mesh.
\begin{figure}[tb]
	\begin{center}
		\def\svgwidth{0.9\textwidth}
		\import{./plots/}{BFS_sketch.pdf_tex}		
		\caption{Schematic representation of the backward-facing step. Both primary and secondary vortices are shown. Sketch is not to scale.}
		\label{BFSsketch}
	\end{center}	
\end{figure} 

\begin{figure}[tb]
	\begin{center}
		\def\svgwidth{0.8\textwidth}
		\import{./plots/}{HBFS_MESH.pdf_tex}
		\caption{Structured mesh used in all calculations. }
		\label{bfsmesh}
	\end{center}	
\end{figure} 

\begin{figure}[bt]
	\centering
	\pgfplotsset{
		group/xticklabels at=edge bottom,
		%		legend style = {
			%			at ={ (1.0,1.0), anchor= north east}
			%		},
		unit code/.code={\si{#1}}
	}
\inputtikz{uvelBFS}
	\caption{Distribution of x-component of velocity in the backward-facing step configuration for a Reynolds number of 400. Solid lines correspond to results obtained with BoSSS}
	\label{fig:uvelBFS}
\end{figure} 



\begin{figure}[tb]
	\pgfplotsset{
		group/xticklabels at=edge bottom,
		legend style = {
			at ={ (0.05,0.9), anchor= north west}
		},
		unit code/.code={\si{#1}}
	}
	\centering
\inputtikz{Re_De_Attachmentlengths}
	\caption{ Detachment and reattachment lengths of the primary (left figure) and secondary (right figure) recirculation zones after the backward-facing step compared to the reference solution \citep{biswasBackwardFacingStepFlows2004}.}
	\label{fig:Re_De_Attachmentlengths}
\end{figure}
The backward-facing step configuration exhibits varying behavior as the number of Reynolds changes. At small Reynolds number, a single vortex, usually called the primary vortex, appears in the vicinity of the step. In addition, as the Reynolds number increases, a second vortex eventually appears on the top wall, as is shown schematically in  \cref{BFSsketch}. 
The detachment and reattachment lengths of the vortices are values usually reported in the literature. It is possible to determine the position of detachment by finding the point along the wall where the velocity gradient normal to the wall acquires a value equal to zero. Cubic splines were used to more accurately find this point.

\cref{fig:Re_De_Attachmentlengths} shows the detachment and reattachment lengths of the primary and secondary vortices obtained with our code for different Reynolds numbers, which are also compared with the results presented in the reference paper from \cite{biswasBackwardFacingStepFlows2004}. It can be observed how the results for the detachment lengths of the primary vortex present very good agreement with those of the reference. In the case of the secondary vortex it is possible to see a very minimal deviation for the reattachment lengths. It is interesting to note that despite the fact that the reference does not report the existence of a secondary vortex for $\gls{Reynolds} = 200$, with our code it was possible to observe it. The results allow us to conclude that it is possible to study flows with complex behavior for high Reynolds numbers, at least in the isothermal case. In the next section we will discuss what happens for the non-isothermal case.


It is worth mentioning that the evaluation of the numerical accuracy of the solver using the two incompressible test cases presented in this section is problematic due to the presence of singularities, specifically on the corners at the coordinates $ \vec{x} = (0,1)$ and $\vec{x} =(1,1)$ of the Lid-driven cavity (where the pressure is not finite according to \cite{botellaBenchmarkSpectralResults1998}), and at the corner of the step $\vec{x} = (0,S)$ of the backward-facing step. The accuracy of the solver will be assessed in following chapters. 


\section{Single-component non-isothermal cases} \label{sec:SinCompNonIsothermCase}
For the test cases presented in this section the equations for continuity, momentum and energy are solved and all systems are assumed to be single-component, thus $N = 1$ and $Y_0 = 1.0$. We start by showing in \cref{ssec:HeatedBackwardFacingStep} a extension of the backward-facing step configuration presented in the last section, considering now a non-isothermal system. Later in \cref{ssec:CouetteFlowTempDiff} a Couette flow configuration presenting a temperature gradient in the vertical direction is studied. Finally in \cref{ss:DHC} a heated square cavity configuration is studied in order to asses the capability of the solver for variable density flows in closed systems. 


\subsection{Heated backward-facing step}\label{ssec:HeatedBackwardFacingStep}
\begin{figure}[tb]
	\begin{center}
		\includegraphics[width=\linewidth]{../plots/HBFS_TemperatureRe700_2.pdf}
		\caption{Temperature profile (top) and streamlines (bottom) corresponding to the backward-facing Step configuration for $\gls{Reynolds} = 400$ and an expansion ratio of two.}
		\label{BFS_Streamlines}
	\end{center}	
\end{figure} 

As an extension to the previous case, we seek to reproduce the results presented by \cite{xieFluidFlowHeat2016}, where the same backward-facing step configuration as presented in \cref{ssec:BackwardFacingStep} is studied, but with the particularity that in this case the bottom wall is heated to a constant temperature higher than the inlet temperature, thus featuring a non-isothermal system. The fluid entering the system has a temperature equal to $T_0 = \SI{283}{\kelvin}$ and the bottom wall is set to a constant temperature of $T_1 =\SI{313}{\kelvin}$
In the work of \cite{xieFluidFlowHeat2016} results are reported for the local Nusselt numbers and local friction coefficients $f_d$  along the bottom wall for different expansion ratios and Reynolds numbers. By putting together the definition of the Nusselt number ( $\gls{Nusselt} = \frac{\gls{HeatTransCoef}\hat{L}}{\gls{HeatConductivity}}$), Newtons law of cooling ($\hat{\vec{q}} = \hat{h} (\hat{T}_0 - \hat{T}_W )$), and Fourier's law of heat  conduction, ($\hat{\vec{q}} = \hat \lambda \hat{\nabla} \glsHat{temp}$) a expression for the local Nusselt number is obtained
\begin{equation}
\gls{NusseltLoc} = \frac{\hat L}{\hat T_0-\hat T_W}\hat \nabla \hat T \cdot \hat {\vec{n}}
\end{equation}
where $\hat L$ is a reference length. We choose $ \hat L = \hat S$. Additionally by recognizing that the wall shear stress along the bottom wall $\tau_{\text{w}} = -\mu \nabla u \cdot \vec{n}$, the local friction factor can be written as
\begin{equation}
f_d = \frac{8\nu} { (U_{\text{mean}})^2}  \nabla u \cdot \vec{n} 
\end{equation}
It should be noted that, for the range of temperature differences involved in this case, the variation of physical parameters such as density, viscosity and thermal conductivity with respect to temperature has no appreciable influence on the calculated flow fields.

Simulations for different Reynolds numbers and expansion ratios were conducted. In \cref{BFS_Streamlines} the temperature field and streamlines corresponding to a calculation with $\gls{Reynolds} = 700$ is shown. Here the apparition of the secondary vortex is appreciated.

It must be noted here that the results obtained by us are substantially different from those reported by \cite{xieFluidFlowHeat2016}, and will not be shown here. However, in the work of \cite{henninkLowMachNumberFlow2022} the same is also pointed out, saying that with his method it was not possible to reproduce the results presented by \cite{xieFluidFlowHeat2016}. Comparing our results with those of Hennink we can observe that the same results are obtained. as demonstrated in \cref{fig:fd_Nu_plot}. We can conclude from these results that the solver is able to deal with non-isothermal flows adequately.  

\begin{figure}[tb]
	\pgfplotsset{
		group/xticklabels at=edge bottom,
		legend style = {
			at ={ (0.59,1.0), anchor= north east}
		},
		unit code/.code={\si{#1}}
	}
\inputtikz{fd_Nu_plot1}
\inputtikz{fd_Nu_plot2}
	\caption{Local friction factor and local Nusselt number along the bottom wall for $\gls{Reynolds} = 700$ and an expansion ratio of two. The solid lines corresponds to our solution and the marks to the reference \citep{henninkLowMachNumberFlow2022}}
	\label{fig:fd_Nu_plot}
\end{figure}



\FloatBarrier

\subsection{Couette flow with vertical temperature gradient} \label{ssec:CouetteFlowTempDiff}
\begin{figure}[tb]
	\begin{center}
		\def\svgwidth{0.5\textwidth}
		\import{./plots}{HeatedCouetteSketch.pdf_tex}
		\caption{Schematic representation of the couette flow with temperature difference test case.}
		\label{fig:CouetteTempDiff_scheme}
	\end{center}	
\end{figure} 
As a following test case for the low-Mach solver we study a Couette-flow with a vertical temperature gradient. This configuration was already studied in \citep{kleinHighorderDiscontinuousGalerkin2016}, where the SIMPLE algorithm in a DG framework was used. We intend in this section to reproduce their results by using our fully coupled solver. Additionally, we show how our implemented solver performs in relation to the SIMPLE based solver. 

In \cref{fig:CouetteTempDiff_scheme} a schematic representation of the test-case is shown. The top wall corresponds to a moving wall ($u = 1$) with a fixed temperature $T=T_h$. The bottom wall is a static one ($u = 0$), and has a constant temperature $T = T_c$. 
The domain is chosen as $\Omega = [0,1]\times[0,1]$, and Dirichlet boundary conditions are used for all boundaries. Additionally,the system is subjected to a gravitational field, where the gravity vector only has a component in the $y$ direction. Under this conditions, the x-component of velocity, pressure and temperature are only dependent on the $y$ coordinate, i.e. $u = u(y)$, $T = T(y)$ and $p = p(y)$. %Thus, the governing equations (\cref{eq:NS-eq}) reduce to 
\begin{align}
&\frac{1}{\gls{Reynolds}} \pfrac{ }{y}\left(\mu\pfrac{u}{y}\right) = 0,\\
&\pfrac{p}{y} = -\frac{\gls{dens}}{\gls{Froude}^2},\\
&\frac{1}{\gls{Reynolds}~\gls{Prandtl}} \pfrac{ }{y}\left(\gls{HeatConductivity}\pfrac{T}{y}\right) = 0.
\end{align}
It is possible to find an analytical solution for this problem.
%%Constant transport properties
%By assuming $\gls{Prandtl} = 1.0$ and constant transport properties ($\mu = \lambda = 1$), we can derive the analytical solution of XXX
%\begin{align}
%u &= y,\\
%p &= - \frac{p0}{\gls{Froude}^2(T_h-T_c)}\ln\left((T_h-T_c)y+T_c\right)+C,\\
%T &= (T_h-T_c)y + T_c,
%\end{align}
%where $C$ is an arbitrary constant which defines the mean value of the pressure TODO rewritethis.
% Variable transport properties
Under this conditions, and by assuming a temperature dependence of the transport properties according to a Power Law ($\mu = \lambda = T^{2/3}$) a solution is obtained as 
\begin{subequations}
\begin{align}
	u(y) &= C_1 + C_2\left(y + \frac{T_c^{5/3}}{T_h^{5/3}-T_c^{5/3}} \right)^{3/5},\label{eq:CouetteU}\\
	p(y) &= -\frac{5p_0}{2\gls{Froude}^2}\frac{\left(y\left(T_h^{5/3}-T_c^{5/3}\right)+T_c^{5/3}\right)^{2/5}}{\left(T_h^{5/3}-T_c^{5/3}\right)}+C,\label{eq:Couettep}\\
	T(y) &= \left(C_3 - \frac{5}{3}C_4 y\right)^{3/5}\label{eq:CouetteT}.
\end{align} 
\end{subequations}
Where the constants $C_1$, $C_2$, $C_3$ and $C_4$ are determined by using the boundary conditions on the top and bottom walls, and are given by 
\begin{align}
C_1 &= \frac{\left(\frac{T_c^{5/3}}{T_h^{5/3}-T_c^{5/3}}\right)^{3/5}}{\left(\frac{T_c^{5/3}}{T_h^{5/3}-T_c^{5/3}}\right)^{3/5}-\left(\frac{T_h^{5/3}}{T_h^{5/3}-T_c^{5/3}}\right)^{3/5}}\\
C_2 &= \frac{1}{\left(\frac{T_h^{5/3}}{T_h^{5/3}-T_c^{5/3}}\right)^{3/5}-\left(\frac{T_c^{5/3}}{T_h^{5/3}-T_c^{5/3}}\right)^{3/5}}\\
C_3 &= T_c^{5/3},\\
C_4 &= \frac{3}{5}\left(T_c^{5/3}-T_h^{5/3}\right)
\end{align}
and $C$ is a real-valued arbitrary constant for the pressure.%
\begin{center}
\begin{figure}[tb]
	\pgfplotsset{
		group/xticklabels at=edge bottom,
	}
\inputtikz{CouetteSolution1}
\inputtikz{CouetteSolution2}
\inputtikz{CouetteSolution3}
	\caption{Solution of the Couette flow with vertical temperature gradient. Viscosity and heat conductivity are calculated with a Power-Law.}
	\label{fig:CouetteSolution}
\end{figure}
\end{center}
\FloatBarrier
For all calculations of this configuration shown the dimensionless parameters are set as $\gls{Reynolds} = 10$ and $\gls{Prandtl} =0.71$, $T_h = 1.6$ and $T_c = 0.4$. Since we are dealing with an open system we set $p_0 =1.0$. The Froude number is calculated as 
\begin{equation}
\text{Fr} = \left( \frac{2\text{Pr}(T_h-T_c)}{(T_h+T_c)}\right)^{1/2}
\end{equation}
In \cref{fig:CouetteSolution} the solution for the velocity, pressure and temperature are shown. The results are for a mesh with $26\times26$ elements and a polynomial degree of three for $u$ and $T$, and a polynomial degree of two for $p$.
\subsubsection{h-convergence study}
The convergence properties of the DG method for this non-isothermal system was studied using the analytical solution described before. The domain is discretized and solved in uniform Cartesian meshes with $16\times16$, $32\times32$, $64\times64$ and $128\times128$ elements. The polynomial degrees for the velocity and temperature are changed from 1 to 4 and for the pressure from 0 to 3. The convergence criteria described in \cref{ssec:TerminationCriterion} was used for all calculations. The  analytical solution given by \cref{eq:CouetteU,eq:Couettep,eq:CouetteT} are used as Dirichlet boundary conditions on all boundaries of the domain. The error is calculated against the analytical solution using the $L^2$ norm. %TODO \todo[inline]{Comment more on the calculation of the l2norm}.
In \cref{fig:ConvergenceDHC} the results of the h-convergence study are shown. We observe how the expected convergence rates are reached for all variables, namely a slope of the order $k+1$ for both velocity components and the temperature, and a slope of $k'+1$ for the pressure.
\begin{figure}[t!]
	\centering
	\pgfplotsset{width=0.34\textwidth, compat=1.3}
	\inputtikz{ConvergenceCFTD}
	\caption{Convergence study of the Couette-flow with temperature difference. A power-law is used for the transport parameters.}\label{fig:ConvergenceCFTD}
\end{figure}

\subsubsection{Comparison with SIMPLE}
As mentioned before, a solver for solving low-Mach flows based on the SIMPLE algorithm (presented in \cite{kleinHighorderDiscontinuousGalerkin2016}) has been already developed and implemented within the BoSSS framework. 
Though the solver was validated and shown to be useful for a wide variety of test cases, there were also disadvantages inherently associated with the SIMPLE algorithm. Within the solution algorithm, Picard-type iterations are used to search for a solution. This requires some prior knowledge from the user in order to select suitable relaxation factor values that provide stability to the algorithm, but at the same time do not slow down the computation substantially.
It was also observed that the calculation times were prohibitively high for some test cases. This point motivated the development of the solver presented in the present work, where the system is solved in a monolytic way and  and the Newton method globalized with a Dogleg-type method is used to solve the system.  
%TODO \todo[inline]{what are exactly the advantages of the present solver? multigrid? Ortogonalization?} 
We intend to show in this section a comparison of runtimes of the calculation of the Couette flow with vertical temperature gradient between the DG-SIMPLE algorithm \citep{kleinHighorderDiscontinuousGalerkin2016} and the present solver (denoted here as XNSEC). Calculations were performed on uniform Cartesian meshes with $16\times16$, $32\times32$, $64\times64$ and $128\times128$ elements, and with varying polynomial degrees between 1 and 3 for the velocity and temperature, and between 0 y 2 for the pressure. All calculations where initialized with a zero velocity and pressure field, and with a temperature equal to one in the whole domain. Are calculations were performed single-core and the convergence criteria is set to $10^{-8}$ for both solvers. The under-relaxation factors for the SIMPLE algorithm were set for all calculations to 0.8, 0.5 and 1.0 for the velocity, pressure and temperature, respectively. 

In figure \cref{fig:RuntimeComparison} a comparison of the runtimes from both solvers is shown. It is clearly  appreciated how the runtimes of the SIMPLE algorithm are higher for almost all of the cases studied. Obviously the under-relaxation parameters of the SIMPLE algorithm have an influence on the calculation times and an appropiate selection of them could decrease the runtimes. This is a clear disadvantage, because the selection of adequate factors is highly problem dependent and requires some previous expertise from the user. On the other hand, the globalized Newton method used by the XNSEC avoid this problem by using a more sophisticated method and heuristics in order to find a better path to the solution. 

%TODO \todo[inline]{I have to somehow highlight even more the positive points of using this solver} 

\begin{center}
	\begin{table}[tb!]
		\begin{tabular}{ccc}
			\inputtikz{RuntimeComparison1}
			&
			\inputtikz{RuntimeComparison2}
			&
					\inputtikz{RuntimeComparison3}
		\end{tabular}%
		\caption{Runtime comparison of the DG-SIMPLE algorithm \citep{kleinHighorderDiscontinuousGalerkin2016} and the present solver (XNSEC) for the Couette flow with vertical temperature gradient configuration}
		\label{fig:RuntimeComparison}
	\end{table}
\end{center} 


\FloatBarrier



\subsection{Differentially heated cavity problem}\label{ss:DHC}

\begin{figure}[bt]
	\begin{center}
		\def\svgwidth{0.53\textwidth}
		\import{./plots/}{diffheatedCavityGeometry.pdf_tex}		
	\caption{Schematic representation of the differentially heated cavity problem.}
		\label{DHCGeom}
	\end{center}	
\end{figure} 



The differentially heated cavity problem corresponds to a classical benchmark case often used to asses the capability of numerical codes to simulate variable density flows. \cite{paillereComparisonLowMach2000,vierendeelsBenchmarkSolutionsNatural2003,tyliszczakProjectionMethodHighorder2014} In this section, we show the basic set-up and compare our results with the ones presented in the work of Vierendeels et al. \cite{vierendeelsBenchmarkSolutionsNatural2003} where benchmark solutions for the differentially heated cavity are presented. They solve the fully compressible Navier-Stokes equations on a $1024\times1024$ stretched grid using a Finite Volume Method with quadratic convergence.

The differentially heated cavity problem consists of a two-dimensional fully enclosed square cavity filled with fluid.  A sketch of the problem is shown in \cref{DHCGeom}. The left and right walls of the cavity have a constant temperature $\hat{T}_h$ and $\hat{T}_c$ respectively, with $\hat{T}_h >\hat{T}_c$, and the top and bottom walls are adiabatic. A gravity field induces fluid movement due to the density differences caused by the difference of temperature between the hot and cold walls.

The natural convection phenomenon is characterized by the Rayleigh number, defined as 
\begin{equation}\label{eq:Rayleigh}
\text{Ra} = \Prandtl \frac{\hat g \RefVal{\rho}^2(\hat T_h-\hat T_c) \RefVal{L}^3}{\RefVal{T}\RefVal{\mu}^2},
\end{equation}
For small values of $\text{Ra}$, conduction dominates the heat transfer process, and a boundary layer covers the whole domain. On the other hand large values of $\text{Ra}$ represent a convection dominated flow. For increasing $\text{Ra}$ number, a thinner boundary layer is formed. 

\paragraph{Set-up}
A reference velocity for buoyancy driven flows can be defined as\cite{vierendeelsBenchmarkSolutionsNatural2003}
\begin{equation}
\RefVal{u} = \frac{\sqrt{\text{Ra}} \RefVal{\mu}}{\RefVal{\rho}\RefVal{L}}.
\end{equation} 
The Rayleigh number is then related to the Reynolds number according to
\begin{equation}
\text{Re} = \sqrt{\text{Ra}}.
\end{equation}
Thus it is sufficient to select a $\Reynolds$ number in our simulation, fixing the value of the $\text{Ra}$ number. The driving temperature difference $(\hat T_h - \hat T_c)$ appearing in \cref{eq:Rayleigh} can be represented as an non-dimensional parameter:
\begin{equation}\label{eq:nondimensionalTemperature}
\varepsilon = \frac{\hat T_h - \hat T_c}{2\RefVal{T}}.
\end{equation}
Using these definitions, the Froude number can be calculated as 
\begin{equation}
\Froude = \sqrt{\Prandtl 2 \varepsilon}.
\end{equation}
All calculations assume a constant Prandtl number equal to 0.71. The viscosity and heat conductivity  dependence on temperature is calculated using \cref{eq:nondim_sutherland}.
Our results are calculated and compared with those of the reference solution for $\RefVal{T} = 600\si{K}$  and $\varepsilon = 0.6$. The non-dimensional length of the cavity is $L=1$. The non-dimensional temperatures $T_h$ and $T_c$ are set to 1.6 and 0.4, respectively. 
Since the cavity contains a single species, it is governed by the equations for continuity, momentum and temperature (\crefrange{eq:LowMachConti}{eq:LowMachEnergy}) and no equation for species transport is needed. Thus $n_s = 1$ and  $Y_1 = 1$ in the whole domain.  Moreover, the non-dimensional equation of state (\cref{eq:ideal_gas}) only depends on the temperature and reduces to
\begin{equation}
\rho = \frac{p_0}{T}.
\end{equation}
The thermodynamic pressure $p_0$ in a closed system and has to be adapted in order to ensure mass conservation. If $m_0$ is the initial total mass of the system, the thermodynamic pressure is given by
\begin{equation}
p_0 = \frac{m_0}{\int_\Omega \frac{1}{T}\text{d}V}, \label{eq:p0Condition}
\end{equation}
where $\Omega$ represents the complete closed domain. The initial mass of the system $m_0$ is constant and we set $m_0 = 1.0$. Within the solution algorithm, \cref{eq:p0Condition} is used to update the value of the thermodynamic pressure after each Newton-Dogleg iteration.
Moreover, the benchmark solution \cite{vierendeelsBenchmarkSolutionsNatural2003} also reports the Nusselt number and thermodynamic pressure associated with a given Rayleigh number. The Nusselt number is defined for a given wall $\Gamma$ in its averaged form as 
\begin{equation}\label{eq:Nusselt}
\text{Nu}_\Gamma = \frac{1}{T_h - T_c}\int_{\Gamma} k \pfrac{T}{x}\text{d}y.
\end{equation}\begin{table}[t]
	\begin{center}
		\begin{tabular}{cccccc}
			\hline
			Rayleigh & $p_0$ &  $p_{0,\text{ref}}$  &$\text{Nu}_{h}$ & $\text{Nu}_{c}$& $\text{Nu}_{\text{ref}}$ \\ \hline
			\parbox[0pt][13pt][c]{0pt}{}$10^2$   & 0.9574 & 0.9573 & 0.9787    & 0.9787 & 0.9787      \\
			$10^3$   & 0.9381 & 0.9381 & 1.1077    & 1.1077  & 1.1077      \\
			$10^4$   & 0.9146 & 0.9146 & 2.2180    & 2.2174  & 2.2180      \\
			$10^5$   & 0.9220 & 0.9220 & 4.4801    & 4.4796   & 4.4800      \\
			$10^6$   & 0.9245 & 0.9245 & 8.6866    & 8.6791  & 8.6870      \\
			$10^7$   & 0.9225 & 0.9226 & 16.2411   & 16.1700 & 16.2400     \\ \hline
		\end{tabular}
	\end{center}
	\caption[Differentially heated cavity: Results of Nusselt number and Thermodynamic pressure]{Comparison of calculated Nusselt numbers of the hot and cold wall and Thermodynamic pressure $p_0$ reported values\cite{vierendeelsBenchmarkSolutionsNatural2003} for the differentially heated cavity. Results are obtained for polynomial degree of four for the velocities and temperature, three for the pressure in an equidistant $128x128$ mesh.}
	\label{tab:p0_Nu_Results}
\end{table}
\paragraph{Comparison of results with benchmark solution}
The benchmark results \cite{vierendeelsBenchmarkSolutionsNatural2003} are presented for $\text{Ra} = \{10^2,10^3,10^4,10^5,10^6,10^7\}$. In this range of Rayleigh numbers the problem has a steady-state solution, thus we are able to use our steady formulation of the problem. The cavity is represented by the domain $[0,1]\times[0,1]$. We use an equidistant Cartesian mesh with $30 \times 30$ elements for each simulation. A polynomial degree of five is used for the velocities and temperature, and a degree of four for the pressure. 
It is observed that for cases until  $\text{Ra} = 10^5$ the solution of the system using Newton's method is possible without further modifications, while for higher values the algorithm diverges. The homotopy strategy mentioned in \cref{sec:CompMethodology} is used to overcome this problem and obtain solutions for higher Rayleigh (and equivalently higher Reynolds) numbers. Here, the Reynolds number is selected as the homotopy parameter and continuously increased until the desired value is reached.
In \cref{fig:TempProfile,fig:VelocityXProfile,fig:VelocityYProfile} temperature and velocity profiles for different Rayleigh numbers are shown. The profiles calculated with \BoSSS agree closely to the benchmark solution. As expected we observe 
an increase of the acceleration of the fluid in the vicinity of the walls for increasing Rayleigh numbers, forming a thin boundary layer. 



\begin{figure}
	\centering
	\pgfplotsset{width=0.45 \textwidth, compat=1.3}
	\inputtikz{HSCStreamlines}
	\caption{Streamlines of the heated cavity configuration with $\epsilon = 0.6$.}\label{fig:HSCStreamlines}
\end{figure}


\begin{figure}[!htb]
	\centering
	\pgfplotsset{width=0.20\textwidth, compat=1.3}
	\inputtikz{VelocityXProfile}
	\caption{ Profiles of the x-velocity component along the vertical line $x=0.5$. Solid lines represents our solution and the marks the benchmark solution. \cite{vierendeelsBenchmarkSolutionsNatural2003}}
	\label{fig:VelocityXProfile}
\end{figure}


\begin{figure}[b!]
	\centering
	\pgfplotsset{width=0.20\textwidth, compat=1.3}
	\inputtikz{VelocityYProfile}
	\caption{ Profiles of the y-velocity component along the horizontal line $y=0.5$. Solid lines represents our solution and marks the benchmark solution. \cite{vierendeelsBenchmarkSolutionsNatural2003}}
	\label{fig:VelocityYProfile}
\end{figure}

We also compare the thermodynamic pressure and the Nusselt numbers to the benchmark solution. The results are shown in \cref{tab:p0_Nu_Results}. The thermodynamic pressure is obtained from \cref{eq:p0Condition}, and the Nusselt number is calculated with \cref{eq:Nusselt}. We observe that our results are in very good agreement with the reference results, and the thermodynamic pressure differ at most in the fourth decimal place. Note that the Nusselt number of the heated  wall $(\text{Nu}_\text{h})$ and the Nusselt number of the cold wall $(\text{Nu}_\text{c})$ are different.  As the Rayleigh number grows, this discrepancy becomes bigger, hinting that at such Rayleigh numbers the used mesh is not refined enough  to represent adequately the thin boundary layer and more complex flow structures appearing at high-Rayleigh cases. While for an energy conservative system $\text{Nu}_\text{h}$ and $\text{Nu}_c$ should be equal, for our formulation this is not the case and the values differ slightly. This discrepancy can be seen as a measure of the discretization error from the DG formulation.\cite{kleinHighorderDiscontinuousGalerkin2016} This hints that the discrepancy between Nusselt numbers should decrease when increasing the mesh resolution, which will be discussed in the next section. 

\begin{figure}[b!]
	\centering
	\pgfplotsset{width=0.20\textwidth, compat=1.3}
	\inputtikz{TempProfile}
	\caption{Temperature profiles for the differentially heated square cavity along different vertical levels. Solid lines represent our solution and marks the benchmark solution. \cite{vierendeelsBenchmarkSolutionsNatural2003}}
	\label{fig:TempProfile}
\end{figure}

\begin{figure}[b!]
	\centering
	\inputtikz{NusseltStudy}
	\caption{Nusselt numbers calculated with \cref{eq:Nusselt} at the hot wall ($\text{Nu}_h$) and the cold wall ($\text{Nu}_c$) for different number of cells and polynomial order $k$. The reference values\cite{vierendeelsBenchmarkSolutionsNatural2003} are shown with dashed lines.}\label{fig:NusseltStudy}
\end{figure}

\paragraph{Convergence study}\label{ssec:ConvStudyHeatedCavity}
An $h-$convergence study of the differentially heated cavity configuration was conducted. Calculations were done for polynomial degrees $k = {1,2,3,4}$ and equidistant regular meshes with respectively $8, 16, 32, 64, 128$ and $256$ elements in each spatial coordinate.  The $L^2$-Norm was used for the calculation of the errors against the solution on the finest mesh. Results of the $h$-convergence study for varying polynomial orders $k$ are shown in \cref{fig:ConvergenceDHC}. Recall that for increasing polynomial order, the expected order of convergence is given by the slope of the line curve when cell length and errors are presented in a log-log plot. Because we are using a mixed order formulation the slopes should be equal to $k$ for the pressure and equal to $k+1$ for all other variables.  It is observed how convergence rates scale approximately as $k+1$. Interestingly, for $k=2$ the rates are higher than expected. On the other hand, some degeneration on the convergence rates is observed for $k = 4$.

As discussed in the last section, the difference on values of the Nusselt number on the hot wall $\text{Nu}_\text{h}$ and the cold wall $\text{Nu}_c$ is a consequence of spatial discretization error. In \cref{fig:NusseltStudy} the convergence behaviour of the Nusselt number for different polynomial degrees $k$, different number of elements and for two different Ra numbers is presented. As expected, it can be observed that this discrepancy is smaller when a higher number of elements is used. 


\begin{figure}[t!]
	\centering
	\pgfplotsset{width=0.34\textwidth, compat=1.3}
	\inputtikz{ConvergenceDHC}
	\caption{Convergence study of the differentially heated cavity problem for $\text{Ra} = 10^3$.}\label{fig:ConvergenceDHC}
\end{figure}




\subsection{Poiseuille–Rayleigh–Bénard instability in a channel}
\blindtext[5]


\subsection{Flow around a heated cylinder}
\subsubsection{Square cylinder}
\blindtext[5]
\subsubsection{Circle? cylinder}
\blindtext[5]



\section{Multi-component non-isothermal cases}\label{sec:MultCompNonIsothermCase}
Later in in \cref{ss:CDF} and \cref{ss:CoFlowFlame} two different configurations for reactive flows are presented. 
In the following, we show results of the simulations of two test cases with combustion, namely the counterflow diffusion flame and the chambered diffusion flame. For both cases, the solution of the flame sheet problem described in \cref{ssec:FlameSheet} is calculated first. This solution is used subsequently as initial estimates for the solution of the finite chemistry rate problem (c.f. \cref{ssec:NonDimLowMachEquations}). In all test cases presented in this section, a smoothing parameter $\sigma = 50$ was used (c.f. \cref{ssec:FlameSheet}). For all test cases methane combustion according to the one-step model shown in \cref{sec:ChemModel} is considered. The mass fraction transport \cref{eq:LowMachMassBalance} is solved for the species \ch{CH4}, \ch{O2}, \ch{CO2} and \ch{H2O}, thus $\vec{Y}' = \left(Y_{\ch{CH4}},Y_{\ch{O2}},Y_{\ch{CO2}},Y_{\ch{H2O}}\right)$. The nitrogen mass fraction $Y_{\ch{N2}}$ is calculated according to \cref{eq:MassFractionConstraint}. 



\subsection{CoFlowing flame?}

As a first test for assessing basic behaviour related to combustion, a cofloing flame configuration used. This test case is the main prototype flame for diffusion regimes \cite{poinsotTheoreticalNumericalCombustion2005}. It is set up by sending a stream of fuel against a stream containing oxidizer. A diagram can be seen in X
%TODO ACA VA UN sketch del coflowing flame con las boundary conditions 
\begin{figure}[t!]
	\centering
	\pgfplotsset{
		compat=1.3,
		tick align = outside,
		yticklabel style={/pgf/number format/fixed}, 
	}
	\inputtikz{CoFlow_ConvergenceStory}
	\caption{Typical convergence history of a diffusion flame in the coflowing flame configuration with mesh refinement.}
	\label{fig:CoFlow_ConvergenceStory}
\end{figure} 
% \begin{figure}[t!]
% 	\begin{center}
% 		\def\svgwidth{0.5\textwidth}

% 		\caption{Schematic representation of strained diffusion flame}
% 		\label{SSDFSketch}
% 	\end{center}	
% \end{figure}
The area where the chemical reaction takes place is usually a thin region, which thickness is defined by the availability of reactants, which at the same time are controled by the velocity which the chemical reaction happens. This factor is governed by the $Da$ number in the non-dimensional formulation (sure??). It is of critical importance for the numerical simulation that the flame is resolved accordingly, which can demand a very fine mesh. Not doing so can provoke non-physically effects to occur, as for example negative values of the reaction term $\omega$ (in the present formulation the reaction is irreversible, and thus only positive values of $\omega$ make sense). 

For avoiding over-resolving in zones where actually no reaction is taking place (or very slowly), an adaptive mesh refinement strategy  (see section X) within a pseudo-time-stepping framework was used.  Here a  suitable strategy for choosing cells to be refined is necesary . For reactive flows, this strategy is based on the variable $\omega$.
Before each refinement step the values of $\omega$ are normalized by the biggest value of the domain, and according to this normalized effects the cells are refined. %In Figure \ref{fig:AMR} the refined meshes can be seen. The legend of the pseudocolor plots is not shown, because for each plot the magnitude scales are different, and just the normalized value accounts for the AMR. 


\begin{figure}
	\centering
	\pgfplotsset{width=0.75 \textwidth, compat=1.3}
	\inputtikz{CoFlowFlameFig1}
	\inputtikz{CoFlowFlameFig2}	
	\caption{Coflowing Flame.} \label{fig:CoFlowFlameFig}
\end{figure}
\subsection{Counterflow diffusion flame}\label{ss:CDF}	

\begin{figure}[b]
	\begin{center}
		\def\svgwidth{0.8\textwidth}
		\import{./plots/}{CounterDiffusionFlame_sketch_rotated2.pdf_tex}		
\caption{Schematic representation of the counterflow configuration.}
\label{fig:CDFScheme}
	\end{center}	
\end{figure} 

The counterflow diffusion flame is a canonical configuration used to study the structure of non-premixed flames. In its most basic configuration it consists of two oppositely situated jets. The fuel (possibly mixed with some inert component such as nitrogen) is fed into the system by one of the jets, while the other jet feeds air to the system, thereby establishing a stagnation point flow. Upon contact, the reactants produce a flame which is located in the vicinity of the stagnation plane. A diagram of the setup can be seen in \cref{fig:CDFScheme}. This simple configuration has been subject of study for decades  because it provides a simple way of creating a strained diffusion flame, which proves to be useful when studying the flame structure, extinction limits or production of pollutants of flames \cite{pandyaStructureFlatCounterFlow1964} \cite{spaldingTheoryMixingChemical1961} \cite{keyesFlameSheetStarting1987}. By assuming an infinite injector diameter and self-similarity of the solution, it is possible to reduce the governing equations to a one-dimensional formulation (see e.g. the textbook of Kee \cite{keeChemicallyReactingFlow2003}). As a means of validating our implementation we compare the results with the solution of the one-dimensional self-similar problem calculated with \lstinline|BVP4|, a fourth order finite difference boundary value problem solver provided by \lstinline|MATLAB|. 

The combustion of a methane-nitrogen mixture with air was simulated using the \BoSSS code. The mass composition of the fuel inlet was assumed to be  $Y^0_{\ch{CH4}} = 0.2$ and $Y^0_{\ch{N2}} = 0.8$, and the oxidizer inlet corresponds to air with   $Y^0_{\ch{O2}} = 0.23$ and $Y^0_{\ch{N2}} = 0.77$. Because we are dealing with an open system, the thermodynamic pressure $\hat p_0$ is constant and set to the ambient pressure of $\SI{101325}{\pascal}$. As noted by Sung et al., \cite{sungStructuralResponseCounterflow1995} although the form of the inlet velocity profiles does have an influence on the solution of the problem, its effect on the solution near the flame zone is rather small. Nevertheless, as mentioned before the solution of the self-similar one-dimensional problem assumed an infinite injector diameter, which implies that the upstream velocity field has to be constant. Based on this fact we set the velocity profile of both inlets as a plug flow. Following combinations of inlet velocities were calculated:
\begin{itemize}
	\item  Low inlet velocities:  $u^0_{\text{fuel}} = \SI{0.048}{\meter \per \second}$ and  $u^0_{\text{oxidizer}} = \SI{0.144}{\meter \per \second}$,
	\item Medium inlet velocities:  $u^0_{\text{fuel}} = \SI{0.12}{\meter \per \second}$ and  $u^0_{\text{oxidizer}} = \SI{0.36}{\meter \per \second}$
	\item High inlet velocities: $u^0_{\text{fuel}}  = \SI{0.24}{\meter \per \second}$ and   $u^0_{\text{oxidizer}} = \SI{0.72}{\meter \per \second}$
\end{itemize}
By using as definition of the strain rate the maximum axial velocity gradient, the calculated strains for the three cases mentioned above are $\SI{34}{\per\second}$, $\SI{76}{\per\second}$ and $\SI{155}{\per\second}$, respectively. The temperature of both inlets is \SI{300}{\kelvin}. The separation between both jets $\hat L$ is equal to $\SI{0.02}{\meter}$, and the length of the inlet opening $\hat D$ is $\SI{0.02}{\meter}$. The left and right domain boundaries are selected to be at a distance $3\hat L$ of the center. A non-unity but constant Lewis number formulation is used,with  
$\Lewis_{\ch{CH4}} =  0.97 $ , $\Lewis_{\ch{O2}} = 1.11 $, $\Lewis_{\ch{H2O}} = 0.83 $ and $\Lewis_{\ch{CO2}} = 1.39 $.\cite{smookePremixedNonpremixedTest1991} The heat capacity of each component is evaluated locally from NASA polynomials, and the mixture heat capacity is calculated with \cref{eq:nondim_cpmixture}.
%\tikzset{external/export next=false}
\begin{figure}[t!]
	\centering
	\pgfplotsset{
		compat=1.3,
		tick align = outside,
		yticklabel style={/pgf/number format/fixed}, 
	}
\inputtikz{CDF_ConvergenceStory}
	\caption{Convergence history of the diffusion flame in the counterflow configuration, with a maximum strain value of $165.1 $\si{s^{-1}}}
	\label{fig:CDF_ConvergenceStory}
\end{figure} 

In \cref{fig:CDF_ConvergenceStory} the convergence history obtained for a typical calculation of the counter diffusion flame is presented. The solution of the flame sheet calculation requires 17 iterations until convergence is reached. The obtained solution is used as a starting value for the finite chemical rate calculation, which only needs 10 iterations until convergence is reached. We note that because the flame sheet calculation is only used as an approximation of the final solution, a low polynomial degree can be used. For the flame sheet calculation $k = 2$ was chosen, resulting in a rather small system with 26,880 degrees of freedom. For the finite rate calculation $k = 4$ was used, which resulted in a system with 174,110 degrees of freedom.  With the above-mentioned another advantage of the approach of using the flame sheet calculation for two-dimensional simulations can be highlighted. The initial estimate can be found relatively easily for a system with few degrees of freedom. Using the solution found as the initial estimate has the consequence that Newton's algorithm for the complete problem (which has many more degrees of freedom) only needs a few iterations to find a solution. 

Because we are solving a two-dimensional configuration, in order to be able to compare our results with the ones obtained from the one-dimensional representation, the temperature, mass fractions and velocity profiles are extracted along the centerline of the system shown as the dashed line in \cref{fig:CDFScheme}. In \cref{fig:BoSSS_1D_Comparison_velocity} a comparison of the axial velocities calculated with \BoSSS and the one-dimensional solution is shown. While for the high strain case the results agree closely, for lower strains a discrepancy is observed. Recall that the derivation of the one-dimensional approximation assumes a constant velocity field incoming to the flame zone in order to obtain a self-similar solution. In the case of the two-dimensional configuration presented here, the border effects do have an influence on the centerline, which disrupts the self-similarity. This effect is more pronounced for low velocities, which explains the discrepancy between curves. Similarly, In \cref{fig:BoSSS_1D_Comparison} the temperature and mass fraction fields are presented. Again, a discrepancy is observed for low strains, but results show a good agreement for higher inlet velocities. It can also be observed how, as expected, \cite{fernandez-tarrazoSimpleOnestepChemistry2006} at higher strains a significant penetration (leakage) of oxygen across the flame is present. Finally, in \cref{fig:TempAndReacFields} the two-dimensional temperature, velocity and reaction rate fields for the case (a) are shown. 

Finally, in \Cref{fig:TemperatureConvergenceDiffFlame} we show how the maximum value of the temperature behaves for different mesh resolutions and polynomial degrees. The temperature tends to a limit value, and we observe how this value is reached already for coarse meshes when using a polynomial degree of three or four. We also observe that for $k=2$ the temperature tends to a limit value, but slower in comparison to $k =3$ or $4$. The values for $k=1$ are not displayed because for the mesh resolutions shown here, the values of the maximum temperature were of the order of 50 \si{K} bigger than the limit value. We note that the solution of this configuration showed singularities in the boundary points where the inlet and wall meet. This fact made hard to realize a $h$-convergence study for the complete domain. Based on this we decided to analyse a flame configuration that doesn't exhibit this behaviour, as shown in the next section.


We note that the solution of this configuration showed singularities in the boundary points where the inlet and wall meet, which induces a pollution phenomenon on the accuracy of the solution. This fact made hard to realize an $h$-convergence study for the complete domain. Based on this fact we decided to analyze a flame configuration that does not exhibit this behaviour, as shown in the next section.
\begin{figure}[t!]
	\pgfplotsset{
		group/xticklabels at=edge bottom,
		legend style = {
			at ={ (0.5,1), anchor= north east}
		},
		unit code/.code={\si{#1}}
	}
	\centering
	\inputtikz{BoSSS_1D_Comparison_velocity}
	\caption{ Comparison of the axial velocity calculated with \BoSSS and the one-dimensional approximation. }
	\label{fig:BoSSS_1D_Comparison_velocity}
\end{figure}
\newpage
\tikzexternaldisable
%\tikzset{external/export next=false}%%%%%%%%%%%%%%%%%%%%%%%%%%%%%%%%%%%%%%%%%%%%%%%%%%%%%%%%%%%%%%%%%%%%%%%%%%%%%%%%%%%%%%%%%%%%%%%%%%%%%%%
\begin{figure}[b!]
	\centering
	\pgfplotsset{
		width=0.85\textwidth,
		height = 0.3\textwidth,
		compat=1.3,
		tick align = outside,
		yticklabel style={/pgf/number format/fixed}, 
	}	
	\inputtikz{BoSSS_1D_Comparison1}
	\inputtikz{BoSSS_1D_Comparison2}
	\inputtikz{BoSSS_1D_Comparison3}
	\caption{Comparison of temperature and mass fraction fields obtained with \BoSSS and the one-dimensional approximation.}
	\label{fig:BoSSS_1D_Comparison}
\end{figure}
\tikzexternalenable
\newpage
\begin{figure}[b]
	\begin{center}
		\def\svgwidth{0.8\textwidth}
		\import{./plots/}{CDF_Results.pdf_tex}		
\caption{Calculated temperature and velocity fields (top picture) and reaction rate (second picture) of the counter diffusion flame configuration, case (a). The unit of the temperature is \si{K} and of the reaction rate \si{\kilo\mole \per \meter \cubed \per \second}. }
	\label{fig:TempAndReacFields}
	\end{center}	
\end{figure} 

\begin{figure}[tbp]
	\centering
	\inputtikz{TemperatureConvergenceDiffFlame}
	\caption{Maximum value of the temperature for the counter diffusion flame configuration, for different mesh sizes in the x-direction and polynomial degrees. Values for $k=1$ are not shown, because for this range of cell lengths the maximum temperature value was of the order of 50K higher than the ones depicted here.}
	\label{fig:TemperatureConvergenceDiffFlame}
\end{figure}
\FloatBarrier
\newpage
\subsection{Chambered diffusion flame}\label{ss:UDF}
\begin{figure}[b]
	\begin{center}
		\def\svgwidth{0.8\textwidth}
		\import{./plots/}{UnstrainedFlameConfig.pdf_tex}		
		\caption{Schematic representation of the chambered diffusion flame configuration. }
\label{fig:chamberedDifFlame}
	\end{center}	
\end{figure} 

The chambered diffusion flame configuration has served as a model for many theoretical studies related to diffusion flames\cite{matalonEffectThermalExpansion2010,rameauNumericalBifurcationChambered1985,matalonDiffusionFlamesChamber1980} A scheme of the configuration can be seen in \cref{fig:chamberedDifFlame}. Fuel is injected at a constant rate into the bottom of a small insulated chamber, while oxidant diffuses into the system against the direction of flow. Constant conditions at the outlet of the chamber are achieved by a rapid renewal of the flow of oxidant.  Under these conditions a planar flame forms far away from the walls, which allows a one-dimensional description of the flame structure.
The fuel inlet into the chamber is modelled with a velocity inlet boundary condition \cref{eq:bc_d}, while the flow outlet at the top is considered an outlet as given by \cref{eq:bc_OD}. Since we are interested in the flame in wall distance, it is sufficient to set the remaining boundary conditions as periodic boundaries. This effectively transforms the problem into a pseudo two-dimensional configuration. 


\begin{figure}[t!]
	\centering
	\pgfplotsset{width=0.34\textwidth, compat=1.3}
\inputtikz{ConvergenceDiffFlame}
	\caption{Convergence study for the chambered diffusion flame configuration.}
	\label{ConvergenceDiffFlame}
\end{figure}

In this test case we study the combustion of a \ch{CH4}-\ch{N2} mixture with air. The thermodynamic pressure $\hat p_0$ is set equal to an ambient pressure of $\SI{101325}{\pascal}$. The inlet velocity of the fuel jet is set to $\SI{0.025}{\meter \per \second}$ and its mass composition is $Y^0_{\ch{CH4}} = 0.2$ and $Y^0_{\ch{N2}} = 0.8$ while air has a composition $Y^0_{\ch{O2}} = 0.23$ and $Y^0_{\ch{N2}} = 0.77$. The temperature of the fuel and air feed streams is $\SI{300}{\kelvin}$. The length of the system $L$ is equal to $\SI{0.015}{\meter}$.
For this configuration an $h$-convergence study is conducted, where uniform Cartesian meshes with  $5\times2^6$, $5\times2^7$, $5\times2^8$,  $5\times2^9$ and $5\times2^{10}$  cells are used. The polynomial degrees are varied from 1 to 4 for velocity, temperature and mass fractions, and from 0 to 3 for pressure.  Errors are calculated using the finest mesh as a reference solution.  The results are shown in \cref{ConvergenceDiffFlame} for variables $u_x$, $T$, $Y_{\ch{CH4}}$ and $p$. The convergence results for other variables are similar and not shown here. We observe the expected slope increase with increasing polynomial degrees. For low polynomial degrees the orders of convergence are very close to the theoretical values. However for higher polynomial degrees we observe a slight deterioration of the convergence rate.  
\FloatBarrier
	