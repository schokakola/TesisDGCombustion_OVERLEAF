
\subsection{Chambered diffusion flame}\label{ss:UDF}
\begin{figure}[b]
	\begin{center}
		\def\svgwidth{0.8\textwidth}
		\import{./plots/}{UnstrainedFlameConfig.pdf_tex}
		\caption{Schematic representation of the chambered diffusion flame configuration. }
		\label{fig:chamberedDifFlame}
	\end{center}
\end{figure}

The chambered diffusion flame configuration has served as a model for many theoretical studies related to diffusion flames\cite{matalonEffectThermalExpansion2010,rameauNumericalBifurcationChambered1985,matalonDiffusionFlamesChamber1980} A scheme of the configuration can be seen in \cref{fig:chamberedDifFlame}. Fuel is injected at a constant rate into the bottom of a small insulated chamber, while oxidant diffuses into the system against the direction of flow. Constant conditions at the outlet of the chamber are achieved by a rapid renewal of the flow of oxidant.  Under these conditions a planar flame forms far away from the walls, which allows a one-dimensional description of the flame structure.
The fuel inlet into the chamber is modelled with a velocity inlet boundary condition \cref{eq:bc_d}, while the flow outlet at the top is considered an outlet as given by \cref{eq:bc_OD}. Since we are interested in the flame in wall distance, it is sufficient to set the remaining boundary conditions as periodic boundaries. This effectively transforms the problem into a pseudo two-dimensional configuration.


\begin{figure}[t!]
	\centering
	\pgfplotsset{width=0.34\textwidth, compat=1.3}
	\inputtikz{ConvergenceDiffFlame}
	\caption{Convergence study for the chambered diffusion flame configuration.}
	\label{ConvergenceDiffFlame}
\end{figure}

In this test case we study the combustion of a \ch{CH4}-\ch{N2} mixture with air. The thermodynamic pressure $\hat p_0$ is set equal to an ambient pressure of $\SI{101325}{\pascal}$. The inlet velocity of the fuel jet is set to $\SI{0.025}{\meter \per \second}$ and its mass composition is $Y^0_{\ch{CH4}} = 0.2$ and $Y^0_{\ch{N2}} = 0.8$ while air has a composition $Y^0_{\ch{O2}} = 0.23$ and $Y^0_{\ch{N2}} = 0.77$. The temperature of the fuel and air feed streams is $\SI{300}{\kelvin}$. The length of the system $L$ is equal to $\SI{0.015}{\meter}$.
For this configuration an $h$-convergence study is conducted, where uniform Cartesian meshes with  $5\times2^6$, $5\times2^7$, $5\times2^8$,  $5\times2^9$ and $5\times2^{10}$  cells are used. The polynomial degrees are varied from 1 to 4 for velocity, temperature and mass fractions, and from 0 to 3 for pressure.  Errors are calculated using the finest mesh as a reference solution.  The results are shown in \cref{ConvergenceDiffFlame} for variables $u_x$, $T$, $Y_{\ch{CH4}}$ and $p$. The convergence results for other variables are similar and not shown here. We observe the expected slope increase with increasing polynomial degrees. For low polynomial degrees the orders of convergence are very close to the theoretical values. However for higher polynomial degrees we observe a slight deterioration of the convergence rate.