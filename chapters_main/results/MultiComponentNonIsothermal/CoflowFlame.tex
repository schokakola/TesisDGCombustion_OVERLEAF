
\subsection{Co-flow laminar diffusion flame}\label{ssec:coflowFlame}
%C:\Users\jfgj8\AppData\Local\BoSSS\plots\sessions\CoFlowFlameIntentoParaTesis2__Full_CoFlowFlamerP3K12smoothfactor0velMult2__e1120753-addc-4671-a7d4-7443132981fb
The co-flowing flame configuration is used as a first test to assess the behavior of the solver for reactive flows applications. It basically consists of two concentric ducts that emit fuel and oxidant into the system, creating a flame. This configuration has been widely studied. In the seminal work of \cite{burkeDiffusionFlames1928} analytical expressions for the flame height and flame shape are obtained by studying a very simplified problem (constant density and velocity field, infinitely fast chemistry, among others). Later, \cite{smookeNumericalModelingAxisymmetric1992} and later works solved this configuration using a 2D-axisymmetric system and also used the flame sheet estimates to find adequate initial conditions for their Newton algorithm. It should be noted that the solution of the axisymmetric system of equations presents numerical difficulties that are not the main concern of the present work. For this reason, it was decided to solve a system with similar characteristics but which is possible to represent in Cartesian coordinates. This is possible if an infinitely long slot burner is considered. A schematic diagram of the configuration can be seen in \cref{fig:CoFlowSketch}. 


The area where the chemical reaction takes place is usually a thin region, whose thickness is defined by the availability of reactants. It is of critical importance for the numerical simulation to have an adequate mesh that allows to solve the flame appropriately. . Not doing so can lead to non-physically effects and the aparition of unwanted numerical artifacts. An example of this are negative values of the reaction term $\omega$ (in the present formulation the reaction is irreversible, and thus only positive values of $\omega$ make sense). 

For avoiding over-resolving in zones where actually no reaction is taking place (or very slowly), an adaptive mesh refinement strategy  (see section X) within a pseudo-time-stepping framework was used.  Here a suitable strategy for choosing cells to be refined. For reactive flows, this strategy is based on the variable $\omega$.
Before each refinement step the values of $\omega$ are normalized by the highest value of the domain, and according to this normalized effects the cells are refined. %In Figure \ref{fig:AMR} the refined meshes can be seen. The legend of the pseudocolor plots is not shown, because for each plot the magnitude scales are different, and just the normalized value accounts for the AMR. 

Set up parameters: 
\begin{figure}[t!]
	\centering
	\pgfplotsset{
		compat=1.3,
		tick align = outside,
		yticklabel style={/pgf/number format/fixed},
	}
	\inputtikz{CoFlow_ConvergenceStory}
	\caption{Typical convergence history of a diffusion flame in the coflowing flame configuration with mesh refinement.}
	\label{fig:CoFlow_ConvergenceStory}
\end{figure}
\begin{figure}[t]
	\centering
	\pgfplotsset{width=0.65\textwidth, compat=1.3}
	\inputtikz{CoFlowFlameFig3}
	\inputtikz{CoFlowFlameFig2}
	\inputtikz{CoFlowFlameFig1}	
	\caption{Coflowing Flame.} \label{fig:CoFlowFlameFig}
\end{figure}


\FloatBarrier