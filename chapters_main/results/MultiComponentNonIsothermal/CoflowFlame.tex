\subsection{Coflow laminar diffusion flame}\label{ssec:coflowFlame}
The coflowing flame configuration is used as a first test to assess the behavior of the solver for reactive flows applications. It consists of two concentric ducts that emit fuel and oxidant into the system, which after ignition forms a flame. This configuration has been widely studied, starting with the seminal work of \textcite{burkeDiffusionFlames1928} and followed by many others (see, for example, \textcite{smookeNumericalModelingAxisymmetric1992, smookeNumericalSolutionTwoDimensional1986,braackAdaptiveFiniteElement1997}). In particular, the work from \textcite{smookeNumericalModelingAxisymmetric1992} and later publications solved this configuration using a 2D-axisymmetric system and also used the flame sheet estimates to find adequate initial conditions for their Newton algorithm. 
\begin{figure}[t]
	\centering
	\def\svgwidth{0.38\textwidth}
	\subcaptionbox{Sketch\label{fig:CoFlowSketch}}{
		\import{./plots/}{CoFlowSketch_withBC.pdf_tex}\vspace{0.5cm}
	}
	\qquad\quad
	\def\svgwidth{0.35\textwidth}
	\subcaptionbox{Refined mesh \label{fig:CoFlowMesh}}{
		\vspace{1.2cm}
		\import{./plots/}{CoFlowMesh.pdf_tex}
	}
	\caption{Geometry of a coflowing flame configuration (not to scale).} \label{fig:CoFlowGeometry}
\end{figure}

Since the solution of the axisymmetric system of equations presents numerical difficulties that are not the main concern of the present work, in this section an infinitely long slot burner configuration is considered. For that kind of configuration, cartesian coordinates describe the problem naturally. 
\subsubsection{Set-up}
A schematic diagram of the configuration can be seen in \cref{fig:CoFlowSketch}. The system consists of a fuel inlet with two oxygen inlets on its sides. These inlets are separated by a finite wall thickness. The inclusion of this separation is observed to be necessary to be able to obtain a converged solution, which makes sense, since a finite separation between both inlets smooths the big gradients due to strong mixing and reaction in that area. Although the system is clearly symmetric around the axis $x = 0$, no symmetry assumption is made and the whole domain is considered for the simulation.

The lengths depicted in \cref{fig:CoFlowSketch} are set as $ r_1 = 1$, $ r_2 = 1.2$, and $ r_3 = 11.763$. Aditionally $ h = 4$ and $ H = 63$. The lengths $ r_3$ and $H$ are set with arbitrarily large values in order to avoid influence of the outer boundary conditions on the solution of the flame zone. Setting a higher value of $ r_3$ or $ L$ did not significantly affect the results. The inlet boundary conditions are set as: 
\begin{itemize}
	\item Oxidizer inlet: $\{\forall (x,y): y = -h \land x \in [-r_3,-r_2]\cup[r_2,r_3]\}$\\
	\begin{equation*}
		u = 0,\qquad v= v_O, \qquad T = T^O, \qquad \vec{Y}' = (0,Y^O_{\ch{O2}},0,0)
	\end{equation*}
\item Fuel Inlet: $\{\forall (x,y): y = -h \land x \in [-r_1,r_1]\} $ \\
\begin{equation*}
	u = 0,\qquad v= v^F(x), \qquad T = T^F, \qquad \vec{Y}' = (Y^F_{\ch{CH4}},0,0,0)
\end{equation*}
\end{itemize}
The oxidizer enters the system as a plug flow with a constant velocity of $v_O = 1 $. The inlet velocity of the fuel stream $v_F$ is a a parabolic profile given by
\begin{equation}
	v^F(x) = \left[1-\left(\frac{x}{X_1}\right)^2\right]v_m^F
\end{equation}
with $v^F_m =0.592$.  The inlet temperatures of both streams is  $T^F = T^O = 1$. Combustion of diluted methane on air is considered, with $Y^F_{\ch{CH4}} = 0.2$ and $Y^F_{\ch{N2}} = 0.8$ for the fuel stream, and  $Y^O_{\ch{O2}} = 0.23$ and $Y^O_{\ch{N2}} = 0.77$ for the oxidyzer stream. The superscripts $F$ and $O$ represent the fuel and oxidizer inlet respectively. The pressure outlet boundary condition is the same as \cref{eq:bc_O}. Finally, the boundary conditions at the tips correspond to adiabatic walls, which are defined as in \cref{eq:bc_dn}, with $\vec{u}_{\text{D}} = (0,0)$.             

The variables are nondimensionalized in the usual way. The reference length is set $\RefVal{L} =  \SI{0.635}{\centi \meter}$ and the reference velocity $\RefVal{u} =\SI{8.19}{\centi \meter \per \second}$. The reference temperature is $\RefVal{T} =\SI{300}{K}$.  All derived variables are nondimensionalized using the air stream as a reference condition, i.e. $\RefVal{\rho} = \SI{1.17}{\kilo \gram \per \cubic \meter}$, $\RefVal{\mu} = \SI{1.85e-5}{\kilo \gram \per \meter \per \second}$ and $\RefVal{W} = \SI{28.82}{\kilo \gram \per \kilo\mole}$, giving the nondimensional numbers $\Reynolds = 33.02$ and $\text{Da} = 2.17\cdot 10^9$. The Prandtl number is assumed to be constant with $\Prandtl = 0.71$. The reference heat capacity is set $\hat{c}_{p,\text{ref}}= \SI{1.3}{\kilo \joule \per \kilo \gram \per \kelvin}$, which is also the constant value used for the flame sheet calculation (i.e. $c_p = 1$). The choice of this value for the heat capacity is important because it gives a solution of the flame sheet which is similar to the actual solution of the full problem. Gravity effects are not taken into account. The transport parameters are calculated using Sutherland's law with $\hat{S} = \SI{110.5}{\kelvin}$. The mixture heat capacity $c_p$ is calculated with \cref{eq:nondim_cpmixture} and using NASA polynomials for the heat capacity of each component. Finally, a nonunity but constant Lewis number formulation is used, with
$\Lewis_{\ch{CH4}} =  0.97 $ , $\Lewis_{\ch{O2}} = 1.11 $, $\Lewis_{\ch{H2O}} = 0.83 $ and $\Lewis_{\ch{CO2}} = 1.39 $ \parencite{smookePremixedNonpremixedTest1991}
\begin{figure}[b!]
	\centering
	\pgfplotsset{
		compat=1.3,
		tick align = outside,
		yticklabel style={/pgf/number format/fixed},
	}
	\inputtikz{CoFlow_ConvergenceStory}
	\caption{Typical convergence history of a diffusion flame in the coflowing flame configuration.A mesh refinement is done at iteration 21. }
	\label{fig:CoFlow_ConvergenceStory}
\end{figure}
\subsubsection{Numerical results}
The purpose of simulating this case is to test the XNSEC solver in a real-world application, with realistic physical parameter values. In particular, it is intended to demonstrate that the strategy of using the flame sheet solution as the initial estimate is adequate for obtaining a converged solution, even for a case where the physical properties of the system ($\rho, \mu, c_p$) are composition and temperature dependent, and for nonunity Lewis numbers. 
\begin{figure}[t!]
	\centering
	\pgfplotsset{width=0.6\textwidth, compat=1.3}
	\inputtikz{CoFlowFlameFigTemperature}%
	\hspace{-2.4cm} 	
	\inputtikz{CoFlowkReact}
	\caption{Temperature and reaction rate fields of the coflow configuration.} \label{fig:CoFlowFlameFig}
\end{figure}
Numerical experiments using the XNSEC solver showed that the solution of the problem is highly mesh-dependent. The presence of very high gradients in some areas requires a higher density of cells to obtain a well-resolved solution.  In \cref{fig:CoFlowMesh} the actual mesh used for the solution of the full problem is shown.  A base mesh with smaller elements in the vicinity of the inlets and larger elements further away from them is used. It is observed that the complex mixing and combustion phenomena that occur in the vicinity of the inlets have a critical effect on the convergence of the solution. For this reason, a special refinement of the base mesh is done in the vicinity of the tips. Another reason that could explain the need for extra refinement in the vicinity of the inlets is that the use of high-order methods introduces very little numerical diffusion into the formulation, which in theory would help to smooth out the large gradients that exist in the area. Furthermore, an adequate mesh resolution  at the flame location is critical. In order to avoid over-solving, an adaptive mesh refinement strategy in a pseudo-time-stepping framework is used. After obtaining a steady-state solution, the mesh is refined and the calculation is started again (see \cref{ssec:MeshRefinement}). In particular, for this case, three pseudo-timesteps were performed for the flame sheet calculation. After each pseudo-timestep, the mesh is refined in the vicinity of the flame sheet, that is, in the cells where $z = z_{\text{st}}$. 
Obviously, a finer grid in the vicinity of the surface $z = z_{\text{st}}$ will be beneficial for the solution of the finite reaction rate problem if the same conditions used to derive the flame sheet equations (namely constant $c_p$ and unity Lewis number) are assumed.  
However, experiments with the XNSEC solver have shown that this refinement strategy is still beneficial for the convergence of the complete problem even with non-constant $c_p$ and nonunity Lewis numbers.

In \cref{fig:CoFlow_ConvergenceStory} the convergence history using the Newton algorithm presented in \cref{sec:newton} is shown. The flame sheet calculation requires 20 Newton iterations to find a solution. It is clearly seen that the residuals $\| \mathcal{A}(\myvector{U}_{n}) \|_2 $  decrease very slowly for about the first 14 iterations, while the parameter $\delta$ of the globalised Newton method is adapted to find an optimal value to reduce the residuals. Around iteration 14 the algorithm starts to increase $\delta$, leading to a faster reduction of the residuals. A solution to the problem according to the termination criterion exposed in \cref{ssec:TerminationCriterion} is found in iteration number 20. In iteration 21 mesh refinement based on the strategy mentioned above is used and now only 6 iterations are required to find a converged solution. Finally, in iteration number 27 the flame sheet solution is used as the initial estimate for the full problem, which requires only 11 iterations to find a solution.  

For the flame sheet calculation a polynomial degree $k = 2$ is chosen, resulting in a rather small system with 49,140 degrees of freedom. For the finite rate calculation $k = 4$ is used, which resulted in a system with 482,310 degrees of freedom.  This highlights another advantage of the approach of using the flame sheet calculation for two-dimensional simulations can be highlighted. The initial estimate can be found relatively easily for a system with few degrees of freedom. Using the solution found as the initial estimate has the consequence that Newton's algorithm for the complete problem (which has many more degrees of freedom) only needs a few iterations to find a solution.

In \cref{fig:CoFlowFlameFig} the obtained temperature and reaction rate fields are shown. Since for the selected inlet velocities the flame corresponds to an overventilated one, the typical jet form is observed. The maximum dimensionless temperature $T$  reached corresponds to 6.04 (meaning $\SI{1812}{\kelvin}$). Magnified plots show that the bottom part of the flame sits on the outside part of the tips. The high reaction rates appearing in the area close to the inlets are also worth highlighting. This could explain why the mesh refinement in the vicinity of the inlets is crucial to obtain a converged solution. 

\begin{figure}[h]
	\centering
	\pgfplotsset{width=0.6\textwidth, compat=1.3}
        \inputtikz{CoFlowMF3_infiniteFinite}
		\caption{Mass fraction field of  \ch{H2O} over the line $y = 10$}
		\label{fig:CoFlowMF3_infiniteFinite}
	\end{figure}


A question to consider is whether indeed the initialization of the finite reaction rate problem with the flame sheet provides adequate estimates for the Newton algorithm, in particular for more complex problems that do not fulfil the assumptions made to obtain the equations of the mixture fraction problem such as systems with nonconstant heat capacity $c_p$ or with nonunity Lewis numbers. 
For illustrative purposes, the simulation of the coflowing flame for unity and nonunity lewis numbers was performed. In \cref{fig:CoFlowMF3_infiniteFinite} the solution obtained for the mass fraction field of \ch{H2O} along the line $y=10$ is shown. As expected, the solution obtained for the case with unity Lewis number is very close to the one obtained using the flame sheet. Furthermore. the case with nonunity Lewis number also presents a solution very similar to that of the flame sheet, with a small deviation in the vicinity of the zones where the chemical reaction occurs.  Both finite chemistry calculations use the same initial estimate and both calculations find the converged solution after 11 Newton iterations.
In all the simulations shown in this thesis, the use of the flame sheet estimation served as a way to initialise the finite-reaction rate problem.



% Otro punto a discutir es la validez del metodo propuesto anteriormente de mesh refinement. Similarmente a lo senalado arriba, 

 
 The simulation of the coflowing flame shows that the strategy of using the flame sheet as an initial condition offers an efficient way to obtain steady-state solutions of combustion problems. Since the flame sheet is only used an estimate for the ignited solution, it is possible to perform the calculations on relatively coarse grids, and use low-order polynomial degrees. The obtained solution can be used as an estimate for calculations with higher polynomial degrees to find a more accurate solution of the full flame problem in a few iterations. A disadvantage, as already mentioned in \cref{ssec:MethodCombustion}, is the requirement to choose  the $c_p$ parameter, since an inappropriate choice gives a solution of the flame sheet problem far away from the solution of the full problem. However, the user's experience and access to experimental information allows estimating this value relatively easily. Note that the value chosen in this section ($\hat{c}_{p,\text{ref}}= \SI{1.3}{\kilo \joule \per \kilo \gram \per \kelvin}$)  served as an adequate estimate for all the simulations presented in this thesis.  
 