\FloatBarrier
\subsection{Combustion over a square cylinder}
As a last testcase of the XNSEC solver, a configuration very similar to the one previously shown for the flow over a cylinder was simulated, but now extending the case to a system where combustion is present. 
Unlike \cref{ssec:FlowCircCyl}, for this test case the flow over a square cylinder is considered.  The simulation is non-stationary and shall serve as a test of the time-steping together with the \Gls{AMR} algorithms. As will be explained later, this was not possible with the current solver implementation, and only some simplified cases will be shown.  First, results for a non-reactive steady state solution are shown. Later, an unsteady case with combustion present is analysed.
\begin{figure}[h]
	\begin{center}
		\def\svgwidth{0.8\textwidth}
		\import{./plots/}{CombustionSquareCylinder.pdf_tex}
		\caption{Temperature field and mesh of the unsteady combustion over a square cylinder.}
		\label{fig:CombustionSquareCylinder}
	\end{center}
\end{figure}
\subsubsection{Flow over a heated square cylinder}
In the work from \textcite{miaoHighOrderSimulationLowMachFlows2022} the XNSEC solver was used for calculating the flow around a heated square cylinder. The results were compared with the data published by \textcite{sharmaHEATFLUIDFLOW2004} for the recirculation lengths after the square cylinder. Simulations were performed with Reynolds numbers ranging from $\Reynolds = 5$ to $\Reynolds = 40$. The square cylinder is modeled by no-slip walls with $(u,v) = (0,0)$ and the incoming flow field (air inlet in \cref{fig:CombustionSquareCylinder}) is $(u,v) = (1,0)$. The fluid considered is air. 
\begin{figure}[b!]
	\centering
	\inputtikz{HeatedSquareCylinderLenghts}
	\caption{Recirculation lengths for different Reynolds numbers for the non-reactive case.}
	\label{fig:RecirculationLength}
\end{figure}
In \cref{fig:RecirculationLength} a comparison of the normalized recirculation length $L_r/B$ obtained with the XNSEC solver and the reference is shown. The results agree very well, deviating slightly at higher Reynolds numbers.

\subsubsection{Unsteady combustion over a square cylinder}
The last test of the XNSEC solver was done with the objective of calculating a non-stationary flame. It is well known that there is a critical Reynolds value (at least for non-reactive systems) from which the system becomes non-stationary. According to \textcite{sharmaHEATFLUIDFLOW2004}, for $\Reynolds > 50$ the flow has a non-stationary periodic character. In this section results from two simulations are shown. For the first one, a constant density is assumed, and in the second one the equation of state is used. 
A sketch of the configuration is shown in \cref{fig:CombustionSquareCylinder}. Fuel is expelled at a constant rate and homogeneously through the cylinder. A constant flow of air in the horizontal direction comes in contact with the fuel, which eventually forms a flame.

The lengths depicted in \cref{fig:CombustionSquareCylinder} are set to $B = 1$, $x_1 = 4$, $x_2 = 22$ and $H = 8$. The air enters with a uniform velocity to the system $(u,v) =(1,0)$, and uniform temperature $T = 1$. Its composition is $Y^O_{\ch{O2}} = 0.23$ and $Y^O_{\ch{N2}} = 0.77$. The fuel inlet enters with a velocity field $(u,v) = (0.2x/B,0.2y/B)$, has a uniform temperature $T = 1$ and composition $Y^F_{\ch{CH4}} = 0.2$ and $Y^F_{\ch{N2}} = 0.8$. A Reynolds number of $\Reynolds = 300$ is set, for which in the non-reactive case the vortex shedding phenomenon occurs. The Prandtl number is set to $\text{Pr} = 0.75$. 

For this simulations only the infinite-reaction rate equations given by \cref{eq:all-eq-mixfrac} are calculated and advanced in time. The initial conditions used are similar to those of the circular cylinder in section \cref{ssec:FlowCircCyl}. Again, an initial vortex is used to trigger the vortex shedding according to
\begin{subequations} 
	\begin{align}
		&u(t=0) = 1 + u^{\text{vortex}},  \\
		&v(t=0) = 0 + v^{\text{vortex}},  \\
		&Z(t=0) = 0,\\
		&p(t=0) = 0.
	\end{align}
\end{subequations}
where the vortices $u^{\text{vortex}}$ and $u^{\text{vortex}}$ are given by \cref{eq:VortexU}. The strength of the vortex is $a=1$ and its center is initially located at $(x^o,y^o) = (-2.5, 0)$. A Cartesian base mesh with $54\times32$ elements is used, which is refined or coarsened as necessary during the simulation. The calculation time is set to $t = 100$, with constant time steps of $\Delta t = 0.05$.  A BDF-2 scheme is used for the temporal discretization.

The time-dependent simulation of a combustion phenomenon proved to be very challenging and the temporal discretization described in \cref{ssec:TemporalDiscretization} did not allow obtaining solutions of this problem. 
In particular, it is observed that the time derivative of the continuity equation $\partial\rho /\partial t$ is a source of numerical instabilities, particularly in systems with large density variations. This is a fact already reported in the literature. In the work of \textcite{nicoudConservativeHighOrderFiniteDifference2000} difficulties are reported for obtaining solutions for density ratios greater than three. In this context density ratios refer to the ratio between maximum and minimum density appearing in the system. In the work of \textcite{rauwoensConservativeDiscreteCompatibilityconstraint2009} and \textcite{cookDirectNumericalSimulation1996} a similar destabilization effect is also reported for high density ratios, reporting also better stability properties when using even-ordered schemes compared to odd-ordered schemes.

It must be mentioned that the non-isothermal unsteady flow configurations presented until this point didn't present any kind of particular problems for its solution. Those test cases presented however only moderate density ratios, the largest of them being 1.5. On the other hand, for a combustion process like the one presented here, the density ratios are much higher, even higher than six for typical combustion cases.  

Nevertheless, simulations of this test case ignoring the  $\partial\rho /\partial t$ term were performed. This is clearly a non-physical approximation, but is nonetheless used to showcase the capability of the solver for calculating unsteady reacting flow.
\subsubsection{Constant density}
First, the problem was calculated with the assumption of a constant density $\rho = 1$. By doing this, the term $\partial\rho /\partial t$ is automatically equal to zero, and the momentum equation is only slightly coupled to the mixture fraction equation, since the viscosity still depends on the temperature. 
In \cref{fig:CombustionOverCylinder_CD} the temperature field is shown at different times. Obviously, since this is a simplified case, using the flame sheet circumvents the need to simulate the ignition of the system, and from the instant $t=0$ onward the system already has areas where the temperature reaches the adiabatic temperature. It can be seen that the vortex-shedding phenomenon appears, as expected for a number $ \Rey = 300$. It is interesting to see that the vortex shedding separation point moves downwards (compared with a square cylinder without outflow) since the incoming air flow is pushed by the fuel exiting the cylinder. The mesh obtained from the mesh refinement process is also shown. The mesh is refined or coarsened after each time-step by using as a criterion the location of the flame sheet, meaning that cells where $z = z_{\text{st}}$ are refined. The calculations were performed using eight cores.  The algorithm performs well for this configuration, needing at most 5 Newton iterations to reach convergence. %and each timestep taking around two minutes. 


\begin{figure}[h]
	\centering
	\inputtikz{ColorBarConstantDensity}
	\subcaptionbox{t=1.25}{
		\inputtikz{CylinderCombustion_ConstDens1}
		\inputtikz{CylinderCombustion_ConstDens1_mesh}
	}
	\par\bigskip%
	\subcaptionbox{t=5}{
		\inputtikz{CylinderCombustion_ConstDens2}
		\inputtikz{CylinderCombustion_ConstDens2_mesh}
	}
	\par\bigskip%
	\subcaptionbox{t=25}{
		\inputtikz{CylinderCombustion_ConstDens3}
		\inputtikz{CylinderCombustion_ConstDens3_mesh}
	}
	\caption{Temperature field and mesh calculated with the Burke-Schumann solution at different times, assuming a constant density.} \label{fig:CombustionOverCylinder_CD}
\end{figure}
\begin{figure}[p]
	\centering
	\inputtikz{ColorBarVariableDensity}
	\subcaptionbox{t=5 }{
		\inputtikz{CylinderCombustion_VarDens1}
		\inputtikz{CylinderCombustion_VarDens1_mesh}	
	}
	\par\bigskip%
	\subcaptionbox{t=15}{
		\inputtikz{CylinderCombustion_VarDens2}
		\inputtikz{CylinderCombustion_VarDens2_mesh}
	}
	\par\bigskip%
	\subcaptionbox{t=80}{
		\inputtikz{CylinderCombustion_VarDens3}
		\inputtikz{CylinderCombustion_VarDens3_mesh}	
	}
	\caption{Temperature field calculated with the Burke-Schumann solution at different times.} \label{fig:CombustionOverCylinder}
\end{figure}

\FloatBarrier
\subsubsection{Variable density}
Finally, the configuration using a variable density was calculated. The temperature fields and meshes are shown in \cref{fig:CombustionOverCylinder}. It is interesting to observe the great effect that the big density variations caused by the combustion have on the flow structure. The zones near the flame sheet are greatly accelerated because of the big reduction in the density. The refinement strategy mentioned before was used, and was observed to be critical for finding converged solutions of the system.  Interestingly, the zone of the domain where the vortex shedding commences in this configuration, is shifted even further to the right compared with the case mentioned above. This is clearly attributable to the flow acceleration due to the high density variations in the zones where chemical reaction takes place.  

The number of DoFs for the simulation range between 62064 for the initial mesh, and 224280 for the mesh corresponding to the last iteration. The relatively low number of DoFs allowed to use the direct solver \textit{PARDISO} during the whole course of the simulation. The calculation time of each timestep ranges between two minutes for the coarse initial mesh and 20 minutes for the finest mesh obtained at the end of the simulation. This relatively high calculation time could be a big drawback of the presented algorithm. The fully coupled solution of the system leads to very big matrices which have to be solved with efficient linear solvers. Although the linear solvers used in this work are very efficient, other specialized solvers could be used in order to reduce the calculation times, particularly for transient simulations.  This is a point which should be addressed in future work. 

The infinite reaction rate equations where used in this test mainly to reduce the calculation time of the transient system. The finite reaction rate equations could have been solved by using a similar strategy as the one used for steady state calculations, meaning that the system \cref{eq:all-eq-mixfrac} can be solved for an initial time interval, after which the finite reaction rate equations are initialized and advanced in time. 
Due to the long calculation times for this kind of simulations, this aspect was not explored in the present work.


