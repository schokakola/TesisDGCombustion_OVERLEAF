\FloatBarrier
\subsection{Combustion over a square cylinder}
As a last testcase of the XNSEC solver, a configuration very similar to the one previously shown for the flow over a cylinder was simulated, but now extending the case to a system where combustion exists. 
Unlike \cref{ssec:FlowCircCyl}, for this test case the flow over a square cylinder is considered.  The simulation is non-stationary. As will be explained later, this was not possible with the current solver implementation, and only some simplified cases will be shown. 


First, results for a non-reactive steady state solution are shown. Later, an unsteady case with combustion present is analyzed.
\begin{figure}[h]
	\begin{center}
		\def\svgwidth{0.8\textwidth}
		\import{./plots/}{CombustionSquareCylinder.pdf_tex}
		\caption{Temperature field and mesh of the unsteady combustion over a square cylinder.}
		\label{fig:CombustionSquareCylinder}
	\end{center}
\end{figure}


\subsubsection{Flow over a heated square cylinder}
In the work from \cite{miaoHighOrderSimulationLowMachFlows2022} the XNSEC-solver was used for calculating the flow around a heated square cylinder. The results were compared with the data published by \cite{sharmaHEATFLUIDFLOW2004}. Simulations were calculated for the test case with Reynolds numbers ranging from $\Reynolds = 5$ to $\Reynolds = 40$. The square cylinder is modeled by no-slip walls with $(u,v) = (0,0)$ and the incoming flow field (air inlet in \cref{fig:CombustionSquareCylinder}) is $(u,v) = (1,0)$. The fluid considered is air. 
\begin{figure}[h!]
	\centering
	\inputtikz{HeatedSquareCylinderLenghts}
	\caption{Recirculation lengths for different Reynolds numbers.}
	\label{fig:RecirculationLength}
\end{figure}
In \cref{fig:RecirculationLength} a comparison of the normalized recirculation length $L_r/B$ obtained with the XNSEC-solver and the reference is shown. The results agree very well, deviating slightly at higher Reynolds numbers.

\subsubsection{Unsteady combustion over a square cylinder}
The last test of the XNSEC solver was aimed at calculating a non-stationary case with combustion. It is well known that there is a critical Reynolds value (at least for non-reactive systems) from which the system becomes non-stationary. According to \cite{sharmaHEATFLUIDFLOW2004}, for $\Reynolds > 50$ the flow has a non-stationary periodic character. In this section results from two simulations are shown. For the first one, a constant density is assumed, and in the second one the equation of state is used. 
A sketch of the configuration is shown in \cref{fig:CombustionSquareCylinder}. Fuel is expelled at a constant rate and homogeneously through the cylinder. A constant flow of air in the horizontal direction comes in contact with the fuel, which eventually forms a flame.

The lengths depicted in \cref{fig:CombustionSquareCylinder} are set to $B = 1$, $x_1 = 4$, $x_2 = 22$ and $H = 8$. The air enters with an uniform velocity to the system $(u,v) =(1,0)$, and an uniform temperature $T = 1$. Its composition is $Y^O_{\ch{O2}} = 0.23$ and $Y^O_{\ch{N2}} = 0.77$. The fuel inlet enters with a velocity field $(u,v) = (0.2x/B,0.2y/B)$, has a uniform temperature $T = 1$ and composition $Y^F_{\ch{CH4}} = 0.2$ and $Y^F_{\ch{N2}} = 0.8$. A Reynolds number of $\Reynolds = 300$, for which in the non-reactive case the vortex shedding phenomenon occurs. The Prandtl number is set to $\text{Pr} = 0.75$. 

For this simulations only the infinite-reaction rate equations are calculated and advanced in time. This is mainly done in order to reduce the calculation time, but also because the goal is to test the capability of the solver to calculate flow fields in an unsteady state that are greatly influenced by the temperature.  

The initial conditions used are similar to those of the circular cylinder in section \cref{ssec:FlowCircCyl}. Again, an initial vortex is used to trigger the vortex shedding according to
\begin{subequations} 
	\begin{align}
		&u(t=0) = 1 + u^{\text{vortex}},  \\
		&v(t=0) = 0 + v^{\text{vortex}},  \\
		&T(t=0) = 1,\\
		&p(t=0) = 0.
	\end{align}
\end{subequations}
where the vortices $u^{\text{vortex}}$ and $u^{\text{vortex}}$ are given by \cref{eq:VortexU} and \cref{eq:VortexV}. The strength of the vortex is $a=1$ and its center is initially located at $(x^o,y^o) = (-2.5, 0)$. A Cartesian base mesh with $54\times32$ elements is used, which is refined or coarsened as necessary during the simulation. The calculation time is set to $t = 100$, with constant time steps of $\Delta t = 0.05$.  A BDF-2 scheme is used for the temporal discretization.

The time-dependent simulation of a combustion phenomenon proved to be very challenging and the temporal discretization described in \cref{ssec:TemporalDiscretization} did not allow to obtain solutions of this problem. It is well known that the inclusion of the $\partial\rho /\partial t$ term of the continuity equation in the source term as done in the present work is a source of numerical instability. In the work of \cite{nicoudNumericalStudyChannel} it is reported that obtaining solutions for density ratios greater than three  becomes difficult. It was previously shown in \cref{ssec:FlowCircCyl} and \cref{ssec:MultipleCellConv} that the XNSEC-solver was able to compute unsteady non-isothermic flows. However, those test cases presented moderate density ratios, the largest of them being 1.5. On the other hand, for a combustion process like the one presented here, the density ratios are much higher, even higher than 6 for typical combustion cases. In \cite{rauwoensConservativeDiscreteCompatibilityconstraint2009} a similar destabilization effect is also reported for high density ratios. 

Nevertheless, simulations of this test case ignoring the  $\partial\rho /\partial t$ term were performed. This is a big and nonphysical approximation, but is nonetheless used to showcase the capability of the solver for calculating unsteady reacting flow.
 
First, the problem was calculated with the assumption of a constant density $\rho = 1$. By doing this, the term $\partial\rho /\partial t$ is automatically equal to zero, and the momentum equation is only slightly coupled to the mixture fraction equation, since the viscosity still depends on the temperature. 
In \cref{fig:CombustionOverCylinder_CD} the temperature field is shown at different times. Obviously, since this is a simplified case, using the flame sheet circumvents the need to simulate the ignition of the system, and from the instant $t=0$ onward the system already has points where the temperature reaches the adiabatic temperature. It can be seen that the vortex-shedding phenomenon appears, as expected for a number $ \Rey = 300$. It is interesting to see how the vortex shedding separation point moves downwards (compared with a square cylinder without outflow) since the incoming air flow is pushed by the fuel exiting the cylinder. The mesh obtained from the mesh refinement process is also shown. The mesh is refined or coarsened after each time-step by using as a criterion the location of the flame sheet. Meaning that cells where $z = z_{\text{st}}$ are refined.

Finally, the configuration using a variable density was calculated. The temperature fields and meshes are shown in \cref{fig:CombustionOverCylinder}. It is interesting to observe the great effect that the big density variations caused by the combustion have on the flow-structure. The zones near the flame-sheet are greatly accelerated because of the big reduction on the density. The same refinement strategy mentioned before was used, and was observed to be critical for finding solutions of the system.
Interestingly the vortex shedding doesnt seem to appear in this case, at least for the times calculated here. 




\begin{figure}[p]
	\centering
	\inputtikz{CombustionOverCylinder_ConstantDensity}
	\caption{Temperature field and mesh calculated with the Burke-Schuhmann solution at different times, assuming a constant density.} \label{fig:CombustionOverCylinder_CD}
\end{figure}


\begin{figure}[p]
	\centering
	\inputtikz{CombustionOverCylinder}
	\caption{Temperature field calculated with the Burke-Schuhmann solution at different times.} \label{fig:CombustionOverCylinder}
\end{figure}

