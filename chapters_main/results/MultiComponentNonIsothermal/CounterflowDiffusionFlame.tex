
\subsection[Counterflow diffusion flame]{Counterflow diffusion flame \footnotemark}\label{ss:CDF}
\footnotetext{Modified version from \cite{gutierrez-jorqueraFullyCoupledHighorder2022}}
\begin{figure}[h!]
	\begin{center}
		\def\svgwidth{0.8\textwidth}
		\import{./plots/}{CounterDiffusionFlame_sketch_rotated2.pdf_tex}
		\caption{Schematic representation (not to scale) of the counterflow diffusion flame configuration.}
		\label{fig:CDFScheme}
	\end{center}
\end{figure}

%This test case is the main prototype flame for diffusion regimes \cite{poinsotTheoreticalNumericalCombustion2005}.
The counterflow diffusion flame is a canonical configuration used to study the structure of nonpremixed flames. This simple configuration has been a subject of study for decades because it provides a simple way of creating a strained diffusion flame, which proves to be useful when studying the flame structure, extinction limits or production of pollutants of flames \citep{pandyaStructureFlatCounterFlow1964,spaldingTheoryMixingChemical1961,keyesFlameSheetStarting1987, leeTwodimensionalDirectNumerical2000}. 

The counterflow diffusion flame consists of two oppositely situated jets. The fuel (possibly mixed with some inert component, such as nitrogen) is fed into the system by one of the jets, while the other jet feeds oxidyzer to the system, thereby establishing a stagnation point flow. On contact and after ignition, the reactants produce a flame that is located in the vicinity of the stagnation plane. A diagram of the setup can be seen in \cref{fig:CDFScheme}. In this section, the solution of a steady two-dimensional flame formed in an infinitely long slot burner will be treated. Simirlarly to the coflow configuration treated before, the infintely long slot burner configuration can be calculated naturally using cartesian coordinates.

First, as a means of verificating the solver for combustion applications, the results obtained with the XNSEC-solver for steady two-dimensional counterflow diffusion flame are compared with the solution of a simplified system of equations for a steady and quasi one-dimensional flame. Later, the influence of the inlet velocities on the maxmimum temperature is studied and finally some remarks concerning the convergence behaviour of the case counter diffusion flame are given. 
\subsubsection{The one-dimensional diffusion flame}
By assuming an infinite injector diameter, a self-similar solution and by neglecting the radial gradients of the scalar variables along the axis of symmetry, it is possible to reduce the three-dimensional governing equations to a one-dimensional formulation along the stagnation streamline $y = 0$ (see the textbook from \cite{keeChemicallyReactingFlow2003} for the derivation).  The governing equations for a steady planar stagnation flow reduce to
\begin{subequations}
\begin{gather}
	%%%%%%%%%%
	\frac{\partial\hat\rho \hat v}{\partial \hat x} +  \hat \rho \hat U = 0\label{eq:OneDimCont}\\ %
	%%%%%%%%%%
	\hat \rho \hat v \frac{\partial \hat U}{\partial \hat x} + \hat \rho \hat U^2 =
	- \hat \Lambda
	+ \frac{\partial}{\partial \hat x}\left(\hat \mu \frac{\partial \hat U}{\partial \hat x}\right)\label{eq:OneDimMom}\\ %
	%%%%%%%%%%
	\hat\rho \hat c_p \hat v \frac{\partial \hat T}{\partial\hat x} =
	\frac{\partial}{\partial \hat x}\left( \hat \lambda \frac{\partial \hat T}{\partial \hat x}\right)
	+\hat\heatRelease~\hat{\mathcal{Q}}\label{eq:OneDimTemp}\\
	%%%%%%%%%%%
	\hat\rho \hat v \frac{\partial Y_k}{\partial \hat x} = 
	\frac{\partial}{\partial \hat x}\left(\hat \rho \hat D \frac{\partial Y_k}{\partial \hat x}\right)
	+ \hat W_k \stoicCoef_k \hat{\mathcal{Q}} \quad (k = 1, \dots,~N - 1) \label{eq:OneDimMF}
	%%%%%%%%%%%
\end{gather}\label{eqs:OneDimEquations}%
\end{subequations}
\cref{eqs:OneDimEquations}.
where $\hat U$ is the scaled velocity and $\hat \Lambda$ is the radial pressure curvature, which is an eigenvalue independant of $x$. Again, the hat sign represent dimensional variables. The equations are written assuming a valid Fick's law and an one-step combustion model. The system of equations needs to be solved for $\hat v$, $\hat U$, $\hat T$ and for $Y_k$, $ (k = 1, \dots,~N - 1)$.  Aditionally an equation of state and expressions for the heat capacity $\hat c_p$ and transport parameters $\hat \mu, \hat \lambda, (\hat \rho \hat D)$ are needed. This formulation is very well known and often used for analysis of flame structure, determination of extintion points, to mention a few.

In order to assess the ability of the XNSEC-solver to simulate such a system, the solution obtained for a two dimentional configuration is compared with the solution of the quasi one-dimensional equations solved with \lstinline|BVP4|, a fourth order finite difference boundary value problem solver provided by \lstinline|MATLAB| \citep{kierzenkaBVPSolverBased2001}. The \lstinline|BVP4| solver provides automatic meshing and error control based on the residuals of the solution, allowing to create with relative easyness a code that solves the one-dimensional equations.


It is important to mention some points regarding the solution of these equations using the \lstinline|BVP4| solver. Analogous to the problem mentioned in \cref{ssec:MethodCombustion}, the solution of the system of  \crefrange{eq:OneDimCont}{eq:OneDimMF} presents the particularity that it is possible to find multiple solutions. One of them is clearly the cold solution and another solution is the burning one. The same idea mentioned in  \cref{ssec:MethodCombustion} is also valid for the quasi one-dimensional configuration. In particular this means, that as a first step for finding a converged solution of \crefrange{eq:OneDimCont}{eq:OneDimMF} is to solve the system
\begin{subequations}
\begin{gather}
	%%%%%%%%%%
\frac{\partial \hat \rho \hat v}{\partial \hat x} +  \hat \rho \hat U = 0\\ \label{eq:OneDimCont2}%
%%%%%%%%%%
\hat \rho \hat v \frac{\partial \hat U}{\partial \hat x} + \hat \rho \hat U^2 =
- \hat \Lambda
+ \frac{\partial}{\partial \hat x}\left(\hat \mu \frac{\partial \hat U}{\partial \hat x}\right)\\ \label{eq:OneDimMom2}%
%%%%%%%%%%
\hat \rho \hat v \frac{\partial Z}{\partial \hat x} = 
\frac{\partial}{\partial \hat x}\left(\hat \rho \hat D \frac{\partial Z}{\partial \hat x}\right)
%%%%%%%%%%%
\end{gather}\label{eqs:OneDimEquationsMixtureFraction}
\end{subequations}
together with the equation of state \cref{eq:ideal_gas} and expressions for the transport parameters. The dependency of the temperature and mass fractions on the mixture fraction $Z$ is given by the Burke-Schuhmann limit. This solution can be used as a initial estimate for the solution of \crefrange{eq:OneDimCont}{eq:OneDimMF}.
An inconvenience is that an initial estimate for the $\hat c_p$ has to be chosen. It was observed that if the value of $\hat c_p$ was chosen too big, the solver delivered solutions without a flame. For the calculations treated here $\hat c_p =\SI{1.3}{\kilo \joule\per\kilogram \kelvin}$ was an adequate value that delivered the ignited solution. 

It was however observed that this flamesheet solution wasnt directly useful as an initial estimation for the solution of the full system of equations. In order to help  the \lstinline|BVP4| solver to find a coverged solution, an intermediate step was necesary. First, the flamesheet solution was used as an initial estimate for the solution of system \cref{eqs:OneDimEquations} and the equation of state and equation for transport parameters but assuming a constant heat capacity. Once the algorithm has found a solution, it can be used for solving the same system but with a variable heat capacity according to \cref{eq:nondim_cpmixture}.


\subsubsection{Set-up of the two-dimensional counterflow diffusion flame}

The combustion of diluted methane with air in a two-dimensional infinitely long slot burner configuration is studied in this part. The solution is obtained by solving the system \cref{eq:all-eq}, making use of the flame-sheet solution as initial estimates. The transport parameters are calculated using Sutherland law with $\hat{S} = \SI{110.5}{\kelvin}$. Gravity effects are not taken into account.  The mixture heat capacity $c_p$ is calculated with \cref{eq:nondim_cpmixture} and using NASA polynomials for the heat capacity of each component.
\begin{table}[b]
	\centering
	\begin{tabular}{lccccc}
		\hline
		& \multicolumn{1}{l}{$\hat v^F_m$ ($\si{\centi \meter \per \second}$)} & \multicolumn{1}{l}{$\hat v^O_m$ ($\si{\centi \meter \per \second}$)} & $a$($\si{\per\second})$ & \multicolumn{1}{l}{$\hat T^F$($\si{\kelvin}$)} & \multicolumn{1}{l}{$\hat T^O$($\si{\kelvin}$)} \\ \hline
		case(a) & 4.85                                                                 & 12.29                                                                & 34                     & 300                                            & 300                                           \\
		case(b) & 12.13                                                                & 30.73                                                                & 76                     & 300                                            & 300                                           \\
		case(c) & 26.69                                                                & 67.62                                                                & 155                    & 300                                            & 300                                           \\ \hline
	\end{tabular}
	\caption{Maximum inlet velocity, strain and temperatures used for the counterflow diffusion flame calculations.}
	\label{tab:cdf_velocities}
\end{table}

For the comparison with the quasi one-dimensional model, three pairs of inlet velocities are considered. They are shown in \cref{tab:cdf_velocities}. Here, $v_m^F$ and $v_m^O$ are the maximal velocity of a parabolic profile for the fuel and air inlets, respectively. Both streams enter at a temperature $\hat T^O = \hat T^F = \SI{300}{\kelvin}$. The mass composition of the fuel inlet is assumed to be  $Y^F_{\ch{CH4}} = 0.2$ and $Y^F_{\ch{N2}} = 0.8$, and the oxidizer inlet is air with  $Y^O_{\ch{O2}} = 0.23$ and $Y^O_{\ch{N2}} = 0.77$. 
Counterflow diffusion flames are usually characterized by the strain rate $a$. Many different definitions for it can be found in the literature \citep{fialaNonpremixedCounterflowFlames2014}. In this work the definition of the strain rate the maximum axial velocity gradient is used. The strains for the three cases mentioned above are $\SI{34}{\per\second}$, $\SI{76}{\per\second}$ and $\SI{155}{\per\second}$, respectively. 

The lengths described in \cref{fig:CDFScheme} are $\hat D = \SI{2}{\centi\meter}$, $\hat H = \SI{2}{\centi\meter}$ and $\hat L = \SI{12}{\centi\meter}$. The variables are non-dimensionalized using $\RefVal{L} = \SI{2}{\centi\meter}$, $\RefVal{T} = \SI{300}{\kelvin}$ and $\RefVal{p} = \SI{101325}{\pascal}$.  For each case, the reference velocity is set to $\RefVal{u} = \hat v^O$.  Again, all derived variables are nondimensionalized using the air stream as a reference condition, i.e. $\RefVal{\rho} = \SI{1.17}{\kilo \gram \per \cubic \meter}$, $\RefVal{\mu} = \SI{1.85e-5}{\kilo \gram \per \meter \per \second}$ and $\RefVal{W} = \SI{28.82}{\kilo \gram \per \kilo\mole}$. The reference heat capacity is set $\hat{c}_{p,\text{ref}}= \SI{1.3}{\kilo \joule \per \kilo \gram \per \kelvin}$. 

Under this conditions, the Reynolds numbers are $\Rey = 156$, $\Rey = 390$ and $\Rey = 858$, for the low, medium and high inlet velocities respectively. The Dahmkoler numbers are $\Da = 4.6\cdot10^9$, $\Da = 1.8\cdot10^9$ and $\Da = 8.3\cdot10^8$. The Prandtl number is assumed to be constant with $\Prandtl = 0.71$. A non-unity but constant Lewis number formulation is used, with $\Lewis_{\ch{CH4}} =  0.97 $ , $\Lewis_{\ch{O2}} = 1.11 $, $\Lewis_{\ch{H2O}} = 0.83 $ and $\Lewis_{\ch{CO2}} = 1.39 $ \citep{smookePremixedNonpremixedTest1991}. The system is considered open, then the thermodynamic pressure is constant and set to  $ p_0 = 1$. 

%%%%%%%%%%%%%%%%%%%%%%%%%%%
%% BoundaryConditions
%%%%%%%%%%%%%%%%%%%%%%%%%%%
The boundary condition of the inlets are,
\begin{itemize}
	\item Oxidizer inlet: $\{\forall (x,y): y = 0 \land x \in [-D/2, D/2]\}$\\
	\begin{equation*}
		u = 0,\qquad v= v^O(y), \qquad T = 1.0, \qquad \vec{Y}' = (0,Y^O_{\ch{O2}},0,0)
	\end{equation*}
	\item Fuel Inlet: $\{\forall (x,y): y = H \land x \in [-D/2, D/2]\} $ \\
	\begin{equation*}
		u = 0,\qquad v= v^F(y), \qquad T = 1.0, \qquad \vec{Y}' = (Y^F_{\ch{CH4}},0,0,0)
	\end{equation*}
\end{itemize}
 
\begin{figure}[p]
	\centering
	\pgfplotsset{width=0.73\textwidth, compat=1.3}
	\inputtikz{CounterFlowFlameMesh}	
	\inputtikz{CounterFlowFlameStreamlines}
	\inputtikz{CounterFlowFlameTemperature}
	\inputtikz{CounterFlowFlameDensity}	
	\inputtikz{CounterFlowFlamekReact}
	\inputtikz{CounterFlowFlamePressure}	
	\caption{Nondimensional solution and derived fields of the counterflow flame configuration for case (a).} \label{fig:CoFlowFlameFig1}
\end{figure}
\begin{figure}[p]
	\ContinuedFloat
	\centering
	\pgfplotsset{width=0.73\textwidth, compat=1.3}		
	\inputtikz{CounterFlowFlameCpMixture}	
	\inputtikz{CounterFlowFlameMF0}
	\inputtikz{CounterFlowFlameMF1}
	\inputtikz{CounterFlowFlameMF2}
	\inputtikz{CounterFlowFlameMF3}
	%	\inputtikz{CounterFlowFlameDensity}	
	\caption{Nondimensional solution and derived fields of the counterflow flame configuration for case (a) (continued).}% \label{fig:CoFlowFlameFig1}
\end{figure} 


The pressure outlet boundary condition is the same as \cref{eq:bc_O}. The pressure outlet boundaries are placed far away of the center of the domain, in order to decrease the effect on the centerline. Placing the boundary further away did not change appreciably the results. Finally the bundary conditions at the walls are defined as in \cref{eq:bc_dn}, with $\vec{u}_{\text{D}} = (0,0)$ and a constant temperature $T = 1.0$.          
 
In \cref{fig:CoFlowFlameFig1} the solution profiles for the case (a) are shown. The used mesh was obtained by a process of mesh refinement. The base mesh is initially created with a larger concentration of elements in the center of the domain. The points of intersection from the velocity inlet and wall boundary conditions are also refined, which was observed to improve the robustness of the algoritm. Similarly to the coflowing flame (\cref{ssec:coflowFlame}), during the solution algorithm of the flame sheet problem, mesh is aditionally refined around the flame sheet making use of a pseudo-timestepping setting.
As expected, a stagnation flow develops and a flame forms close to it. For this strain rate, a maximum temperature of $T = 6.05$ is obtained (1815 $\si{\kelvin}$). This big increase in the temperature is also reflected in a big decrease in the density field, where a decrease of almost six times on the density values is apreciated. This change of density also provokes the acceleration of fluid, as observed in \cref{fig:CounterFlowStreamlines}. 

In \cref{fig:CounterFlowReactionRate} the reaction rate given by \cref{eq:NonDimArr} is ploted. It is interesting to see that the actual reacting zone is very small, which clearly demonstrates why adequate meshing is necessary to capture the steep gradients resulting from the strong and highly localized heat sources. Finally, and as expected, the fuel and oxidizers field seem to only be found on either side of the flame. Altought it can not be seen here, some reactant leaking occurs, meaning that there exists a small zone where both especies coexist. This point will be adressed later. 


\subsubsection{Comparison of two-dimensional and the quasi one-dimensional counterflow flames}
In this section a comparison of the results obtained with the XNSEC-solver for a two dimensional counterflow diffusion flame, and the results obtained with the \lstinline|BVP4| solver for a quasi one-dimensional flame is done. The comparison of both set of results is done along the centerline of the domain (see \cref{fig:CDFScheme}). In this section, only dimensional variables will be considered. The transport parameters, chemical model and the equation of state are exactly the same for both formulations. For all calculations in this section, a polynomial degree of four is used for the velocity components, temperature and mass fractions. A polynomial degree of three is used for the pressure. This resulted on systems with approximately 439000 degrees of freedom. 
 
\begin{figure}[t!]
	\centering
	\inputtikz{CounterFlowFlame_DifferentBoundaryConditions}
	\caption{Velocity profiles of the counterflow diffusion flame for parabolic and plug inlet boundary conditions.}\label{fig:CounterFlowFlame_DifferentBoundaryConditions}
\end{figure}
%TODO Still calculating => same plot but with vel mult 11 '\\hpccluster\hpccluster-scratch\gutierrez\CounterFlowFlame_BCComparison6'.
The choice of the type of velocity boundary conditions for the inlets requires some attention. Different possiblities exist to describe the velocity profiles. One posibility is to characterize the velocity boundary conditions by assuming a Hiemenz potential flow, where a single parameter defines the flow field. Other posibilities are also a constant velocity value (plug flow) or a parabolic profile, allowing to define different velocity values for each jet inlet.
The effect of boundary conditions on the flame structure has been estudied by \cite{chelliahExperimentalTheoreticalInvestigation1991} and \cite{johnsonAxisymmetricCounterflowFlame2015}, where it was concluded that both plug and potential are able to adequately describe experimental data. 


The question whether a plug or parabolic flow profile allows a better representation of the quasi one-dimensional equations was treated in the work from \cite{frouzakisTwodimensionalDirectNumerical1998}. There is stated that the one and two dimensional formulations deliver very similar results, provided that the inlets of the two-dimensional configurations are uniform. Furthermore, preliminary calculations with the XNSEC-solver showed that the selection of a plug flow or parabolic have an influence on the solution, as shown in \cref{fig:CounterFlowFlame_DifferentBoundaryConditions}. Based on these results, the plug flow boundary condition is adopted for all following test cases.

\begin{figure}[t]
	\pgfplotsset{
		width=0.95\textwidth,
		group/xticklabels at=edge bottom,
		legend style = {
			at ={ (0.49,1), anchor= north east}
		},
		unit code/.code={\si{#1}}
	}
	\centering
	\inputtikz{BoSSS_1D_Comparison_velocity}
	\caption{Comparison of the axial velocity calculated with the XNSEC-solver and the one-dimensional approximation.}
	\label{fig:BoSSS_1D_Comparison_velocity}
\end{figure}
 In \cref{fig:BoSSS_1D_Comparison_velocity} a comparison of the axial velocities calculated with the XNSEC-solver and the one-dimensional solution is shown. While for the high strain case the results agree closely, for lower strains a discrepancy is observed. Recall that the derivation of the one-dimensional approximation assumes a constant velocity field incoming to the flame zone in order to obtain a self-similar solution. In the case of the two-dimensional configuration presented here, the border effects do have an influence on the centerline, which disrupts the self-similarity. This effect is more pronounced for low velocities, which explains the discrepancy between curves.
 
 
 \tikzexternaldisable
 \begin{figure}[p]
 	\centering
 	\pgfplotsset{
 		width=0.85\textwidth,
 		height = 0.33\textwidth,
 		compat=1.3,
 		tick align = outside,
 		yticklabel style={/pgf/number format/fixed},
 	}
 	\inputtikz{BoSSS_1D_Comparison1}
 	\inputtikz{BoSSS_1D_Comparison2}
 	\inputtikz{BoSSS_1D_Comparison3}
 	\caption[Comparison of temperature and mass fraction fields obtained with the XNSEC-solver and the one-dimensional approximation.]{Comparison of temperature and mass fraction fields obtained with the XNSEC-solver (solid lines) and the one-dimensional approximation (dashed lines).}
 	\label{fig:BoSSS_1D_Comparison}
 \end{figure}
 \tikzexternalenable
Similarly, In \cref{fig:BoSSS_1D_Comparison} the temperature and mass fraction fields are presented. Again, a discrepancy is observed for low strains, but results show a good agreement for higher inlet velocities. It can also be observed how, as expected, at higher strains a significant leakage of oxygen across the flame is present. This is a typical behaviour of a flame that is getting closer to its extintion point \cite{fernandez-tarrazoSimpleOnestepChemistry2006}.  

This is a drawback from usual one-step models with constant activation temperature, because they tend to overpredict fuel leakage. This behaviour is not appreciated in the one-step model with variable activation temperature used here.  In \cref{fig:VarParams} the comparison is shown for the configuration (c) between the mass fractions fields obtained using a chemical model with variable kinetic parameters given by \cref{eq:ActivationTemperatureOneStep,eq:heatReleaseOneStep} and other with constant kinetic parameters using $\hat T_a = \hat T_{a0}$ and  $\hat Q = \hat Q_{0}$.  The oxygen leakage obtained by using the chemical model with variable parameters is evident, confirming 
 
 
 \begin{figure}[h]
 	\centering
 	\begin{tikzpicture} 
 		\pgfplotsset{width=0.35\textwidth, compat=1.3}
 		\begin{axis}[
 			name=plot1,%	
 			xlabel = {$x$ [cm]},
 			ylabel= {Mass fraction},		 
 			x filter/.code={\pgfmathparse{#1*100}\pgfmathresult},
 			%		y filter/.code={\pgfmathparse{#1*100}\pgfmathresult},
 			legend style={at={(0.95,0.65)}, anchor=north east},
 			]
 			\addplot+[no marks] table {data/StudyVariableParameters/11/FullMassFraction0_vp0.txt}; \addlegendentry{\ch{CH4}, CK}
 			\addplot+[no marks] table {data/StudyVariableParameters/11/FullMassFraction0_vp1.txt};\addlegendentry{\ch{CH4}, VK}
 			\addplot+[no marks] table {data/StudyVariableParameters/11/FullMassFraction1_vp0.txt};\addlegendentry{\ch{O2}, CK}
 			\addplot+[no marks] table {data/StudyVariableParameters/11/FullMassFraction1_vp1.txt};\addlegendentry{\ch{O2}, VK}	
 		\end{axis}
 	\end{tikzpicture}
 	\begin{tikzpicture} 
 		\pgfplotsset{width=0.35\textwidth, compat=1.3}
 		\begin{axis}[
 			name=plot1,%	
 			xlabel = {$x$[cm]},
 			ylabel= {Mass fraction},		 
 			x filter/.code={\pgfmathparse{#1*100}\pgfmathresult},
 			%		y filter/.code={\pgfmathparse{#1*100}\pgfmathresult},
 			%		legend style={at={(0.84,0.45)}, anchor=south west},
 			xmin = 0.4,xmax=0.80,
 			ymax = 0.05,
 			]
 			\addplot+[no marks] table {data/StudyVariableParameters/11/FullMassFraction0_vp0.txt}; 
 			\addplot+[no marks] table {data/StudyVariableParameters/11/FullMassFraction1_vp0.txt};
 			\addplot+[no marks] table {data/StudyVariableParameters/11/FullMassFraction0_vp1.txt};
 			\addplot+[no marks] table {data/StudyVariableParameters/11/FullMassFraction1_vp1.txt};
 		\end{axis}
 	\end{tikzpicture}
 	\caption[Comparison of fuel and oxidizer mass fraction profiles using constant kinetic parameters and variable kinetic parameters]{Mass fraction profiles of methane and oxygen along the centerline of a counterflow diffusion flame configuration  using variable kinetic parameters (VK) and constant kinetic parameters (CK). Right picture is zoomed in near the flame zone.} \label{fig:VarParams}
 \end{figure}

 
In \cref{fig:TemperatureStrainPlot} the maximum temperature obtained in the centerline for different strain rates is ploted. Qualitatively speaking, the solution obtained with the XNSEC-solver agrees with the expectations. As the strain rate increases, the maximum temperature decreases (see \cref{fig:Sshaped}). On the other hand, the comparison of values obtained with the XNSEC-solver and those of the quasi one-dimensional approximation clearly shows a discrepancy in the results. For low strain rates this discrepancy is small, being for $a = \SI{20}{\per\second}$ only 10K, approximately a difference of 0.5\%. As the strain rate increases so does the discrepancy. For $a = \SI{200}{\per\second}$ the difference is almost 50K, which is a 9\% disagreement. A similar behaviour is also reported in \cite{frouzakisTwodimensionalDirectNumerical1998}, where a difference of 50K was reported. 

It is worth noting that the XNSEC-solver wasnt able to find a converged solution for $a > \SI{202}{\per\second}$, and the newton algorithm stagnates. This is most probably a sign of underresolution of the mesh, and that the used refinement strategy did not help for such high strain rates. A better mesh refinement strategy is necesary for calculating the flame at conditions near the extintion point. 
Moreover, for high strain rates the flame will be far from the thermochemical equilibrium, and it is likely that the solution obtained for the flame sheet will be far away from the solution with finite rates. A posibility would be to use one of the well-known continuation methods (see for example \cite{nishiokaFlamecontrollingContinuationMethod1996}) to progressively move in the direction of the extinction point. Obviously, the Homotopy Methodology presented in \cref{sec:HomotopyMethod} can be visualized as one of those methods, and would be useful when looking for solutions of systems that are close to the extinction point. A complexity that arises then is how to create a dynamical mesh that is suitable for obtaining the intermediate solutions while searching for the final result in a robust way. This issue is beyond the scope of this thesis, and may be the subject of future research.


The difference between the results obtained for the two-dimensional configuration and the quasi-one dimensional approximation could be explained by some condition within the 1D system assumptions being violated in the 2D configuration. It is known that in addition to the boundary conditions, the ratio between the slot width and the separation between the two slots ($D/H$) also has an influence on the solution, and that a high ratio is desirable \citep{frouzakisTwodimensionalDirectNumerical1998}. Another point that was not addressed here is whether the boundary conditions chosen for cold walls have an effect on the solution along the centerline. Other posibilities could have been using outlet boundary conditions or an adiabatic wall. It is nevertheless expected that its effect on the centerline would not be big. The points mentioned here should be adressed in future work.

\begin{figure}[h]
	\centering
	\inputtikz{TemperatureStrainPlot}
	\caption{Maximum centerline temperature of a counterflow flame for different strains.}
	\label{fig:TemperatureStrainPlot}
\end{figure}
\subsubsection{Temperature convergence study}
\begin{figure}[h]
	\centering
	\inputtikz{TemperatureConvergenceDiffFlame}
	\caption{Convergence study of the maximum value of the temperature for the counterflow diffusion flame configuration.}
	\label{fig:TemperatureConvergenceDiffFlame}
\end{figure}
Similar to problems presented in earlier sections, the presence of singularities caused by non-consistent boundary conditions causes a degenerative effect on the global error values, making a global convergence study for this configuration is problematic. However, it is still possible to study the behaviour of some characteristic point value under different conditions to prove the mesh independence of the solution.

In \cref{fig:TemperatureConvergenceDiffFlame} it is shown how the maximum temperature along the centerline obtained for configuration (b)  behaves under different mesh resolutions and polynomial degrees. Values for $k=1$ are not shown, because for this range of cell elements the maximum temperature value was of the order of 60K higher than the ones depicted here. The temperature tends to a limit value, and its posible to observe how this value is reached already for coarse meshes when using a polynomial degree of three or four. For $k=2$ the temperature also tends to a limit value, but at a slower rate when compared to $k =3$ or $k = 4$. 

In the next section a simplified one-dimensional flame configuration will be used in order to be able to realize a $h$-convergence study of the whole sytem operator.

