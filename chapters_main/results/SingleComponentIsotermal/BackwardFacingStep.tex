
\subsection{Backward-facing step}\label{ssec:BackwardFacingStep}
The backward-facing step problem is another classical configuration widely used for validation of incompressible CFD codes. It has been widely studied theoretically, experimentally, and numerically by many authors in the last decades (see, for example, \cite{armalyExperimentalTheoreticalInvestigation1983,barkleyThreedimensionalInstabilityFlow2000,biswasBackwardFacingStepFlows2004} ).  In \cref{BFSsketch} a schematic representation of the problem is shown. It consists of a channel flow (usually considered fully developed) that is subjected to a sudden change in geometry that causes separation and reattachment phenomena. For these reasons, this case can be considered more challenging than the one presented in the previous section, since special care of the mesh used has to be taken in order to capture accurately all complex phenomena taking place.

Although the backward-facing step problem is known to be inherently three-dimensional, it has been shown that it can be studied as a two-dimensional configuration along the symmetry plane for moderate Reynolds numbers. For the range of Reynolds numbers used in the calculations presented here, the two-dimensional assumption is justified \citep{barkleyThreedimensionalInstabilityFlow2000, biswasBackwardFacingStepFlows2004}.  The origin of the coordinate system is set in the bottom part of the step. The step height \gls{StepHeigth} and channel height \gls{ChannelHeight} characterize the system. Results in the literature are often reported as a function of the expansion ratio, defined as $\gls{ExpansionRatio} = (\gls{ChannelHeight}+\gls{StepHeigth})/\gls{ChannelHeight}$.

A series of simulations were performed with the objective of reproducing the results reported by \cite{biswasBackwardFacingStepFlows2004}, where the backward-facing step was calculated for Reynolds numbers up to $400$ and for an expansion ratio of $1.9423$. In particular, the reported lengths of deattachment and reattachment are used as a means of comparison with the results from the XNSEC-solver.

The Reynolds number for the backward-facing step configuration is defined in the literature in many forms. Here, the definition based on the step height $\glsHat{StepHeigth}$ and the mean inlet velocity $\hat U_{\text{mean}}$ is adopted as the reference length and velocity, resulting in
\begin{equation}
	\gls{Reynolds}= \frac{\glsHat{StepHeigth}\hat U_{\text{mean}}}{\glsHat{kinVisc}}.
\end{equation}
The boundary at $x = - L_0$ is an inlet boundary condition, where a parabolic profile is defined with %, with a kinematic viscosity  $\nu = \SI{15.52e-6 }{\meter \squared \per \second}$
\begin{equation}
	u(y) = -6\frac{( y- S)( y-( h+ S))}{h^2} %= \frac{\hat u(y)}{\hat U_{\text{mean}}} 
\end{equation}
The system is isothermal, and the fluid is assumed to be air. The step length is set $S=1$ and $h = 1.061$. To minimize the effects of the outlet boundary condition on the part of interest in the system, the length $L$ of the domain is set to $L = 70 \gls{StepHeigth}$. All other boundaries are fixed walls. From prior calculations, the effect of the domain length before the step was found to have almost no impact on the results and is set to $L_0 = \gls{StepHeigth}$. Preliminary calculations showed that the calculated reattachment and detachment lengths are highly sensitive to the mesh resolution. For all calculations in this section, a structured grid with 88400 elements is useda. To better resolve the complex structures that occur in this configuration, smaller elements are used in the vicinity of the step, as seen in \cref{bfsmesh}.  A polynomial degree of three was chosen for both velocity components and two for pressure.


\begin{figure}[tb]
	\begin{center}
		\def\svgwidth{0.9\textwidth}
		\import{./plots/}{BFS_sketch.pdf_tex}
		\caption[Schematic representation of the backward-facing step.]{Schematic representation (not to scale) of the backward-facing step. Both primary and secondary vortices are shown.}
		\label{BFSsketch}
	\end{center}
\end{figure}

\begin{figure}[tb]
	\begin{center}
		\def\svgwidth{0.8\textwidth}
		\import{./plots/}{HBFS_MESH.pdf_tex}
		\caption{Mesh used for the backward-facing step configuration.}
		\label{bfsmesh}
	\end{center}
\end{figure}

\begin{figure}[bt]
	\centering
	\pgfplotsset{
		group/xticklabels at=edge bottom,
		%		legend style = {
		%			at ={ (1.0,1.0), anchor= north east}
		%		},
		unit code/.code={\si{#1}}
	}
	\inputtikz{uvelBFS}
	\caption[Distribution of x-component of velocity in the backward-facing step configuration for a Reynolds number of 400.]{Distribution of x-component of velocity in the backward-facing step configuration for a Reynolds number of 400. Solid lines correspond to results obtained with the XNSEC solver.}
	\label{fig:uvelBFS}
\end{figure}



\begin{figure}[tb]
	\pgfplotsset{
		group/xticklabels at=edge bottom,
		legend style = {
				at ={ (0.05,0.9), anchor= north west}
			},
		unit code/.code={\si{#1}}
	}
	\centering
	\inputtikz{Re_De_Attachmentlengths}
	\caption[Detachment and reattachment lengths of the primary and secondary recirculation zones after the backward-facing step compared to the reference solution]{ Detachment and reattachment lengths of the primary (left figure) and secondary (right figure) recirculation zones after the backward-facing step compared to the reference solution \citep{biswasBackwardFacingStepFlows2004}.}
	\label{fig:Re_De_Attachmentlengths}
\end{figure}
The backward-facing step configuration exhibits varying behavior as the number of Reynolds changes. For small Reynolds numbers, a single vortex, usually called the primary vortex, appears in the vicinity of the step. Furthermore, as the Reynolds number increases, a second vortex eventually appears on the top wall, as shown schematically in \cref{BFSsketch}.
The detachment and reattachment lengths of the vortices are values that are usually reported in the literature. It is possible to determine the detachment position by finding the point along the wall where the velocity gradient normal to the wall acquires a value equal to zero. 

\cref{fig:Re_De_Attachmentlengths} shows the detachment and reattachment lengths of the primary and secondary vortices obtained with the XNSEC solver for different Reynolds numbers, which are also compared with the results presented in the reference paper from \cite{biswasBackwardFacingStepFlows2004}. Cubic splines have been used to accurately locate this point. It can be seen that the results for the detachment lengths of the primary vortex $R_1$ are in very good agreement with those of the reference. In the case of the secondary vortex, it is possible to see a very minimal deviation for the lengths of the reattachment $R_3$, hinting at a possible spatial underresolution far away from the step. It is interesting to note that, despite the fact that the reference does not report the existence of a secondary vortex for $\gls{Reynolds} = 200$, it was possible to observe it with the XNSEC-solver. The results allow us to conclude that it is possible to study flows with complex behavior for low- to moderate Reynolds numbers, at least in the isothermal case. In the next section, a non-isothermal case of this configuration will be studied.

It is worth mentioning that the evaluation of the global order of accuracy of the solver using the two incompressible test cases presented in this section is problematic due to the presence of singularities, specifically at the corners at the coordinates $ \vec{x} = (0,1)$ and $\vec{x} =(1,1)$ of the Lid-driven cavity (where the pressure is not finite according to \cite{botellaBenchmarkSpectralResults1998}), and at the corner of the step $\vec{x} = (0,S)$ of the backward-facing step. The accuracy of the solver will be assessed later in  \cref{ssec:CouetteFlowTempDiff} making use of a analytical solution and \cref{ssec:ConvStudyHeatedCavity} using a solution obtained with a high spatial resolution.
