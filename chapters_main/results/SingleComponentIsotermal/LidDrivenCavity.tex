\subsection{Lid-driven cavity flow}

\begin{figure}[t]
	\begin{center}
		\def\svgwidth{0.3\textwidth}
		\import{./plots/}{LidDrivvenGeometry.pdf_tex}
		\caption{Schematic representation of the Lid-Driven cavity flow.}
		\label{fig:LidDrivenCavity}
	\end{center}
\end{figure}

The lid-driven cavity flow is a classic test problem used for the validation of Navier-Stokes solvers. The system configuration is shown in \cref{fig:LidDrivenCavity}. It consists simply of a two-dimensional square cavity enclosed by walls whose upper boundary moves at constant velocity, causing the fluid to move. Benchmark results can be found widely in the literature for different Reynolds numbers. In this section, the results obtained with the XNSEC solver are compared with those published by \textcite{botellaBenchmarkSpectralResults1998}. 

The problem is defined in the domain $\Omega = [0,1]\times [0,1]$. The system is solved for the velocity vector $\gls{velVec} = (u,v)$ and the pressure $p$. All boundary conditions are Dirichlet-type, particularly with $\gls{velVec} = (-1,0)$ for the boundary at $y = 1$ and $\gls{velVec} = (0,0)$ for all other sides. The gravity vector is set to $\gls{gravityVec} = (0,0)$.
A Cartesian mesh with extra refinement at both upper corners is used, and is shown in \cref{fig:LiddrivenMesh}. The refinement was done to better represent the complex effects that take place in the corners. The streamline plot presented in \cref{fig:LiddrivenMesh} shows the different vortex structures typical of this kind of system, where in addition to the main vortex of the cavity, smaller structures appear in the corners.

\begin{figure}[b]
	\centering
	\pgfplotsset{width=0.35 \textwidth, compat=1.3}
	\inputtikz{LiddrivenMesh1}
	\inputtikz{LiddrivenMesh2}
	\caption{Mesh and streamlines of the lid-driven cavity flow with $\gls{Reynolds} = 1000$} \label{fig:LiddrivenMesh}
\end{figure}

The lid-driven cavity was calculated for a Reynolds number $\gls{Reynolds} = 1000$. For the calculations presented here, the polynomial degree is set to four for both velocity components and three for the pressure. A regular Cartesian mesh with $16\times16$ elements is used with extra refinement in the corners. In \cref{fig:LidVelocities} a comparison of the calculated velocity with the DG-Solver and the velocities provided by the benchmark is shown. Clearly, very good agreement is obtained, even by using a relatively coarse mesh (the benchmark result uses a grid with $160\times160$ elements).

A more rigorous comparison of results is presented in \cref{tab:LidCavityExtrema}, where the extreme values of the velocity components calculated through the centerline of the cavity are compared with the results presented by \textcite{botellaBenchmarkSpectralResults1998}. Different mesh resolutions were used for this comparison, particularly meshes with $16\times16$, $32\times32$, $64\times64$, $128\times128$ and $256\times256$ elements, each with extra refinement at the corners. It can be clearly seen how for the finest mesh the results obtained with the DG-solver are extremely close to the reference. A difference is only appreciated at the fifth digit after the decimal point for the velocity components. In case of the position of the extremal values no difference is observed. It can also be seen that the results obtained with the coarser meshes are still very close to those of the reference %. It is worth mentioning that this comparison only considering number of elements could be considered unfair, since the reference uses another method for discretizing and solving the governing equations. One of the advantages of the DG method is that choosing higher-order polynomials allows more information to be packed into each cell. %A better comparison would be achieved by comparing results based on the number of degrees of freedom (DOF) used in the simulation, as will be done later in section %TODO XXX.

\begin{figure}[tb]
	\pgfplotsset{
		group/xticklabels at=edge bottom,
		legend style = {
				at ={ (0.09,0.3), anchor= north east}
			},
	}
	\inputtikz{LidVelocities1}
	\pgfplotsset{
		group/xticklabels at=edge bottom,
		legend style = {
				at ={ (0.59,0.3), anchor= north east}
			},
	}
	\inputtikz{LidVelocities2}
	\caption{Calculated velocities along the centerlines of the cavity and reference values. Left plot shows the x-velocity for $x = 0.5$. Right plot shows the y-velocity for $y = 0.5$  }
	\label{fig:LidVelocities}
\end{figure}

%	\begin{figure}[tb]
%		\pgfplotsset{
%			group/xticklabels at=edge bottom,
%			legend style = {
%				at ={ (0.09,0.2), anchor= north west}
%			},
%		}
%		\begin{tikzpicture}
%			\begin{axis}[
%				width= 0.4\textwidth ,
%				height= 0.3\textwidth ,
%				xlabel = $y$,
%				ylabel= $\omega$, 
%				]
%				\addplot+[color=black, only marks] table {data/LidDrivenCavity/omegaRe1000_x_ref.txt}; \addlegendentry{Reference}
%				\addplot[color=black, no marks] table {data/LidDrivenCavity/omegaRe1000_x.txt}; \addlegendentry{BoSSS}
%			\end{axis}
%		\end{tikzpicture}
%		\pgfplotsset{
%			group/xticklabels at=edge bottom,
%			legend style = {
%				at ={ (0.59,1.0), anchor= north east}
%			},
%		}
%		\begin{tikzpicture}
%			\begin{axis}[
%				width= 0.4\textwidth ,
%				height= 0.3\textwidth ,
%				xlabel = $x$,
%				ylabel= $\omega$, 
%				]
%				\addplot+[color=black, only marks] table {data/LidDrivenCavity/omegaRe1000_y_ref.txt}; \addlegendentry{Reference}
%				\addplot[color=black, no marks] table {data/LidDrivenCavity/omegaRe1000_y.txt}; \addlegendentry{BoSSS}
%			\end{axis}
%		\end{tikzpicture}
%		\caption{Calculated vorticity along the centerlines of the cavity and reference values. Left plot shows the vorticity for $x = 0.5$. Right plot shows the vorticity for $y = 0.5$  }
%		\label{fig:LidVorticities}
%	\end{figure}

\begin{table}[]
	\centering
	\begin{tabular}{lllllrr}
		\hline
		Mesh           & $u_{\text{max}}$ & $y_{\text{max}}$ & $v_{\text{max}}$ & $x_{\text{max}}$ & \multicolumn{1}{l}{$v_{\text{min}}$} & \multicolumn{1}{l}{$x_{\text{min}}$} \\ \hline
		$16\times16$   & 0.3852327        & 0.1820           & 0.3737295        & 0.8221           & -0.5056627                           & 0.0941                               \\
		$32\times32$   & 0.3872588        & 0.1821           & 0.3760675        & 0.8227           & -0.5080496                           & 0.0943                               \\
		$64\times64$   & 0.3897104        & 0.1748           & 0.3774796        & 0.8408           & -0.5248360                           & 0.0937                               \\
		$128\times128$ & 0.3886452        & 0.1720           & 0.3770127        & 0.8422           & -0.5271487                           & 0.0907                               \\
		$256\times256$ & 0.3885661        & 0.1717           & 0.3769403        & 0.8422           & -0.5270653                           & 0.0907                               \\\hline
		Reference      & 0.3885698        & 0.1717           & 0.3769447        & 0.8422           & \multicolumn{1}{l}{-0.5270771}       & \multicolumn{1}{l}{0.0908}           \\ \hline
	\end{tabular}
	\caption{Extrema of velocity components through the centerlines of the lid-driven cavity for $\gls{Reynolds} = 1000$. Reference values obtained from \textcite{botellaBenchmarkSpectralResults1998} }
	\label{tab:LidCavityExtrema}
\end{table}
\FloatBarrier
% \newpage