\subsection{Differentially heated cavity problem}\label{ss:DHC}
\begin{figure}[bt]
	\begin{center}
		\def\svgwidth{0.53\textwidth}
		\import{./plots/}{diffheatedCavityGeometry.pdf_tex}
		\caption{Schematic representation of the differentially heated cavity problem.}
		\label{DHCGeom}
	\end{center}
\end{figure}
The differentially heated cavity problem is a classical benchmark case that is often used to assess the ability of numerical codes to simulate variable density flows \parencite{paillereComparisonLowMach2000,vierendeelsBenchmarkSolutionsNatural2003,tyliszczakProjectionMethodHighorder2014}.
The test case has the particularity that deals with a closed system, where the thermodynamic pressure $p_0$ is a parameter that must be adjusted so that the mass is conserved.The thermodynamic pressure $p_0$ determines the density field, which in turn appears in the momentum equation and the energy equation, making it necessary to use an adequate algorithm to solve the system. This point presents a special difficulty for the solution, since the calculation of $p_0$ requires knowledge of the temperature field on the whole computational domain, inducing a global coupling of the variables. 

The system is a fully enclosed two-dimensional square cavity filled with fluid.  A sketch of the problem is shown in \cref{DHCGeom}. The left and right walls of the cavity have constant temperatures $\hat{T}_h$ and $\hat{T}_c$, respectively, with $\hat{T}_h >\hat{T}_c$, and the top and bottom walls are adiabatic. A gravity field induces fluid movement because of density differences caused by the difference in temperature between the hot and cold walls.
The natural convection phenomenon is characterized by the Rayleigh number, defined as
\begin{equation}\label{eq:Rayleigh}
	\text{Ra} = \Prandtl \frac{\hat g \RefVal{\rho}^2(\hat T_h-\hat T_c) \RefVal{L}^3}{\RefVal{T}\RefVal{\mu}^2},
\end{equation}
For small values of $\text{Ra}$, conduction dominates the heat transfer process, and a boundary layer covers the entire domain. On the other hand, large values of $\text{Ra}$ represent a flow dominated by convection. When the number $\text{Ra}$ increases, a thinner boundary layer is formed.
Following \textcite{vierendeelsBenchmarkSolutionsNatural2003}, a reference velocity for buoyancy-driven flows can be defined as
\begin{equation}
	\RefVal{u} = \frac{\sqrt{\text{Ra}} \RefVal{\mu}}{\RefVal{\rho}\RefVal{L}}.
\end{equation}
The Rayleigh number is then related to the Reynolds number according to
\begin{equation}
	\text{Re} = \sqrt{\text{Ra}}.
\end{equation}
Thus, it is sufficient to select a $\Reynolds$ number in the simulation, fixing the value of the $\text{Ra}$ number. The driving temperature difference $(\hat T_h - \hat T_c)$ appearing in \cref{eq:Rayleigh} can be represented as a non-dimensional parameter:
\begin{equation}\label{eq:nondimensionalTemperature}
	\varepsilon = \frac{\hat T_h - \hat T_c}{2\RefVal{T}}.
\end{equation}
Using these definitions, the Froude number can be calculated as
\begin{equation}
	\Froude = \sqrt{\Prandtl 2 \varepsilon}.
\end{equation}
The results of the XNSEC solver are compared with those of the reference solution for $\RefVal{T} = 600\si{K}$  and $\varepsilon = 0.6$. All calculations assume a constant Prandtl number equal to 0.71. The dependence of viscosity and heat conductivity on temperature is calculated using Sutherland's law (\cref{eq:nondim_sutherland}). The non-dimensional length of the cavity is $L=1$. The non-dimensional temperatures $T_h$ and $T_c$ are set to 1.6 and 0.4, respectively. The non-dimensional equation of state (\cref{eq:ideal_gas}) depends only on the temperature and reduces to
\begin{equation}
	\rho = \frac{p_0}{T}.
\end{equation}
The thermodynamic pressure $p_0$ in a closed system must be adjusted to ensure mass conservation. For a closed system is given by
\begin{equation}
	p_0 =\frac{\int_\Omega \rho_0\text{d}V}{\int_\Omega \frac{1}{T}\text{d}V}= \frac{m_0}{\int_\Omega \frac{1}{T}\text{d}V}, \label{eq:p0Condition}
\end{equation}
where $\Omega$ represents the complete closed domain. The initial mass of the system $m_0$ is constant and is set $m_0 = 1.0$. Note that the thermodynamic pressure is a parameter with a dependence on the temperature of the entire domain. This makes necessary the use of an iterative solution algorithm, so that the solution obtained respects the conservation of mass. Within the solution algorithm of the XNSEC-solver, \cref{eq:p0Condition} is used to update the value of the thermodynamic pressure after each Newton iteration.
The average Nusselt number is defined for a given wall $\Gamma$  as
\begin{equation}\label{eq:Nusselt}
	\text{Nu}_\Gamma = \frac{1}{T_h - T_c}\int_{\Gamma} k \pfrac{T}{x}\text{d}y.
\end{equation}
%%%%%%%%%%%%%%%
%%% Result comparison
%%%%%%%%%%%%%%%
\subsubsection{Comparison of results with the benchmark solution}

Here a comparison of the results obtained with the XNSEC solver and the results presented in the work of \textcite{vierendeelsBenchmarkSolutionsNatural2003} is made. They solved the fully compressible Navier-Stokes equations on a stretched grid with $1024\times1024$ using a finite-volume method with quadratic convergence, providing very accurate results that can be used as reference.
The benchmark results are presented for $\text{Ra} = \{10^2,10^3,10^4,10^5,10^6,10^7\}$. In this range of Rayleigh numbers, the problem has a steady-state solution. 
The cavity is represented by the domain $[0,1]\times[0,1]$. For all calculations in this subsection, the simulations are done with a polynomial degree of four for both velocitiy components and temperature and three for the pressure. The mesh is in an equidistant $128\times128$ mesh.

Preliminar calculations showed that for cases up to $\text{Ra} = 10^5$ the solution of the system using Newton's method presented in \cref{sec:newton} is possible without further modifications, while for higher values the algorithm couldnt find a solution and stagnates after certain number of iterations. The homotopy strategy mentioned in \cref{sec:CompMethodology} is used to overcome this problem and obtain solutions for higher Rayleigh numbers. Here, the Reynolds number is selected as the homotopy parameter and continuously increased until the desired value is reached.

In \cref{fig:TempProfile,fig:VelocityXProfile,fig:VelocityYProfile} the temperature and velocity profiles across the cavity for different Rayleigh numbers are shown. The profiles calculated with the XNSEC solver agree closely to the benchmark solution. As expected, an increase of the acceleration of the fluid in the vicinity of the walls for increasing Rayleigh numbers is observed.
\begin{figure}[h]
	\centering
	\pgfplotsset{width=0.3 \textwidth, compat=1.3}
	\inputtikz{HSCStreamlines}
	\caption{Streamlines of the heated cavity configuration with $\epsilon = 0.6$ for different Reynold numbers.}\label{fig:HSCStreamlines}
\end{figure}


\begin{figure}[h]
	\centering
	\pgfplotsset{width=0.22\textwidth, compat=1.3} 
	\inputtikz{TempProfile}
	\caption[Temperature profiles for the differentially heated square cavity along different vertical levels.]{Temperature profiles for the differentially heated square cavity along different vertical levels. Solid lines represent the XNSEC solver solution and marks the benchmark solution.}
	\label{fig:TempProfile}
\end{figure}
%
\begin{figure}[h]
	\centering
	\pgfplotsset{width=0.22\textwidth, compat=1.3}
	\inputtikz{VelocityXProfile}
	\caption[Profiles of the x-velocity component for the differentially heated square cavity along the vertical line $x=0.5$.]{Profiles of the x-velocity component for the differentially heated square cavity along the vertical line $x=0.5$. Solid lines represent the XNSEC solver solution and marks the benchmark solution.}
	\label{fig:VelocityXProfile}
\end{figure}
%
\begin{figure}[h]
	\centering
	\pgfplotsset{width=0.22\textwidth, compat=1.3}
	\inputtikz{VelocityYProfile}
	\caption[Profiles of the y-velocity component for the differentially heated square cavity along the horizontal line $y=0.5$.]{Profiles of the y-velocity component for the differentially heated square cavity along the horizontal line $y=0.5$. Solid lines represents the XNSEC solver solution and marks the benchmark solution.}
	\label{fig:VelocityYProfile}
\end{figure}
\FloatBarrier
A comparison of the thermodynamic pressure and the Nusselt numbers to the benchmark solution was also made. The results are shown in \cref{tab:p0_Nu_Results}.  The thermodynamic pressure is obtained from \cref{eq:p0Condition}, and the average Nusselt number is calculated with \cref{eq:Nusselt}. The results obtained with the XNSEC solver agree very well with the reference results, as can be seen for the thermodynamic pressure, which differs at most in the fourth decimal place. Note that the average Nusselt number of the heated wall $(\text{Nu}_\text{h})$ and the Nusselt number of the cold wall $(\text{Nu}_\text{c})$ are different. As the Rayleigh number grows, this discrepancy becomes larger, hinting that, at such Rayleigh numbers, the mesh used is not refined enough to adequately represent the thin boundary layer and more complex flow structures appearing in high-Rayleigh number cases. While for an energy conservative system $\text{Nu}_\text{h}$ and $\text{Nu}_c$ should be equal, for the DG-formulation this is not the case, since conservation is only ensured locally and the global values can differ. This discrepancy can be seen as a measure of the discretization error of the DG formulation and should decrease as the mesh resolution increases. This point will be discussed in the next section.
\begin{table}[t!]
	\begin{center}
		\begin{tabular}{cccccc}
			\hline
			Rayleigh                           & $p_0$  & $p_{0,\text{ref}}$ & $\text{Nu}_{h}$ & $\text{Nu}_{c}$ & $\text{Nu}_{\text{ref}}$ \\ \hline
			\parbox[0pt][13pt][c]{0pt}{}$10^2$ & 0.9574 & 0.9573             & 0.9787          & 0.9787          & 0.9787                   \\
			$10^3$                             & 0.9381 & 0.9381             & 1.1077          & 1.1077          & 1.1077                   \\
			$10^4$                             & 0.9146 & 0.9146             & 2.2180          & 2.2174          & 2.2180                   \\
			$10^5$                             & 0.9220 & 0.9220             & 4.4801          & 4.4796          & 4.4800                   \\
			$10^6$                             & 0.9245 & 0.9245             & 8.6866          & 8.6791          & 8.6870                   \\
			$10^7$                             & 0.9225 & 0.9226             & 16.2411         & 16.1700         & 16.2400                  \\ \hline
		\end{tabular}
	\end{center}
	\caption[Differentially heated cavity: Results of Nusselt number and Thermodynamic pressure]{Comparison of calculated Nusselt numbers of the hot and cold wall and Thermodynamic pressure $p_0$ reported values by \textcite{vierendeelsBenchmarkSolutionsNatural2003} for the differentially heated cavity.}
	\label{tab:p0_Nu_Results}
\end{table}
%%%%%%%%%%%%%%%
%%% Convergence study
%%%%%%%%%%%%%%%


\subsubsection{Convergence study}\label{ssec:ConvStudyHeatedCavity}
An $h-$convergence study of the XNSEC solver was conducted using the heated cavity configuration. Calculations were performed for polynomial degrees $k = {1,2,3,4}$ and equidistant regular meshes with, respectively, $8\times8$, $16\times16$, $32\times32$, $64\times64$, $128\times128$ and $256\times256$ elements.  The $L^2$ -Norm was used to calculate errors against the solution in the finest mesh. The results of the $h$-convergence study for varying polynomial orders $k$ are shown in \cref{fig:ConvergenceDHC}. It is observed how the convergence rates scale approximately as $k+1$. Interestingly, for $k=2$ the rates are higher than expected. On the other hand, some degeneration is observed in convergence rates for $k = 4$. This strange behavior can be explained if one considers that the heated cavity presents a singular behavior at the corners (similar to the problem previously exposed for the lid-driven cavity), which causes global pollution in the convergence behavior of the algorithm. 
 
As discussed in the previous section, the difference in the average values of the Nusselt number on the hot wall $\text{Nu}_\text{h}$  and the cold wall $\text{Nu}_\text{c}$ is a direct consequence of the spatial discretization error and should decrease for finer meshes. In \cref{fig:NusseltStudy} the convergence behavior of the Nusselt number is presented for different polynomial degrees $k$, different number of elements and for two Rayleigh numbers. As expected, it can be observed that this discrepancy is smaller when a larger number of elements is used. It can also be seen that  $\text{Nu}_\text{h}$ reaches the expected solution of cells for a much smaller number of elements. This can be explained if one thinks that more complex phenomena take place near the cold wall (see \cref{fig:HSCStreamlines}), which makes necessary a finer mesh resolution in that area.



\begin{figure}[tb]
	\centering
	\pgfplotsset{width=0.34\textwidth, compat=1.3}
	\inputtikz{ConvergenceDHC}
	\caption{Convergence study of the differentially heated cavity problem for $\text{Ra} = 10^3$.}\label{fig:ConvergenceDHC}
\end{figure}
\begin{figure}[tb]
	\centering
	\inputtikz{NusseltStudy}
	\caption[Nusselt numbers of the differentially heated square cavity at the hot wall ($\text{Nu}_h$) and the cold wall ($\text{Nu}_c$) for different number of cells and polynomial order $k$.]{Nusselt numbers of the differentially heated square cavity at the hot wall ($\text{Nu}_h$) and the cold wall ($\text{Nu}_c$) for different number of cells and polynomial order $k$. The reference values from \textcite{vierendeelsBenchmarkSolutionsNatural2003} are shown with dashed lines.}\label{fig:NusseltStudy}
\end{figure}
\FloatBarrier
\subsubsection{Influence of the penalty factor}
\begin{table}[h]
\centering
\begin{tabular}{lllllll}
	\hline \vspace{0.1cm}
		$\eta_0$                  &  $k$ & DOFs& $\text{Nu}_c$ &  $\frac{\text{Nu}_c - \text{Nu}_{c,\text{ref}}} {\text{Nu}_{c,\text{ref}}}\times 10^2 $   & $p_0$    & $\frac{p_0 - p_{0,\textbf{ref}}} {p_{0,\textbf{ref} }}\times 10^4 $ \\ \hline
\multirow{3}{*}{0.01} & 2 & 6804 & 0.549483* & 50.39424 & 0.899757* & 408.7292 \\
                      & 3 & 7056 & 0.722593* & 34.76637 & 0.936085* & 21.48436 \\
                      & 4 & 6655 & -0.50954* & 146      & 1.016691* & 837.7683 \\ \hline
\multirow{3}{*}{1}    & 2 & 6804 & 1.090047 & 1.593667 & 0.938192 & 0.980674 \\
                      & 3 & 7056 & 1.102072 & 0.508037 & 0.938057 & 0.453674 \\
                      & 4 & 6655 & 1.105225 & 0.22348  & 0.938046 & 0.570494 \\ \hline
\multirow{3}{*}{4}    & 2 & 6804 & 1.089332 & 1.65817  & 0.93843  & 3.521926 \\
                      & 3 & 7056 & 1.102261 & 0.491027 & 0.938076 & 0.25384  \\
                      & 4 & 6655 & 1.105359 & 0.211372 & 0.938047 & 0.561709 \\ \hline
\multirow{3}{*}{16}   & 2 & 6804 & 1.08694  & 1.874166 & 0.939109 & 10.75641 \\
                      & 3 & 7056 & 1.102266 & 0.490563 & 0.938124 & 0.251627 \\
                      & 4 & 6655 & 1.105439 & 0.204153 & 0.93805  & 0.537265 \\ \hline
\end{tabular}
\caption[Thermodynamic pressure and cold-side Nusselt number for different penalty safety factors in a heated cavity with Ra $=10^3$.]{Thermodynamic pressure and cold-side Nusselt number for different penalty safety factors in a heated cavity with Ra $=10^3$. Values marked with an asterisk are from problems not converged after 100 iterations.} \label{fig:EtaInfluence}
\end{table}
A point not still discussed is the choice of the safety parameter $\eta_0$ of the penalty terms from the SIP discretization (see \cref{eq:PenaltyFactor}).  \Cref{fig:EtaInfluence} shows results obtained for Ra $=10^3$, for different polynomial degrees and penalty safety factor.  For the tests presented here, the penalty terms of the diffusive terms from the momentum and energy equations are considered equal. Furthermore, the number of elements in the mesh is selected in such a way that the number of degrees of freedom remains approximately constant for each simulation. 

It is possible to see that the penalty safety factor (and therefore the penalty term) can have a great influence on the solution. If the value chosen is very small, as in the case of the table for $\eta_0 = 0.01$, the algorithm is not able to find a solution. On the other hand, if the chosen value is too high, the error also increases. It can be concluded that an optimal value for the penalty factor exists.

It is also noticeable that, maintaining a constant penalty safety factor, increasing the polynomial degree for an approximately constant number of DOFs gives an improvement in the results compared to the literature. Although for this testcase the effect of the penalty factor on the solution is not very large, the effect could be considerable, especially when dealing with more complex geometries and coarser meshes. The value $\eta_0 = 4$ has shown to be a value that gives stability to the scheme and is used for all simulations in this thesis, as already has been done in many works \parencite{krauseIncompressibleImmersedBoundary2017,kummerExtendedDiscontinuousGalerkin2017,smudamartinDirectNumericalSimulation2021} and is used for all calculations in this work.
 
The results presented in this section allows to conclude that the implemented solver is capable of dealing with flows with variable densities, and in particular in closed spaces. Additionally, it was observed that even for this complex test, convergence properties close to those expected from the DG method are obtained. Until this point only systems with a steady state solution were treated. Later in \cref{ssec:FlowCircCyl} the ability of the solver to compute flows with varying densities in non-steady state will be shown.
