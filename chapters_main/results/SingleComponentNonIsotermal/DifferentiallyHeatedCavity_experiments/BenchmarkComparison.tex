\subsubsection{Comparison of results with the benchmark solution}

Here a comparison of the results obtained with the XNSEC-solver and the results presented in the work of \cite{vierendeelsBenchmarkSolutionsNatural2003} is done. They solved the fully compressible Navier-Stokes equations on a stretched grid with $1024\times1024$ using a finite-volume method with quadratic convergence, providing very accurate results that can be used as a reference.
The benchmark results are presented for $\text{Ra} = \{10^2,10^3,10^4,10^5,10^6,10^7\}$. In this range of Rayleigh numbers, the problem has a steady-state solution. 
The cavity is represented by the domain $[0,1]\times[0,1]$. An equidistant Cartesian mesh with $30 \times 30$ elements is used for each simulation. A polynomial degree of five is used for the velocities and temperature and a degree of four for the pressure.
%
\begin{figure}
	\centering
	\pgfplotsset{width=0.35 \textwidth, compat=1.3}
	\inputtikz{HSCStreamlines}
	\caption{Streamlines of the heated cavity configuration with $\epsilon = 0.6$.}\label{fig:HSCStreamlines}
\end{figure}
% \FloatBarrier
\begin{figure}[h]
	\centering
	\pgfplotsset{width=0.20\textwidth, compat=1.3} 
	\inputtikz{TempProfile}
	\caption{Temperature profiles for the differentially heated square cavity along different vertical levels. Solid lines represent the XNSEC-solver solution and marks the benchmark solution.}
	\label{fig:TempProfile}
\end{figure}
%
\begin{figure}[h]
	\centering
	\pgfplotsset{width=0.20\textwidth, compat=1.3}
	\inputtikz{VelocityXProfile}
	\caption{Profiles of the x-velocity component along the vertical line $x=0.5$. Solid lines represent the XNSEC-solver solution and marks the benchmark solution.}
	\label{fig:VelocityXProfile}
\end{figure}
%
\begin{figure}[h]
	\centering
	\pgfplotsset{width=0.20\textwidth, compat=1.3}
	\inputtikz{VelocityYProfile}
	\caption{Profiles of the y-velocity component along the horizontal line $y=0.5$. Solid lines represents our solution and marks the benchmark solution.}
	\label{fig:VelocityYProfile}
\end{figure}

\begin{table}[h]
	\begin{center}
		\begin{tabular}{cccccc}
			\hline
			Rayleigh                           & $p_0$  & $p_{0,\text{ref}}$ & $\text{Nu}_{h}$ & $\text{Nu}_{c}$ & $\text{Nu}_{\text{ref}}$ \\ \hline
			\parbox[0pt][13pt][c]{0pt}{}$10^2$ & 0.9574 & 0.9573             & 0.9787          & 0.9787          & 0.9787                   \\
			$10^3$                             & 0.9381 & 0.9381             & 1.1077          & 1.1077          & 1.1077                   \\
			$10^4$                             & 0.9146 & 0.9146             & 2.2180          & 2.2174          & 2.2180                   \\
			$10^5$                             & 0.9220 & 0.9220             & 4.4801          & 4.4796          & 4.4800                   \\
			$10^6$                             & 0.9245 & 0.9245             & 8.6866          & 8.6791          & 8.6870                   \\
			$10^7$                             & 0.9225 & 0.9226             & 16.2411         & 16.1700         & 16.2400                  \\ \hline
		\end{tabular}
	\end{center}
	\caption[Differentially heated cavity: Results of Nusselt number and Thermodynamic pressure]{Comparison of calculated Nusselt numbers of the hot and cold wall and Thermodynamic pressure $p_0$ reported values by \cite{vierendeelsBenchmarkSolutionsNatural2003} for the differentially heated cavity.}
	\label{tab:p0_Nu_Results}
\end{table}
It was observed that for cases up to $\text{Ra} = 10^5$ the solution of the system using Newton's method presented in \cref{sec:newton} is possible without further modifications, while for higher values the algorithm diverges. The homotopy strategy mentioned in \cref{sec:CompMethodology} is used to overcome this problem and obtain solutions for higher Rayleigh numbers. Here, the Reynolds number is selected as the homotopy parameter and continuously increased until the desired value is reached.

In \cref{fig:TempProfile,fig:VelocityXProfile,fig:VelocityYProfile} temperature and velocity profiles for different Rayleigh numbers are shown. The profiles calculated with the XNSEC-solver agree closely to the benchmark solution. As expected,  an increase of the acceleration of the fluid in the vicinity of the walls for increasing Rayleigh numbers is observed.
We also compare the thermodynamic pressure and the Nusselt numbers to the benchmark solution. The results are shown in \cref{tab:p0_Nu_Results}.  The results are obtained for a polynomial degree of four for the velocities and temperature, three for the pressure in an equidistant $128\times128$ mesh. The thermodynamic pressure is obtained from \cref{eq:p0Condition}, and the average Nusselt number is calculated with \cref{eq:Nusselt}. We observe that our results are in very good agreement with the reference results, and the thermodynamic pressure differs at most in the fourth decimal place. Note that the average Nusselt number of the heated wall $(\text{Nu}_\text{h})$ and the Nusselt number of the cold wall $(\text{Nu}_\text{c})$ are different. As the Rayleigh number grows, this discrepancy becomes bigger, hinting that at such Rayleigh numbers, the mesh used is not refined enough to adequately represent the thin boundary layer and more complex flow structures appearing at high-Rayleigh cases. While for an energy conservative system $\text{Nu}_\text{h}$ and $\text{Nu}_c$ should be equal, for our formulation this is not the case, and the values differ slightly. This discrepancy can be seen as a measure of the discretization error from the DG formulation. This hints that the discrepancy between the average Nusselt numbers should decrease when the mesh resolution is increased, which will be discussed in the next section.
\FloatBarrier