

\subsubsection{Convergence study}\label{ssec:ConvStudyHeatedCavity}
An $h-$convergence study of the XNSEC solver was conducted using the heated cavity configuration. Calculations were performed for polynomial degrees $k = {1,2,3,4}$ and equidistant regular meshes with, respectively, $8\times8$, $16\times16$, $32\times32$, $64\times64$, $128\times128$ and $256\times256$ elements.  The $L^2$ -Norm was used to calculate errors against the solution in the finest mesh. The results of the $h$-convergence study for varying polynomial orders $k$ are shown in \cref{fig:ConvergenceDHC}. It is observed how the convergence rates scale approximately as $k+1$. Interestingly, for $k=2$ the rates are higher than expected. On the other hand, some degeneration is observed in convergence rates for $k = 4$. This strange behavior can be explained if one considers that the heated cavity presents a singular behavior at the corners %TODO. am i sure of this?
-similar to the problem previously exposed for the lid-driven cavity -, which causes global pollution in the convergence behavior of the algorithm. %Figure XXX shows in an elevated plane the difference obtained by subtracting the pressure field for a simulation with $k=3$ and the meshes of $64times64$ and $128times128$ elements,  which can be understood as a measure of the error in the simulation. It is clearly seen that the corners of the system present a very high error, which could be explained by the inconsistency between boundary conditions in that area. 
 
As discussed in the previous section, the difference in the average values of the Nusselt number on the hot wall $\text{Nu}_\text{h}$  and the cold wall $\text{Nu}_\text{c}$ is a direct consequence of the spatial discretization error and should decrease for finer meshes. In \cref{fig:NusseltStudy} the convergence behavior of the Nusselt number is presented for different polynomial degrees $k$, different number of elements and for two different Ra numbers. As expected, it can be observed that this discrepancy is smaller when a larger number of elements is used. It can also be seen that  $\text{Nu}_\text{h}$ reaches the expected solution of cells for a much smaller number of elements. This can be explained if one thinks that more complex phenomena take place near the cold wall (see \cref{fig:HSCStreamlines}), which makes necessary a finer mesh resolution in that area.

The results presented in this section allow us to conclude that the implemented solver is capable of dealing with flows with variable densities, and in particular in closed spaces. Additionally, it was observed that even for this complex test, convergence properties close to those expected from the DG-method are obtained. In this section only systems with a steady state solution were treated. In the next section the ability of the solver to compute flows with varying densities in non-steady state will be shown.

\begin{figure}[tb]
	\centering
	\pgfplotsset{width=0.34\textwidth, compat=1.3}
	\inputtikz{ConvergenceDHC}
	\caption{Convergence study of the differentially heated cavity problem for $\text{Ra} = 10^3$.}\label{fig:ConvergenceDHC}
\end{figure}
\begin{figure}[tb]
	\centering
	\inputtikz{NusseltStudy}
	\caption[Nusselt numbers of the differentially heated square cavity at the hot wall ($\text{Nu}_h$) and the cold wall ($\text{Nu}_c$) for different number of cells and polynomial order $k$.]{Nusselt numbers of the differentially heated square cavity at the hot wall ($\text{Nu}_h$) and the cold wall ($\text{Nu}_c$) for different number of cells and polynomial order $k$. The reference values from \textcite{vierendeelsBenchmarkSolutionsNatural2003} are shown with dashed lines.}\label{fig:NusseltStudy}
\end{figure}
\FloatBarrier
\subsubsection{Influence of the penalty factor}
\begin{table}[h]
\centering
\begin{tabular}{lllllll}
	\hline \vspace{0.1cm}
		$\eta_0$                  & DoFs & $k$ & $\text{Nu}_c$ &  $\frac{\text{Nu}_c - \text{Nu}_{c,\text{ref}}} {\text{Nu}_{c,\text{ref}}}\times 10^2 $   & $p_0$    & $\frac{p_0 - p_{0,\textbf{ref}}} {p_{0,\textbf{ref} }}\times 10^4 $ \\ \hline
\multirow{3}{*}{0.01} & 2 & 6804 & 0.549483* & 50.39424 & 0.899757* & 408.7292 \\
                      & 3 & 7056 & 0.722593* & 34.76637 & 0.936085* & 21.48436 \\
                      & 4 & 6655 & -0.50954* & 146      & 1.016691* & 837.7683 \\ \hline
\multirow{3}{*}{1}    & 2 & 6804 & 1.090047 & 1.593667 & 0.938192 & 0.980674 \\
                      & 3 & 7056 & 1.102072 & 0.508037 & 0.938057 & 0.453674 \\
                      & 4 & 6655 & 1.105225 & 0.22348  & 0.938046 & 0.570494 \\ \hline
\multirow{3}{*}{4}    & 2 & 6804 & 1.089332 & 1.65817  & 0.93843  & 3.521926 \\
                      & 3 & 7056 & 1.102261 & 0.491027 & 0.938076 & 0.25384  \\
                      & 4 & 6655 & 1.105359 & 0.211372 & 0.938047 & 0.561709 \\ \hline
\multirow{3}{*}{16}   & 2 & 6804 & 1.08694  & 1.874166 & 0.939109 & 10.75641 \\
                      & 3 & 7056 & 1.102266 & 0.490563 & 0.938124 & 0.251627 \\
                      & 4 & 6655 & 1.105439 & 0.204153 & 0.93805  & 0.537265 \\ \hline
\end{tabular}
\caption[Thermodynamic pressure and cold-side Nusselt number for different penalty safety factors in a heated cavity with Ra $=10^3$.]{Thermodynamic pressure and cold-side Nusselt number for different penalty safety factors in a heated cavity with Ra $=10^3$. Values marked with an asterisk are from problems not converged after 100 iterations.} \label{fig:EtaInfluence}
\end{table}
A point not still discussed is the choice of the safety parameter $\eta_0$ of the penalty terms from the SIP discretization (see \cref{eq:PenaltyFactor}).  \Cref{fig:EtaInfluence} shows results obtained for Ra $=10^3$, for different polynomial degrees and penalty safety factor.  For the tests presented here, the penalty terms of the diffusive terms from the momentum and energy equations are considered equal. Furthermore, the number of elements in the mesh is selected in such a way that the number of degrees of freedom remains approximately constant for each simulation. 

It is possible to see that the penalty safety factor (and therefore the penalty term) can have a great influence on the solution. If the value chosen is very small, as in the case of the table for $\eta_0 = 0.01$, the algorithm is not able to find a solution. On the other hand, if the chosen value is too high, the error also increases. It can be concluded that an optimal value for the penalty factor exists.

It is also noticeable that, maintaining a constant penalty safety factor, increasing the polynomial degree for an approximately constant number of dofs gives an improvement in the results compared to the literature. Although for this testcase the effect of the penalty factor on the solution is not very large, the effect could be considerable, especially when dealing with more complex geometries and coarser meshes. The value $\eta_0 = 4$ has shown to be a value that gives stability to the scheme and is used for all simulations in this thesis, as already has been done in many works \parencite{krauseIncompressibleImmersedBoundary2017,kummerExtendedDiscontinuousGalerkin2017,smudamartinDirectNumericalSimulation2021}.
 

