
\begin{figure}[b!]
	\centering
	\inputtikz{NusseltStudy}
	\caption{Nusselt numbers calculated with \cref{eq:Nusselt} at the hot wall ($\text{Nu}_h$) and the cold wall ($\text{Nu}_c$) for different number of cells and polynomial order $k$. The reference values from \cite{vierendeelsBenchmarkSolutionsNatural2003} are shown with dashed lines.}\label{fig:NusseltStudy}
\end{figure}

\subsubsection{Convergence study}\label{ssec:ConvStudyHeatedCavity}
An $h-$convergence study of the XNSEC solver was conducted using the heated cavity configuration. Calculations were performed for polynomial degrees $k = {1,2,3,4}$ and equidistant regular meshes with, respectively, $8\times8$, $16\times16$, $32\times32$, $64\times64$, $128\times128$ and $256\times256$ elements.  The $L^2$ -Norm was used to calculate errors against the solution in the finest mesh. The results of the $h$-convergence study for varying polynomial orders $k$ are shown in \cref{fig:ConvergenceDHC}. It is observed how the convergence rates scale approximately as $k+1$. Interestingly, for $k=2$ the rates are higher than expected. On the other hand, some degeneration is observed in convergence rates for $k = 4$. This strange behaviour can be explained by the fact that similarly to the lid-driven cavity configuration

Este comportamiento extranio puede ser explicado si se piensa en que la cavidad calentada presenta un comportamiento singular en las esquinas -problema similar al expuesto anteriormente para el lid-driven cavity-. En la figura X se muestra 

As discussed in the previous section, the difference in the average values of the Nusselt number on the hot wall $\text{Nu}_\text{h}$  and the cold wall $\text{Nu}_\text{c}$ is a direct consequence of the spatial discretization error.

In \cref{fig:NusseltStudy} the convergence behavior of the Nusselt number for different polynomial degrees $k$, different number of elements and for two different Ra numbers is presented. As expected, it can be observed that this discrepancy is smaller when a larger number of elements is used.


\begin{figure}[t!]
	\centering
	\pgfplotsset{width=0.34\textwidth, compat=1.3}
	\inputtikz{ConvergenceDHC}
	\caption{Convergence study of the differentially heated cavity problem for $\text{Ra} = 10^3$.}\label{fig:ConvergenceDHC}
\end{figure}

\FloatBarrier