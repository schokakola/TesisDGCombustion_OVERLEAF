\subsection{Flow over a circular cylinder}\label{ssec:FlowCircCyl}

A further thest case of the XNSEC solver is the simulation of flow over a circular cylinder. The results shown section are based on the work by \textcite{miaoHighOrderSimulationLowMachFlows2022}

The simulation of a 2D flow over an obstacle exhibits different flows states depending on the Reynolds number. For low Reynolds numbers, the flow is laminar and stationary. When the Reynolds number reaches a critical value, the flow becomes non-stationary and eventually turbulent. In particular, for $50 \leq \Reynolds \leq 160$ the behavior of the fluid is laminar with an non-stationary periodic character, and the well-known vortex shedding phenomenon appears \parencite{sharmaHEATFLUIDFLOW2004}. In this section the ability of the XNSEC solver for dealing with isothermal and non-isothermal unsteady flow will be assessed. 
This is done by calculating the Strouhal number, average drag and lift coefficients and the Nusselt number and comparing them to reference solutions.
\begin{figure}[t!]
	\begin{center}
		\def\svgwidth{0.88\textwidth}
		\import{./plots/}{HeatedCylinderGeom.pdf_tex}
		\caption{Schematic representation of the heated circular cylinder. Figure adapted from \parencite{miaoHighOrderSimulationLowMachFlows2022}} 
		\label{fig:CircularCylinderGeom}%
	\end{center}%
\end{figure}%
\subsubsection{Set-up}
\cref{fig:CircularCylinderGeom} shows the geometry of the problem. In the center of the domain is located a heated cylinder of diameter $d = 1$, which is subjected to a flow of constant velocity magnitude. The size of the computational domain is chosen as $D = 59d$, which is large enough so that the outlet boundary conditions do not influence the solution. The left half of the boundaries of the computational domain corresponds to a velocity inlet with constant velocity $(u,v) = (1,0)$ and temperature $T = T_\infty$. The right half of the domain corresponds to a pressure outlet. The boundary condition corresponding to the heated cylinder is a no-slip wall defined with $(u,v) = (0,0)$, and with a constant temperature $T = T_h$. The system is an open system, and the thermodynamic pressure has a constant value set to $p_0 = 1$.

As mentioned, a isothermal and non-isothermal system is considered in this section. A reference temperature of $\RefVal{T} = \SI{298.15}{\kelvin}$ is selected. The conditions for the velocity inlet are for the isothermal case a dimensionless temperature of $T_h = 1.0$ and $T_\infty = 1$, while for the non-isothermal $T_h = 1.5$ and $T_\infty = 1.0$ . A Reynolds number of 100 is used, for which the flow is known to be unsteady and periodic. Gravity effects are taken as negligible. The Prandtl number has a constant value of $\text{Pr} = 0.71$ and the heat capacity ratio is $\gamma = 1.4$. The transport parameters are calculated with Sutherland's law which is given by \cref{eq:nondim_sutherland} and the equation of state for the density is \cref{eq:ideal_gas}.

For all calculations a polynomial degree of order three is used for both velocity components and the temperature, and two for pressure. In this simulation a curved mesh consisting of 64 elements in the radial direction and 64 elements in the angular direction is used. The use of a curved mesh for this problem was done primarily to adequately represent the geometry of the cylinder. Another possibility could have been to use a method such as a Immersed Boundary Method -which is also supported in the BoSSS-framework-, but it is beyond the scope of this work.

One point to note regarding the use of curved meshes within the DG method (and in all High-order methods), is that special care has to be taken in the representation of curved elements in order to use correctly quadrature rules \parencite{bassiHighOrderAccurateDiscontinuous1997}. This is critical to preserve the convergence properties of DG-methods. In particular, since the highest degree of the polynomials used in this simulation is three, elements of the bi-cubic type are used, where 16 nodes are used per element for the discretization for a two-dimensional element. Furthermore, the time discretization is performed with a BDF-3 scheme, and the time derivative of the continuity equation is calculated with a second order backward difference scheme (see \cref{ssec:TemporalDiscretization}). The simulation time corresponds to $t= 100$, and constant timesteps of $\Delta t = 0.2$ are used. The initial conditions are
\begin{subequations} 
\begin{align}
&u(t=0) = 1 + u^{\text{vortex}},  \\
&v(t=0) = 0 + v^{\text{vortex}},  \\
&T(t=0) = 1,\\
&p(t=0) = 0.
\end{align}
\end{subequations}
Here, $u^{\text{vortex}}$ and $u^{\text{vortex}}$ are the velocity component of a vortex field of radius $r$, strength $a$  and central point $(x^o,y^o)$ defined by 
{
\begin{subequations}
\begin{equation}
	u^{\text{vortex}}(x,y) = 
	\begin{cases}
		-a(y-y^o) & \text{if} \quad\sqrt{(x-x^o)^2+(y-y^o)^2} \leq r \\
		0 & \text{if} \quad\sqrt{(x-x^o)^2+(y-y^o)^2} > r
	\end{cases}
\end{equation}\label{eq:VortexU}
\begin{equation}
	v^{\text{vortex}}(x,y) = 
	\begin{cases}
		a(x-x^o) & \text{if} \quad\sqrt{(x-x^o)^2+(y-y^o)^2} \leq r \\
		0 & \text{if} \quad\sqrt{(x-x^o)^2+(y-y^o)^2} > r
	\end{cases}
\end{equation}	\label{eq:VortexV}
\end{subequations}
}%
The reason for placing a vortex in the initial conditions is to include a perturbation in the system which triggers the vortex shredding phenomenon. The vortex moves with the flow and is eventually advected from the calculation domain. Specifically, for this simulation, a vortex of radius $r=1$ and strength $a = 1$ is placed at $(x^o,y^o) = ( 2,0)$. It is worth mentioning that the inclusion of the vortex in the initial conditions is not imperative to make the vortex-shedding phenomenon emerge, but it is a way to accelerate its appearance. For the range of Reynolds numbers mentioned, even instabilities of the numerical method should be enough to cause the phenomenon to arise, but in a much slower way. 

The variables lift coefficient $C_L$, drag coefficient $C_D$, Strouhal number St, and Nusselt number Nu are calculated and compared with reference results. These characteristic quantities are defined as
\begin{equation} 
	C_L = \frac{2F_L}{\rho_\infty u^2_\infty d}
\end{equation}
\begin{equation}
	C_D = \frac{2F_D}{\rho_\infty u^2_\infty d}
\end{equation}
\begin{equation}
	\text{St} = \frac{fd}{u_\infty}
\end{equation}
\begin{equation}
	\text{Nu} = \frac{d}{\Delta T}\frac{1}{\Vert \partial S\Vert_{\textbf{leb}}}\oint_{\partial S} \nabla T \cdot \vec{n} dS
\end{equation}
%TODO como escribir de forma consistente y simple el numero de nusselt average?
%TODO como se definen las fuerzas FL y FD? 
$F_L$ and $F_D$ are the lift and drag force respectively, $f$ is the vortex shedding frequency and ${\Vert \partial S\Vert_{\textbf{leb}}}= \pi d$ is the circunference of the cylinder.
\begin{figure}[t]
	\centering	
	\inputtikz{HeatedCylinderCl}	
	\inputtikz{HeatedCylinderCd}	    
	\inputtikz{HeatedCylinderNu}
	\caption{Temporal evolution of average Nusselt number, lift coefficient and drag coefficient of the heated cylinder}	\label{fig:HeatedCylinderResults}
\end{figure}
\subsubsection{Isothermal case.}
The isothermal case results are compared with the work of \textcite{sharmaHEATFLUIDFLOW2004}. They reported results for the case $\Reynolds = 100$ and $T_h = T_\infty$. The Strouhal number and average drag coefficient calculated with the XNSEC solver are $\text{St} = 0.1639$ and $C_{D,\textbf{avg}} = 1.3103$, while the reference values $\text{St} = 0.164$ and $C_{D,\textbf{avg}} = 1.3183$, which means a error of less than a 0.6\% on both quantities. Its worth noting that preliminary calculations with a implicit Euler scheme for the discretization of the temporal terms didn't cause the vortex shedding phenomenon to appear, making necessary a scheme of higher order. This is probably due to an excess of numerical dissipation by using a low-order temporal discretization, which smooths out and removes the natural perturbation effects that trigger vortex shedding.
\subsubsection{Non-isothermal case.}
A comparison of the calculation results for the non-isothermal case was performed based on the results reported in \textcite{shiHeatingEffectSteady2004}, \textcite{wangRelationshipEffectiveReynolds2000} and \textcite{henninkLowMachNumberFlow2022}.
The temporal evolution of the characteristic quantities $C_D$, $C_L$ and Nu is shown in \cref{fig:HeatedCylinderResults}. The results agree very closely with the references. The average Nusselt number is  $\text{Nu} = 3.8434$, while the reference value from \textcite{henninkLowMachNumberFlow2022} is $\text{Nu} = 3.804$, which corresponds to a difference of 1.04\%. The Strouhal number is $\text{St} = 0.1538$ and the references report values of $\text{St} = 0.152$ and $\text{St} = 0.1536$, a difference of approximately 1\%. It is possible then to conclude that the XNSEC solver allows to simulate adequately unsteady non-isothermal flows, where the fluid properties do present a scalar dependence (in this case, on the temperature). 

It is worth noting that in this test-case the temperature changes causes only a moderate variation in the density. As will be seen later, the temporal term appearing in the continuity equation, as implemented in the present solver, is problematic in the case of larger density differences \parencite{knikkerComparativeStudyHighorder2011}, as it is a cause of numerical instabilities.
\FloatBarrier