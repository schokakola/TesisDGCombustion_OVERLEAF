
\subsection{Heated backward-facing step}\label{ssec:HeatedBackwardFacingStep}


As an extension to the previous casem the backward-facing step configuration in a non-isothermal configuration is studied, where the bottom wall is heated to a constant temperature. 


In this section the configuration for a heated backward-facing step proposed in \cite{xieFluidFlowHeat2016} is solved. 
The fluid entering the system has a temperature equal to $\hat T_0 = \SI{283}{\kelvin}$ and the bottom wall is set to a constant temperature of $\hat T_1 =\SI{313}{\kelvin}$. The inlet temperature is used as the reference temperature, obtaining $T_0 = 1.0$ and $T_1 = 1.106$.
In the work of \cite{xieFluidFlowHeat2016} results are reported for the local Nusselt numbers and the local friction coefficients $f_d$  along the bottom wall ($y = 0$) for different expansion ratios and Reynolds numbers.
By combining the definition of the Nusselt number ($\gls{Nusselt} = \gls{HeatTransCoef}\hat{L}/\glsHat{HeatConductivity}$), Newton's law of cooling ($\hat{\vec{q}} = \hat{h} (\hat{T}_0 - \hat{T}_1 )$), and Fourier's law of heat conduction ($\hat{\vec{q}} = \hat \lambda \hat{\nabla} \glsHat{temp}$) a expression for the local Nusselt number is obtained.
\begin{equation}
	\gls{NusseltLoc} = \frac{\hat L}{\hat T_0-\hat T_1}\hat \nabla \hat T \cdot \hat {\vec{n}}
\end{equation}
where $\hat L$ is the reference length. $ \hat L = \hat S$ is chosen to be consistent with the definition of the Reynolds number of the reference. Furthermore, the local friction factor can be written as % recognizing that the wall shear stress along the bottom wall $\tau_{\text{w}} = -\mu \nabla u \cdot \vec{n}$,
\begin{equation}
	f_d = \frac{8\hat \nu} { (\hat U_{\text{mean}})^2}  \hat \nabla \hat u \cdot \hat {\vec{n}}
\end{equation}

\begin{figure}[t]
	\centering
	\pgfplotsset{width=0.81\textwidth, compat=1.3}
	\inputtikz{BackwardFacingStepTemperatureField}	
	\inputtikz{BackwardFacingStepStreamlines}
	\caption{Temperature profile and streamlines corresponding to the backward-facing Step configuration for $\gls{Reynolds} = 400$ and an expansion ratio of two.} \label{fig:BFS_Temperature_Streamlines}
\end{figure}

 Simulations were conducted for different Reynolds numbers and expansion ratios. In \cref{fig:BFS_Temperature_Streamlines} the temperature field and the streamlines corresponding to a calculation with $\gls{Reynolds} = 700$ are shown. Here, the apparition of the secondary vortex is seen in the top wall. Note that only a small part of the computational domain is shown. Far away from the step, a lightly skewed parabolic velocity profile is obtained, which is influenced by the density variations on the vertical direction.
 
 For this range of temperature differences, the temperature profile is just influenced by conductive effects, since no appreciable natural convection phenomena appears. For larger temperature differences, Rayleigh-Bénard type instabilities would appear in the flow. This type of system will be treated later in \cref{ssec:RayBer}.
 
It should be noted here that the results obtained using the XNSEC-solver are substantially different from those reported by \cite{xieFluidFlowHeat2016}, and will not be shown here. However, in the work of \cite{henninkLowMachNumberFlow2022} the same is also reported, stating that with his method it was not possible to reproduce the results presented by \cite{xieFluidFlowHeat2016}. 

In \cref{fig:fd_Nu_plot} the local friction factor and local Nusselt number along the wall $y = 0$ are plotted for $\gls{Reynolds} = 700$ and ER $= 2$. Comparing the results from the XNSEC-solver with those reported in \cite{henninkLowMachNumberFlow2022} a very good agreement can be observed.
With this test it is possible to confirm that the XNSEC solver is able to deal with complex systems where heat transfer is present. However, for the range of temperature differences involved in this case, the variation of physical parameters such as density, viscosity and thermal conductivity with respect to temperature has no appreciable influence on the simulated flow fields. The next two test-cases will show how the XNSEC-solver is able to simulate low-Mach number flows with a considerable temperature difference.
\begin{figure}[tb]
	\pgfplotsset{
		group/xticklabels at=edge bottom,
		legend style = {
				at ={ (0.59,1.0), anchor= north east}
			},
		unit code/.code={\si{#1}}
	}
	\inputtikz{fd_Nu_plot1}
	\inputtikz{fd_Nu_plot2}
	\caption[Local friction factor and local Nusselt number along the bottom wall of the backward-facing step for $\gls{Reynolds} = 700$ and an expansion ratio of two.]{Local friction factor and local Nusselt number along the bottom wall of the backward-facing step for $\gls{Reynolds} = 700$ and an expansion ratio of two. The solid lines corresponds to our solution and the marks to the reference \citep{henninkLowMachNumberFlow2022}}
	\label{fig:fd_Nu_plot}
\end{figure}%
\FloatBarrier%