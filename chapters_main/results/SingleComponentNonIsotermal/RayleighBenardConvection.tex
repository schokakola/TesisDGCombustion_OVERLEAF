\subsection{Rayleigh-Bénard Convection}\label{ssec:RayBer}

The Rayleigh-Bénard convection is a configuration similar to the heated cavity (see \cref{ss:DHC}) as the flow is also induced by bouyancy effects. The configuration consists of a fluid located between two horizontal plates maintained at different temperatures, with the lower temperature being higher than the upper temperature. It is known that under a certain Rayleigh number the flow presents a macroscopically stable behavior and the heat is simply transferred by conductive phenomena. When the Rayleigh number exceeds a certain critical value $\text{Ra}_{\text{crit}}$, the system becomes unstable, which causes a perturbation of the system to give rise to fluid motion, creating the so-called convection cells, also usually also referred to as Bénard cells.

In the following sections two aspects of this situation will be discussed by making use of different type of boundary conditions on the sides of the domain. First in \cref{ssec:SingleCellConv} periodic boundary conditions are used, which allows to study a single pair of convection rolls and to calculate numerically the critical value $\text{Ra}_{\text{crit}}$, which is compared with theoretical values. Subsequently, in \cref{ssec:MultipleCellConv} a transient simulation of this configuration is performed by using pressure-outlet boundary conditions. The results shown section are based on the work by \textcite{miaoHighOrderSimulationLowMachFlows2022} %TODOOOOOOOOOOOOOOOOOOOOOOOOOOOOOOOOOO

\subsubsection{Periodic boundary conditions}\label{ssec:SingleCellConv}
\begin{figure}[bt]
	\begin{center}
		\def\svgwidth{0.83\textwidth}
		\import{./plots/}{RayleighBen_Geometry.pdf_tex}
		\caption{Geometry of the Rayleigh-Bénard convection problem. Convection rolls are sketched}
		\label{fig:RayBenGeometryPeriodic}
	\end{center}
\end{figure}
In this section the fluid behavior will be analyzed by studying a system where a single convection-cell appears. Making use of the linear stability theory for the governing equations under the boussinesq assumption, it is possible to determine that for a system with rigid boundaries the critical Rayleigh number is $\text{Ra}_{\text{crit}} = 1707.762$ \parencite{chandrasekharHydrodynamicHydromagneticStability1961}.  Moreover, the wave number is $a_c =3.117$, which implies that the convective rolls develop at an aspect ratio of $2 \pi/a_c = 2.016$. For the simulations presented in this section an aspect ratio of $L/H = 2$ is used, as is done by \textcite{kaoSimulatingOscillatoryFlows2007}. The geometry and boundary conditions of the analyzed problem can be found in \cref{fig:RayBenGeometryPeriodic}. The upper wall corresponds to a no-slip wall that is maintained at a constant temperature $T_c$. Similarly, the bottom wall is also a no-slip wall with a temperature $T_h$, with $T_h > T_c$. The boundary conditions on the left and right of the computational domain are periodic boundary conditions. Gravity has only one component in the negative direction of $y$. %The chosen configuration makes it possible to study a single convection roll. 

For the range of Rayleigh numbers treated here, the simulations are steady state simulations. For all simulations in this section a regular Cartesian grid consisting of $32\times64$ cells, with dimensionless lengths $L = 2$ and $H = 1$ was used. The polynomial degree of both velocity components and temperature is four, and for pressure it is three. An open system is assumed, and $p_0 = 1.0$ throughout the simulation.


First, the stability of the XNSEC-solver will be studied by determining the critical value $\text{Ra}_{\text{crit}}$. As mentioned above, the theoretical value $\text{Ra}_{\text{crit}} = 1707.762$ was determined using the Boussinesq  assumption in the governing equations, which, unlike the low mach equations, is only valid if the variation of temperature within the system is very small. For this reason, an analysis is performed for a dimensionless temperature difference $\epsilon = (T_h-T_c)/(T_h+T_c) = 0.0001$, meaning $T_h =1.0001$ and $T_c = 0.9999$. The reference temperature is chosen as $\RefVal{T} = \SI{600}{\kelvin}$. Since the XNSEC-solver is based on the low-Mach equations, this choice of low temperature difference allows to compare the present results with those obtained analytically using the Boussinesq approximation. The definition of the dimensionless numbers as well as the reference velocity are exactly the same as those mentioned in \cref{ss:DHC}. The transport coefficients are calculated with Sutherlands Law.

Similarly to the case exposed in \cref{ssec:FlowCircCyl}, velocity fields are initialized with vortices, which play the role within the simulation of a trigger for the inherent instabilities of the problem. In particular, two vortices with opposite directions of rotation are added. The velocity components of these are again given by equations \cref{eq:VortexU} and \cref{eq:VortexV} . The coordinates of the first vortex are $(x^o,y^o) = (-0.5,0)$ and it has a strength $a = 1$ and a radius $r=0.4$. For the second vortex $(x^o,y^o) = (0.5,0)$, $a = -1$ and $r=0.4$. The initial conditions are
\begin{subequations} 
	\begin{align}
		&u(t=0) = 1 + u^{\text{vortex-left}} + u^{\text{vortex-right}},  \\
		&v(t=0) = 0 + v^{\text{vortex-left}}+ v^{\text{vortex-right}},  \\
		&T(t=0) = 1,\\
		&p(t=0) = -\frac{\rho y}{\text{Fr}^2}.
	\end{align}
\end{subequations}
It is worth noting that the initialization of the solver without the vortices leads the solver to a stationary solution where there is no fluid motion. Clearly, this is a solution of the equations, but, as already mentioned, it is an unstable solution.
\begin{figure}[bt]
	\centering
	%	\pgfplotsset{
		%		group/xticklabels at=edge bottom,
		%		unit code/.code={\si{#1}}
		%	}
	\pgfplotsset{width=0.31\textwidth, compat=1.3}
	\inputtikz{RayBerStability_a}%
	\hspace{0.2cm}
	\inputtikz{RayBerStability_r}
	\caption{Maximum x-velocity in the Rayleigh-Bénard convection configuration for different $a$ and $r$.}
	\label{fig:RayBerMaxVel}
\end{figure}

\begin{figure}[bt]
	\centering
	\inputtikz{RayBerCalculationRa}
	\caption{Stability behaviour of the Rayleigh-Bénard convection with $\epsilon = 0.0001$.} \label{fig:ReyBerCritRa}
\end{figure}
\begin{figure}[bt]
	\pgfplotsset{width=0.45\textwidth, compat=1.3}
	\centering
    \inputtikz{RayBenTemperatureRa}
\caption{Critical Rayleigh number at different temperatures and different reference temperatures.}\label{fig:RayBenardTemperatureRaPlot}
\end{figure}
\begin{figure}[h]
	\centering
	\pgfplotsset{width=0.31\textwidth, compat=1.3}
	\inputtikz{RayBerTemperature2e3}%
	\inputtikz{RayBerTemperature3e3}%
	\inputtikz{RayBerTemperature5e3}%
	\par\bigskip
	\inputtikz{RayBerTemperature1e4}%
	\inputtikz{RayBerTemperature5e4}%
	\inputtikz{RayBerTemperature1e5}%
	\caption{Temperature field and contours of a Rayleigh-Bénard convection roll} \label{fig:RayBenTemperatureField}
\end{figure}
\begin{figure}[h!]
	\centering
	\pgfplotsset{width=0.31\textwidth, compat=1.3}
	\inputtikz{RayBerStreamLine2e3}%
	\inputtikz{RayBerStreamLine3e3}%
	\inputtikz{RayBerStreamLine5e3}%
	\par\bigskip
	\inputtikz{RayBerStreamLine1e4}%
	\inputtikz{RayBerStreamLine5e4}%
	\inputtikz{RayBerStreamLine1e5}%
	\caption{Streamlines of a Rayleigh-Bénard convection roll} \label{fig:RayBenStreamlines}
\end{figure}
First, a study was performed to demonstrate the independence of the steady-state solution from the chosen initial conditions. This is demonstrated in \cref{fig:RayBerMaxVel}, where for a given Rayleigh number the maximum velocity obtained inside the system is shown, varying the strength $a$ or the radius $r$ of the vortices. It is also apparent that the case $\text{Ra} = 1000$ does not exhibit a macroscopic fluid motion, while $\text{Ra} = 2000$ does. This points to the fact that the critical value is effectively in this range. 

Subsequently, a series of simulations were performed with $a = 0.5$ y $r = 0.4$, and a bisection algorithm was used to find the critical Rayleigh value, obtaining $\Ra = 1707.922$, as shown in \cref{fig:ReyBerCritRa}. Compared to the theoretical value $\Ra = 1707.762$, it presents a difference of only $0.009\%$. In addition, the proportionality  $u \propto \sqrt{\text{Ra}- \text{Ra}_\text{Ra}}$  that is expected by analytical arguments is also fulfilled, obtaining, in particular, $u_{\text{max}}  = 0.008015\sqrt{\text{Ra} - 1707.922}$. It is possible to conclude that, at least for small temperature differences, the low-Mach approximation has a behavior similar to that of the equations with the Boussinesq approximation. 

In relation to this, it is interesting to analyze the influence of the temperature difference within the system with respect to the critical Rayleigh value. This is demonstrated in \cref{fig:RayBenardTemperatureRaPlot}, where the calculation is done for different values of the reference temperature. Clearly, for low $\epsilon$  the critical value is very close to the theoretical value obtained under the Boussinesq approximation. As $\epsilon$ increases, a larger deviation becomes apparent. This can be attributed to the dependence of the viscosity, thermal conductivity, and density coefficients on the temperature. 

Finally, in \cref{fig:RayBenTemperatureField} and \cref{fig:RayBenStreamlines} the temperature fields and streamlines obtained with the XNSEC-solver for $\epsilon = 0.5$ and $\RefVal{T} = \SI{600}{\kelvin}$ for different Rayleigh numbers are shown. An interesting point to mention is the difference between the results obtained using the Boussinesq approximation and the low-Mach approximation. Comparing the results reported here with publications that use the Boussinesq approximation ( \textcite{shishkinaRayleighBenardConvectionContainer2021,zhouNumericalSimulationLaminar2004}), a clear difference in the streamlines is observed. In particular, the centers of the streamlines obtained with the low-Mach solver are slightly shifted towards the colder zones, while the streamlines obtained with the Boussinesq approximation do not present this deviation and remain close to the center. A possible explanation for this is that with the Boussinesq approximation the expansion effects of the fluid are ignored, whereas in the low-Mach equations this is not the case. When the latter is used, the hot fluid moves the center of the streamline towards the cold fluid.
\subsubsection{Pressure outlet boundary condition}\label{ssec:MultipleCellConv}
\begin{figure}[t!]
	\begin{center}
		\def\svgwidth{0.93\textwidth}
		\import{./plots/}{RayleighBen_Geometry_PressOutlet.pdf_tex}
		\caption{Geometry of the Rayleigh-Bénard convection with pressure outlet boundary conditions. }
		\label{fig:RayBenGeometry}
	\end{center}
\end{figure}
Now a transient calculation of the Rayleigh-Bénard configuration is done. The configuration is similar to the one described in the last part, with the difference that the boundaries of the left and right sides are now pressure outlet boundary conditions and the length $L$ of the system is chosen considerably longer. A sketch showing the system is shown in \cref{fig:RayBenGeometry}. Theoretically,  an infinitely long system should be used to avoid the influence of the boundary condition to be able to adequately represent the convection rolls. The nondimensional lengths are chosen as $H=1$ and $L=10$. A grid with $32\times320$ cells is used. The polynomial degrees for the velocity components and temperature are set to four and for the hydrodinamic pressure to three. The time discretization is done again with a BDF-3 scheme and the calculation time is 150, using timesteps of $\Delta t = 0.5$. The temperatures are set to $T_h = 1.5$ and $T_c = 0.5$, with a reference temperature of $\RefVal{T} = \SI{600}{\kelvin}$. The Rayleigh number is $\Ra = 5659$, which is above the critical value.
The initial conditions are chosen as
\begin{subequations} 
	\begin{align}
		&u(t=0) = 0, \\
		&v(t=0) = 0, \\
		&T(t=0) = 1, \\
		&p(t=0) = -\frac{\rho y}{\text{Fr}^2}.
	\end{align}
\end{subequations}

Note that no vortex is included in the initial conditions and that only a fluid at rest is considered. A perturbation-like effect caused by the pressure outlet boundaries triggers the movement of the fluid. At the first stages of the simulation the perturbations induce a vortex-like structure close to the left and right boundaries, and start gradually filling the whole domain, reaching finally a steady solution. This can be seen in \cref{fig:RayBerUnsteadySol}. Clearly in the center of the domain (far away from the outlet boundary conditions), the structure of the solution is very similar to the ones shown in the last subsection. In fact, if the domain length is chosen sufficiently large, they should be equal. 
\begin{figure}[t]
	\centering
	\pgfplotsset{width=0.96\textwidth, compat=1.3}
	\inputtikz{RayBerPressureOutlet_Temperature10}
	\par\bigskip%
	\inputtikz{RayBerPressureOutlet_Streamline10}
	\par\bigskip%
	\inputtikz{RayBerPressureOutlet_Temperature30}
	\par\bigskip%
	\inputtikz{RayBerPressureOutlet_Streamline30}
	\par\bigskip%
	\inputtikz{RayBerPressureOutlet_Temperature50}
	\par\bigskip%
	\inputtikz{RayBerPressureOutlet_Streamline50}
	\par\bigskip%
	\inputtikz{RayBerPressureOutlet_Temperature100}
	\par\bigskip%		
	\inputtikz{RayBerPressureOutlet_Streamline100}
	\par\bigskip%
	\caption{Temperature and streamlines of the Rayleigh-Bénard flow with pressure outlets.}\label{fig:RayBerUnsteadySol}
\end{figure}
\FloatBarrier