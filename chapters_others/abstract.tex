\glsresetall
In this work, a high-order fully coupled numerical solver based on the \Gls{DG} method for simulating reactive flows is presented. The discretized set of equations of continuity, momentum, energy and chemical species are solved in a fully coupled manner using a globalized Newton algorithm. The main objective of the solver is to create a framework for investigating diffusion flames using the \Gls{DG} method. For this, the low-Mach approximation of the Navier--Stokes equations is used, which enables the simulation of non-Boussinesq flows. The chemical reaction is modeled using a one-step combustion model with variable kinetic parameters, which is specifically tailored for hydrocarbon combustion. The temperature and concentration dependence of the density, heat capacity, and transport parameters is considered in the formulation. 

A detailed description of the governing equations utilized in this study is presented, along with a comprehensive discussion of their derivation and the assumptions involved.
The general procedure involved in the temporal and spatial discretization of the \Gls{DG} method is illustrated using a basic transport equation. Subsequently, the DG discretization of the governing equations for reacting flows is presented, and numerical fluxes are thoroughly described. 

The computational methods developed for solving the governing equations are presented in detail. In particular, the strategy for the solution of the nonlinear problem by means of the globalized Dogleg-Newton method is explained, together with an efficient method for calculating the Jacobian matrix. Furthermore, various strategies that improve the convergence properties of the algorithm are presented. These include a fully automatized homotopy continuation method for the solution of highly nonlinear systems, an Adaptive Mesh Refinement strategy used for generating adequate meshes on critical areas of the simulation, and a solver safeguard for avoiding unphysical solutions during the calculation. 

For steady reacting flows an additional strategy is applied, which allows to find adequate initial estimates for the simulation of diffusion flames. This approach involves solving a simplified set of equations obtained under the assumption of an infinitely fast chemical reaction and is a robust method for finding the solution of combustion systems.

A thorough validation of the solver using several test cases is shown, which also highlight important advantages of the DG method and the algorithms introduced in this work. The test cases validate the solver against various benchmark solutions obtaining results in very good agreement with the literature. Additionally, the accuracy of the method is assessed in various flow settings, all of them demonstrating the expected high convergence rates of the \Gls{DG} method. However, stability problems are observed in transient simulations of low-Mach flows in which density exhibits large variations.