The aim of this work is to present a high-order fully coupled numerical solver based on the Discontinuous Galerkin method for simulating reactive flows. The discretized set of equations of continuity, momentum, energy, and chemical species are solved in a fully coupled manner, using specialized nonlinear and linear solvers together with adequate preconditioning. The low-Mach approximation of the governing equations is used which allows the simulation of non-Boussinesq flow and the chemical reaction is modeled using a one-step combustion model with variable kinetic parameters.
A detailed derivation of the governing equations is presented, with emphasis on the 

 The temporal discretization is realized in a implicit manner using BDF-schemes. 






The solution of the nonlinear problem is done by means of the Dogleg-Newton method, a globalized algorithm which increases the likelihood of finding a convergence solution even in cases where no adequate initial is avaliable. Various strategies that improve the convergence properties of the algorithm are also presented. 
Fully automatized homotopy method for the solution of highly nonlinear systems, based in the idea solving a series of simpler problems until convergence of the original problem is reached.

An adaptive mesh refinement framework is also presented, which is used for obtaining suitable meshes for a better representation of high gradients present in some of the problems presented.

Steady state simulation of diffusion flames are initialized by means of flame-sheet estimates, which correspond to the solution of a system with infinitely fast chemistry. These estimates serve as suitable initial conditions that guide the algorithm towards the solution where the flame is present.



The use of a fully coupled method as an alternative to segregated approaches is explored. 
The fully coupled method proves to be a robust approach 

===================================

 Details on the spatial dis-cretization and the nonlinear solver are presented. 
 The method is tested withreactive and nonreactive benchmark cases. Convergence studies are presented,and we show that the expected convergence rates are obtained. 
 The solver forthe low-Mach equations is used for calculating a differentially heated cavityconfiguration, which is validated against benchmark solutions. 
 
 Additionally,a two-dimensional counter diffusion flame is calculated, and the results arecompared with the self-similar one dimensional solution of said configuration.

