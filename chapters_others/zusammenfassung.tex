\glsresetall
In dieser Arbeit wird ein vollständig gekoppelter numerischer Löser hoher Ordnung vorgestellt, der auf der \Gls{DG}-Methode zur Simulation von reaktiven Strömungen basiert. Die diskretisierten Gleichungen der Kontinuität, des Impulses, der Energie und der chemischen Spezies werden auf vollständig gekoppelte Weise unter Verwendung eines globalisierten Newton-Algorithmus gelöst. Das Hauptziel des Lösers ist die Erstellung eines Frameworks für die Untersuchung von Diffusionsflammen mit der \Gls{DG}-Methode. Dazu wird die low-Mach-Approximation der Navier--Stokes-Gleichungen verwendet. Die chemische Reaktion wird mit einem einstufigen Verbrennungsmodell mit variablen kinetischen Parametern modelliert, das speziell auf die Verbrennung von Kohlenwasserstoffen zugeschnitten ist. Die Temperatur- und Konzentrationsabhängigkeit der Dichte-, Wärmekapazitäts- und Transportparameter wird bei der Formulierung berücksichtigt. 

Eine detaillierte Darstellung der in dieser Studie verwendeten Gleichungen wird zusammen mit einer umfassenden Diskussion ihrer Herleitung und der damit verbundenen Annahmen präsentiert.
Das allgemeine Verfahren für die zeitliche und räumliche Diskretisierung mit der DG-Methode wird anhand einer allgemeinen Transportgleichung erläutert. Anschließend wird die DG-Diskretisierung der Gleichungen für reaktive Strömungen vorgestellt, und die verwendeten numerischen Flüsse werden beschrieben. 

Die entwickelten Berechnungsmethoden zur Lösung der herrschenden Gleichungen werden im Detail vorgestellt. Insbesondere wird die Strategie zur Lösung des nichtlinearen Problems mit Hilfe der globalisierten Dogleg-Newton-Methode erläutert, zusammen mit einer effizienten Methode zur Berechnung der Jacobimatrix. Darüber hinaus werden verschiedene Strategien vorgestellt, die die Konvergenzeigenschaften des Algorithmus verbessern. Dazu gehören eine vollautomatisierte Homotopie-Fortsetzungsmethode für die Lösung von stark nichtlinearen Systemen, eine adaptive Netzverfeinerungsstrategie, die für adäquate Netze in kritischen Bereichen der Simulation verwendet wird, und eine Solver-Safeguard zur Vermeidung von unphysikalischen Lösungen während der Berechnung. 

Für stationär reaktive Strömungen wird eine zusätzliche Strategie verwendet, die es ermöglicht, geeignete Anfangsschätzungen zu finden, die für die Simulation einer Diffusionsflamme verwendet werden können. Dieser Ansatz erfordert die Lösung eines vereinfachten Satzes von Gleichungen, die unter der Annahme einer unendlich schnellen chemischen Reaktion aufgestellt werden, und ist eine robuste Methode zur Lösung von Verbrennungssystemen.

Eine gründliche Validierung des Lösers anhand mehrerer Testfälle wird gezeigt, wodurch auch zentrale Vorteile der DG-Methode und der in dieser Arbeit vorgestellten Algorithmen hervorgehoben werden können. Die Testfälle ermöglichen es, den Löser gegen verschiedene Benchmark-Lösungen zu validieren, wobei Ergebnisse in sehr guter Übereinstimmung mit der Literatur erhalten werden. Darüber hinaus wird die Genauigkeit der Methode in verschiedenen Strömungssituationen bewertet, wobei für alle die erwarteten hohen Konvergenzraten der \Gls{DG}-Methode erzielt werden.  Stabilitätsprobleme werden jedoch bei instationären Simulationen von low-Mach-Strömungen beobachtet, bei denen die Dichte große Schwankungen aufweist.
